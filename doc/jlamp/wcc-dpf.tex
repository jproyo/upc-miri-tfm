%
\section{Enumerating Weakly Connected Component on the DPF}\label{sec:wcc-dpf}
This section presents the most relevant details of the implementation of the $\dpwcc$ using DPF-Haskell. The $\dpwcc$ implementation has been made as a proof of concept to understand and explore
the limitations and challenges that could be found in the development of a \acrshort{dpf} in \acrshort{hs}.  Indeed, the development of the $\dpwcc$ is as general as possible and it uses most of the constructs and abstractions required by the component \acrshort{idl}. In Section 4, we emphasize that the focus of \acrshort{dpf} in Haskell is on the \acrshort{idl} component. Next, we present the minimal code needed for encoding any \acrshort{dp} using \acrshort{dpfh}. \footnote{All the code that we expose here can be accessed publicly in \url{https://github.com/jproyo/dynamic-pipeline/tree/main/examples/Graph}}

\begin{listing}[H]
  \begin{minted}[fontsize=\fontsize{10}{11}\selectfont,numbers=left,breaklines,frame=lines,framerule=2pt,framesep=2mm,baselinestretch=1.2,highlightlines={1-3,6}]{haskell}
    
    type DPConnComp = Source (Channel (Edge :<+> ConnectedComponents :<+> Eof))
                :=> Generator (Channel (Edge :<+> ConnectedComponents :<+> Eof))
                :=> Sink

    program :: FilePath -> IO ()
    program file = runDP $ mkDP @DPConnComp (source' file) generator' sink'
        
  \end{minted}
  \caption[{[\mintinline{shell}{ConnectedComp.hs}] Main entry point of the program}]{In this code we can appreciate the main construct of our $\dpwcc$ which is a combination of $\iwc$, $\gwc$ and $\owc$}
  \label{src:dpwcc:1}
\end{listing}

In \autoref{src:dpwcc:1}, there are two important declarations. First, the \textit{Type Level} declaration of the $\dpwcc$ to indicate \acrshort{dpfh} how our stages are going be connected, and
using that \textit{Type Level} construct, we use the \acrshort{idl} to allow the framework interpret the type representation of our \acrshort{dp} and ensuring at compilation time that we provide the correct stages,  \textit{Source} ($\iwc$), \textit{Generator} ($\gwc$) and \textit{Sink} ($\owc$), that matches those declaration. As we can see in \autoref{src:dpwcc:1}, highlighted lines $1$-$3$ matches one to one the definition in \autoref{fig:dp_example_0}, although in the case of the framework it is not required to provide \textit{Filter} ($\fwc$) definition because \acrshort{dpfh} will deducted from \mintinline[breaklines]{haskell}{Generator}.
According to this declaration what we need to provide is the correct implementation of \mintinline{haskell}{source'}, \mintinline{haskell}{generator'} and \mintinline{haskell}{sink'}
which \textit{Type checked} the \acrshort{dp} type definition\footnote{The names of the functions are completely chosen by the user of the framework and it should not be confused with the internal framework combinators.}.

\begin{listing}[H]
  \begin{minted}[fontsize=\fontsize{10}{11}\selectfont,numbers=left,breaklines,frame=lines,framerule=2pt,framesep=2mm,baselinestretch=1.2,highlightlines={4-5,8}]{haskell}
    
    source' :: FilePath
            -> Stage
              (WriteChannel Edge -> WriteChannel ConnectedComponents -> DP st ())
    source' filePath = withSource @DPConnComp
      $ \edgeOut _ -> unfoldFile filePath edgeOut (toEdge . decodeUtf8)

    sink' :: Stage (ReadChannel Edge -> ReadChannel ConnectedComponents -> DP st ())
    sink' = withSink @DPConnComp $ \_ cc -> withDP $ foldM_ cc print

    generator' :: GeneratorStage DPConnComp ConnectedComponents Edge st
    generator' =
      let gen = withGenerator @DPConnComp genAction
      in  mkGenerator gen filterTemplate
        
  \end{minted}
  \caption[{[\mintinline{shell}{ConnectedComp.hs}] $\iwc$, $\gwc$ $\owc$ Code}]{In this code we can appreciate the $\iwc$, $\gwc$ and $\owc$ functions that matches the type level definition of the $\DP$. $\iwc$ and $\owc$ are completely trivial but $\gwc$ will be analyzed later due to its internal complexity.}
  \label{src:dpwcc:2}
\end{listing}

As we appreciate in \autoref{src:dpwcc:2}, $\iwc$ and $\owc$ are trivial. In the case of \mintinline{haskell}{source'} the only work it needs to do is to read the input data edge by edge and downstream to the next stages. 
That process is achieved with a \acrshort{dpfh} combinator called \mintinline{haskell}{unfoldFile} which is a \emph{catamorphism} of the input data to the stream.
\mintinline{haskell}{sink'} delivers to the output of the program the upstream connected components received from previous stages. $\owc$ implementation is done using an \emph{anamorphism} combinator provided by the framework as well, which is \mintinline{haskell}{foldM_}.
The $\gwc$ Stage is a little more complex because it contains the core of the algorithm explained in \autoref{prole}. According to what we described in \autoref{sec:dp}, \textit{Generator} stage spawns a \textit{Filter} on each received edge in our case of $\dpwcc$.
Therefore, it needs to contain that recipe on how to generate a new \textit{Filter} instance -- in our case of \acrshort{hs} it is a defunctionalized Data Type or Function --. 
Then, there are two important functions to describe: \mintinline{haskell}{genAction} which tells how to spawn a new \textit{Filter} and under what circumstances, and \mintinline{haskell}{filterTemplate} which carries the function to be spawn.

\begin{listing}[H]
  \begin{minted}[fontsize=\fontsize{10}{11}\selectfont,numbers=left,breaklines,frame=lines,framerule=2pt,framesep=2mm,baselinestretch=1.2,highlightlines={8-10}]{haskell}
    
    genAction :: Filter DPConnComp ConnectedComponents Edge st
              -> ReadChannel Edge
              -> ReadChannel ConnectedComponents
              -> WriteChannel Edge
              -> WriteChannel ConnectedComponents
              -> DP st ()
    genAction filter' readEdge readCC _ writeCC = do
      let unfoldFilter = mkUnfoldFilterForAll filter' toConnectedComp readEdge (readCC .*. HNil) 
      results <- unfoldF unfoldFilter
      foldM_ (hHead results) (`push` writeCC)
        
  \end{minted}
  \caption[{[\mintinline{shell}{ConnectedComp.hs}] Generator Action Code}]{In this code we can appreciate the Generator Action code which will expand all the filters in runtime in front of it and downstream all the connected components calculated for those, to the Sink}
  \label{src:dpwcc:3}
\end{listing}

\acrshort{dpfh} provides several combinators to help the user with the \textit{Generator} code, in particular with the spawning process as it has been describe in \autoref{dp-hs}.
\mintinline{haskell}{genAction} for $\dpwcc$ will use the combinator \mintinline{haskell}{mkUnfoldFilterForAll} which will spawn one \textit{Filter} per received edge in the channel, expanding dynamically the stages on runtime.
In line $10$, we can appreciate how after expanding the filters, the generator will downstream to the \textit{Sink}, the received Connected Components calculated from previous filters.

\begin{listing}[H]
 \scriptsize{
  \begin{minted}[fontsize=\fontsize{10}{11}\selectfont,numbers=left,breaklines,frame=lines,framerule=2pt,framesep=2mm,baselinestretch=1.2,highlightlines={2,11-15,24-31}]{haskell}
    
    filterTemplate :: Filter DPConnComp ConnectedComponents Edge st
    filterTemplate = actor actor1 |>> actor actor2
    
    actor1 :: Edge
           -> ReadChannel Edge
           -> ReadChannel ConnectedComponents
           -> WriteChannel Edge
           -> WriteChannel ConnectedComponents
           -> StateT ConnectedComponents (DP st) ()
    actor1 _ readEdge _ writeEdge _ = 
      foldM_ readEdge $ \e -> get >>= doActor e
     where
      doActor v conn
        | toConnectedComp v `intersect` conn = modify' (toConnectedComp v <>)
        | otherwise = push v writeEdge
    
    actor2 :: Edge
           -> ReadChannel Edge
           -> ReadChannel ConnectedComponents
           -> WriteChannel Edge
           -> WriteChannel ConnectedComponents
           -> StateT ConnectedComponents (DP st) ()
    actor2 _ _ readCC _ writeCC = do 
      foldWithM_ readCC pushMemory $ \e -> get >>= doActor e
    
     where
       pushMemory = get >>= flip push writeCC
    
       doActor cc conn
        | cc `intersect` conn = modify' (cc <>)
        | otherwise = push cc writeCC
    
  \end{minted}
  }
  \caption[{[\mintinline{shell}{ConnectedComp.hs}] Filter Template Code}]{Filter template code composed by the two (sequential) actors corresponding to the two phases of the $\dpwcc$ algorithm.}
  \label{src:dpwcc:4}
\end{listing}

Finally, the \textit{Filter} template code is defined in \autoref{src:dpwcc:4}. 
As we have seen in \autoref{prole}, $\dpwcc$ \textit{Filter} is composed of two actors. The first actor collects all the possible vertices that are connected to some vertex in the filter's set of connected vertices. Once the filter does not receive any more edges, this actor starts downstream the filter's set of vertices to the following stages. \mintinline{haskell}{actor2} -- according to algorithm in \autoref{prole} -- passes sets of connected vertices  to the following stage. As we show in \autoref{src:dpwcc:4} with the help of the \acrlong{dpfh}, building a \acrshort{dp} algorithm like \acrshort{wcc} enumeration consist in few lines of codes with the \textit{Type Safety} that \acrshort{hs} provides.



