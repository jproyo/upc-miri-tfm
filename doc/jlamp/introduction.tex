\section{Introduction}\label{intro}
Effective streaming processing of large amounts of data has been studied for several years \cite{exploiting, onthefly}  as a critical factor providing fast and incremental results in big data algorithmic problems. 
Parallel techniques that exploit computational power as much as possible represent one of the most explored techniques, regardless of the approach.
In that regard, the \acrfull{dp} \cite{dpdef} has lately emerged as one of the models that exploit data streaming processing using a dynamic pipeline parallelism approach \cite{onthefly}. 
This computational model has been designed with a functional focus. The main components of this paradigm are functional stages or pipes that dynamically enlarge and shrink depending on incoming data.  
%The details of the specific requirements of the Haskell system according to the results of the proof of concept will be presented in \autoref{dp-hs}.

One of the biggest challenges of implementing a \acrfull{dpf} is to find a proper set of tools and a programming language which can take advantage of both of its primary aspects: \begin{inparaenum}[i\upshape)]
\item  \emph{fast parallel} processing and 
\item  \emph{strong theoretical} foundations that manage computations as first-class citizens.
 \end{inparaenum}
Haskell is a statically typed pure functional language  
designed and evolved on solid theoretical foundations where computations are primary entities. At the same time, Haskell makes available a robust set of tools for writing multithreading and parallel programs with optimal performance \cite{parallelbook, monadpar}.

In the context of this research, we first assess the suitability of Haskell  to implement \acrshort{dp} solutions to stream processing problems.
To be concrete, we conduct a proof of concept implementing a Dynamic Pipeline in  Haskell for solving a particular and very relevant problem as the computation/e\-nu\-me\-ra\-tion of the weakly connected components of a graph.
In particular, the main objective of our proof of concept is to study the critical features required in \acrshort{hs} for a \acrshort{dpf} implementation,  the real possibilities of emitting incrementally results, and the performance of such kinds of im\-ple\-men\-ta\-tions. 
Indeed, we explore the basis of an implementation of a \acrshort{dpf}  in a pure (parallel) functional language as Haskell. This is, we determine the particular features (i.e., versions and libraries) that will allow for an efficient implementation of a \acrshort{dpf}. Moreover, we conduct an empirical evaluation to analyze the performance of the Dynamic Pipeline implemented in Haskell for enumerating \acrshort{wcc}. To assess the incremental delivery of results, we measure the Diefficiency metrics \cite{diefpaper}, i.e., the continuous efficiency of the implementation of an algorithm for generating incremental results.
Since continuous performance results are encouraging and the programming basis is clearly stated, we develop a general \acrlong{dpf} written in \acrlong{hs} which allows for implementing algorithms under the \acrlong{dp} approach. Then, we conduct, analyze, and report experiments to measure the performance of using this framework to compute weakly connected components incrementally w.r.t. the \textit{ad hoc}  Dynamic Pipeline solution.  
Obtained results satisfy our expectations and encourage us to keep using the Dynamic Pipeline Framework for solving some families of graph stream processing problems where it is critical not to have to wait until the whole results are emitted. Furthermore, the experiments give us insights about the characteristics this family of problems should meet.

\textbf{Problem Research and Objective:}\label{research:obj} The main objective of this work is to design and implement a Dynamic Pipeline Framework using Haskell as programming language. Through a particular and very relevant problem as the computation of the \acrfull{wcc} of a graph, we study the critical features required in \acrshort{hs} for a \acrshort{dpf} implementation and set the basis of an implementation of a \acrshort{dpf}  in \acrshort{hs}. 
This is,  we determine the particular features (i.e., versions and libraries) of this language that will allow for an efficient implementation of the \acrshort{dpf}. 


\textbf{Contributions:} 
This paper is an extension to our work previously reported by Royo-Sales et al. \cite{prole}; we present a solution (a.k.a. \acrshort{dpwcc}) to the problem of enumerating the weakly connected components of a graph as a dynamic pipeline implemented in \acrshort{hs}. We also empirically show the behavior of \acrshort{dpwcc} concerning a solution for enumerating the weakly connected components using \acrshort{hs} libraries; we name this solution \acrfull{blwcc}. 
Additionally,  we present the following novel contributions in this paper:
\begin{inparaenum}[i\upshape)]
\item the Dynamic Pipeline Framework (DPF-Haskell) implemented in Haskell. 
\item A program for enumerating weakly connected component implemented on top of the Dynamic Pipeline Framework (\acrshort{dpfh}).
\item An empirical evaluation, comparing the performance of 
\acrfull{dpfhwcc} with respect to \acrshort{blwcc} and \acrfull{dpwcc}. 
The continuous  performance of these im\-ple\-men\-ta\-tions is measured in terms of the diefficiency metric dief\@t \cite{diefpaper}. The results of this study suggests that  \acrshort{hs} is a suitable language for implementing \acrshort{dp} and the programming basis for implementing DP solutions is settled. Lastly, it has been shown that implementing \acrshort{dpfhwcc} performs well with respect to \acrshort{dpwcc}. 
Hence, we envision that the overhead of \acrshort{dpfh} does not degrade the performance of dynamic pipelines implemented on the framework. 
\end{inparaenum}

The rest of this paper is organized as follows. 
Section \ref{section:related-work} analyzes the related work and positions the \acrlong{dpfh} with respect to existing approaches. Section \ref{prelim} describes basic notions used in this work, the dynamic pipeline paradigm, the problem of weakly connected components, and the baseline solution  \acrshort{dpwcc}.  
Section \ref{dp-hs}  defines the \acrlong{dpfh} 
and details the most relevant points of its Haskell implementation. Additionally, the implementation of the weakly connected component on top of  \acrlong{dpfh} (a.k.a. \acrfull{dpfhwcc}) is sketched in \autoref{sec:wcc-dpf}. Section \ref{sec:new:eval} reports the results of the empirical evaluation and compares the performance and continuous behavior of the solutions. Finally, \autoref{conc} summaries our conclusions and outlooks our future work. 
