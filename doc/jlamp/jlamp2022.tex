\documentclass[preprint]{elsarticle}
\usepackage[utf8]{inputenc}
\usepackage[T1]{fontenc}
\usepackage{amsmath}
\usepackage{amsfonts}
\usepackage{amssymb}
\usepackage{amsthm}
\usepackage{array}
\usepackage{parskip}
\usepackage{wrapfig}
\usepackage{float}
\usepackage{paralist}
\usepackage{listings}
\usepackage{babel}
\usepackage{color}
\usepackage{caption}
\usepackage{multirow}
\usepackage{caption}
\usepackage{subcaption}
\usepackage{tikz}
\usepackage[a4paper, total={6in, 8in}]{geometry}
\usepackage{graphicx}
\usepackage[pdfencoding=auto]{hyperref}
\usepackage{xcolor}
\usepackage{colortbl}
\usepackage[ruled,linesnumbered,lined,boxed,commentsnumbered]{algorithm2e}
\usepackage[acronym]{glossaries}
\usepackage[nottoc]{tocbibind}
\usepackage[cache=true, section]{minted}
\usepackage{tikz}
\usetikzlibrary{shapes.misc,shadows}
\usetikzlibrary{quotes,positioning,arrows,decorations.markings}
\usetikzlibrary{positioning} 
\usemintedstyle{default}
\newminted{haskell}{frame=lines,framerule=2pt}
\newminted{R}{frame=lines,framerule=2pt}

%

\graphicspath{{./images/}}

\bibliographystyle{abbrvnat}

\glsdisablehyper

\usetikzlibrary{shapes.misc,shadows}
\usetikzlibrary{quotes,positioning,arrows,decorations.markings}
\usetikzlibrary{positioning} 
\usemintedstyle{default}
\newminted{haskell}{frame=lines,framerule=2pt}
\newminted{R}{frame=lines,framerule=2pt}
\graphicspath{{./images/}}
\tikzstyle{bag} = [align=center]
\newcommand{\dw}{\mathbb{DW}}
\newcommand{\aw}{\mathbb{AW}}
\newcommand{\bt}{\mathbb{BT}}
\newcommand{\bti}{BT_{(l_1, l_2,l_3)}^{(u_1, u_2, u_3)}}
\newcommand{\at}{\mathbb{AT}}
\newcommand{\st}{ST}
\newcommand{\sw}{\mathtt{spawn}}
\newcommand{\fd}{\mathtt{killFilter}}
\newcommand{\fid}{\mathtt{filterIsDied}}
\newcommand{\us}{\mathtt{updateState}}
\newcommand{\gs}{\mathtt{getState}}
\newcommand{\p}{\mathtt{push}}
\newcommand{\mt}{\mathtt{matchQ}}
\newcommand{\io}{\mathtt{indexOf}}
\newcommand{\la}{\left\langle}
\newcommand{\ra}{\right\rangle}
\newcommand{\DP}{\mathsf{DP}}
\newcommand{\dpwcc}{\mathsf{DP_{WCC}}}
\newcommand{\iwcc}{\mathsf{Sr}}
\newcommand{\iwc}{\mathsf{Sr_{WCC}}}
\newcommand{\owcc}{\mathsf{Sk}}
\newcommand{\owc}{\mathsf{Sk_{WCC}}}
\newcommand{\fwcc}{\mathsf{F}} 
\newcommand{\fwc}{\mathsf{F_{WCC}}} 
\newcommand{\gwcc}{\mathsf{G}}
\newcommand{\gwc}{\mathsf{G_{WCC}}}
\newcommand{\ice}{\mathsf{IC_E}}
\newcommand{\csofv}{\mathsf{IC_{set(V)}}}
\newcommand{\sgen}{\mathsf{S_G}}
\newcommand{\sfilter}{\mathsf{S_F}}
\newcommand{\sinp}{\mathsf{S_I}}
\newcommand{\sout}{\mathsf{S_O}}
\newcommand{\istream}{\mathsf{D}}
\newcommand{\wccout}{\mathsf{R}}
\newcommand{\fmem}{\mathsf{M_F}}
\newcommand{\eof}{\mathsf{eof}}
\newcommand{\Act}{\mathsf{actor_1}}
\newcommand{\Actt}{\mathsf{actor_2}}
\newcommand{\gdsl}{G_{dsl}}

\renewcommand\listingscaption{Source Code}
\providecommand*{\listingautorefname}{Source Code}

\DeclareMathOperator*{\argmax}{arg\,max}
\DeclareMathOperator*{\argmin}{arg\,min}

\newacronym{prole21}{PROLE21}{Jornadas de la Sociedad de Ingeniería de Software y Tecnologías de Desarrollo de Software}
\newacronym{bt}{BT}{Bitriangle}
\newacronym{bg}{BG}{Bipartite Graph}
\newacronym{ug}{UG}{Unipartite Graph}
\newacronym{dp}{DPP}{Dynamic Pipeline Paradigm}
\newacronym{dpf}{DPF}{Dynamic Pipeline Framework}
\newacronym{dpfh}{DPF-Haskell}{Haskell Dynamic Pipeline Framework}
\newacronym{ds}{DS}{Data Streaming}
\newacronym{dap}{DAP}{Data Parallelism}
\newacronym{pip}{PIP}{Pipeline Parallelism}
\newacronym{mr}{MR}{MapReduce}
\newacronym{er}{ER}{Entity Resolution}
\newacronym{hack}{Hackage}{The Haskell Package Repository}
\newacronym{dpbt}{DP-BT-Haskell}{$DP_{BT}$ in Haskell}
\newacronym{dpwcc}{DP-WCC-Haskell}{$DP_{WCC}$ in Haskell}
\newacronym{dpl}{DPL}{dynamic-pipeline}
\newacronym{bfs}{BFS}{Breadth-First Search}
\newacronym{dfs}{DFS}{Depth-First Search}
\newacronym{wcc}{WCC}{Weak Connected Components}
\newacronym{hs}{Haskell}{Haskell Programming Language}
\newacronym{fp}{FP}{Functional Programming}
\newacronym{stm}{STM}{Software Transactional Memory}
\newacronym{rl}{R}{R Language}
\newacronym{os}{OS}{Operative System}
\newacronym{dm}{Dm}{Diefficency Metrics}
\newacronym{tfft}{TFFT}{Time for the first tuple}
\newacronym{et}{ET}{Execution Time}
\newacronym{comp}{Comp}{Completeness}
\newacronym{tt}{T}{Throughput}
\newacronym{dt}{dief$@$t}{Diefficiency first $t$ time units}
\newacronym{snap}{SNAP}{Stanford Network Data Set Collection}
\newacronym{ghc}{GHC}{Glasgow Haskell Compiler}
\newacronym{dsl}{DSL}{Domain-specific Language}
\newacronym{edsl}{EDSL}{Embedded Domain-specific Language}
\newacronym{idl}{IDL}{Interpreter of DSL}
\newacronym{rs}{RS}{Runtime System}
\newacronym{go}{Go}{Go Programming Language}
\newacronym{hl}{HL}{Host Language}

\glsdisablehyper

\newtheorem{hyp}{Hypothesis}


%

\title{A Dynamic Pipeline Framework implemented in Haskell\tnoteref{t1}}

\tnotetext[t1]{This work was partially supported by  MCIN/ AEI /10.13039/501100011033 [grant number PID2020-112581GB-C21].}
%
\author[1]{Juan Pablo Royo Sales}
\ead{juan.pablo.royo@estudiantat.upc.edu}

\author[1]{Edelmira Pasarella}
\ead{edelmira@cs.upc.edu}

\author[1]{Cristina Zoltan}
\ead{zoltan@cs.upc.edu}

\author[2]{Maria-Esther Vidal}
\ead{maria.vidal@tib.eu}

\affiliation[1]{organization={Universitat Politècnica de Catalunya},
postcode={08034},
city={Barcelona},
country={Spain}}

\affiliation[2]{organization={TIB/L3S Research Centre at the University of Hannover},
city={Hannover},
country={Germany}}

\begin{document}

\begin{abstract}
Streaming processing has given rise to new computation paradigms to provide effective and efficient data stream processing. 
The Dynamic Pipeline Paradigm is a computational model for stream processing. In particular, this paradigm is suitable to solving problems where the incremental emission of results is a critical  issue. In the implementation of a Dynamic Pipeline Framework computations  must correspond to natural primary entities. This fact suggests that a proper im\-ple\-men\-ta\-tion language  must allow for manipulating functions  as first citizens as implementation language.  Haskell accomplishes this requirement. It is a pure functional programming language relaying on solid theoretical foundations that provides the possibility of manipulating computations as primary entities. Moreover, from a practical point of view, it has a robust set of tools for writing multithreading and parallel computations with optimal performance.  In this work we tackle the problem of specifying and developing a general and algorithm-parametric Dynamic Pipeline Paradigm using Haskell as development language. Additionally, we conduct, analyze and report experiments to measure the performance of computing the weakly connected components of large networks using the Dynamic Pipeline Framework.  To assess the incremental delivery of results we measure the Diefficiency metrics, i.e. the continuous efficiency of the implementation of an algorithm for generating incremental results. Obtained results satisfy our expectations and encourage us to keep using the Dynamic Pipeline Framework for solving some graph stream processing problems 
where is critical not to have to wait until the whole results are emitted. 
%The most important features of these new paradigms are the exploitation of parallelism, the capacity to adapt execution schedulers, reconfigure computational structures, adjust the use of resources according to the characteristics of the input stream and produce  results incrementally. 
%This is a relevant feature of functional programming languages. 
%We think that a pure functional programming language like Haskell  is suitable for solving stream processing problems using the DPP approach.  The justification of this choice is twofold. From a formal point of view, Haskell has solid theoretical foundations providing the possibility of manipulating computations as primary entities and  formally prove the correctnes of implementations. From a practical perspective,  it has a robust set of tools for writing multithreading and parallel computations with optimal performance.  

%Based on the encouraging results obtained in a a proof of concept to assess the suitability of using Haskell to implement a Dynamic Pipeline solution for the problem of  compute incrementally the weakly connected components of a graph parallel

%In this work, we firstly  conduct a proof of concept to assess the suitability of using Haskell to implement a Dynamic Pipeline Framework. to be concrete, we implement a dynamic pipeline to compute incrementally the weakly connected components of a graph parallel Haskell and  empirically evaluated and compared it performance with a solution provided by a Haskell library. The results of the experiment  are competitive with the baseline solution available in a graph algorithm Haskell library.

%As proof of concept, we present an implementation of a dynamic pipeline to compute the weakly connected components of a graph (WCC) in Haskell (a.k.a. $\dpwcc$). The  $\dpwcc$ behavior is empirically evaluated and compared with a solution provided by a Haskell library. The evaluation is assessed in three networks of different sizes and topology. Performance is measured in terms of the time of the first emitted result, continuous generation of results, total time, and consumed memory. The results suggest that $\dpwcc$, even naive, is competitive with the baseline solution available in a Haskell library. In particular, $\dpwcc$ exhibits a higher continuous behavior and can produce the first result faster than the baseline. Once the proof of concept put in perspective the suitability of Haskell's abstractions for the implementation of DPF, we tackle the problem of specifying and developing a general and parametric DPF. 
%Additionally, we conduct, analyze and report experiments to measure the performance of computing the weakly connected components of large networks using the Dynamic Pipeline Framework. %having up to X millions of weakly connected components.
%Obtained results satisfy our expectations. To assess the incremental delivery of results we measure the Diefficiency metrics, i.e. the continuous efficiency of the implementation of an algorithm for generating incremental results.

%Finally, we conduct experiments to measure the performance of computing the weakly connected components of a graph implemented on the DPF. The obtained results of these experiments encourage us to use the DPF for benchmarking some important graph stream processing problems requiring results are produced incrementally. This is the case of hard problems as mining small patterns of graphs in which is critical not to have to wait until the whole results are emitted.

\end{abstract}

%Research highlights
%\begin{highlights}
%\item Research highlight 1
%\item Research highlight 2
%\end{highlights}

\begin{keyword}
Stream Processing Frameworks \sep Dynamic Pipeline \sep Parallelism \sep Concurrency \sep Haskell
\end{keyword}

\maketitle
%
\tableofcontents
%
\section{Introduction}\label{intro}
Effective streaming processing of large amounts of data has been studied for several years \cite{enumeratingsg, exploting, onthefly} 
as a key factor providing fast and incremental results in big data algorithmic problems. One of the most explored techniques, 
regardless of the approach, is the exploitation of parallel techniques to take advantage of the available computational power as much as possible. 
In that regard, the \acrfull{dp} \cite{dpdef} has lately emerged as one of the models that exploit data streaming processing using a dynamic pipeline parallelism approach \cite{onthefly}. 
This computational model has been designed with a functional focus, where the main components of the paradigm are functional stages or pipes which dynamically enlarge and shrink depending on incoming data.  
The details of the specific requirements of the Haskell system according to the results of the proof of concept will be presented in \autoref{dp-hs}.

One of the biggest challenges of implementing a \acrfull{dpf} is to find a proper set of tools and programming language which can take advantage of both of its primary aspects: \begin{inparaenum}[i\upshape)]
\item  \emph{fast parallel} processing and 
\item  \emph{strong theoretical} foundations that manage computations as first-class citizens.
 \end{inparaenum}
\acrfull{hs} is a statically typed pure functional language that has been designed and evolved from its birth in 1987, 
on strong theoretical foundations where computations are primary entities, and at the same time has been providing a powerful set of tools for writing multithreading and parallel programs with optimal performance \cite{parallelbook, monadpar}.

\textbf{Problem Research and Objective:}\label{research:obj} The main objective of this work is to explore the feasibility of using a  \acrfull{fp} language to implement a \acrshort{dpf}. In particular, we tackle the problem of establishing the basis of an implementation of a \acrshort{dpf}  in \acrshort{hs}, a pure functional language. 
This is,  our aim is to determine the particular features (i.e., versions and libraries) of this language that will allow for an efficient implementation of the \acrshort{dpf}. 
To be concrete, through a particular and very relevant problem as the computation of the \acrfull{wcc} of a graph,  we study the critical features required in \acrshort{hs} for a \acrshort{dpf} implementation.

\textbf{Contributions:} A proof of concept of the implementation of a $\DP$ for \acrshort{wcc} using \acrshort{hs}; the results of the empirical study suggests that   \acrshort{hs} is a suitable language for implementing \acrshort{dp}. This work also contributes to building the first abstraction approximation of a future framework/library of this computational model in that language. 

The rest of this paper is organized as follows. 
Next section  presents  the basic notions used through this work. 
In \autoref{prole}  the proof of concept is analyzed. To be concrete, in this section an algorithm for enumerating \acrshort{wcc} using \acrshort{dp} and its implementation using parallel \acrshort{hs} are introduced. 
In \autoref{sec:evaluation}, the experiments conducted to assess how  Haskell supports the dynamic pipeline implementation for enumerating \acrshort{wcc} are described and their results reported. 
In \autoref{dp-hs} the design of the \acrlong{dpfh} and the most relevant details of its implementation are deeply explained. In particular, all the \acrshort{hs} data types and language techniques used in the implementation are detailed. 
The justification of the used  external libraries for the runtime system are presented at the end of this section. Section \ref{sec:wcc-dpf} the results of the experiments conducted to evaluate the performance of  using the \acrshort{dpf} are presented.
In \ref{section:related-work} we present the related work and, finally, conclusions and further work are presented in  \autoref{conc}. 
Part of the content presented through this work has been published in \cite{prole}. 

\section{Preliminaries}\label{prelim}
Before moving forward with the core of the research, we describe the fundamental concepts that support the different parts of our study. That is, the metrics used to capture continuous behavior, pipeline parallelism processing in general, and the dynamic pipeline paradigm. Next, we present an implementation of weakly connected components in Haskell following this paradigm. 
\subsection{Basic Concepts}
%
\subsubsection{Diefficiency Metrics}\label{prem:dief}
In this work, we use two relevant metrics to measure the diefficiency, i.e., continuous efficiency of a program to generate incremental results. The metrics to measure diefficiency are \acrshort{dt} and \acrshort{dk}~\cite{diefpaper}. The metric \acrshort{dt} quantifies the continuous efficiency during the first $t$ time units of execution regarding the results generated by the program. The higher value of the \acrshort{dt} metric, the better the continuous behavior. The metric \acrshort{dk} measures the continuous efficiency while producing the first $k$ answers. The lower the value of the \acrshort{dk} metric, the better the continuous behavior.
Both metrics can be computed using \acrfull{dm} Tool \acrshort{dtkp}\footnote{\url{https://github.com/SDM-TIB/diefpy/}}, given the traces of the execution of each of the approaches.
Additionally, \acrshort{dtkp} generates two different kinds of plots from an execution trace: A bi-dimensional plot  and a radial plot. In the bi-dimensional plot, the x-axis represents the time when answers  are generated and the y-axis represents the number of generated answers. Points  $(x,y)$  are taken from traces.  The radial plot contains the visual comparison of \acrshort{dt} the metric with respects to other non-continuos metrics, such as, 
\begin{inparaenum}[\bf i\upshape)]
  \item \acrfull{comp} which is the total number of  produced answers. 
  \item \acrfull{tfft} which measure the elapsed time spent to produce the first answer. 
  \item \acrfull{et} which measures the elapsed time spent  to complete the execution of a query. 
  \item \acrfull{tt} which measure the number of total answers produced after evaluating a query divided by its execution time \acrshort{et}
\end{inparaenum}.

\subsubsection{\acrfull{pip}} \label{prelim-pip}
A pipeline parallelism approach divides a process computation into sequential stages that are stateful operators~\cite{hr19}. Each stage takes the result of the previous one as an input and downstream its results to the next stage. Each \emph{Pipeline Stage} is parallelized. The communication between stages takes place through some means, typically channels. One of the main advantages of this model is that the stages are non-blocking, i.e., there is no need to wait to process all data to run the next stage. This kind of paradigm enables computational algorithms that can generate incremental results, preventing users wait until the end of the whole data stream processing to get a result. This feature corresponds with the nature of the Dynamic Pipeline Paradigm. Hence, the \acrshort{pip} approach is a proper parallelization streaming computational model for a Dynamic Pipeline Framework implementation. It is worth noting that although pipeline stages are parallelized in the \acrshort{pip} approach,  unbalanced intensive computation in one stage w.r.t. the other ones might delay the whole processing as a natural consequence of the sequential dependency among stages. As a result, users must be sure each stage runs high-speed computations. 
\subsubsection{Dynamic Pipeline Paradigm}\label{sec:dp}
The \textit{Dynamic Pipeline Paradigm} (DPP) \cite{dpdef} is a \acrshort{pip} computational model based on a one-dimensional and unidirectional chain of stages connected by means of channels synchronized by data availability. 
This chain of stages is a computational structure called \textit{Dynamic Pipeline} ($\DP$). A $\DP$ stretches and shrinks depending on the spawning and the lifetime of its stages, respectively. Modeling an algorithmic 
solution as a $\DP$ corresponds to define a dynamic computational structure in terms of four kinds of stages: \textit{Source} ($\iwcc$), \textit{Generator} ($\gwcc$), \textit{Sink} ($\owcc$) and \textit{Filter} ($\fwcc$) stages. 
In particular, the specific behavior of each stage to solve a particular problem must be defined, as well as the number and the type of channels connecting them. Channels are unidirectional according to the flow of the data. 
The \textit{Generator} stage is in charge of spawning \textit{Filter} stage instances. This particular behavior of the \textit{Generator} gives the elastic capacity to DPs. \textit{Filter} stage instances are stateful operators in the 
sense described in \cite{hr19}, i.e., \textit{Filter} instances have a state.  
The deployment of a $\DP$ consists in setting up the initial configuration depicted in \autoref{fig:initialDP}. 
%\begin{wrapfigure}{r}{0.5\textwidth}
\begin{figure}[h]
\centering
\inputtikz{genericDP-Initial}
\caption[{[Pre] Initial configuration of a DP}]{ An initial DP consists of three stages: $\iwcc$, $\gwcc$ together its filter parameter $\fwcc$, and $\owcc$. These stages are connected through its channels --represented by right arrows-- as shown in this figure.}
\label{fig:initialDP}
\end{figure}
%\end{wrapfigure}
The activation of a $\DP$ starts when a stream of data items arrives at the initial configuration of the $\DP$. 
When a data stream arrives to the \textit{Source} stage. During the execution, the \textit{Generator} stage spawns \textit{Filter} stage instances according to incoming data and the \textit{Generator} defined behavior; \autoref{fig:activeDP} depicts this evolution. 
If the data stream is bounded, the computation ends when the lifetime of all the stages of $\DP$ has finished. Otherwise, if the stream data is unbounded, 
the $\DP$ remains active and results are output incrementally. 


\begin{figure}[h]
 \centering
 \inputtikz{genericDP-Evol}
 \caption[{[Pre] Evolution of DP}]{Evolution of a DP. After creating some filter instances (shadow Filter squares) of the filter parameter (light Filter square) in the Generator, the DP has stretched.}
\label{fig:activeDP}
\end{figure}

\iffalse
\subsubsection{Streaming in Haskell Language}
Streaming computational models have been implemented in \acrlong{hs} during the last $10$ years. One of the first libraries in the ecosystem was \mintinline{shell}{conduit}\footnote{\url{https://hackage.haskell.org/package/conduit}} in 2011.
After that, several efforts on improving streaming processing on the language has been made not only at abstraction level for the user but as well as performance execution 
improvements like \mintinline{shell}{pipes}~\footnote{\url{https://hackage.haskell.org/package/pipes}} and \mintinline{shell}{streamly}\footnote{\url{https://hackage.haskell.org/package/streamly}} lately.
Moreover, there is an empirical comparison between those three, where a benchmark analysis has been conducted~\cite{benchstreamhs}.

Although most of those libraries offer the ability to implement \acrshort{dap} and \acrshort{pip}, none of them provide clear abstractions to create \acrshort{dp} models because
the setup of the stages should be provided beforehand. In the context of this work, we have done a proof of concept at the beginning, 
but it was not possible to adapt any of those libraries to implement properly \acrshort{dp}. 
The closest we have been to implement \acrshort{dp} with some of those libraries was when we explored \mintinline{shell}{streamly}.
In this case, there is a \mintinline{haskell}{foldrS} combinator that could have been proper for the purpose of generating a dynamic pipeline of stages based on the data flow. However, it was not possible to manipulate the channels between the stages to control the flow of the data. It is important to remark that, even though, the  library  \mintinline{shell}{streamly} implements channels, they are hidden from the end-user, and there is not a  clear way to manipulate them.

To the best of our knowledge, no similar library under the  \acrshort{dp} approach has been written in \acrlong{hs}. 
One important motivation to develop our own framework is that  we not only  want to satisfy our research needs but, as a novel contribution, we aim at providing a \acrshort{dpf} to the \acrshort{hs} community as well. We hope this contribution encourages and helps writing algorithms under the Dynamic Pipeline Paradigm. 
\fi


\section{Proof of concept: Weakly Connected Components of a Graph}\label{prole}

One of the biggest challenges of implementing a Dynamic Pipeline is to find  a programming language with a proper set of tools supporting both of the  primary features of the \acrshort{dp}: \begin{inparaenum}[i\upshape)]
\item  \emph{parallel} processing and 
\item  \emph{strong theoretical} foundations to manage computations as first-class citizens.
 \end{inparaenum}
Haskell is a statically typed pure functional language designed on strong theoretical foundations where com\-pu\-ta\-tions are primary entities.
This pure functional language has evolved from its birth in 1987 and nowadays provides a powerful set of tools for writing multithreading and parallel programs with optimal performance \cite{parallelbook, monadpar}. 

In the context of this research, we first assessed the suitability of \acrlong{hs}  to implement a Dynamic Pipeline. 
To be concrete, we conducted a proof of concept implementing a Dynamic Pipeline in  Haskell for solving a particular and very relevant problem as the computation/e\-nu\-me\-ra\-tion of the Weakly Connected Components of a graph.
In particular, the main objective of our proof of concept was to study the critical features required in \acrshort{hs} for a \acrshort{dpf} implementation,  the real possibilities of emitting incrementally results, and the performance of such kinds of im\-ple\-men\-ta\-tions. 
Indeed, we explored the basis of an implementation of a \acrshort{dpf}  in a pure (parallel) functional language as Haskell.
This is, we determined the particular features (i.e., versions and libraries) that will allow for an efficient implementation of a \acrshort{dpf}. 
Moreover, we conducted an empirical evaluation to analyze the performance of the Dynamic Pipeline implemented in Haskell for enumerating \acrshort{wcc} with special attention to the emission of results, i.e. \acrshort{wcc}, incrementally. 
In this chapter, we focus on the presentation of our algorithm for enumerating \acrshort{wcc} using \acrshort{hs} as well as the results obtained in the empirical evaluation of its implementation. 
 
\subsection{$\dpwcc$ Algorithm}
Let us consider the problem of computing/enumerating the (weak) connected components of a graph $G$ using \acrshort{dp}. 
A connected component of a graph is a subgraph in which any two vertices are connected by paths.  
Thus, finding connected components of an undirected graph implies obtaining the minimal partition of the set of nodes induced by the relationship \textit{connected}, i.e., there is a path between each pair of nodes. An example of that graph can be seen in \autoref{fig:example_dp_graph}.
The input of the Dynamic Pipeline for computing the WCC of a graph, $\dpwcc$, is a sequence of edges ending with $\eof$. In the source graph there are neither isolated vertices nor loops. The connected components are output as soon as they are computed, i.e., they are produced incrementally. 
Roughly speaking the idea of the algorithm is that the weakly connected components are built in two phases. In the first phase filter instance stages receive the edges of the input graph and create sets of connected vertices. 
During the second phase, these filter instances construct maximal subsets of connected vertices, i.e. the vertices corresponding to (weakly) connected components.
%
$\dpwcc$ is defined in terms of the behavior of its four kinds stages: \textit{Source} ($\iwc$),  \textit{Generator} ($\gwc$),  \textit{Sink} ($\owc$), and \textit{Filter}($\fwc$) stages. Additionally,  the channels connecting these stages must be defined. 
In $\dpwcc$, stages are connected linearly and unidirectionally through the channels $\ice$ and  $\csofv$. Channel $\ice$ carries edges while channel  $\csofv$ conveys sets of connected vertices. Both channels end by the $\eof$ mark. 
The behavior of $\fwc$ is given by a sequence of two actors (scripts). Each actor corresponds to a phase of the algorithm. In what follows, we denote these actors by $\Act$ and $\Actt$, respectively. 
The script $\Act$ keeps a set of connected vertices ($CV$) in the state of the $\fwc$ instance. When an edge $e$ arrives, if an endpoint of $e$ is present in the state, then the other endpoint of $e$ is added to $CV$. 
Edges without incident endpoints are passed to the next stage. When $\eof$ arrives at channel $\ice$, it is passed to the next stage, and the script $\Actt$ starts its execution. 
If script $\Actt$ receives a set of connected vertices $CV$ in $\csofv$, it determines if the intersection between $CV$ and the nodes in its state is not empty. If so, it adds the nodes in $CV$  to its state. 
Otherwise, the $CV$ is passed to the next stage.  Whenever $\eof$ is received, $\Actt$ passes--through $\csofv$-- the set of vertices in its state and the $\eof$ mark to the next stage; then, it dies.
The behavior of $\iwc$ corresponds to the identity transformation over the data stream of edges.  As edges arrive, they are passed through  $\ice$ to the next stage. When receiving $\eof$ on $\ice$, this mark is put on both channels. 
Then, $\iwc$ dies. 

%\begin{wrapfigure}{r}{0.4\textwidth}
\begin{figure}
 \begin{center}
\inputtikz{graph_example_wcc}
\end{center}
\caption[{[PoC] Graph WCC Example}]{Example of a graph with two weakly connected components: $\{1,2\}$ and $\{3,4,5,6\}$}
\label{fig:example_dp_graph}
\end{figure}
%\end{wrapfigure}

Let us describe this behavior with the example of the graph shown in \autoref{fig:example_dp_graph}.

\begin{figure}[h!]
  \centering
\inputtikz{dp_example_0}
\caption[{[PoC] $\dpwcc$ Initial Setup}]{$\dpwcc$ Initial setup. Stages Source, Generator, and Sink are represented by the squares labeled by $\mathsf{Sr_{WCC}}$, $\mathsf{G_{WCC}}$ and $\mathsf{Sk_{WCC}}$, respectively.  The square $\fwc$ corresponding to the Filter stage template is the parameter of $\gwc$. Arrows $\rightrightarrows$ between represents the connection of stages through two channels, $\ice$, and $\csofv$. The arrow  $\rightarrow$ represents the channel $\csofv$ connecting the stages $\mathsf{G_{WCC}}$ and $\mathsf{Sk_{WCC}}$. The arrow $\Longrightarrow$ stands for I/O data flow. Finally, the input stream comes between the dotted lines on the left and the WCC computed incrementally will be placed between the solid lines on the right.}
\label{fig:dp_example_0}
\end{figure}

\autoref{fig:dp_example_0} depicts the initial configuration of $\dpwcc$. 
The interaction of $\dpwcc$ with the "external" world is done through the stages $\iwc$ and $\owc$. 
Indeed, once activated the initial $\dpwcc$, the input stream -- consisting of a sequence containing all the edges in the graph in \autoref{fig:example_dp_graph} -- feeds $\iwc$ while  $\owc$ emits incrementally the resulting weakly connected components.  
In what follows \autoref{fig:dp_example_1_2}, \autoref{fig:dp_example_3_4}, \autoref{fig:dp_example_5_6}, \autoref{fig:dp_example_7_8} and \autoref{fig:dp_example_9_10} depict the evolution of the $\dpwcc$.
 
\begin{figure}[h!]
\centering
\begin{subfigure}[b]{\textwidth}
 \centering
  \inputtikz{dp_example_1}
  \caption{The edge $(1,2)$ is arriving to $\gwc$.}
  \label{fig:dp_example_1_2a}
\end{subfigure}
\vspace{.3cm}

\begin{subfigure}[b]{\textwidth}
 \centering
  \inputtikz{dp_example_2}
  \caption{When the edge $(1,2)$ arrives to $\gwc$, it  spawns a new instance of $\fwc$ before $\gwc$. Filter instance $F_{\{1,2\}}$ is connected to  $\gwc$ through channels $\ice$ and  $\csofv$. The state of the new filter instance $F_{\{1,2\}}$ is initialized with the set of vertices $\{1,2\}$. The edge $(3,6)$ arrives to the new filter instance $F_{\{1,2\}}$.}
  \label{fig:dp_example_1_2b}
\end{subfigure}
\caption[{[PoC] $\dpwcc$ Evolving first state}]{Evolution of the $\dpwcc$: First state}
\label{fig:dp_example_1_2}
\end{figure}
\vspace{.5cm}

\begin{figure}[h!]
\centering
\begin{subfigure}[b]{\textwidth}
 \centering
  \inputtikz{dp_example_3}
  \caption{None of the vertices in the edge $(3,6)$ is in the set of vertices $\{1,2\}$ in the state of $F_{\{1,2\}}$, hence it is passed through $\ice$ to $\gwc$.}
  \label{fig:dp_example_3_4a}
\end{subfigure}
\vspace{.3cm}

\begin{subfigure}[b]{\textwidth}
 \centering
  \inputtikz{dp_example_4}
  \caption{When the edge $(3,6)$ arrives to $\gwc$, it spawns the filter instance $F_{\{3,6\}}$  between $F_{\{1,2\}}$ and $\gwc$. Filter instance $F_{\{1,2\}}$ is connected to the new filter instance $F_{\{3,6\}}$ and this one is connected to  $\gwc$ through channels $\ice$ and  $\csofv$. The state of the new filter instance $F_{\{3,6\}}$ is initialized with the set of vertices $\{3,6\}$. The edge $(3,4)$ arrives to $F_{\{1,2\}}$  and $\mathsf{Sr_{WCC}}$ is fed with the mark $\eof$. Edges $(3,4)$ and $(4,5)$ remain passing through $\ice$.}
  \label{fig:dp_example_3_4b}
\end{subfigure}
\caption[{[PoC] $\dpwcc$ Evolving second state}]{Evolution of the $\dpwcc$: Second state}
\label{fig:dp_example_3_4}
\end{figure}
\vspace{.5cm}

\begin{figure}[h!]
\centering
\begin{subfigure}[b]{\textwidth}
 \centering
  \inputtikz{dp_example_5}
  \caption{$\mathsf{Sr_{WCC}}$  fed both, $\ice$ and $\csofv$, channels with the mark $\eof$ received from the input stream in previous state and then, it died. The edge $(4,5)$ is arriving to $\gwc$ and the edge $(3,4)$ is arriving to $F_{\{3,6\}}$. }
  \label{fig:dp_example_5_6a}
\end{subfigure}
\vspace{.3cm}

\begin{subfigure}[b]{\textwidth}
 \centering
  \inputtikz{dp_example_6}
  \caption{When the edge $(4,5)$ arrives to $\gwc$, it spawns the filter instance $F_{\{4,5\}}$  between $F_{\{3,6\}}$ and $\gwc$. Filter instance $F_{\{3,6\}}$ is connected to the new filter instance $F_{\{4,5\}}$ and this one is connected to  $\gwc$ through channels $\ice$ and  $\csofv$.  Since the edge $(3,4)$ arrived to $F_{\{3,6\}}$ at the same time and  vertex $3$ belongs to the set of connected vertices of the filter $F_{\{3,6\}}$,  the vertex $4$ is added to the state of $F_{\{3,6\}}$. Now, the state of $F_{\{3,6\}}$ is the connected set of vertices $\{3,4,6\}$. When the mark $\eof$ arrives to the first filter instance, $F_{\{1,2\}}$, through  $\csofv$, this stage passes  its partial set of connected vertices,  $\{1,2\}$, through $\csofv$ and dies.  This action will activate $\Actt$ in next  filter instances to start building  maximal connected components. In this example, the state in  $F_{\{3,6\}}$, $\{3,4,6\}$, and the arriving set $\{1,2\}$ do not intersect and, hence, both sets of vertices, $\{1,2\}$ and $\{3,4,6\}$ will be passed  to the next filter instance through $\csofv$.}
  \label{fig:dp_example_5_6b}
\end{subfigure}
\caption[{[PoC] $\dpwcc$ Evolving third state}]{Evolution of the $\dpwcc$: Third state}
\label{fig:dp_example_5_6}
\end{figure}
\vspace{.5cm}

\begin{figure}[h!]
\centering
\begin{subfigure}[b]{\textwidth}
 \centering
  \inputtikz{dp_example_7}
  \caption{The set of connected vertices  $\{3,4,6\}$ is arriving to $F_{\{4,5\}}$. The mark $\eof$ continues passing to next stages through the channel $\ice$.}
  \label{fig:dp_example_7_8a}
\end{subfigure}
\vspace{.3cm}

\begin{subfigure}[b]{\textwidth}
 \centering
  \inputtikz{dp_example_8}
  \caption{Since the intersection of the set of connected vertices $\{3,4,6\}$ arrived to  $F_{\{4,5\}}$ and its state is not empty, this state is enlarged to be $\{3,4,5,6\}$. The set of connected vertices $\{1,2\}$ is arriving to  $F_{\{4,5\}}$}
  \label{fig:dp_example_7_8b}
\end{subfigure}
\caption[{[PoC] $\dpwcc$ Evolving fourth state}]{Evolution of the $\dpwcc$:  Fourth state}
\label{fig:dp_example_7_8}
\end{figure}
\vspace{.5cm}

\begin{figure}[h!]
\centering
\begin{subfigure}[b]{\textwidth}
 \centering
  \inputtikz{dp_example_9}
  \caption{$F_{\{4,5\}}$ has passed the set of connected vertices  $\{1,2\}$ and it is arriving to $\mathsf{Sk_{WCC}}$. The mark $\eof$ is arriving to  $F_{\{4,5\}}$ through $\csofv$.}
  \label{fig:dp_example_9_10a}
\end{subfigure}
\vspace{.3cm}

\begin{subfigure}[b]{\textwidth}
 \centering
  \inputtikz{dp_example_10}
  \caption{Since the mark $\eof$ arrived to $F_{\{4,5\}}$ through $\csofv$, it passes its state, the set $\{3,4,5,6\}$ through $\csofv$ to next stages and died. The set of connected vertices  $\{1,2\}$ arrived to $\mathsf{Sk_{WCC}}$ and this implies  that $\{1,2\}$ is a maximal set of connected vertices, i.e. a connected component of the input graph. Hence,  $\mathsf{Sk_{WCC}}$ output this first weakly connected component.}
 \label{fig:dp_example_9_10b}
\end{subfigure}
\vspace{.5cm}

\begin{subfigure}[b]{\textwidth}
 \centering
  \inputtikz{dp_example_11}
  \caption{Finally, the set of connected vertices  $\{3,4,5,6\}$ arrived to $\mathsf{Sk_{WCC}}$ and was output as a new weakly connected component. Besides, the mark $\eof$ also arrived to $\mathsf{Sk_{WCC}}$ through $\csofv$ and thus, it dies.}
  \label{fig:dp_example_9_10c}
\end{subfigure}
\vspace{.3cm}

\begin{subfigure}[b]{\textwidth}
 \centering
  \inputtikz{dp_example_12}
  \caption{The weakly connected component of in the graph \autoref{fig:example_dp_graph} such as they have been emitted by $\dpwcc$.}
  \label{fig:dp_example_9_10d}
\end{subfigure}
\caption[{[PoC] $\dpwcc$ Evolving last state}]{Last states in the evolution of the $\dpwcc$}
\label{fig:dp_example_9_10}
\end{figure}

It is important to highlight that during the states shown in \autoref{fig:dp_example_1_2a},  \autoref{fig:dp_example_1_2b},  \autoref{fig:dp_example_3_4a},  \autoref{fig:dp_example_3_4b} and  \autoref{fig:dp_example_5_6a} the only actor executed in any filter instance is $\Act$ (constructing sets of connected vertices). Afterwards, although $\Act$ can continue being executed in some filter instances, there are some instances that start executing $\Actt$ (constructing sets of maximal connected vertices). This is shown from \autoref{fig:dp_example_5_6a}  to \autoref{fig:dp_example_9_10a}.
%
\clearpage

\section{Empirical Evaluation}
For the empirical evaluation we consider the following research questions: 
\begin{inparaenum}[\bf {\bf RQ}1\upshape)]
\label{res:question}
    \item Does $\dpwcc$ in \acrshort{hs} support the dynamic parallelization level that $\dpwcc$ requires?
    \item Is $\dpwcc$ in \acrshort{hs} competitive compared with default implementations on base libraries for the same problem?
    \item Does $\dpwcc$ in \acrshort{hs} handle memory efficiently?
\end{inparaenum}

We have conducted different kinds of experiments to test our assumptions and verify the correctness of the implementation.
First, we have performed an \emph{Implementation Analysis} in which we have selected some graphs from \acrfull{snap} \cite{stanford} 
and analyze how the implementation behaves under real-world graphs if it timeouts or not and if it is producing correct results in terms of the amount of \acrshort{wcc} that we know beforehand.
We have also tested the implementation doing a \emph{Benchmark Analysis} where we focus on two different types of benchmarks. On the one hand, 
using \texttt{criterion} library \cite{criterion}, we have evaluated a benchmark between our solution and \acrshort{wcc} algorithm implemented in \texttt{containers} \acrshort{hs} library \cite{containers} 
using \mintinline{haskell}{Data.Graph}. On the other hand, we have compared if the results are being generated incrementally in both cases and how that is done during the pipeline execution time. 
This last analysis has been conducted using \texttt{diefpy} tool \cite{diefpaper,diefpy}.
Finally, we have executed a \textit{Performance Analysis} in which we have to gather profiling data from \acrfull{ghc} for one of the real-world graphs to measure how the program performs regarding multithreading and memory allocation.

\paragraph{Implementation analysis} The following represents the execution for running these graphs on our \acrshort{dp} implementation.

\begin{table}[H]
  \centering
  \resizebox{\textwidth}{!}{
  \begin{tabular}{|l|r|r|r|r|}
   \hline
   \textbf{Network} & \textbf{Exec Param} & \textbf{MUT Time} & \textbf{GC Time} & \textbf{Total Time}\\
   \hline
   Enron Emails & \mintinline{bash}{+RTS -N4 -s} & 2.797s & 0.942s & 3.746s \\
   \hline
   Astro Physics Coll Net & \mintinline{bash}{+RTS -N4 -s} & 2.607s & 1.392s & 4.014s \\
   \hline
   Google Web Graph & \mintinline{bash}{+RTS -N8 -s} & 137.127s & 218.913s & 356.058s \\
   \hline
  \end{tabular}
  }
 \caption[{[PoC] Execution times}]{This table shows the \acrshort{ghc} execution time measurement of selected networks. Column \texttt{Exec Param} describe the runtime flags provided to the running program. \texttt{MUT Time} is the time in seconds the program was executing computations (a.k.a. program time). \texttt{GC} Time is garbage collector time. Total time is the sum of \texttt{MUT} $+$ \texttt{GC} time.}
 \label{table:5}
 \end{table}

It is important to point out that since the first two networks are smaller in the number of edges compared with \emph{web-Google}, 
executing those with $8$ cores as the \mintinline{bash}{-N} parameters indicates does not affect the final speed-up since \acrshort{ghc} 
is not distributing threads on extra cores because it handles the load with $4$ cores only.
As we can see in \autoref{table:5}, we are obtaining remarkable execution times for the first two graphs, and it seems not to be the case 
for \textit{web-Google} due to the topology of the graph; it is denser in terms of connected components than the others.

\paragraph{Benchmark Analysis} Regarding mean execution times for each implementation on each case measure by \texttt{criterion} library \cite{criterion}, we can display the following results:

\begin{table}[H]
  \centering
  \resizebox{\textwidth}{!}{
  \begin{tabular}{|l|l|l|l|}
   \hline
   \textbf{Network} & \textbf{\acrshort{dpwcc}} & \textbf{\acrshort{hs} \texttt{containers}} & \textbf{Speed-up}\\
   \hline
   Enron Emails & 4.68s &  6.46s & 1.38\\
   \hline
   Astro Physics Coll Net & 4.98s & 6.95s  & 1.39\\
   \hline
   Google Web Graph & 386s & 106s & 0.27\\
   \hline
  \end{tabular}
  }
 \caption[{[PoC] Mean Execution times}]{Mean execution time of each network running under \texttt{criterion} library comparing both implementations in \acrshort{hs}: \acrshort{dpwcc} and \texttt{containers} lib. \texttt{criterion} runs $1000$ times each implementations and takes mean execution times of each. \texttt{Speed-up} column shows the ratio between \texttt{Haskell containers} and \texttt{\acrshort{dpwcc}}}
 \label{table:6}
 \end{table}

These results allow for answering Question [Q2], where we have seen that the graph topology is affecting the performance and the parallelization, penalizing \acrfull{dpwcc} for this particular case. In this benchmark, 
the solution against a non-parallel \texttt{containers} \mintinline{haskell}{Data.Graph} confirms the hypothesis. 

\paragraph{Diefficency metrics} The definition of this metric has been discussed on \autoref{prem:dief}. Some considerations are needed before starting to analyze the data gathered with \acrfull{dtkp} tool. Firstly, the tool is plotting the results according to the traces generated by the implementation, 
both \acrshort{dpwcc} and \acrshort{hs} \emph{containers}. By the nature of \acrshort{dp} model, we can gather or register that timestamps as long as the model is generating results. In the case of \acrshort{hs} \texttt{containers}, this is not possible since it 
calculates \acrshort{wcc} at once. This is not an issue and we still can check at what point in time all \acrshort{wcc} in \acrshort{hs} \texttt{containers} are generated. In those cases, we are going to see a straight vertical line. 

It is important to remark that we needed to scale the timestamps because we have taken the time in nanoseconds. After all, the incremental generation between one \acrshort{wcc} and the other is very small but significant enough to be taken into consideration. 
Thus, if we left the time scale in integer milliseconds, microseconds, or nanoseconds integer part, it cannot be appreciated. In case of escalation, we are discounting the nanosecond integer of the first generated results resulting in a time scale that starts close to $0$. 
This does not mean that the first result is generated at $0$ time, but we are discarding the previous time to focus on how the results are incrementally generated.

\begin{figure}[!htb]
    \centering
    \begin{subfigure}{0.33\textwidth}
     \includegraphics[width=1\linewidth, height=0.2\textheight]{email_enron0}
      \caption[{[PoC] \acrshort{dt} Results: email-Enron}]{email-Enron \acrshort{dt}}
      \label{fig:dief:1}
    \end{subfigure}%
    \begin{subfigure}{0.33\textwidth}
     \includegraphics[width=1\linewidth, height=0.2\textheight]{ca_astroph0}
      \caption[{[PoC] \acrshort{dt} Results: ca-AstroPh}]{ca-AstroPh \acrshort{dt}}
      \label{fig:dief:2}
    \end{subfigure}%
    \begin{subfigure}{0.33\textwidth}
     \includegraphics[width=1\linewidth, height=0.2\textheight]{web_google0}
      \caption[{[PoC] \acrshort{dt} Results: web-Google}]{web-Google \acrshort{dt}}
      \label{fig:dief:3}
    \end{subfigure}
    \caption[{[PoC] \acrshort{dt} Metrics}]{This plots are showing the \acrshort{dt} on the three networks comparing both \acrshort{hs} implementations \acrshort{dpwcc} and \texttt{containers} lib. Red lines indicates \texttt{containers} \acrshort{hs} \acrshort{dt} metric. Yellow points indicates \acrshort{dpwcc} \acrshort{dt} metric}
    \label{fig:poc:dief}
\end{figure}

\begin{figure}[!htb]
  \centering
  \begin{subfigure}{0.33\textwidth}
   \includegraphics[width=1\linewidth, height=0.2\textheight]{email_enron_radar0}
    \caption[{[PoC] \acrshort{dt} Results: email-Enron radar}]{email-Enron \acrshort{dt}}
    \label{fig:dief:rad:1}
  \end{subfigure}%
  \begin{subfigure}{0.33\textwidth}
   \includegraphics[width=1\linewidth, height=0.2\textheight]{ca_astroph_radar0}
    \caption[{[PoC] \acrshort{dt} Results: ca-AstroPh radar}]{ca-AstroPh \acrshort{dt}}
    \label{fig:dief:rad:2}
  \end{subfigure}%
  \begin{subfigure}{0.33\textwidth}
   \includegraphics[width=1\linewidth, height=0.2\textheight]{web_google_radar0}
    \caption[{[PoC] \acrshort{dt} Results: web-Google radar}]{web-Google \acrshort{dt}}
    \label{fig:dief:rad:3}
  \end{subfigure}
  \caption[{[PoC] \acrshort{dt} Metrics (Radial)}]{Radial plot shows how the different metrics provided by \acrshort{dtkp} tool such as \acrshort{tt}, \acrshort{tfft}, \acrshort{dt}, \acrshort{et} and \acrshort{comp} are related each other for each \acrshort{hs} implementation: \acrshort{dpwcc} and \texttt{containers}. Red area indicates \texttt{containers} \acrshort{hs} \acrshort{dt} metric. Yellow area indicates \acrshort{dpwcc} \acrshort{dt} metric}
  \label{fig:poc:dief:radial}
\end{figure}

Based on the results shown in \autoref{fig:poc:dief} and \autoref{fig:poc:dief:radial} above, all the solutions in \acrshort{dpwcc} indicates continuous behavior, 
but there is some difference that we would like to remark. In the case of \emph{email-Enron} and \emph{ca-AstroPh} graphs 
as we can see in \autoref{fig:dief:1} and \autoref{fig:dief:2}, there seems to be a more incremental generation of results. 
This behavior is measured with the values of \acrlong{dt}. \emph{ca-AstroPh} as it can be seen in \autoref{fig:dief:2}, is even more incremental, and it is showing a clear separation between some results and others. 
The \emph{web-Google} network, which is shown in \autoref{fig:dief:3}, is a little more linear, and that is because all the results are being generated with very little difference in time between them. 
Having the biggest \acrshort{wcc} at the end of \emph{web-Google} \acrshort{dp} algorithm 
it is retaining results until the biggest \acrshort{wcc} can be solved, which takes longer. 


\paragraph{Multithreading} For analyzing parallelization and multithreading we have used \textit{ThreadScope} \cite{threadscope} which allows us to see how the parallelization is taking place on \acrshort{ghc} at a fine grained level and how the threads are distributed throughout the different cores requested with the \mintinline{bash}{-N} execution \texttt{ghc-option} flag.
The distribution of the load is more intensive at the end of the execution, where \mintinline{haskell}{actor2} filter stage of the algorithm is taking place and different filters are reaching execution of that second actor.
We can appreciate how many threads are being spawned and by the tool and if they are evenly distributed among cores. 

\begin{figure}[!htb]
  \centering
  \includegraphics[width=0.7\textwidth, height=0.3\textheight]{screen_2}
  \caption[{[PoC] Thread Metrics: Fraction of Time}]{Threadscope Image of Zoomed Fraction of $10$ nanoseconds. Upper green area shows the amount of core used during that fraction of time. The lower are where it shows four separated green bars describe the behavior on each core. The number inside the green bar show the amount of threads running on that core at that moment. Finally orange bars are GC time.}
  \label{fig:4}
 \end{figure}

~\autoref{fig:4} zooms in on \textit{ThreadScope} output in a particular moment, approximately in the middle of the execution. 
The numbers inside green bars represent the number of threads that are being executed on that particular core (horizontal line) at that execution slot. 
Thus, the number of threads varies among slot execution times, because as it is already known, \acrshort{ghc} implements \emph{Preemptive Scheduling} \cite{lightweightghc}.
It can be appreciated in \autoref{fig:4} our first assumption that the load is evenly distributed because the mean number of executing threads per core is $571$.


\paragraph{Memory allocation} Another important aspect in our case is how the memory is being managed to avoid memory leaks or other non-desired behavior that increases memory allocation during the execution time. 
This is even more important in the particular implementation of \acrshort{wcc} using \acrshort{dp} model because it requires to maintain the set of connected components in memory throughout the execution of the program or at least until we can output the calculated \acrshort{wcc} if we reach to the last \textit{Filter} and we know that this \acrshort{wcc} cannot be enlarged anymore.
In order to verify this, we measure memory allocation with \textit{eventlog2html} \cite{eventlog2html} which converts generated profiling memory eventlog files into graphical HTML representation. 

\begin{figure}[!htb]
  \centering
  \includegraphics[width=1\linewidth, height=0.3\textheight]{visualization}
  \caption[{[PoC] Memory Metrics: Allocation by Data Type}]{This plot is showing the accumulated memory allocation size of each \acrshort{hs} Data Type throughout the execution of the program. The dark blue area shows the \texttt{ARR\_WORDS} data type which is \texttt{String} values. There are many of them because all that it comes from a file is in \texttt{String} format and need to be converted to the proper Data type. Rest of the light blue areas belong to \texttt{ByteString} which is the format treated in the input file as well, and \texttt{Maybe} type which is the type of data transfer between channels.}
  \label{fig:5}
\end{figure}

As we can see in \autoref{fig:5}, \acrshort{dpwcc} does efficient work on allocating memory since we are not using more than $57$ MB of memory during the whole execution of the program.
On the other hand, if we analyze how the memory is allocated during the execution of the program, it can also be appreciated that most of the memory is allocated at the beginning of the program and steadily decrease over time, with a small peak at the end that does not overpass even half of the initial peak of $57$ MB. 
The explanation for this behavior is quite straightforward because, in the beginning, we are reading from the file and transforming a \mintinline{haskell}{ByteString} buffer to \mintinline{haskell}{(Int, Int)} edges. 
This is seen in the image in which the dark blue that is on top of the area is \mintinline{haskell}{ByteString} allocation. 
Light blue is allocation of \mintinline{haskell}{Maybe a} type which is the type that is returned by the \textit{Channels} because it can contain a value or not. 
Data value \mintinline{haskell}{Nothing} is indicating end of the \textit{Channel}. 
Another important aspect is the green area which represents \mintinline{haskell}{IntSet} allocation, which in the case of our program is the data structure that we use to gather the set of vertices that represents a \acrshort{wcc}. 
This means that the amount of memory used for gathering the \acrshort{wcc} itself is minimum, and it is decreasing over time, which is another empirical indication that we are incrementally releasing results to the user. 
It can be seen as well that as long the green area reduces the lighter blue (\texttt{MUT\_ARR\_PTRS\_CLEAN} \cite{ghcheap}) increases at the same time indicating that the computations for the output (releasing results) is taking place. 
Finally, according to what we have stated above, we can answer the question [Q3], showing that not only the memory management was efficient, but at the same time, the memory was not leaking or increasing across the running execution program.

The empirical evaluation of the \acrshort{dpwcc} implementation to compute weakly connected components of a graph, evidence suitability, 
and robustness to provide a Dynamic Pipeline Framework in that language. Measuring using \par\bigskip metrics reveals some advantageous capability of $\dpwcc$ implementation to deliver incremental results compared with default containers library implementation. 
Regarding the main aspects where DPP is strong, i.e., pipeline parallelism and time processing, the $\dpwcc$ performance shows that Haskell 
can deal with the requirements for the \acrshort{wcc} problem without penalizing neither execution time nor memory allocation. 
In particular, the $\dpwcc$ implementation outperforms in those cases where the topology of the graph is sparse and where the number of vertices in the largest \acrshort{wcc} is not big enough. 
To conclude, the proof of concept has gathered enough evidence to show that the implementation of Dynamic Pipeline in Haskell Programming Language is feasible. 
This fact opens a wide range of algorithms to be explored using the Dynamic Pipeline Paradigm, supported by purely functional programming language.

\section{Chapter Summary}
In this chapter, we have presented a proof of concept that allows us to assess the feasibility of implementing \acrshort{dp} using \acrshort{hs}. 
The obtained results gave us insights about how to proceed for implementing a first version of a DPF using (parallel) Haskell and, afterward, to implement an algorithm for enumerating incrementally the bitriangles of a bipartite graph based on the  \acrshort{dp}.


%

\section{Proof of concept: Weakly Connected Components of a Graph}\label{prole}

One of the biggest challenges of implementing a Dynamic Pipeline is to find  a programming language with a proper set of tools supporting both of the  primary features of the \acrshort{dp}: \begin{inparaenum}[i\upshape)]
\item  \emph{parallel} processing and 
\item  \emph{strong theoretical} foundations to manage computations as first-class citizens.
 \end{inparaenum}
Haskell is a statically typed pure functional language designed on strong theoretical foundations where com\-pu\-ta\-tions are primary entities.
This pure functional language has evolved from its birth in 1987 and nowadays provides a powerful set of tools for writing multithreading and parallel programs with optimal performance \cite{parallelbook, monadpar}. 

In the context of this research, we first assessed the suitability of \acrlong{hs}  to implement a Dynamic Pipeline. 
To be concrete, we conducted a proof of concept implementing a Dynamic Pipeline in  Haskell for solving a particular and very relevant problem as the computation/e\-nu\-me\-ra\-tion of the Weakly Connected Components of a graph.
In particular, the main objective of our proof of concept was to study the critical features required in \acrshort{hs} for a \acrshort{dpf} implementation,  the real possibilities of emitting incrementally results, and the performance of such kinds of im\-ple\-men\-ta\-tions. 
Indeed, we explored the basis of an implementation of a \acrshort{dpf}  in a pure (parallel) functional language as Haskell.
This is, we determined the particular features (i.e., versions and libraries) that will allow for an efficient implementation of a \acrshort{dpf}. 
Moreover, we conducted an empirical evaluation to analyze the performance of the Dynamic Pipeline implemented in Haskell for enumerating \acrshort{wcc} with special attention to the emission of results, i.e. \acrshort{wcc}, incrementally. 
In this chapter, we focus on the presentation of our algorithm for enumerating \acrshort{wcc} using \acrshort{hs} as well as the results obtained in the empirical evaluation of its implementation. 
 
\subsection{$\dpwcc$ Algorithm}
Let us consider the problem of computing/enumerating the (weak) connected components of a graph $G$ using \acrshort{dp}. 
A connected component of a graph is a subgraph in which any two vertices are connected by paths.  
Thus, finding connected components of an undirected graph implies obtaining the minimal partition of the set of nodes induced by the relationship \textit{connected}, i.e., there is a path between each pair of nodes. An example of that graph can be seen in \autoref{fig:example_dp_graph}.
The input of the Dynamic Pipeline for computing the WCC of a graph, $\dpwcc$, is a sequence of edges ending with $\eof$. In the source graph there are neither isolated vertices nor loops. The connected components are output as soon as they are computed, i.e., they are produced incrementally. 
Roughly speaking the idea of the algorithm is that the weakly connected components are built in two phases. In the first phase filter instance stages receive the edges of the input graph and create sets of connected vertices. 
During the second phase, these filter instances construct maximal subsets of connected vertices, i.e. the vertices corresponding to (weakly) connected components.
%
$\dpwcc$ is defined in terms of the behavior of its four kinds stages: \textit{Source} ($\iwc$),  \textit{Generator} ($\gwc$),  \textit{Sink} ($\owc$), and \textit{Filter}($\fwc$) stages. Additionally,  the channels connecting these stages must be defined. 
In $\dpwcc$, stages are connected linearly and unidirectionally through the channels $\ice$ and  $\csofv$. Channel $\ice$ carries edges while channel  $\csofv$ conveys sets of connected vertices. Both channels end by the $\eof$ mark. 
The behavior of $\fwc$ is given by a sequence of two actors (scripts). Each actor corresponds to a phase of the algorithm. In what follows, we denote these actors by $\Act$ and $\Actt$, respectively. 
The script $\Act$ keeps a set of connected vertices ($CV$) in the state of the $\fwc$ instance. When an edge $e$ arrives, if an endpoint of $e$ is present in the state, then the other endpoint of $e$ is added to $CV$. 
Edges without incident endpoints are passed to the next stage. When $\eof$ arrives at channel $\ice$, it is passed to the next stage, and the script $\Actt$ starts its execution. 
If script $\Actt$ receives a set of connected vertices $CV$ in $\csofv$, it determines if the intersection between $CV$ and the nodes in its state is not empty. If so, it adds the nodes in $CV$  to its state. 
Otherwise, the $CV$ is passed to the next stage.  Whenever $\eof$ is received, $\Actt$ passes--through $\csofv$-- the set of vertices in its state and the $\eof$ mark to the next stage; then, it dies.
The behavior of $\iwc$ corresponds to the identity transformation over the data stream of edges.  As edges arrive, they are passed through  $\ice$ to the next stage. When receiving $\eof$ on $\ice$, this mark is put on both channels. 
Then, $\iwc$ dies. 

%\begin{wrapfigure}{r}{0.4\textwidth}
\begin{figure}
 \begin{center}
\inputtikz{graph_example_wcc}
\end{center}
\caption[{[PoC] Graph WCC Example}]{Example of a graph with two weakly connected components: $\{1,2\}$ and $\{3,4,5,6\}$}
\label{fig:example_dp_graph}
\end{figure}
%\end{wrapfigure}

Let us describe this behavior with the example of the graph shown in \autoref{fig:example_dp_graph}.

\begin{figure}[h!]
  \centering
\inputtikz{dp_example_0}
\caption[{[PoC] $\dpwcc$ Initial Setup}]{$\dpwcc$ Initial setup. Stages Source, Generator, and Sink are represented by the squares labeled by $\mathsf{Sr_{WCC}}$, $\mathsf{G_{WCC}}$ and $\mathsf{Sk_{WCC}}$, respectively.  The square $\fwc$ corresponding to the Filter stage template is the parameter of $\gwc$. Arrows $\rightrightarrows$ between represents the connection of stages through two channels, $\ice$, and $\csofv$. The arrow  $\rightarrow$ represents the channel $\csofv$ connecting the stages $\mathsf{G_{WCC}}$ and $\mathsf{Sk_{WCC}}$. The arrow $\Longrightarrow$ stands for I/O data flow. Finally, the input stream comes between the dotted lines on the left and the WCC computed incrementally will be placed between the solid lines on the right.}
\label{fig:dp_example_0}
\end{figure}

\autoref{fig:dp_example_0} depicts the initial configuration of $\dpwcc$. 
The interaction of $\dpwcc$ with the "external" world is done through the stages $\iwc$ and $\owc$. 
Indeed, once activated the initial $\dpwcc$, the input stream -- consisting of a sequence containing all the edges in the graph in \autoref{fig:example_dp_graph} -- feeds $\iwc$ while  $\owc$ emits incrementally the resulting weakly connected components.  
In what follows \autoref{fig:dp_example_1_2}, \autoref{fig:dp_example_3_4}, \autoref{fig:dp_example_5_6}, \autoref{fig:dp_example_7_8} and \autoref{fig:dp_example_9_10} depict the evolution of the $\dpwcc$.
 
\begin{figure}[h!]
\centering
\begin{subfigure}[b]{\textwidth}
 \centering
  \inputtikz{dp_example_1}
  \caption{The edge $(1,2)$ is arriving to $\gwc$.}
  \label{fig:dp_example_1_2a}
\end{subfigure}
\vspace{.3cm}

\begin{subfigure}[b]{\textwidth}
 \centering
  \inputtikz{dp_example_2}
  \caption{When the edge $(1,2)$ arrives to $\gwc$, it  spawns a new instance of $\fwc$ before $\gwc$. Filter instance $F_{\{1,2\}}$ is connected to  $\gwc$ through channels $\ice$ and  $\csofv$. The state of the new filter instance $F_{\{1,2\}}$ is initialized with the set of vertices $\{1,2\}$. The edge $(3,6)$ arrives to the new filter instance $F_{\{1,2\}}$.}
  \label{fig:dp_example_1_2b}
\end{subfigure}
\caption[{[PoC] $\dpwcc$ Evolving first state}]{Evolution of the $\dpwcc$: First state}
\label{fig:dp_example_1_2}
\end{figure}
\vspace{.5cm}

\begin{figure}[h!]
\centering
\begin{subfigure}[b]{\textwidth}
 \centering
  \inputtikz{dp_example_3}
  \caption{None of the vertices in the edge $(3,6)$ is in the set of vertices $\{1,2\}$ in the state of $F_{\{1,2\}}$, hence it is passed through $\ice$ to $\gwc$.}
  \label{fig:dp_example_3_4a}
\end{subfigure}
\vspace{.3cm}

\begin{subfigure}[b]{\textwidth}
 \centering
  \inputtikz{dp_example_4}
  \caption{When the edge $(3,6)$ arrives to $\gwc$, it spawns the filter instance $F_{\{3,6\}}$  between $F_{\{1,2\}}$ and $\gwc$. Filter instance $F_{\{1,2\}}$ is connected to the new filter instance $F_{\{3,6\}}$ and this one is connected to  $\gwc$ through channels $\ice$ and  $\csofv$. The state of the new filter instance $F_{\{3,6\}}$ is initialized with the set of vertices $\{3,6\}$. The edge $(3,4)$ arrives to $F_{\{1,2\}}$  and $\mathsf{Sr_{WCC}}$ is fed with the mark $\eof$. Edges $(3,4)$ and $(4,5)$ remain passing through $\ice$.}
  \label{fig:dp_example_3_4b}
\end{subfigure}
\caption[{[PoC] $\dpwcc$ Evolving second state}]{Evolution of the $\dpwcc$: Second state}
\label{fig:dp_example_3_4}
\end{figure}
\vspace{.5cm}

\begin{figure}[h!]
\centering
\begin{subfigure}[b]{\textwidth}
 \centering
  \inputtikz{dp_example_5}
  \caption{$\mathsf{Sr_{WCC}}$  fed both, $\ice$ and $\csofv$, channels with the mark $\eof$ received from the input stream in previous state and then, it died. The edge $(4,5)$ is arriving to $\gwc$ and the edge $(3,4)$ is arriving to $F_{\{3,6\}}$. }
  \label{fig:dp_example_5_6a}
\end{subfigure}
\vspace{.3cm}

\begin{subfigure}[b]{\textwidth}
 \centering
  \inputtikz{dp_example_6}
  \caption{When the edge $(4,5)$ arrives to $\gwc$, it spawns the filter instance $F_{\{4,5\}}$  between $F_{\{3,6\}}$ and $\gwc$. Filter instance $F_{\{3,6\}}$ is connected to the new filter instance $F_{\{4,5\}}$ and this one is connected to  $\gwc$ through channels $\ice$ and  $\csofv$.  Since the edge $(3,4)$ arrived to $F_{\{3,6\}}$ at the same time and  vertex $3$ belongs to the set of connected vertices of the filter $F_{\{3,6\}}$,  the vertex $4$ is added to the state of $F_{\{3,6\}}$. Now, the state of $F_{\{3,6\}}$ is the connected set of vertices $\{3,4,6\}$. When the mark $\eof$ arrives to the first filter instance, $F_{\{1,2\}}$, through  $\csofv$, this stage passes  its partial set of connected vertices,  $\{1,2\}$, through $\csofv$ and dies.  This action will activate $\Actt$ in next  filter instances to start building  maximal connected components. In this example, the state in  $F_{\{3,6\}}$, $\{3,4,6\}$, and the arriving set $\{1,2\}$ do not intersect and, hence, both sets of vertices, $\{1,2\}$ and $\{3,4,6\}$ will be passed  to the next filter instance through $\csofv$.}
  \label{fig:dp_example_5_6b}
\end{subfigure}
\caption[{[PoC] $\dpwcc$ Evolving third state}]{Evolution of the $\dpwcc$: Third state}
\label{fig:dp_example_5_6}
\end{figure}
\vspace{.5cm}

\begin{figure}[h!]
\centering
\begin{subfigure}[b]{\textwidth}
 \centering
  \inputtikz{dp_example_7}
  \caption{The set of connected vertices  $\{3,4,6\}$ is arriving to $F_{\{4,5\}}$. The mark $\eof$ continues passing to next stages through the channel $\ice$.}
  \label{fig:dp_example_7_8a}
\end{subfigure}
\vspace{.3cm}

\begin{subfigure}[b]{\textwidth}
 \centering
  \inputtikz{dp_example_8}
  \caption{Since the intersection of the set of connected vertices $\{3,4,6\}$ arrived to  $F_{\{4,5\}}$ and its state is not empty, this state is enlarged to be $\{3,4,5,6\}$. The set of connected vertices $\{1,2\}$ is arriving to  $F_{\{4,5\}}$}
  \label{fig:dp_example_7_8b}
\end{subfigure}
\caption[{[PoC] $\dpwcc$ Evolving fourth state}]{Evolution of the $\dpwcc$:  Fourth state}
\label{fig:dp_example_7_8}
\end{figure}
\vspace{.5cm}

\begin{figure}[h!]
\centering
\begin{subfigure}[b]{\textwidth}
 \centering
  \inputtikz{dp_example_9}
  \caption{$F_{\{4,5\}}$ has passed the set of connected vertices  $\{1,2\}$ and it is arriving to $\mathsf{Sk_{WCC}}$. The mark $\eof$ is arriving to  $F_{\{4,5\}}$ through $\csofv$.}
  \label{fig:dp_example_9_10a}
\end{subfigure}
\vspace{.3cm}

\begin{subfigure}[b]{\textwidth}
 \centering
  \inputtikz{dp_example_10}
  \caption{Since the mark $\eof$ arrived to $F_{\{4,5\}}$ through $\csofv$, it passes its state, the set $\{3,4,5,6\}$ through $\csofv$ to next stages and died. The set of connected vertices  $\{1,2\}$ arrived to $\mathsf{Sk_{WCC}}$ and this implies  that $\{1,2\}$ is a maximal set of connected vertices, i.e. a connected component of the input graph. Hence,  $\mathsf{Sk_{WCC}}$ output this first weakly connected component.}
 \label{fig:dp_example_9_10b}
\end{subfigure}
\vspace{.5cm}

\begin{subfigure}[b]{\textwidth}
 \centering
  \inputtikz{dp_example_11}
  \caption{Finally, the set of connected vertices  $\{3,4,5,6\}$ arrived to $\mathsf{Sk_{WCC}}$ and was output as a new weakly connected component. Besides, the mark $\eof$ also arrived to $\mathsf{Sk_{WCC}}$ through $\csofv$ and thus, it dies.}
  \label{fig:dp_example_9_10c}
\end{subfigure}
\vspace{.3cm}

\begin{subfigure}[b]{\textwidth}
 \centering
  \inputtikz{dp_example_12}
  \caption{The weakly connected component of in the graph \autoref{fig:example_dp_graph} such as they have been emitted by $\dpwcc$.}
  \label{fig:dp_example_9_10d}
\end{subfigure}
\caption[{[PoC] $\dpwcc$ Evolving last state}]{Last states in the evolution of the $\dpwcc$}
\label{fig:dp_example_9_10}
\end{figure}

It is important to highlight that during the states shown in \autoref{fig:dp_example_1_2a},  \autoref{fig:dp_example_1_2b},  \autoref{fig:dp_example_3_4a},  \autoref{fig:dp_example_3_4b} and  \autoref{fig:dp_example_5_6a} the only actor executed in any filter instance is $\Act$ (constructing sets of connected vertices). Afterwards, although $\Act$ can continue being executed in some filter instances, there are some instances that start executing $\Actt$ (constructing sets of maximal connected vertices). This is shown from \autoref{fig:dp_example_5_6a}  to \autoref{fig:dp_example_9_10a}.
%
\clearpage

\section{Empirical Evaluation}
For the empirical evaluation we consider the following research questions: 
\begin{inparaenum}[\bf {\bf RQ}1\upshape)]
\label{res:question}
    \item Does $\dpwcc$ in \acrshort{hs} support the dynamic parallelization level that $\dpwcc$ requires?
    \item Is $\dpwcc$ in \acrshort{hs} competitive compared with default implementations on base libraries for the same problem?
    \item Does $\dpwcc$ in \acrshort{hs} handle memory efficiently?
\end{inparaenum}

We have conducted different kinds of experiments to test our assumptions and verify the correctness of the implementation.
First, we have performed an \emph{Implementation Analysis} in which we have selected some graphs from \acrfull{snap} \cite{stanford} 
and analyze how the implementation behaves under real-world graphs if it timeouts or not and if it is producing correct results in terms of the amount of \acrshort{wcc} that we know beforehand.
We have also tested the implementation doing a \emph{Benchmark Analysis} where we focus on two different types of benchmarks. On the one hand, 
using \texttt{criterion} library \cite{criterion}, we have evaluated a benchmark between our solution and \acrshort{wcc} algorithm implemented in \texttt{containers} \acrshort{hs} library \cite{containers} 
using \mintinline{haskell}{Data.Graph}. On the other hand, we have compared if the results are being generated incrementally in both cases and how that is done during the pipeline execution time. 
This last analysis has been conducted using \texttt{diefpy} tool \cite{diefpaper,diefpy}.
Finally, we have executed a \textit{Performance Analysis} in which we have to gather profiling data from \acrfull{ghc} for one of the real-world graphs to measure how the program performs regarding multithreading and memory allocation.

\paragraph{Implementation analysis} The following represents the execution for running these graphs on our \acrshort{dp} implementation.

\begin{table}[H]
  \centering
  \resizebox{\textwidth}{!}{
  \begin{tabular}{|l|r|r|r|r|}
   \hline
   \textbf{Network} & \textbf{Exec Param} & \textbf{MUT Time} & \textbf{GC Time} & \textbf{Total Time}\\
   \hline
   Enron Emails & \mintinline{bash}{+RTS -N4 -s} & 2.797s & 0.942s & 3.746s \\
   \hline
   Astro Physics Coll Net & \mintinline{bash}{+RTS -N4 -s} & 2.607s & 1.392s & 4.014s \\
   \hline
   Google Web Graph & \mintinline{bash}{+RTS -N8 -s} & 137.127s & 218.913s & 356.058s \\
   \hline
  \end{tabular}
  }
 \caption[{[PoC] Execution times}]{This table shows the \acrshort{ghc} execution time measurement of selected networks. Column \texttt{Exec Param} describe the runtime flags provided to the running program. \texttt{MUT Time} is the time in seconds the program was executing computations (a.k.a. program time). \texttt{GC} Time is garbage collector time. Total time is the sum of \texttt{MUT} $+$ \texttt{GC} time.}
 \label{table:5}
 \end{table}

It is important to point out that since the first two networks are smaller in the number of edges compared with \emph{web-Google}, 
executing those with $8$ cores as the \mintinline{bash}{-N} parameters indicates does not affect the final speed-up since \acrshort{ghc} 
is not distributing threads on extra cores because it handles the load with $4$ cores only.
As we can see in \autoref{table:5}, we are obtaining remarkable execution times for the first two graphs, and it seems not to be the case 
for \textit{web-Google} due to the topology of the graph; it is denser in terms of connected components than the others.

\paragraph{Benchmark Analysis} Regarding mean execution times for each implementation on each case measure by \texttt{criterion} library \cite{criterion}, we can display the following results:

\begin{table}[H]
  \centering
  \resizebox{\textwidth}{!}{
  \begin{tabular}{|l|l|l|l|}
   \hline
   \textbf{Network} & \textbf{\acrshort{dpwcc}} & \textbf{\acrshort{hs} \texttt{containers}} & \textbf{Speed-up}\\
   \hline
   Enron Emails & 4.68s &  6.46s & 1.38\\
   \hline
   Astro Physics Coll Net & 4.98s & 6.95s  & 1.39\\
   \hline
   Google Web Graph & 386s & 106s & 0.27\\
   \hline
  \end{tabular}
  }
 \caption[{[PoC] Mean Execution times}]{Mean execution time of each network running under \texttt{criterion} library comparing both implementations in \acrshort{hs}: \acrshort{dpwcc} and \texttt{containers} lib. \texttt{criterion} runs $1000$ times each implementations and takes mean execution times of each. \texttt{Speed-up} column shows the ratio between \texttt{Haskell containers} and \texttt{\acrshort{dpwcc}}}
 \label{table:6}
 \end{table}

These results allow for answering Question [Q2], where we have seen that the graph topology is affecting the performance and the parallelization, penalizing \acrfull{dpwcc} for this particular case. In this benchmark, 
the solution against a non-parallel \texttt{containers} \mintinline{haskell}{Data.Graph} confirms the hypothesis. 

\paragraph{Diefficency metrics} The definition of this metric has been discussed on \autoref{prem:dief}. Some considerations are needed before starting to analyze the data gathered with \acrfull{dtkp} tool. Firstly, the tool is plotting the results according to the traces generated by the implementation, 
both \acrshort{dpwcc} and \acrshort{hs} \emph{containers}. By the nature of \acrshort{dp} model, we can gather or register that timestamps as long as the model is generating results. In the case of \acrshort{hs} \texttt{containers}, this is not possible since it 
calculates \acrshort{wcc} at once. This is not an issue and we still can check at what point in time all \acrshort{wcc} in \acrshort{hs} \texttt{containers} are generated. In those cases, we are going to see a straight vertical line. 

It is important to remark that we needed to scale the timestamps because we have taken the time in nanoseconds. After all, the incremental generation between one \acrshort{wcc} and the other is very small but significant enough to be taken into consideration. 
Thus, if we left the time scale in integer milliseconds, microseconds, or nanoseconds integer part, it cannot be appreciated. In case of escalation, we are discounting the nanosecond integer of the first generated results resulting in a time scale that starts close to $0$. 
This does not mean that the first result is generated at $0$ time, but we are discarding the previous time to focus on how the results are incrementally generated.

\begin{figure}[!htb]
    \centering
    \begin{subfigure}{0.33\textwidth}
     \includegraphics[width=1\linewidth, height=0.2\textheight]{email_enron0}
      \caption[{[PoC] \acrshort{dt} Results: email-Enron}]{email-Enron \acrshort{dt}}
      \label{fig:dief:1}
    \end{subfigure}%
    \begin{subfigure}{0.33\textwidth}
     \includegraphics[width=1\linewidth, height=0.2\textheight]{ca_astroph0}
      \caption[{[PoC] \acrshort{dt} Results: ca-AstroPh}]{ca-AstroPh \acrshort{dt}}
      \label{fig:dief:2}
    \end{subfigure}%
    \begin{subfigure}{0.33\textwidth}
     \includegraphics[width=1\linewidth, height=0.2\textheight]{web_google0}
      \caption[{[PoC] \acrshort{dt} Results: web-Google}]{web-Google \acrshort{dt}}
      \label{fig:dief:3}
    \end{subfigure}
    \caption[{[PoC] \acrshort{dt} Metrics}]{This plots are showing the \acrshort{dt} on the three networks comparing both \acrshort{hs} implementations \acrshort{dpwcc} and \texttt{containers} lib. Red lines indicates \texttt{containers} \acrshort{hs} \acrshort{dt} metric. Yellow points indicates \acrshort{dpwcc} \acrshort{dt} metric}
    \label{fig:poc:dief}
\end{figure}

\begin{figure}[!htb]
  \centering
  \begin{subfigure}{0.33\textwidth}
   \includegraphics[width=1\linewidth, height=0.2\textheight]{email_enron_radar0}
    \caption[{[PoC] \acrshort{dt} Results: email-Enron radar}]{email-Enron \acrshort{dt}}
    \label{fig:dief:rad:1}
  \end{subfigure}%
  \begin{subfigure}{0.33\textwidth}
   \includegraphics[width=1\linewidth, height=0.2\textheight]{ca_astroph_radar0}
    \caption[{[PoC] \acrshort{dt} Results: ca-AstroPh radar}]{ca-AstroPh \acrshort{dt}}
    \label{fig:dief:rad:2}
  \end{subfigure}%
  \begin{subfigure}{0.33\textwidth}
   \includegraphics[width=1\linewidth, height=0.2\textheight]{web_google_radar0}
    \caption[{[PoC] \acrshort{dt} Results: web-Google radar}]{web-Google \acrshort{dt}}
    \label{fig:dief:rad:3}
  \end{subfigure}
  \caption[{[PoC] \acrshort{dt} Metrics (Radial)}]{Radial plot shows how the different metrics provided by \acrshort{dtkp} tool such as \acrshort{tt}, \acrshort{tfft}, \acrshort{dt}, \acrshort{et} and \acrshort{comp} are related each other for each \acrshort{hs} implementation: \acrshort{dpwcc} and \texttt{containers}. Red area indicates \texttt{containers} \acrshort{hs} \acrshort{dt} metric. Yellow area indicates \acrshort{dpwcc} \acrshort{dt} metric}
  \label{fig:poc:dief:radial}
\end{figure}

Based on the results shown in \autoref{fig:poc:dief} and \autoref{fig:poc:dief:radial} above, all the solutions in \acrshort{dpwcc} indicates continuous behavior, 
but there is some difference that we would like to remark. In the case of \emph{email-Enron} and \emph{ca-AstroPh} graphs 
as we can see in \autoref{fig:dief:1} and \autoref{fig:dief:2}, there seems to be a more incremental generation of results. 
This behavior is measured with the values of \acrlong{dt}. \emph{ca-AstroPh} as it can be seen in \autoref{fig:dief:2}, is even more incremental, and it is showing a clear separation between some results and others. 
The \emph{web-Google} network, which is shown in \autoref{fig:dief:3}, is a little more linear, and that is because all the results are being generated with very little difference in time between them. 
Having the biggest \acrshort{wcc} at the end of \emph{web-Google} \acrshort{dp} algorithm 
it is retaining results until the biggest \acrshort{wcc} can be solved, which takes longer. 


\paragraph{Multithreading} For analyzing parallelization and multithreading we have used \textit{ThreadScope} \cite{threadscope} which allows us to see how the parallelization is taking place on \acrshort{ghc} at a fine grained level and how the threads are distributed throughout the different cores requested with the \mintinline{bash}{-N} execution \texttt{ghc-option} flag.
The distribution of the load is more intensive at the end of the execution, where \mintinline{haskell}{actor2} filter stage of the algorithm is taking place and different filters are reaching execution of that second actor.
We can appreciate how many threads are being spawned and by the tool and if they are evenly distributed among cores. 

\begin{figure}[!htb]
  \centering
  \includegraphics[width=0.7\textwidth, height=0.3\textheight]{screen_2}
  \caption[{[PoC] Thread Metrics: Fraction of Time}]{Threadscope Image of Zoomed Fraction of $10$ nanoseconds. Upper green area shows the amount of core used during that fraction of time. The lower are where it shows four separated green bars describe the behavior on each core. The number inside the green bar show the amount of threads running on that core at that moment. Finally orange bars are GC time.}
  \label{fig:4}
 \end{figure}

~\autoref{fig:4} zooms in on \textit{ThreadScope} output in a particular moment, approximately in the middle of the execution. 
The numbers inside green bars represent the number of threads that are being executed on that particular core (horizontal line) at that execution slot. 
Thus, the number of threads varies among slot execution times, because as it is already known, \acrshort{ghc} implements \emph{Preemptive Scheduling} \cite{lightweightghc}.
It can be appreciated in \autoref{fig:4} our first assumption that the load is evenly distributed because the mean number of executing threads per core is $571$.


\paragraph{Memory allocation} Another important aspect in our case is how the memory is being managed to avoid memory leaks or other non-desired behavior that increases memory allocation during the execution time. 
This is even more important in the particular implementation of \acrshort{wcc} using \acrshort{dp} model because it requires to maintain the set of connected components in memory throughout the execution of the program or at least until we can output the calculated \acrshort{wcc} if we reach to the last \textit{Filter} and we know that this \acrshort{wcc} cannot be enlarged anymore.
In order to verify this, we measure memory allocation with \textit{eventlog2html} \cite{eventlog2html} which converts generated profiling memory eventlog files into graphical HTML representation. 

\begin{figure}[!htb]
  \centering
  \includegraphics[width=1\linewidth, height=0.3\textheight]{visualization}
  \caption[{[PoC] Memory Metrics: Allocation by Data Type}]{This plot is showing the accumulated memory allocation size of each \acrshort{hs} Data Type throughout the execution of the program. The dark blue area shows the \texttt{ARR\_WORDS} data type which is \texttt{String} values. There are many of them because all that it comes from a file is in \texttt{String} format and need to be converted to the proper Data type. Rest of the light blue areas belong to \texttt{ByteString} which is the format treated in the input file as well, and \texttt{Maybe} type which is the type of data transfer between channels.}
  \label{fig:5}
\end{figure}

As we can see in \autoref{fig:5}, \acrshort{dpwcc} does efficient work on allocating memory since we are not using more than $57$ MB of memory during the whole execution of the program.
On the other hand, if we analyze how the memory is allocated during the execution of the program, it can also be appreciated that most of the memory is allocated at the beginning of the program and steadily decrease over time, with a small peak at the end that does not overpass even half of the initial peak of $57$ MB. 
The explanation for this behavior is quite straightforward because, in the beginning, we are reading from the file and transforming a \mintinline{haskell}{ByteString} buffer to \mintinline{haskell}{(Int, Int)} edges. 
This is seen in the image in which the dark blue that is on top of the area is \mintinline{haskell}{ByteString} allocation. 
Light blue is allocation of \mintinline{haskell}{Maybe a} type which is the type that is returned by the \textit{Channels} because it can contain a value or not. 
Data value \mintinline{haskell}{Nothing} is indicating end of the \textit{Channel}. 
Another important aspect is the green area which represents \mintinline{haskell}{IntSet} allocation, which in the case of our program is the data structure that we use to gather the set of vertices that represents a \acrshort{wcc}. 
This means that the amount of memory used for gathering the \acrshort{wcc} itself is minimum, and it is decreasing over time, which is another empirical indication that we are incrementally releasing results to the user. 
It can be seen as well that as long the green area reduces the lighter blue (\texttt{MUT\_ARR\_PTRS\_CLEAN} \cite{ghcheap}) increases at the same time indicating that the computations for the output (releasing results) is taking place. 
Finally, according to what we have stated above, we can answer the question [Q3], showing that not only the memory management was efficient, but at the same time, the memory was not leaking or increasing across the running execution program.

The empirical evaluation of the \acrshort{dpwcc} implementation to compute weakly connected components of a graph, evidence suitability, 
and robustness to provide a Dynamic Pipeline Framework in that language. Measuring using \par\bigskip metrics reveals some advantageous capability of $\dpwcc$ implementation to deliver incremental results compared with default containers library implementation. 
Regarding the main aspects where DPP is strong, i.e., pipeline parallelism and time processing, the $\dpwcc$ performance shows that Haskell 
can deal with the requirements for the \acrshort{wcc} problem without penalizing neither execution time nor memory allocation. 
In particular, the $\dpwcc$ implementation outperforms in those cases where the topology of the graph is sparse and where the number of vertices in the largest \acrshort{wcc} is not big enough. 
To conclude, the proof of concept has gathered enough evidence to show that the implementation of Dynamic Pipeline in Haskell Programming Language is feasible. 
This fact opens a wide range of algorithms to be explored using the Dynamic Pipeline Paradigm, supported by purely functional programming language.

\section{Chapter Summary}
In this chapter, we have presented a proof of concept that allows us to assess the feasibility of implementing \acrshort{dp} using \acrshort{hs}. 
The obtained results gave us insights about how to proceed for implementing a first version of a DPF using (parallel) Haskell and, afterward, to implement an algorithm for enumerating incrementally the bitriangles of a bipartite graph based on the  \acrshort{dp}.


%

\section{Dynamic Pipeline Framework in Haskell}\label{dp-hs}
The design and implementation of \acrfull{dpfh} is a fundamental piece of the present work. 
A \acrlong{dpf} written in \acrlong{hs} which allow \acrshort{hs} users to implement any suitable algorithm for \acrlong{dp}.
During the process of conducting this research, we have implemented \acrshort{dpfh}~\cite{dynamic-pipeline} and publishes it into \acrlong{hack}~\cite{hackage}.
In this section, we describe the design and implementation details of \acrshort{dpfh}.

\subsection{Framework Design}

\subsubsection{Background}
A suitable framework should provide the user the right level of abstraction that removes and hides underlying complexity, 
allowing the developer to focus on the problem that needs to be solved.
There are several design approximations to implement a framework: \begin{inparaenum}[i\upshape)]
  \item  \emph{Configuration Based} where the user only focuses on completing a specific configuration either on a file or a database or both. Once this configuration is completed, the user provides it to the framework's runtime system in order to execute the program. An example of this could be WordPress~\cite{wordpress},
  \item  \emph{Convention over Configuration (CoC)} where the user writes his code and definition following certain patterns in naming or source code location. Using the source code and content-specific information, the framework interprets the execution flow that needs to be executed. This technique has been deeply explored in the last $10$ years. One of the first framework that introduce this design paradigm was Ruby on Rails~\cite{rubyonrails}. Other examples are Spring Boot and Cake PHP for example \cite{springboot, cakephp},
  \item \emph{Application Programing Interface (API)} where the framework or library provides a certain amount of functionality implemented in terms of functions or interfaces, and the user needs to compose those functions or implement those abstractions to achieve the desired results. This has been the traditional design paradigm for building any library or framework, and finally
  \item \emph{\acrfull{dsl}}~\cite{Fowler10} where the framework or library provides a new language that represents the domain problem, encoding the solution in terms of that \acrshort{dsl} language. An example of this type of design is Hibernate Query Language~\cite{hql}
   \end{inparaenum}.

There exists two types of \acrlong{dsl}s~\cite{dsl}: External \acrfull{dsl} and \acrfull{edsl}. The purpose of \acrshort{dsl}, is to create a completely new language with its own semantic, syntax, and interpreter. 
\acrshort{dsl}s are not general-purpose languages, because as their name indicates, they are domain-specific. \acrshort{edsl} are syntactically embedded in the host language of the library, and the user writes in that host language, but restricted by the \acrshort{edsl} abstractions.
\acrshort{dpfh} follows a \acrshort{edsl} approach taking advantage of the strong type \acrshort{hs} system providing correctness at type-level~\cite{curryhoward}.

\subsubsection{Architectural Design}
In this section we focus on the architectural design of the \acrshort{dpfh} using a \acrshort{edsl} approach. We have built a framework that contains
three important components: \acrshort{dsl}, \acrshort{idl} and \acrshort{rs}. 

\begin{figure}[!ht]
  \centering
   \includegraphics[width=1\textwidth, height=0.6\textheight]{dpf_haskell_v3.png}
    \caption[{[\acrshort{dpfh}] Architectural design of \acrshort{dpfh}}]{This diagram shows the architectural design of \acrshort{dpfh}. \acrshort{dpfh} is a \acrshort{dsl} which is built on three main components: \acrshort{dsl}, \acrshort{idl} and \acrshort{rs}. In the \acrshort{dsl} we can see how the user can compose the main stages of the \acrshort{dp}. \acrshort{idl} is showing how the frameworks is helping the user to transform that definition into real function or computations. Finally \acrshort{rs} execute all that definition plus functions. Execution layer indicates an example of a \acrshort{dp} running after being executed.}
    \label{fig:dpfh:1}
\end{figure}

In \autoref{fig:dpfh:1} we can appreciate the different components mentioned before that are the grey boxes.

\paragraph{DSL} The user interacts with the \acrshort{dsl} component where defines how the \acrshort{dp} flow
should be. Defining the flow consists on to provide a type-level specification about the channels that communicate each stage of the pipeline, as well as the data types those channels carry. 
For example, in the case of \autoref{prole} that we develop the \acrshort{wcc} algorithm, the user knows stages $\iwcc$, $\gwcc$, and $\owcc$ need to be connected with two channels. 
One of those channels is carrying the edges -- \texttt{Edge} data type -- and the other the accumulated connected components -- \texttt{ConnectedComp} data type --. 

\paragraph{IDL} Based on the definition provided in the \acrshort{dsl}, the user interacts with the \acrshort{idl} to build the functions with the algorithms needed for each stage: $\iwcc$, $\gwcc$, $\owcc$, $\fwcc$, and actors. 

\paragraph{RS} \acrshort{rs} is fed with the \acrshort{dp} definition and the functions implementations to finally execute the program. 

\subsection{Implementation}
In this section, we describe the implementation details of each architectural layer: \acrshort{dsl}, \acrshort{idl} and \acrshort{rs}.
As we have explained in \autoref{sec:contrib}, this library was published on Hackage~\cite{dynamic-pipeline}, the source code is open and can be found on this Github Repository~\cite{dynamic-pipeline-git}.

\subsubsection{DSL Grammar}\label{sub:sec:dsl-gram}
In order to provide correctness verification at compilation level, we define a \acrfull{cfg} that generates a \acrshort{dp} \acrshort{dsl} language. 
\acrshort{cfg} enables the user to define a \acrshort{dp} at type-level. 

\begin{definition}[\acrshort{dsl} \acrshort{cfg}]\label{def:cfg:dsl}
Lets $\gdsl = (N, \Sigma, DB, P)$ be a Context-Free Grammar, such that $N$ is the set of non-terminal symbols, $\Sigma$ the set of terminal symbols,
$DP \in N$ is the start symbol and $P$ are the generation rules. \autoref{fig:def:dpfh:dsl} shows the formal definition of the grammar.
\begin{figure}[H]
\begin{equation*}
    \boxed{
      \begin{aligned}
    N &= \{DP,S_r,S_k,G,F_b,CH,CH_s\},\\
    \Sigma &= \{\text{\mintinline{haskell}{Source}},\text{\mintinline{haskell}{Generator}},\text{\mintinline{haskell}{Sink}},\text{\mintinline{haskell}{FeedbackChannel}},\text{\mintinline{haskell}{Type}},\text{\mintinline{haskell}{Eof}},\text{\mintinline{haskell}{:=>}},\text{\mintinline{haskell}{:<+>}}\},
    \end{aligned}
    }
\end{equation*}
\begin{equation*}
  \boxed{
    \begin{aligned}
  P = \{\\
  DP  &\rightarrow S_r\ \text{\mintinline{haskell}{:=>}}\ G\ \text{\mintinline{haskell}{:=>}}\ S_k\ |\ S_r\ \text{\mintinline{haskell}{:=>}}\ G\ \text{\mintinline{haskell}{:=>}}\ F_b\ \text{\mintinline{haskell}{:=>}}\ S_k,\\
  S_r &\rightarrow \text{\mintinline{haskell}{Source}}\ CH_s,\\
  G   &\rightarrow \text{\mintinline{haskell}{Generator}}\ CH_s,\\
  S_k &\rightarrow \text{\mintinline{haskell}{Sink}},\\
  F_b &\rightarrow \text{\mintinline{haskell}{FeedbackChannel}} CH,\\
  CH_s &\rightarrow \text{\mintinline{haskell}{Channel}}\ CH,\\
  CH &\rightarrow \text{\mintinline{haskell}{Type :<+>}}\ CH\ |\ \text{\mintinline{haskell}{Eof}}\}
\end{aligned}
}
\end{equation*}
\caption[{[\acrshort{dpfh}] DSL Grammar definition}]{This is the Context-Free Grammar defined for the DSL. In the first box we can see $N$ which is the set of non-terminals symbols of the Grammar. $\Sigma$ which is the set of the terminal symbols and $P$ the production rules of the grammar.}
\label{fig:def:dpfh:dsl}
\end{figure}
\end{definition}

For encoding $\gdsl$ on the \acrshort{hs}, we use an \emph{Index type}~\cite{type-index} to keep track, at type-level, of the extra information required by the \acrshort{dp} definition such as channels and data types the channels carry. 

\begin{listing}[H]
  \begin{minted}[fontsize=\fontsize{10}{11}\selectfont,numbers=left,breaklines,frame=lines,framerule=2pt,framesep=2mm,baselinestretch=1.2,highlightlines={3,4}]{haskell}

data Source (a :: Type)
data Generator (a :: Type)
data Sink
data Eof
data Channel (a :: Type)
data FeedbackChannel (a :: Type)

  \end{minted}
  \caption[{[\mintinline{shell}{Flow.hs}] $\Sigma$ enconding of $G_{dsl}$}]{This code is showing most of the data types that represent the same terminal symbols $\Sigma \in G_{dsl}$. Those types that are indexed by another kind \mintinline{haskell}{Type}, allows to store information at type-level needed for interpret the DSL}
  \label{src:dpfh:1}
\end{listing}
  
In \autoref{src:dpfh:1}, there is an \emph{Index type} for each element of $\Sigma$ encoded in \acrshort{hs} Types.
The highlighted lines in \autoref{src:dpfh:1} shows the terminal symbols $\Sigma$ that are not indexed, because neither \mintinline{haskell}{Sink} nor \mintinline{haskell}{Eof} are carrying extra type-level information. 
In the case of \mintinline{haskell}{Sink}, since it is the last stage that does not connect further with any other stage, we do not need to indicate any channel information. 
\mintinline{haskell}{Eof} it is just a terminal type to disambiguate the \mintinline{haskell}{Channel (a :: Type)} subtree for the full parser tree. 
\mintinline{haskell}{Channel} can carry any type because it needs to be polymorphic to support a different number of channels and data types.

\begin{listing}[H]
  \begin{minted}[fontsize=\fontsize{10}{11}\selectfont,numbers=left,breaklines,frame=lines,framerule=2pt,framesep=2mm,baselinestretch=1.2,highlightlines={1,5}]{haskell}
    
    data chann1 :<+> chann2 = chann1 :<+> chann2
    deriving (Typeable, Eq, Show, Functor, Traversable, Foldable, Bounded)
    infixr 5 :<+>
    
    data a :=> b = a :=> b
    deriving (Typeable, Eq, Show, Functor, Traversable, Foldable, Bounded)
    infixr 5 :=>
    
  \end{minted}
  \caption[{[\mintinline{shell}{Flow.hs}] $\Sigma$ enconding of $G_{dsl}$ - Especial non-terminals}]{Special terminal symbols $\{\text{\mintinline{haskell}{:<+>}}, \text{\mintinline{haskell}{:=>}}\} \in \Sigma$. This terminal symbols allows to index two types in order to combine several of them and build a chain of stages (\mintinline{haskell}{:=>}) and a set of channels (\mintinline{haskell}{:<+>}).}
  \label{src:dpfh:2}
\end{listing}

There are two important terminal symbols in $\Sigma$: \mintinline{haskell}{:=>} and \mintinline{haskell}{:<+>}.
In \autoref{src:dpfh:2}, the definition shows how \mintinline{haskell}{:=>} and \mintinline{haskell}{:<+>} can combine 2 (two) types. 
The propose of writing \mintinline{haskell}{:=>} and \mintinline{haskell}{:<+>} as types is to have a syntactic sugar type combinator for writing the \acrshort{dsl} according to the \acrshort{cfg}. 
Apart from that, they are different because two distinguishable terminal symbols $\Sigma$ are needed to separate the encoding of the pipeline stage ($\iwcc$, $\gwcc$, $\owcc$)
from the encoding of channel composition in the same stage, as we can appreciate in \dref{def:cfg:dsl}.

Now, we can start defining our pipelines at type-level. For example, if we want to generate a \acrshort{dp} that eliminates duplicated elements in a stream, we know that we only need one channel connecting the stages that carries out the type of the element, in this case, \mintinline{haskell}{Int} (see \autoref{src:dpfh:3}).

\begin{listing}[H]
  \begin{minted}[fontsize=\fontsize{10}{11}\selectfont,numbers=left,breaklines,frame=lines,framerule=2pt,framesep=2mm,baselinestretch=1.2,highlightlines={}]{haskell}

type DPExample = Source (Channel (Int :<+> Eof)) 
              :=> Generator (Channel (Int :<+> Eof)) 
              :=> Sink
   
  \end{minted}
  \caption[{[\mintinline{shell}{Repeated.hs} Example of \acrshort{dp} encoded in $G_{dsl}$}]{This example shows the \acrshort{dsl} encoding in \acrshort{dp} of repeated elements problems}
  \label{src:dpfh:3}
\end{listing}

\subsubsection{DSL Validation}\label{sub:sec:dsl-val}
The language generated by the grammar needs to be validated to avoid errors or provide an incorrect \acrshort{dp} definition.
Fortunately, \acrshort{hs} provides several Type-level techniques~\cite{type-haskell} which allows to verify properties of programs before running them, 
preventing the users to introduce bugs, reducing errors. This verification done by the compiler establish a Curry-Howard Isomorphism~\cite{curryhoward}, i.e. 
\emph{Propositions as Types - Programs as Proof}. It is important to remark here that \acrshort{hs} is not a theorem prover System like Coq~\cite{coq}, but some verifications, as we present in this work, can be done with \acrshort{ghc} to ensure correctness on programs.
Although \acrshort{hs} provides tools to build advanced type-level verifications, all these techniques require the addition of \emph{Haskell Language Extensions}.

Once we have the encoded \acrshort{dp} problem in the \acrshort{dsl} grammar -- see \autoref{sub:sec:dsl-gram} --, we can proceed on validating that encoded grammar. 
The implementation of the validation of the \acrshort{dsl} \acrshort{cfg} at type-level, has been done using \emph{Associated Type Families}~\cite{associated-types}.

\begin{listing}[H]
  \begin{minted}[fontsize=\fontsize{10}{11}\selectfont,numbers=left,breaklines,frame=lines,framerule=2pt,framesep=2mm,baselinestretch=1.2,highlightlines={6,18}]{haskell}

type family And (a :: Bool) (b :: Bool) :: Bool where
    And 'True 'True = 'True
    And a b         = 'False
  

type family IsDP (dpDefinition :: k) :: Bool where
    IsDP (Source (Channel inToGen) :=> Generator (Channel genToOut) :=> Sink)
        = And (IsDP (Source (Channel inToGen))) (IsDP (Generator (Channel genToOut)))
    IsDP ( Source (Channel inToGen) :=> Generator (Channel genToOut) :=> FeedbackChannel toSource :=> Sink)
        = And (IsDP (Source (Channel inToGen))) (IsDP (Generator (Channel genToOut)))
    IsDP (Source (Channel (a :<+> more)))     
        = IsDP (Source (Channel more))
    IsDP (Source (Channel Eof))               = 'True
    IsDP (Generator (Channel (a :<+> more)))  = IsDP (Generator (Channel more))
    IsDP (Generator (Channel a))              = 'True
    IsDP x                                    = 'False
     
type family ValidDP (a :: Bool) :: Constraint where
  ValidDP 'True = ()
  ValidDP 'False = TypeError
                    ( 'Text "Invalid Semantic for Building DP Program"
                      ':$$: 'Text "Language Grammar:"
                      ':$$: 'Text "DP       -> Source CHANS :=> Generator CHANS :=> Sink"
                      ':$$: 'Text "DP       -> Source CHANS :=> Generator CHANS :=> FEEDBACK :=> Sink"
                      ':$$: 'Text "CHANS    -> Channel CH"
                      ':$$: 'Text "FEEDBACK -> FeedbackChannel CH"
                      ':$$: 'Text "CH       -> Type :<+> CH | Eof"
                      ':$$: 'Text "Example: 'Source (Channel (Int :<+> Int)) :=> Generator (Channel (Int :<+> Int)) :=> Sink'"
                    )
  \end{minted}
  \caption[{[\mintinline{shell}{Stage.hs}] Validating encoded in $G_{dsl}$ - FCF}]{Type Families \mintinline{haskell}{And}, \mintinline{haskell}{IsDP} and \mintinline{haskell}{ValidDP} which allows to perform a type-level validation over a \acrshort{dsl} \acrshort{cfg} definition.}
  \label{src:dpfh:4}
\end{listing}

In \autoref{src:dpfh:4}, there are 3(three) Type families that helps to validate the \acrshort{dsl} \acrshort{cfg}. 
\mintinline{haskell}{IsDP} associated type family is checking the production rules $P$ of the grammar defined in \autoref{fig:def:dpfh:dsl}, returning a promoted data type~\cite{promoted-types} (not a boolean value) \mintinline{haskell}{'True} in case
the production rule matches all the generated language, or \mintinline{haskell}{'False} otherwise. 
\mintinline{haskell}{ValidDP} is taking the result of \mintinline{haskell}{IsDP} type application, associating \mintinline{haskell}{'True} promoted boolean type to empty \mintinline{haskell}{()} constraint. An empty constraint is an 
indication of nothing to be restricted, meaning that if \mintinline{haskell}{ValidDP} is used as a constraint, and it is fully applied to \mintinline{haskell}{()}, it will give the compiler the evidence that there is no error at type-level.
\mintinline{haskell}{ValidDP} is also associating \mintinline{haskell}{'False} to a custom \mintinline{haskell}{TypeError} which will appear at compilation time if the \acrshort{dp} \acrshort{dsl} definition fully applies to that.

\begin{listing}[H]
  \begin{minted}[fontsize=\fontsize{10}{11}\selectfont,numbers=left,breaklines,frame=lines,framerule=2pt,framesep=2mm,baselinestretch=1.2,highlightlines={2}]{haskell}

mkDP :: forall dpDefinition filterState filterParam st.
    ( ValidDP (IsDP dpDefinition)
    , DPConstraint dpDefinition filterState st filterParam)
 => Stage (WithSource dpDefinition (DP st)) 
 -> GeneratorStage dpDefinition filterState filterParam st  
 -> Stage (WithSink dpDefinition (DP st))  
 -> DP st ()
mkDP = ...

someFunc = mkDP @DPExample ...

  \end{minted}
  \caption[{[\mintinline{shell}{Stage.hs}] Using validation of \acrshort{dp} encoded in $G_{dsl}$}]{Definition of \mintinline{haskell}{mkDP} function of the Framework which uses type-level validation of the grammar \mintinline{haskell}{ValidDP (IsValid Type)}. Last line of the code is showing that using that function will compile-time check the definition of \mintinline{haskell}{DPExample} type.}
  \label{src:dpfh:5}
\end{listing}

\subsubsection{\acrfull{idl}}
\acrshort{idl} component takes the \acrshort{dp} definition made on with \acrshort{dsl} component to interpret and generate the function definitions
that the user needs to fill in for solving a specific problem. In \autoref{sec:dp}, we have described what the user needs to provide in a \acrshort{dp} algorithm: $\iwcc$, $\gwcc$, $\owcc$, and the $\fwcc$ with the non-empty set of Actors.
The \acrshort{idl} generates the function definitions with an empty implementation to be completed by the user, ensuring that those functions will give "Proof" -- in terms of Curry-Howard Correspondence~\cite{curryhoward} --  of the "Propositions" defined on the \acrshort{dsl}.

Similar techniques that we used on \autoref{sub:sec:dsl-val} are also used here. 
On the first hand, we use \emph{Type-level Defunctionalization}~\cite{defunctionalization, fun-type-function-haskell} to let the compiler generates the signatures of the required functions. 
On the other hand, we use \emph{Term-level Defunctionalization} to interpret those functions.
Moreover, \emph{Indexed Types}~\cite{type-index} and \emph{Heterogeneous List}~\cite{hlist} are used to keep track of the dynamic number and polymorphic types of the functions parameters. 

\begin{listing}[H]
  \begin{minted}[fontsize=\fontsize{10}{11}\selectfont,numbers=left,breaklines,frame=lines,framerule=2pt,framesep=2mm,baselinestretch=1.2,highlightlines={2,6,10}]{haskell}
withSource :: forall (dpDefinition :: Type) st. WithSource dpDefinition (DP st) 
            -> Stage (WithSource dpDefinition (DP st))
withSource = mkStage' @(WithSource dpDefinition (DP st))

withGenerator :: forall (dpDefinition :: Type) (filter :: Type) st. WithGenerator dpDefinition filter (DP st) 
              -> Stage (WithGenerator dpDefinition filter (DP st))
withGenerator = mkStage' @(WithGenerator dpDefinition filter (DP st))

withSink :: forall (dpDefinition :: Type) st. WithSink dpDefinition (DP st) 
           -> Stage (WithSink dpDefinition (DP st))
withSink = mkStage' @(WithSink dpDefinition (DP st))
  \end{minted}
  \caption[{[\mintinline{shell}{Stage.hs}] Using with Interpreters of \acrshort{dp} encoded in $G_{dsl}$}]{This code is showing the different interpreters combinators to help the user to generate the functions of the principal stages of \acrshort{dp}}
  \label{src:dpfh:6}
\end{listing}

In \autoref{src:dpfh:6} we can appreciate the different combinators of the \acrshort{idl} that helps the user of the framework to interpret the \acrshort{dsl} to generate the function definitions.
\mintinline{haskell}{Stage} data type will be cover in \autoref{src:dpfh:8}, but it is a wrapper type of a pipeline stage -- minimal unit of execution --, containing the function to be executed -- here is the use \emph{Term-level Defunctionalization} --.
\mintinline{haskell}{withSource}, \mintinline{haskell}{withGenerator}, and \mintinline{haskell}{withSink} are aliases of the function \mintinline{haskell}{mkStage'} which is the combinator that is applying the \emph{Associated Type} related to that stage. For example \mintinline{haskell}{withSource}, is equivalent to \mintinline{haskell}{mkStage' @(WithSource dpDefinition (DP st))}.
For each \emph{Associated Type Family} defintion, there exist an equivalent term-level definition: \mintinline{haskell}{WithSource} type with \mintinline{haskell}{withSource} term , \mintinline{haskell}{WithGenerator} type with \mintinline{haskell}{withGenerator} term, and \mintinline{haskell}{WithSink} type with \mintinline{haskell}{withSink} term -- notice the capital case letter "W" indicating the type and not the term --.

\begin{listing}[H]
  \begin{minted}[fontsize=\fontsize{10}{11}\selectfont,numbers=left,breaklines,frame=lines,framerule=2pt,framesep=2mm,baselinestretch=1.2,highlightlines={7,11}]{haskell}
type family WithSource (dpDefinition :: Type) (monadicAction :: Type -> Type) :: Type where
  WithSource (Source (Channel inToGen) :=> Generator (Channel genToOut) :=> Sink) monadicAction
      = WithSource (ChanIn inToGen) monadicAction
  WithSource (Source (Channel inToGen) :=> Generator (Channel genToOut) :=> FeedbackChannel toSource :=> Sink) monadicAction 
      = WithSource (ChanOutIn toSource inToGen) monadicAction
  WithSource (ChanIn (dpDefinition :<+> more)) monadicAction         
      = WriteChannel dpDefinition -> WithSource (ChanIn more) monadicAction
  WithSource (ChanIn Eof) monadicAction                              
      = monadicAction ()
  WithSource (ChanOutIn (dpDefinition :<+> more) ins) monadicAction  
      = ReadChannel dpDefinition -> WithSource (ChanOutIn more ins) monadicAction
  WithSource (ChanOutIn Eof ins) monadicAction                       
      = WithSource (ChanIn ins) monadicAction
  WithSource dpDefinition _                                          
      = TypeError
          ( 'Text "Invalid Semantic for Source Stage"
            ':$$: 'Text "in the DP Definition '"
            ':<>: 'ShowType dpDefinition
            ':<>: 'Text "'"
            ':$$: 'Text "Language Grammar:"
            ':$$: 'Text "DP       -> Source CHANS :=> Generator CHANS :=> Sink"
            ':$$: 'Text "DP       -> Source CHANS :=> Generator CHANS :=> FEEDBACK :=> Sink"
            ':$$: 'Text "CHANS    -> Channel CH"
            ':$$: 'Text "FEEDBACK -> FeedbackChannel CH"
            ':$$: 'Text "CH       -> Type :<+> CH | Eof"
            ':$$: 'Text "Example: 'Source (Channel (Int :<+> Int)) :=> Generator (Channel (Int :<+> Int)) :=> Sink'"
          )
  \end{minted}
  \caption[{[\mintinline{shell}{Stage.hs}] WithSource Associate Type Details}]{An example of the Associated Type Family \mintinline{haskell}{WithSource} that allows to implement \emph{Type-level Defunctionalization} technique that will be the Type-level verification of the term \mintinline{haskell}{withSource}}
  \label{src:dpfh:7}
\end{listing}

In \autoref{src:dpfh:7}, in the highlighted lines, it can be seen how \emph{Type-level Defunctionalization} is being expanded in a signature function definition with the form \mintinline{haskell}{WriteChannel a -> ReadChannel b -> ... -> monadicAction ()} depending on \acrshort{dp} language definition. 

\begin{listing}[H]
  \begin{minted}[fontsize=\fontsize{10}{11}\selectfont,numbers=left,breaklines,frame=lines,framerule=2pt,framesep=2mm,baselinestretch=1.2,highlightlines={12,16}]{haskell}

data Stage a where
  Stage :: Proxy a -> a -> Stage a

mkStage' :: forall a. a -> Stage a
mkStage' = Stage (Proxy @a)
    
  \end{minted}
  \caption[{[\mintinline{shell}{Stage.hs}] Stage Data Type}]{\mintinline{haskell}{Stage} data type for implementing \emph{Term-level Defunctionalization} providing evidence to the Type-Level Associated types}
  \label{src:dpfh:8}
\end{listing}

In \autoref{src:dpfh:8}, \mintinline{haskell}{Stage} data type uses a \mintinline{haskell}{Proxy} phantom type. 
This phantom type allow \mintinline{haskell}{Stage} to index the type definition generated by \mintinline{haskell}{a}.
For example, in \autoref{src:dpfh:6}, when \mintinline{haskell}{withSource} interpreter is applied to \mintinline{haskell}{WithSource dpDefinition},  
the compiler is provided with \mintinline{haskell}{dpDefinition} \acrshort{dsl} type, it expands the function signature belonging to that \acrshort{dp} definition inside the \mintinline{haskell}{Stage}.

\paragraph{Generator and Filter}
According to \acrshort{dp} definition in \autoref{sec:dp}, $\gwcc$ has a $\fwcc$ template in order to know how to dynamically interpose a new $\fwcc$ during the runtime execution of the program.
Let's first study $\fwcc$ Data Type in the context of the framework.

\begin{listing}[H]
  \begin{minted}[fontsize=\fontsize{10}{11}\selectfont,numbers=left,breaklines,frame=lines,framerule=2pt,framesep=2mm,baselinestretch=1.2,highlightlines={2,5}]{haskell}

newtype Actor dpDefinition filterState filterParam monadicAction =
    Actor {  unActor :: MonadState filterState monadicAction => Stage (WithFilter dpDefinition filterParam monadicAction) }

newtype Filter dpDefinition filterState filterParam st =
    Filter { unFilter :: NonEmpty (Actor dpDefinition filterState filterParam (StateT filterState (DP st))) }
    deriving Generic
    
  \end{minted}
  \caption[{[\mintinline{shell}{Stage.hs}] Filter / Actor Data Type}]{This code shows the definition of the \mintinline{haskell}{Filter} data type which contains a non-empty set of \mintinline{haskell}{Actor}. The \mintinline{haskell}{Actor} data type is an \mintinline{haskell}{Stage} in the Context of the \mintinline{haskell}{MonadState} to allow keeping a local memory in the execution context of the filter.}
  \label{src:dpfh:9}
\end{listing}

In \autoref{src:dpfh:9} the definition of the \mintinline{haskell}{Filter} data type contains a non-empty set of \mintinline{haskell}{Actor}.
An \mintinline{haskell}{Actor} is a \mintinline{haskell}{Stage}, because an actor is the minimal unit of execution of a filter. 
A \mintinline{haskell}{Filter} has a \mintinline{haskell}{NonEmpty Actor} -- Non-empty List -- because a filter is built by a sequence of actors calls. 
Moreover, \mintinline{haskell}{Actor} Stage is defunctionalized with \mintinline{haskell}{WithFilter} \emph{Associated Type Family}. 
\mintinline{haskell}{Filter} runs in an explicit \mintinline{haskell}{StateT} monadic context. This is because the $\fwcc$ instance should have an state, according to \acrshort{dp} definition in \autoref{sec:dp}.
For example, in the case of $\dpwcc$, as we have seen in \autoref{prole}, $\fwcc$ keeps an updated list of connected components that updates as long as it receives more edges that are connected with the current list of vertices.
\mintinline{haskell}{Actor} data type -- see \autoref{src:dpfh:9} --, is constrained by \mintinline{haskell}{MonadState} which is in the same execution context of the whole \mintinline{haskell}{NonEmpty Actor} list of the \mintinline{haskell}{Filter}. 
This means the \mintinline{haskell}{StateT} is executed for each \mintinline{haskell}{Actor} of that filter, sharing the same state between them. 

\begin{listing}[H]
  \begin{minted}[fontsize=\fontsize{10}{11}\selectfont,numbers=left,breaklines,frame=lines,framerule=2pt,framesep=2mm,baselinestretch=1.2,highlightlines={}]{haskell}

mkFilter :: forall dpDefinition filterState filterParam st. WithFilter dpDefinition filterParam (StateT filterState (DP st)) 
         -> Filter dpDefinition filterState filterParam st
mkFilter = Filter . single

single :: forall dpDefinition filterState filterParam st. WithFilter dpDefinition filterParam (StateT filterState (DP st)) 
       -> NonEmpty (Actor dpDefinition filterState filterParam (StateT filterState (DP st)))
single = one . actor

actor :: forall dpDefinition filterState filterParam st. WithFilter dpDefinition filterParam (StateT filterState (DP st)) 
      -> Actor dpDefinition filterState filterParam (StateT filterState (DP st))
actor = Actor . mkStage' @(WithFilter dpDefinition filterParam (StateT filterState (DP st)))

(|>>>) :: forall dpDefinition filterState filterParam st. Actor dpDefinition filterState filterParam (StateT filterState (DP st)) 
       -> Filter dpDefinition filterState filterParam st 
       -> Filter dpDefinition filterState filterParam st
(|>>>) a f = f & _Wrapped' %~ (a <|)
infixr 5 |>>>

(|>>) :: forall dpDefinition filterState filterParam st. Actor dpDefinition filterState filterParam (StateT filterState (DP st)) 
      -> Actor dpDefinition filterState filterParam (StateT filterState (DP st)) 
      -> Filter dpDefinition filterState filterParam st
(|>>) a1 a2 = Filter (a1 <|one a2)
infixr 5 |>>
  \end{minted}
  \caption[{[\mintinline{shell}{Stage.hs}] Filter / Actor smart constructors and combinators}]{Combinators and small constructor to enable building actors and filter.}
  \label{src:dpfh:10}
\end{listing}

Finally, in \autoref{src:dpfh:10}, some combinators and smart constructors are provided in the framework to enable the construction of \mintinline{haskell}{Filter} and \mintinline{haskell}{Actor}.
\mintinline{haskell}{mkFilter} is a smart constructor for \mintinline{haskell}{Filter} Data Constructor. \mintinline{haskell}{single} wraps one actor inside a \mintinline{haskell}{Filter}.
\mintinline{haskell}{actor} is a smart constructor for \mintinline{haskell}{Actor} Data Constructor. \mintinline{haskell}{(|>>>)} is an appending combinator of an \mintinline{haskell}{Actor} to a \mintinline{haskell}{Filter}. 
\mintinline{haskell}{(|>>>)} also ensures actor execution order, i.e. the latest actor added is the latest to be executed.


\begin{listing}[H]
  \begin{minted}[fontsize=\fontsize{10}{11}\selectfont,numbers=left,breaklines,frame=lines,framerule=2pt,framesep=2mm,baselinestretch=1.2,highlightlines={}]{haskell}
    data GeneratorStage dpDefinition filterState filterParam st = GeneratorStage
    { _gsGenerator      :: Stage (WithGenerator dpDefinition (Filter dpDefinition filterState filterParam st) (DP st))
    , _gsFilterTemplate :: Filter dpDefinition filterState filterParam st
    }  
  \end{minted}
  \caption[{[\mintinline{shell}{Stage.hs}] Generator}]{\mintinline{haskell}{Generator} Data type which contains the \mintinline{haskell}{Stage} code of the generator itself, and the \mintinline{haskell}{Filter} template that it can be spawned by the \mintinline{haskell}{Generator}.}
  \label{src:dpfh:11}
\end{listing}

In \autoref{src:dpfh:11}, $\gwcc$ contains a $\fwcc$ template and its own stage behavior.
\mintinline{haskell}{Generator} data type has a field with the \mintinline{haskell}{Filter} template that could be spawned by the algorithm defined by the user according to the data received from its input channels.
\mintinline{haskell}{Generator} has also another field with the behavior of the $\gwcc$ -- a \mintinline{haskell}{Stage} --. 

\subsubsection{\acrfull{rs}}
The \acrshort{rs} can be divided into two parts: the mechanism to generate stages dynamically in runtime, and the execution entry point of the \acrshort{dp}.
Regarding execution entry point, all the stages that we have seen in previous sections are the pieces needed to build an executable \mintinline{haskell}{DP st a} monad.
This executable monad has an existential type similar to \mintinline{haskell}{ST} monad to not escape out from the context on different stages.
Once the dynamic pipeline starts to execute, the core of the framework dynamically generates stages between $\gwcc$ and previous stages, according to the user definition, i.e. an \emph{anamorphism}~\cite{lenses} that creates $\fwcc$ instances until some condition is met.

\begin{listing}[H]
  \begin{minted}[fontsize=\fontsize{10}{11}\selectfont,numbers=left,breaklines,frame=lines,framerule=2pt,framesep=2mm,baselinestretch=1.2,highlightlines={2,12,14,15,16,17,18}]{haskell}
unfoldF :: forall dpDefinition readElem st filterState filterParam l. SpawnFilterConstraint dpDefinition readElem st filterState filterParam l
        => UnFoldFilter dpDefinition readElem st filterState filterParam l 
        -> DP st (HList l) 
unfoldF = loopSpawn

where
  loopSpawn uf@UnFoldFilter{..} =
    maybe (pure _ufRsChannels) (loopSpawn <=< doOnElem uf) =<< DP (pull _ufReadChannel)

  doOnElem uf@UnFoldFilter{..} elem' = do
    _ufOnElem elem'
    if _ufSpawnIf elem'
     then do
       (reads', writes' :: HList l3) <- getFilterChannels <$> DP (makeChansF @(ChansFilter dpDefinition))
       let hlist = elem' .*. _ufReadChannel .*. (_ufRsChannels `hAppendList` writes')
       void $ runFilter _ufFilter (_ufInitState elem') hlist (_ufReadChannel .*. (_ufRsChannels `hAppendList` writes'))
       return $ uf { _ufReadChannel = hHead reads', _ufRsChannels = hTail reads' }
     else return uf

  \end{minted}
  \caption[{[\mintinline{shell}{Stage.hs}] unfoldF}]{\mintinline{haskell}{unfolF} is the \emph{anamorphism} combinator to spawn new \mintinline{haskell}{Filter} types between the \mintinline{haskell}{Generator} and previous stages.}
  \label{src:dpfh:12}
\end{listing}

In \autoref{src:dpfh:12}, it is presented how is the \emph{anamorphism} mechanims that generates dynamic stages between $\gwcc$ and the previous stages.
That \emph{anamorphism} is implemented with the function \mintinline{haskell}{unfoldF}. That function receives an \mintinline{haskell}{UnFoldFilter} Data type, which contains the recipe for controlling that unfold recursive call. 
In line $12$, \mintinline{haskell}{_ufSpawnIf} field of \mintinline{haskell}{UnFoldFilter}, indicates when to stop the recursion. 
Inside the conditional, in line $14$, new channels are created for the new filter to be spawned. Those new channels connect the new filter with the previous stages and with \mintinline{haskell}{Generator}. 
After that, in line $16$ \mintinline{haskell}{runFilter} starts the monadic computation, spawning the filter stage with its actors.
Finally, the new list of channels are returned for the next recursive step to allow further channel connections.

\begin{listing}[H]
  \begin{minted}[fontsize=\fontsize{10}{11}\selectfont,numbers=left,breaklines,frame=lines,framerule=2pt,framesep=2mm,baselinestretch=1.2,highlightlines={}]{haskell}
mkUnfoldFilter :: (readElem -> Bool) 
    -> (readElem -> DP st ()) 
    -> Filter dpDefinition filterState filterParam st 
    -> (readElem -> filterState)
    -> ReadChannel readElem
    -> HList l 
    -> UnFoldFilter dpDefinition readElem st filterState filterParam l


mkUnfoldFilterForAll' :: (readElem -> DP st ())
                      -> Filter dpDefinition filterState filterParam st
                      -> (readElem -> filterState)
                      -> ReadChannel readElem
                      -> HList l
                      -> UnFoldFilter dpDefinition readElem st filterState filterParam l

mkUnfoldFilterForAll :: Filter dpDefinition filterState filterParam st
                      -> (readElem -> filterState)
                      -> ReadChannel readElem
                      -> HList l
                      -> UnFoldFilter dpDefinition readElem st filterState filterParam l
   \end{minted}
  \caption[{[\mintinline{shell}{Stage.hs}] UnfoldFilter combinators}]{Combinators for building \mintinline{haskell}{UnfoldFilter} types indicating the type of the \mintinline{haskell}{unfold} that the user want to achieve.}
  \label{src:dpfh:13}
\end{listing}

Several smart constructors are also provided for building \mintinline{haskell}{UnfoldFilter} Data Type.
In \autoref{src:dpfh:13} the first combinator is the default smart constructor.  \begin{inparaenum}[i\upshape)]
  \item First field \mintinline{haskell}{(readElem -> Bool)} indicate if the a new filter should be spawn or not.
  \item Second field \mintinline{haskell}{(readElem -> DP st ())} is a monadic optional computation to do when received a new element, for example logging.
  \item Third field \mintinline{haskell}{Filter} data type to be spawned.
  \item Fourth field \mintinline{haskell}{(readElem -> filterState)} is initialization of the \mintinline{haskell}{Filter} State.
  \item Fifth field \mintinline{haskell}{(ReadChannel readElem)} that feeds the filter instance.
  \item Last field is the \emph{Heterogeneous List} with the rest of the channels to connect with other stages.
\end{inparaenum}.
The combinator \mintinline{haskell}{mkUnfoldFilterForAll} is an smart constructor of \mintinline{haskell}{UnfoldFilter} that allows to spawn a new filter for each element received in the $\gwcc$.

\subsection{Libraries and Tools}
\subsection{Parallelization} 
One of the most important components of the implementation is the selection of concurrency libraries to support an intensive parallelization workload. Parallelization techniques and tools have been intensively studied and implemented in \acrshort{hs} \cite{monadpar}. 
Indeed, it is well known that green threads and sparks allow spawning thousands to millions of parallel computations. 
These parallel computations do not penalize performance when compare with \acrfull{os} level threading \cite{parallelbook}. 
A straightforward assumption to achieve here, is to use \texttt{monad-par} library \cite{monadparlib, monadpar}. 
Nevertheless, in this work, we have discarded the use of sparks \cite{sparks} because we can achieve the level of required parallelism spawning green threads only.
The next obvious choice is to use \mintinline{haskell}{forkIO :: IO () -> IO ThreadId} from \texttt{base} library \cite{forkio}. 
However, that would imply handling all the threads lifecycles and errors programmatically without any abstraction to facilitate that complex task. 
Therefore, we choose \texttt{async} library \cite{async} which enables to spawn asynchronous computations \cite{parallelbook} on \acrshort{hs} using green threads, and at the same time, it provides useful combinators to managing thread terminations and errors.

\subsubsection{Channels\label{section:channels}} 
We have several techniques to our disposal to communicate between threads or sparks in \acrshort{hs} like \mintinline{haskell}{MVar} or concurrent safe mechanisms like \acrfull{stm} \cite{stm}. 
At the same time, in \acrshort{hs} library ecosystem, we dispose of \texttt{Channels} abstractions based on both mentioned communication techniques. 
In that sense, for conducting the communication between dynamic stages and data flowing in a \acrshort{dp}, we have selected \texttt{unagi-chan} library \cite{unagi} which provides the following advantages to our solution: Firstly, \mintinline{haskell}{MVar} channel without using \acrshort{stm} reducing overhead. 
\acrshort{stm} is not required in a \acrshort{dp} because one specific stage which is running in a separated thread, can only access to its \texttt{I/O} channels for reading/writing accordingly, and those operations are not concurrently shared by other threads (stages) for the same channels. 
Second, non-blocking channels. \texttt{unagi-chan} library contains blocking and non-blocking channels for reading. This aspect is key to gain speed up on the implementation. Third, the library is optimized for $x86$ architectures with use of low-level \texttt{fetch-and-add} instructions. Finally, \texttt{unagi-chan} is $100x$ faster~\cite{unagi-bench} on Benchmarking compare with \acrshort{stm} and default base \mintinline{haskell}{Chan} implementations.


\section{Enumerating Weakly Connected Component on the DPF}\label{sec:wcc-dpf}

Having the \acrshort{dpfh} in place as we describe in the previous \autoref{dp-hs}, we implement the algorithm for enumerating weakly components of a graph using the \acrshort{dpfh}.
We describe first the general algorithm in the context of \acrshort{dp}.
Secondly, we show the implementation details of that algorithm in the context of the \acrshort{dpfh} and finally we show the experiments and results conducted over that implementation.

\iffalse
\subsection{\texorpdfstring{$\dpwcc$}{Lg} Algorithm}\label{sub:sec:wcc:algo}
Let us consider the problem of computing/enumerating the (weak) connected components of a graph $G$ using \acrshort{dp}. 
A connected component of a graph is a subgraph in which any two vertices are connected by paths.  
Thus, finding connected components of an undirected graph implies obtaining the minimal partition of the set of nodes induced by the relationship \textit{connected}, i.e., there is a path between each pair of nodes. 
An example of that graph can be seen in \autoref{fig:example_dp_graph}.
The input of the Dynamic Pipeline for computing the WCC of a graph, $\dpwcc$, is a sequence of edges ending with $\eof$\footnote{Note that there are neither isolated vertices nor loops in the source graph $G$.}. 
The connected components are output as soon as they are computed, i.e., they are produced incrementally. 
Roughly speaking the idea of the algorithm is that the weakly connected components are built in two phases. 
In the first phase filter instance stages receive the edges of the input graph and create sets of connected vertices. 
During the second phase, these filter instances construct maximal subsets of connected vertices, i.e. the vertices corresponding to (weakly) connected components.
%
$\dpwcc$ is defined in terms of the behavior of its four kinds stages: \textit{Source} ($\iwc$),  \textit{Generator} ($\gwc$),  \textit{Sink} ($\owc$), and \textit{Filter}($\fwc$) stages. Additionally,  the channels connecting these stages must be defined. 
In $\dpwcc$, stages are connected linearly and unidirectionally through the channels $\ice$ and  $\csofv$. Channel $\ice$ carries edges while channel  $\csofv$ conveys sets of connected vertices. Both channels end by the $\eof$ mark. 
The behavior of $\fwc$ is given by a sequence of two actors (scripts). Each actor corresponds to a phase of the algorithm. In what follows, we denote these actors by $\Act$ and $\Actt$, respectively. 
The script $\Act$ keeps a set of connected vertices ($CV$) in the state of the $\fwc$ instance. When an edge $e$ arrives, if an endpoint of $e$ is present in the state, then the other endpoint of $e$ is added to $CV$. 
Edges without incident endpoints are passed to the next stage. When $\eof$ arrives at channel $\ice$, it is passed to the next stage, and the script $\Actt$ starts its execution. 
If script $\Actt$ receives a set of connected vertices $CV$ in $\csofv$, it determines if the intersection between $CV$ and the nodes in its state is not empty. If so, it adds the nodes in $CV$  to its state. 
Otherwise, the $CV$ is passed to the next stage. Whenever $\eof$ is received, $\Actt$ passes--through $\csofv$-- the set of vertices in its state and the $\eof$ mark to the next stage; then, it dies.
The behavior of $\iwc$ corresponds to the identity transformation over the data stream of edges.  As edges arrive, they are passed through  $\ice$ to the next stage. When receiving $\eof$ on $\ice$, this mark is put on both channels. 
Then, $\iwc$ dies. 

%\begin{wrapfigure}{r}{0.4\textwidth}
\begin{figure}
 \begin{center}
\inputtikz{graph_example_wcc}
\end{center}
\caption[{[PoC] Graph WCC Example}]{Example of a graph with two weakly connected components: $\{1,2\}$ and $\{3,4,5,6\}$}
\label{fig:example_dp_graph}
\end{figure}
%\end{wrapfigure}

Let us describe this behavior with the example of the graph shown in \autoref{fig:example_dp_graph}.

\begin{figure}[h!]
  \centering
\inputtikz{dp_example_0}
\caption[{[PoC] $\dpwcc$ Initial Setup}]{$\dpwcc$ Initial setup. Stages Source, Generator, and Sink are represented by the squares labeled by $\mathsf{Sr_{WCC}}$, $\mathsf{G_{WCC}}$ and $\mathsf{Sk_{WCC}}$, respectively.  The square $\fwc$ corresponding to the Filter stage template is the parameter of $\gwc$. Arrows $\rightrightarrows$ between represents the connection of stages through two channels, $\ice$, and $\csofv$. The arrow  $\rightarrow$ represents the channel $\csofv$ connecting the stages $\mathsf{G_{WCC}}$ and $\mathsf{Sk_{WCC}}$. The arrow $\Longrightarrow$ stands for I/O data flow. Finally, the input stream comes between the dotted lines on the left and the WCC computed incrementally will be placed between the solid lines on the right.}
\label{fig:dp_example_0}
\end{figure}

\autoref{fig:dp_example_0} depicts the initial configuration of $\dpwcc$. 
The interaction of $\dpwcc$ with the "external" world is done through the stages $\iwc$ and $\owc$. 
Indeed, once activated the initial $\dpwcc$, the input stream -- consisting of a sequence containing all the edges in the graph in \autoref{fig:example_dp_graph} -- feeds $\iwc$ while  $\owc$ emits incrementally the resulting weakly connected components.  
In what follows \autoref{fig:dp_example_1_2}, \autoref{fig:dp_example_3_4}, \autoref{fig:dp_example_5_6}, \autoref{fig:dp_example_7_8} and \autoref{fig:dp_example_9_10} depict the evolution of the $\dpwcc$.
 
\begin{figure}[h!]
\centering
\begin{subfigure}[b]{\textwidth}
 \centering
  \inputtikz{dp_example_1}
  \caption{The edge $(1,2)$ is arriving to $\gwc$.}
  \label{fig:dp_example_1_2a}
\end{subfigure}
\vspace{.3cm}

\begin{subfigure}[b]{\textwidth}
 \centering
  \inputtikz{dp_example_2}
  \caption{When the edge $(1,2)$ arrives to $\gwc$, it  spawns a new instance of $\fwc$ before $\gwc$. Filter instance $F_{\{1,2\}}$ is connected to  $\gwc$ through channels $\ice$ and  $\csofv$. The state of the new filter instance $F_{\{1,2\}}$ is initialized with the set of vertices $\{1,2\}$. The edge $(3,6)$ arrives to the new filter instance $F_{\{1,2\}}$.}
  \label{fig:dp_example_1_2b}
\end{subfigure}
\caption[{[PoC] $\dpwcc$ Evolving first state}]{Evolution of the $\dpwcc$: First state}
\label{fig:dp_example_1_2}
\end{figure}
\vspace{.5cm}

\begin{figure}[h!]
\centering
\begin{subfigure}[b]{\textwidth}
 \centering
  \inputtikz{dp_example_3}
  \caption{None of the vertices in the edge $(3,6)$ is in the set of vertices $\{1,2\}$ in the state of $F_{\{1,2\}}$, hence it is passed through $\ice$ to $\gwc$.}
  \label{fig:dp_example_3_4a}
\end{subfigure}
\vspace{.3cm}

\begin{subfigure}[b]{\textwidth}
 \centering
  \inputtikz{dp_example_4}
  \caption{When the edge $(3,6)$ arrives to $\gwc$, it spawns the filter instance $F_{\{3,6\}}$  between $F_{\{1,2\}}$ and $\gwc$. Filter instance $F_{\{1,2\}}$ is connected to the new filter instance $F_{\{3,6\}}$ and this one is connected to  $\gwc$ through channels $\ice$ and  $\csofv$. The state of the new filter instance $F_{\{3,6\}}$ is initialized with the set of vertices $\{3,6\}$. The edge $(3,4)$ arrives to $F_{\{1,2\}}$  and $\mathsf{Sr_{WCC}}$ is fed with the mark $\eof$. Edges $(3,4)$ and $(4,5)$ remain passing through $\ice$.}
  \label{fig:dp_example_3_4b}
\end{subfigure}
\caption[{[PoC] $\dpwcc$ Evolving second state}]{Evolution of the $\dpwcc$: Second state}
\label{fig:dp_example_3_4}
\end{figure}
\vspace{.5cm}

\begin{figure}[h!]
\centering
\begin{subfigure}[b]{\textwidth}
 \centering
  \inputtikz{dp_example_5}
  \caption{$\mathsf{Sr_{WCC}}$  fed both, $\ice$ and $\csofv$, channels with the mark $\eof$ received from the input stream in previous state and then, it died. The edge $(4,5)$ is arriving to $\gwc$ and the edge $(3,4)$ is arriving to $F_{\{3,6\}}$. }
  \label{fig:dp_example_5_6a}
\end{subfigure}
\vspace{.3cm}

\begin{subfigure}[b]{\textwidth}
 \centering
  \inputtikz{dp_example_6}
  \caption{When the edge $(4,5)$ arrives to $\gwc$, it spawns the filter instance $F_{\{4,5\}}$  between $F_{\{3,6\}}$ and $\gwc$. Filter instance $F_{\{3,6\}}$ is connected to the new filter instance $F_{\{4,5\}}$ and this one is connected to  $\gwc$ through channels $\ice$ and  $\csofv$.  Since the edge $(3,4)$ arrived to $F_{\{3,6\}}$ at the same time and  vertex $3$ belongs to the set of connected vertices of the filter $F_{\{3,6\}}$,  the vertex $4$ is added to the state of $F_{\{3,6\}}$. Now, the state of $F_{\{3,6\}}$ is the connected set of vertices $\{3,4,6\}$. When the mark $\eof$ arrives to the first filter instance, $F_{\{1,2\}}$, through  $\csofv$, this stage passes  its partial set of connected vertices,  $\{1,2\}$, through $\csofv$ and dies.  This action will activate $\Actt$ in next  filter instances to start building  maximal connected components. In this example, the state in  $F_{\{3,6\}}$, $\{3,4,6\}$, and the arriving set $\{1,2\}$ do not intersect and, hence, both sets of vertices, $\{1,2\}$ and $\{3,4,6\}$ will be passed  to the next filter instance through $\csofv$.}
  \label{fig:dp_example_5_6b}
\end{subfigure}
\caption[{[PoC] $\dpwcc$ Evolving third state}]{Evolution of the $\dpwcc$: Third state}
\label{fig:dp_example_5_6}
\end{figure}
\vspace{.5cm}

\begin{figure}[h!]
\centering
\begin{subfigure}[b]{\textwidth}
 \centering
  \inputtikz{dp_example_7}
  \caption{The set of connected vertices  $\{3,4,6\}$ is arriving to $F_{\{4,5\}}$. The mark $\eof$ continues passing to next stages through the channel $\ice$.}
  \label{fig:dp_example_7_8a}
\end{subfigure}
\vspace{.3cm}

\begin{subfigure}[b]{\textwidth}
 \centering
  \inputtikz{dp_example_8}
  \caption{Since the intersection of the set of connected vertices $\{3,4,6\}$ arrived to  $F_{\{4,5\}}$ and its state is not empty, this state is enlarged to be $\{3,4,5,6\}$. The set of connected vertices $\{1,2\}$ is arriving to  $F_{\{4,5\}}$}
  \label{fig:dp_example_7_8b}
\end{subfigure}
\caption[{[PoC] $\dpwcc$ Evolving fourth state}]{Evolution of the $\dpwcc$:  Fourth state}
\label{fig:dp_example_7_8}
\end{figure}
\vspace{.5cm}

\begin{figure}[h!]
\centering
\begin{subfigure}[b]{\textwidth}
 \centering
  \inputtikz{dp_example_9}
  \caption{$F_{\{4,5\}}$ has passed the set of connected vertices  $\{1,2\}$ and it is arriving to $\mathsf{Sk_{WCC}}$. The mark $\eof$ is arriving to  $F_{\{4,5\}}$ through $\csofv$.}
  \label{fig:dp_example_9_10a}
\end{subfigure}
\vspace{.3cm}

\begin{subfigure}[b]{\textwidth}
 \centering
  \inputtikz{dp_example_10}
  \caption{Since the mark $\eof$ arrived to $F_{\{4,5\}}$ through $\csofv$, it passes its state, the set $\{3,4,5,6\}$ through $\csofv$ to next stages and died. The set of connected vertices  $\{1,2\}$ arrived to $\mathsf{Sk_{WCC}}$ and this implies  that $\{1,2\}$ is a maximal set of connected vertices, i.e. a connected component of the input graph. Hence,  $\mathsf{Sk_{WCC}}$ output this first weakly connected component.}
 \label{fig:dp_example_9_10b}
\end{subfigure}
\vspace{.5cm}

\begin{subfigure}[b]{\textwidth}
 \centering
  \inputtikz{dp_example_11}
  \caption{Finally, the set of connected vertices  $\{3,4,5,6\}$ arrived to $\mathsf{Sk_{WCC}}$ and was output as a new weakly connected component. Besides, the mark $\eof$ also arrived to $\mathsf{Sk_{WCC}}$ through $\csofv$ and thus, it dies.}
  \label{fig:dp_example_9_10c}
\end{subfigure}
\vspace{.3cm}

\begin{subfigure}[b]{\textwidth}
 \centering
  \inputtikz{dp_example_12}
  \caption{The weakly connected component of in the graph \autoref{fig:example_dp_graph} such as they have been emitted by $\dpwcc$.}
  \label{fig:dp_example_9_10d}
\end{subfigure}
\caption[{[PoC] $\dpwcc$ Evolving last state}]{Last states in the evolution of the $\dpwcc$}
\label{fig:dp_example_9_10}
\end{figure}

It is importat to highlight that during the states shown in \autoref{fig:dp_example_1_2a},  \autoref{fig:dp_example_1_2b},  \autoref{fig:dp_example_3_4a},  \autoref{fig:dp_example_3_4b} and  \autoref{fig:dp_example_5_6a} the only actor executed in any filter instance is $\Act$ (constructing sets of connected vertices). Afterwards, although $\Act$ can continue being executed in some filter instances, there are some instances that start executing $\Actt$ (constructing sets of maximal connected vertices). This is shown from \autoref{fig:dp_example_5_6a}  to \autoref{fig:dp_example_9_10a}.
\fi
%
\clearpage
%
\subsection{\texorpdfstring{$\dpwcc$}{Lg} Implementation}
%
As we said before, the $\dpwcc$ implementation has been made as a proof of concept to understand and explore the limitations and challenges that we could find in the development of a future \acrshort{dpf} in \acrshort{hs}. 
In Section \ref{dp-hs} we emphasize that the focus of \acrshort{dpf} in \acrshort{hs} is on the \acrshort{idl} component. 
Hence, the development of the $\dpwcc$ is as general as possible using most of the constructs and abstractions required by the \acrshort{idl}. 
Lets introduce the minimal code needed for encoding any \acrshort{dp} using \acrshort{dpfh}. \footnote{All the code that we expose here can be accessed publicly in \url{https://github.com/jproyo/dynamic-pipeline/tree/main/examples/Graph}}

\begin{listing}[H]
  \begin{minted}[fontsize=\fontsize{10}{11}\selectfont,numbers=left,breaklines,frame=lines,framerule=2pt,framesep=2mm,baselinestretch=1.2,highlightlines={1-3,6}]{haskell}
    
    type DPConnComp = Source (Channel (Edge :<+> ConnectedComponents :<+> Eof))
                :=> Generator (Channel (Edge :<+> ConnectedComponents :<+> Eof))
                :=> Sink

    program :: FilePath -> IO ()
    program file = runDP $ mkDP @DPConnComp (source' file) generator' sink'
        
  \end{minted}
  \caption[{[\mintinline{shell}{ConnectedComp.hs}] Main entry point of the program}]{In this code we can appreciate the main construct of our $\dpwcc$ which is a combination of $\iwc$, $\gwc$ and $\owc$}
  \label{src:dpwcc:1}
\end{listing}

In \autoref{src:dpwcc:1} there are two important declarations. First, the \textit{Type Level} declaration of the $\dpwcc$ to indicate \acrshort{dpfh} how our stages are going be connected, and
using that \textit{Type Level} construct, we use the \acrshort{idl} to allow the framework interpret the type representation of our \acrshort{dp} and ensuring at compilation time that we provide the correct stages,  \textit{Source} ($\iwc$), \textit{Generator} ($\gwc$) and \textit{Sink} ($\owc$), that matches those declaration.
According to this declaration what we need to provide is the correct implementation of \mintinline{haskell}{source'}, \mintinline{haskell}{generator'} and \mintinline{haskell}{sink'}
which \textit{Type checked} the \acrshort{dp} type definition\footnote{The names of the functions are completely choosen by the user of the framework and it should not be confused with the internal framework combinators.}.

\begin{listing}[H]
  \begin{minted}[fontsize=\fontsize{10}{11}\selectfont,numbers=left,breaklines,frame=lines,framerule=2pt,framesep=2mm,baselinestretch=1.2,highlightlines={4-5,8}]{haskell}
    
    source' :: FilePath
            -> Stage
              (WriteChannel Edge -> WriteChannel ConnectedComponents -> DP st ())
    source' filePath = withSource @DPConnComp
      $ \edgeOut _ -> unfoldFile filePath edgeOut (toEdge . decodeUtf8)

    sink' :: Stage (ReadChannel Edge -> ReadChannel ConnectedComponents -> DP st ())
    sink' = withSink @DPConnComp $ \_ cc -> withDP $ foldM_ cc print

    generator' :: GeneratorStage DPConnComp ConnectedComponents Edge st
    generator' =
      let gen = withGenerator @DPConnComp genAction
      in  mkGenerator gen filterTemplate
        
  \end{minted}
  \caption[{[\mintinline{shell}{ConnectedComp.hs}] $\iwc$, $\gwc$ $\owc$ Code}]{In this code we can appreciate the $\iwc$, $\gwc$ and $\owc$ functions that matches the type level definition of the $\DP$. $\iwc$ and $\owc$ are completely trivial but $\gwc$ will be analyzed later due to its internal complexity.}
  \label{src:dpwcc:2}
\end{listing}

As we appreciate in \autoref{src:dpwcc:2}, $\iwc$ and $\owc$ are trivial. In the case of \mintinline{haskell}{source'} the only work it needs to do is to read the input data edge by edge and downstream to the next stages. 
Thats is a achieve with a \acrshort{dpfh} combinator called \mintinline{haskell}{unfoldFile} which is a catamorphism of the input data to the stream.
In the case of $\owc$ it is also a simple implementation but doing the opposite as $\iwc$ using an anamorphism combinator provided by the framework as well, which is \mintinline{haskell}{foldM_}.
The $\gwc$ Stage is a little more complex because it contains the core of the algorithm explained in \autoref{sub:sec:wcc:algo}. According to what we described in \autoref{sec:dp}, \textit{Generator} stage spawns a \textit{Filter} on each received edge in our case of $\dpwcc$.
Therefore, it needs to contain that recipe on how to generate a new \textit{Filter} instance -- in our case of \acrshort{hs} it is a defunctionalized Data Type or Function --. 
Then, we have two things \mintinline{haskell}{genAction} which tells how to spawn a new \textit{Filter} and under what circumstances, and \mintinline{haskell}{filterTemplate} with the function to be spawn.

\begin{listing}[H]
  \begin{minted}[fontsize=\fontsize{10}{11}\selectfont,numbers=left,breaklines,frame=lines,framerule=2pt,framesep=2mm,baselinestretch=1.2,highlightlines={8-10}]{haskell}
    
    genAction :: Filter DPConnComp ConnectedComponents Edge st
              -> ReadChannel Edge
              -> ReadChannel ConnectedComponents
              -> WriteChannel Edge
              -> WriteChannel ConnectedComponents
              -> DP st ()
    genAction filter' readEdge readCC _ writeCC = do
      let unfoldFilter = mkUnfoldFilterForAll filter' toConnectedComp readEdge (readCC .*. HNil) 
      results <- unfoldF unfoldFilter
      foldM_ (hHead results) (`push` writeCC)
        
  \end{minted}
  \caption[{[\mintinline{shell}{ConnectedComp.hs}] Generator Action Code}]{In this code we can appreciate the Generator Action code which will expand all the filters in runtime in front of it and downstream all the connected components calculated for those, to the Sink}
  \label{src:dpwcc:3}
\end{listing}

\acrshort{dpfh} provides several combinators to help the user with the \textit{Generator} code, in particular with the spawning process as it has been describe in \autoref{dp-hs}.
\mintinline{haskell}{genAction} for $\dpwcc$ will use the combinator \mintinline{haskell}{mkUnfoldFilterForAll} which will spawn one \textit{Filter} per received edge in the channel, expanding dynamically the stages on runtime.
In the third highlighted line we can appreciate how after expanding the filters, the generator will downstream to the \textit{Sink}, the received Connected Components calculated from previous filters.

\begin{listing}[H]
  \begin{minted}[fontsize=\fontsize{10}{11}\selectfont,numbers=left,breaklines,frame=lines,framerule=2pt,framesep=2mm,baselinestretch=1.2,highlightlines={2,11-15,24-31}]{haskell}
    
    filterTemplate :: Filter DPConnComp ConnectedComponents Edge st
    filterTemplate = actor actor1 |>> actor actor2
    
    actor1 :: Edge
           -> ReadChannel Edge
           -> ReadChannel ConnectedComponents
           -> WriteChannel Edge
           -> WriteChannel ConnectedComponents
           -> StateT ConnectedComponents (DP st) ()
    actor1 _ readEdge _ writeEdge _ = 
      foldM_ readEdge $ \e -> get >>= doActor e
     where
      doActor v conn
        | toConnectedComp v `intersect` conn = modify' (toConnectedComp v <>)
        | otherwise = push v writeEdge
    
    actor2 :: Edge
           -> ReadChannel Edge
           -> ReadChannel ConnectedComponents
           -> WriteChannel Edge
           -> WriteChannel ConnectedComponents
           -> StateT ConnectedComponents (DP st) ()
    actor2 _ _ readCC _ writeCC = do 
      foldWithM_ readCC pushMemory $ \e -> get >>= doActor e
    
     where
       pushMemory = get >>= flip push writeCC
    
       doActor cc conn
        | cc `intersect` conn = modify' (cc <>)
        | otherwise = push cc writeCC
    
  \end{minted}
  \caption[{[\mintinline{shell}{ConnectedComp.hs}] Filter Template Code}]{Filter template code composed by 2 Sequential Actors that will calculate the Connected Components and downstream them.}
  \label{src:dpwcc:4}
\end{listing}

Finally, in \autoref{src:dpwcc:4} the \textit{Filter} template code is defined. 
As we have seen in \autoref{sub:sec:wcc:algo}, $\dpwcc$ \textit{Filter} is composed of 2 Actors. The first actor collect all the possible vertices that are incidence some of the vertices edge that was instantiate with.
Once it does not receive any more edges, it starts downstream it set of vertices to the following filters in order to build a maximal connected components, that is \mintinline{haskell}{actor2}. At the end \mintinline{haskell}{actor2} will downstream 
its connected component to the following stages.
As we show, with the help of the \acrlong{dpfh}, building a \acrshort{dp} algorithm like \acrshort{wcc} enumeration consist in few lines of codes with the \textit{Type Safety} that \acrshort{hs} provides.

\subsection{Empirical Evaluation}\label{sec:new:eval}
The empirical study aims at evaluating $\dpwcc$ implemented \acrshort{dpfh} to verify that the performance is suitable compared with the empirical evaluation conducted on \autoref{prole}. 
Our goal is to answer the following research questions: 

\begin{inparaenum}[\bf {\bf RQ}1\upshape)]
\label{res:question}
    \item Does $\dpwcc$ in \acrshort{dpfh} exhibit similar execution time performance compared with \textit{Proof of Concept} evaluation?
    \item Does $\dpwcc$ in \acrshort{dpfh} exhibit continuous behavior enumerating connected components incrementally?
\end{inparaenum}

In order to answer the former questions, we have conducted the same experiments that we did on \textit{Proof of Concept} evaluation, but not focusing on the comparison with default \acrshort{hs} \texttt{containers} implementation. 
Another topic that we haven't measured in this empirical analysis is the performance of \textit{Memory and Threads} allocation. 
The networks selected for this evaluation are the same as the networks used on \autoref{prole}.

\subsubsection{Running Architecture}
All the experiments have been executed in a $x86$ $64$ bits architecture with a \textit{$6$-Core Intel Core i7} processor of $2,2$ GHz which can emulate up to $12$ virtual cores. This processor has \emph{hyper-threading} enable. Regarding memory, the machine has $32 GB$ \emph{DDR4} of RAM, $256\ KB$ of L2 cache memory, and $9\ MB$ of L3 cache.

\subsubsection{Haskell Setup}
Regarding specific libraries and compilations flags used on \acrshort{hs}, we have used \acrshort{ghc} version $8.10.4$. 
We have also used the following set of libraries: \mintinline{bash}{bytestring 0.10.12.0} \cite{bytestring}, \mintinline{bash}{containers 0.6.2.1} \cite{containers}, \mintinline{bash}{relude 1.0.0.1} \cite{relude} and \mintinline{bash}{dynamic-pipeline 0.3.2.0} \cite{dynamic-pipeline}. 
The use of \texttt{relude} library is because we disabled \mintinline{haskell}{Prelude} from the project with the language extension \mintinline{haskell}{NoImplicitPrelude} \cite{extensions}. 
Regarding compilation flags (\acrshort{ghc} options) we have compiled our program with \mintinline{bash}{-threaded}, \mintinline{bash}{-O3}, \mintinline{bash}{-rtsopts}, \mintinline{bash}{-with-rtsopts=-N}. 
Since we have used \texttt{stack} version $2.5.1$ \cite{stack} as a building tool on top of \acrshort{ghc} the compilation command is \mintinline{bash}{stack build}\footnote{For more information about package.yaml or cabal file please check https://github.com/jproyo/upc-miri-tfm/tree/main/connected-comp}.

\subsection{Experiments Definition}\label{sub:new:exp:def}
\paragraph{E1: Implementation Analysis}
Similar to what we have measure on \autoref{prole}, in this experiment we measure \acrshort{ghc} statistics running time enabling \mintinline{bash}{+RTS -s} flags.
The metrics that we measure are \emph{MUT Time} which is the amount of time in seconds \acrshort{ghc} is running computations and \emph{GC Time} which is the number of seconds that \acrshort{ghc} is running garbage collector. 
\emph{Total execution time} is the sum of both in seconds. This experiment will help to answer research question [RQ1].

\paragraph{E2: Benchmark Analysis}
This experiment measures \emph{Average Running Time}.
The \emph{Average Running Time} is the average running time of $1000$ resamples using \texttt{criterion} tool~\cite{criterion}. 
In each sample, the running time is measure from the beginning of the execution of the program until when the last answer is produced.
This experiment will help to answer research question [RQ1].

\paragraph{E3: Continuous Behavior - Diefficiency Metrics}
This experiment measures \acrlong{dm} described in \autoref{prem:dief}, and will help to answer research question [RQ2].

\subsection{Discussion of Observed Results}\label{new:experiments}
\subsubsection{Experiment: E1}\label{sub:new:sec:e1}
The following represents the \emph{Total execution time} for each of the networks using the new $\dpwcc$ algorithm, implemented with \acrshort{dpfh}.
 
\begin{table}[H]
  \centering
  \begin{tabular}{|l|r|r|r|r|}
   \hline
   \textbf{Network} & \textbf{Exec Param} & \textbf{MUT Time} & \textbf{GC Time} & \textbf{Total Time}\\
   \hline
   Enron Emails & \mintinline{bash}{+RTS -N4 -s} & 1.795s & 0.505s & 2.314s \\
   \hline
   Astro Physics Coll Net & \mintinline{bash}{+RTS -N4 -s} & 2.294s & 1.003s & 3.311s \\
   \hline
   Google Web Graph & \mintinline{bash}{+RTS -N8 -s} & 169.381s & 270.784s & 440.176s \\
   \hline
  \end{tabular}
 \caption{Total Execution times of each of the networks using $\dpwcc$ algorithm implemented with \acrshort{dpfh}. \textit{MUT Time} is the time of running or executing code and \textit{GC Time} is the time that the program spent doing Garbage collection. \textit{Total Execution time} is the sum of both times}
 \label{table:new:wcc:dpfh:5}
 \end{table}

In \autoref{table:new:wcc:dpfh:5}, we are obtaining similar execution times compared with \autoref{table:5} in \autoref{prole}.
In fact, for \textit{Enron Emails} and \textit{Astro Physics Coll} networks, all times are better than the implementation of the \textit{Proof of Concept}.
Regarding \textit{Google Web} network, the time is slighlity worse but there is the fact that \acrshort{dpfh} is adding some overhead over plain code execution.
According to this results we can partially answer [RQ1], because the implementation of $\dpwcc$ with \acrshort{dpfh} has similar performance compared with \textit{Proof of Concept} implementation. 

\subsubsection{Experiment: E2}\label{sub:new:exp:2}
In \autoref{fig:new:1}, we can appreciate the \textit{Average Execution Time} for each of the networks after running \texttt{criterion}~\cite{criterion} tool.

\begin{minipage}[t]{\linewidth}
  \includegraphics[width=\textwidth]{bench_1_new.png}
  \captionsetup{type=figure}
  \captionof{figure}{$\dpwcc$ implemented with \acrshort{dpfh}. Average Execution time of running $1000$ samples over each of the networks.}
  \label{fig:new:1}
\end{minipage}

If we compare those \textit{Average Execution Times} with the previous obtained in \autoref{prole}, we have the following comparison table.

 \begin{table}[H]
  \centering
  \begin{tabular}{|l|l|l|l|}
   \hline
   \textbf{Network} & \textbf{DPF-WCC} & \textbf{\acrshort{dpwcc}} & \textbf{Speed-up}\\
   \hline
   Enron Emails & 4.30s & 4.68s &  0.91\\
   \hline
   Astro Physics Coll Net & 4.76s  & 4.98s & 0.95\\
   \hline
   Google Web Graph & 456s & 386s & -1.18\\
   \hline
  \end{tabular}
 \caption{Comparison of Average Execution Times between $\dpwcc$ implemented with \acrshort{dpfh} and the implementation of Proof of Concept (\acrshort{dpwcc})}
 \label{table:new:6}
 \end{table}

As we can see in \autoref{table:new:6}, all the \textit{Average Execution times} are better for the new implementation with \acrshort{dpfh} compared with the \acrshort{dpwcc}.
As it is consistent with the \textit{Total Execution time} of the previous experiment in \autoref{sub:new:sec:e1}, the only networks that performs slightly worse is \textit{Google Web}, but the difference is not significant enough taking into consideration the overhead introduced by \acrshort{dpfh}.
These results allow for answering research question [RQ1] completely ensuring that the performance in terms of execution is better for smaller networks and almost similar for bigger networks. 

\subsubsection{Experiment: E3}\label{sub:new:sec:e2}
In the case of continuous behavior analysis, we have run the comparison against \acrshort{hs} \texttt{containers} again in order to visually appreciate the incremental generation of results. 

\begin{figure}[!htp]
  \centering
  \begin{subfigure}[t]{0.3\textwidth}
   \includegraphics[width=1\linewidth, height=0.2\textheight]{email_enron}
   \caption{email-Enron \acrlong{dm} Line Plot}
    \label{fig:dief:1}
  \end{subfigure}\hfill
  \begin{subfigure}[t]{0.3\textwidth}
   \includegraphics[width=1\linewidth, height=0.2\textheight]{ca_astroph}
   \caption{ca-AstroPh \acrlong{dm} Line Plot}
    \label{fig:dief:2}
  \end{subfigure}\hfill
  \begin{subfigure}[t]{0.3\textwidth}
   \includegraphics[width=1\linewidth, height=0.2\textheight]{web_google}
   \caption{web-Google \acrlong{dm} Line Plot}
    \label{fig:dief:3}
  \end{subfigure}\hfill
   \caption{These figures show \acrshort{dt} observed results after running all the scenarios for each network. $y$ axis represents the number of Answers produced and $x$ axis is the $t$ time of the \acrshort{dt} metric describe in \autoref{prem:dief}. The more data points distributed throughout the $x$ axis, the higher, the continuous behavior.}
   \label{fig:dief:all}
 \end{figure}

\begin{table}[htp!]
  \centering
  \begin{tabular}{|p{0.25\linewidth}|c|c|c|}
    \hline
   \textbf{Network} & \textbf{Algorithm} & \textbf{dief@t Metric}  & \textbf{dief@k Metric}\\
   \hline
   ca-AstroPh & $\dpwcc$ in \acrshort{dpfh} & $5.48 \times 10^2$ & $3.94 \times 10^2$\\
   \hline
   email-Enron & $\dpwcc$ in \acrshort{dpfh} & $4.35 \times 10^3$ & $4.02 \times 10^3$\\
   \hline
   web-Google & $\dpwcc$ in \acrshort{dpfh} & $3.48 \times 10^4$ & $3.48 \times 10^4$ \\
  \hline
  \end{tabular}
  \caption{This tables shows the \acrshort{dt} and \acrshort{dk} values gather for  $\dpwcc$ in \acrshort{dpfh}. We can appreciate that in all cases $\dpwcc$ has a higher value of \acrshort{dt} and a lower value of \acrshort{dk} showing continuos behavior}
 \label{table:e1:dm:values}
 \end{table}

 \begin{figure}[!htp]
  \centering
  \begin{subfigure}[t]{0.3\textwidth}
   \includegraphics[width=1\linewidth, height=0.2\textheight]{email_enron_radar}
   \caption{email-Enron \acrlong{dm} Radar Plot}
    \label{fig:dief:4}
  \end{subfigure}\hfill
  \begin{subfigure}[t]{0.3\textwidth}
   \includegraphics[width=1\linewidth, height=0.2\textheight]{ca_astroph_radar}
   \caption{ca-AstroPh \acrlong{dm} Radar Plot}
    \label{fig:dief:5}
  \end{subfigure}\hfill
  \begin{subfigure}[t]{0.3\textwidth}
   \includegraphics[width=1\linewidth, height=0.2\textheight]{web_google_radar}
   \caption{web-Google \acrlong{dm} Radar Plot}
    \label{fig:dief:6}
  \end{subfigure}\hfill
   \caption{Radial plots show how the different dimensions values provided by \acrshort{dtkp} tool such as \acrshort{tt}, \acrshort{tfft}, \acrshort{dt}, \acrshort{et} and \acrshort{comp} are related each other for each experimental case. These figures show radial plot observed results after running for each network. \acrshort{dt} is described in \autoref{prem:dief}.}
   \label{fig:dief:radial:all}
 \end{figure}

Based on the results shown in \autoref{fig:dief:all}, all the solutions in \acrshort{dpwcc} show continuous behavior.
Moreover in \autoref{table:e1:dm:values}, $\dpwcc$ has a higher value of \acrshort{dt} and a lower value of \acrshort{dk} confirming the continuos behavior that we can see in the images.
As we can appreciate in \autoref{fig:dief:radial:all} radar plots our previous analysis can be confirmed.

In conclusion, we can say that regarding [RQ2] (\autoref{res:question}) although \acrshort{dpwcc} is faster than the traditional approach, the speed-up dimension execution factor is not always the most interest analysis that we can have, because as we have seen even when in the case of \emph{web-Google} Graph \acrshort{dpwcc} is slower at execution, it is at least generating incremental results without the need to wait for the rest of the computations.

\iffalse
\subsubsection{Experiment: E3}
For this type of analysis, our experiment focuses on \emph{email-Enron} network \cite{netenron} only because profiling data generated by \acrshort{ghc} is big enough to conduct the analysis and on the other, and enabling profiling penalize execution time.

\paragraph{Multithreading} For analyzing parallelization and multithreading we have used \textit{ThreadScope} \cite{threadscope} which allows us to see how the parallelization is taking place on \acrshort{ghc} at a fine grained level and how the threads are distributed throughout the different cores requested with the \mintinline{bash}{-N} execution \texttt{ghc-option} flag.

\begin{minipage}[t!]{\linewidth}

  \includegraphics[width=\textwidth]{screen_1}
  \captionsetup{type=figure}
  \captionof{figure}{Threadscope Image of General Execution}
  \label{fig:3}
\end{minipage}

In \autoref{fig:3}, we can see that the parallelization is being distributed evenly among the $4$ Cores that we have set for this execution.
The distribution of the load is more intensive at the end of the execution, where \mintinline{haskell}{actor2} filter stage 
%as it can be seen in \autoref{src:haskell:3}, 
of the algorithm is taking place and different filters are reaching execution of that second actor.

\begin{wrapfigure}{r}{0.5\textwidth}
  \begin{center}
     \includegraphics[width=0.48\textwidth] {screen_2}
     %[width=10cm, height=9cm]
       \end{center}
     \caption{Threadscope Image of Zoomed Fraction}
     \label{fig:4}
 %\end{figure}
 \end{wrapfigure}
Another important aspect shown in \autoref{fig:3}, is that this work is not so significant for \acrshort{ghc} and the threads and distribution of the work keeps between 1 or 2 cores during the execution time of the \mintinline{haskell}{actor1}. However, the usages increase on the second actor as pointed out before. In this regard, we can answer research questions [Q1] and [Q3] (\autoref{res:question}), verifying that \acrshort{hs} not only supports the required parallelization level but is evenly distributed across the program execution too.

Finally, it can also be appreciated that there is no sequential execution on any part of the program because the $4$ cores have \textit{CPU} activity during the whole execution time. This is because as long the program start, and because of the nature of the \acrshort{dp} model, it is spawning the \textit{Source} stage in a separated thread. This is a clear advantage for the model and the processing of the data since the program does not need to wait to do some sequential processing like reading a file, before start computing the rest of the stages.




%\begin{minipage}[t!]{\linewidth}
%\begin{center}
%  \includegraphics[width=0.6\textwidth]{screen_2}
%  \captionsetup{type=figure}
%  \captionof{figure}{Threadscope Image of Zoomed Fraction}
%  \label{fig:4}
%  \end{center}
%\end{minipage}

\autoref{fig:4} zooms in on \textit{ThreadScope} output in a particular moment, approximately in the middle of the execution. We can appreciate how many threads are being spawned and by the tool and if they are evenly distributed among cores. The numbers inside green bars represent the number of threads that are being executed on that particular core (horizontal line) at that execution slot. Thus, the number of threads varies among slot execution times because as it is already known, \acrshort{ghc} implements \emph{Preemptive Scheduling} \cite{lightweightghc}.

Having said that, it can be appreciated in \autoref{fig:4} our first assumption that the load is evenly distributed because the mean number of executing threads per core is $571$.

\paragraph{Memory allocation} Another important aspect in our case is how the memory is being managed to avoid memory leaks or other non-desired behavior that increases memory allocation during the execution time. This is even more important in the particular implementation of \acrshort{wcc} using \acrshort{dp} model because it requires to maintain the set of connected components in memory throughout the execution of the program or at least until we can output the calculated \acrshort{wcc} if we reach to the last \textit{Filter} and we know that this \acrshort{wcc} cannot be enlarged anymore. 

In order to verify this, we measure memory allocation with \textit{eventlog2html} \cite{eventlog2html} which converts generated profiling memory eventlog files into graphical HTML representation. 

\begin{wrapfigure}{r}{0.5\textwidth}
  \begin{center}
     \includegraphics[width=0.5\textwidth] {visualization}
     %[width=10cm, height=9cm]
       \end{center}
     \caption{Memory Allocation}
     \label{fig:5}
 %\end{figure}
 \end{wrapfigure}
 
%\begin{minipage}[t]{\linewidth}
%\begin{center}
%  \includegraphics[width=0.6\textwidth]{visualization}
%  \captionsetup{type=figure}
%  \captionof{figure}{Memory Allocation}
%  \label{fig:5}
%  \end{center}
%\end{minipage}

As we can see in \autoref{fig:5}, \acrshort{dpwcc} does an efficient work on allocating memory since we are not using more than $57$ MB of memory during the whole execution of the program.

On the other hand, if we analyze how the memory is allocated during the execution of the program, it can also be appreciated that most of the memory is allocated at the beginning of the program and steadily decrease over time with a small peak at the end that does not overpass even half of the initial peak of $57$ MB. The explanation for this behavior is quite straightforward because at the beginning we are reading from the file and transforming a \mintinline{haskell}{ByteString} buffer to \mintinline{haskell}{(Int, Int)} edges. This is seen in the image in which the dark blue that is on top of the area is \mintinline{haskell}{ByteString} allocation. Light blue is allocation of \mintinline{haskell}{Maybe a} type which is the type that is returned by the \textit{Channels} because it can contain a value or not. Data value \mintinline{haskell}{Nothing} is indicating end of the \textit{Channel}. 
%as we can see in \autoref{src:haskell:f3}.

Another important aspect is the green area which represents \mintinline{haskell}{IntSet} allocation, which in the case of our program is the data structure that we use to gather the set of vertices that represents a \acrshort{wcc}. This means that the amount of memory used for gathering the \acrshort{wcc} itself is minimum and it is decreasing over time, which is another empirical indication that we are incrementally releasing results to the user. It can be seen as well that as long the green area reduces the lighter blue (\mintinline{haskell}{MUT_ARR_PTRS_CLEAN} \cite{ghcheap}) increases at the same time indicating that the computations for the output (releasing results) is taking place. 

Finally, according to what we have stated above, we can answer the question [Q3] (\autoref{res:question}) showing that not only memory management was efficient, but at the same time, the memory was not leaking or increasing across the running execution program.
\fi
\section{Related Work}\label{section:related-work}

\paragraph{Streaming in Haskell Language}
Streaming computational models have been implemented in \acrlong{hs} during the last $10$ years. One of the first libraries in the ecosystem was \mintinline{shell}{conduit}\footnote{\url{https://hackage.haskell.org/package/conduit}} in 2011.
After that, several efforts on improving streaming processing on the language has been made not only at abstraction level for the user but as well as performance execution 
improvements like \mintinline{shell}{pipes}~\footnote{\url{https://hackage.haskell.org/package/pipes}} and \mintinline{shell}{streamly}\footnote{\url{https://hackage.haskell.org/package/streamly}} lately.
Moreover, there is an empirical comparison between those three, where a benchmark analysis has been conducted\footnote{\url{https://github.com/composewell/streaming-benchmarks}}.

Although most of those libraries offer the ability to implement \acrshort{dap} and \acrshort{pip}, none of them provide clear abstractions to create \acrshort{dp} models because
the setup of the stages should be provided beforehand. In the context of this work, we have done a proof of concept at the beginning, 
but it was not possible to adapt any of those libraries to implement properly \acrshort{dp}. 
The closest we have been to implement \acrshort{dp} with some of those libraries was when we explored \mintinline{shell}{streamly}.
In this case, there is a \mintinline{haskell}{foldrS} combinator that could have been proper for the purpose of generating a dynamic pipeline of stages based on the data flow. However, it was not possible to manipulate the channels between the stages to control the flow of the data. It is important to remark that, even though, the  library  \mintinline{shell}{streamly} implements channels, they are hidden from the end-user, and there is not a  clear way to manipulate them.

To the best of our knowledge, no similar library under the  \acrshort{dp} approach has been written in \acrlong{hs}. 
One important motivation to develop our own framework is that  we not only  want to satisfy our research needs but, as a novel contribution, we aim at providing a \acrshort{dpf} to the \acrshort{hs} community as well. We hope this contribution encourages and helps writing algorithms under the Dynamic Pipeline Paradigm. 
\iffalse
Several implementations for streaming processing models \footnote{\url{https://hackage.haskell.org/package/conduit}}\footnote{\url{https://hackage.haskell.org/package/pipes}}\footnote{\url{https://hackage.haskell.org/package/streamly}} in \acrshort{hs} have arisen over the years. All these libraries have their abstractions and can do data streaming processing in a fast way with different performance according to recent benchmarks\footnote{\url{https://github.com/composewell/streaming-benchmarks}}. Although they seem to be suitable for implementing a $\DP$, it is required to know pipeline stages disposition beforehand, and it is hard to achieve a succinct and expressive implementation of a \acrshort{dpf}. Moreover, since they have been conceived as a data parallel streaming model \cite{hr19} by design instead of pipeline parallel streaming, implementing $\DP$ using these tools becomes counter-intuitive and hard to achieve.
\fi
Another kind of streaming implementation in \acrshort{hs} is described in \cite{parallelbook}. In that work, the author describes how to encode pipeline parallelism with \mintinline{haskell}{Par} \textit{Monad}. Although this could have been a suitable alternative for implementing $\DP$, the parallelization level used by \mintinline{haskell}{Par} \textit{Monad} is sparks \cite{sparks}. As we have explained in section \autoref{section:prob:dp:haskell}, we do not require to reach that level of parallelization in our current model.

In regards to other $\DP$ language implementations, a significant contribution on \cite{dpp_triangles} has been done, where a $\DP$ implementation in \acrfull{go} for counting triangles of graphs is compared against MapReduce. Those experiment results have shown how $\DP$ in \acrshort{go} improves the performance in terms of execution time and memory depending on the graph topology. It would be interesting and a matter of future work, to compare different language implementations of $\DP s$, taking into consideration those promising results and the ones presented in this article.

\iffalse
In particular, the problem presented in \autoref{sub:sec:mot:ex} is one of the algorithms in which the amount of stages that could run in parallel is the worst case having one Stage per edge at most, but still in that scenario the number of threads can be efficiently handled by \acrshort{ghc}. Therefore, there is no need for such a fine grained parallelization level as it could be required when the data should be split into the smallest processing units as possible. 
\fi

\begin{frame}[fragile]{Conclusions and Future Work}
  \begin{block}{Conclusions}      
    \begin{itemize}
      \item \textbf{Dynamic Pipeline Paradigm} is a suitable computational model to build an Algorithm for Incrementally Enumerating Bitriangles in large Bipartite Networks. 
    \end{itemize}
  \end{block}
\end{frame}

\begin{frame}[fragile]{Conclusions and Future Work}
  \begin{block}{Conclusions}      
    \begin{itemize}
      \item {\color{light}\textbf{Dynamic Pipeline Paradigm} is a suitable computational model to build an Algorithm for Incrementally Enumerating Bitriangles in large Bipartite Networks. }
      \item \textbf{Haskell} has behaved accordingly and efficiently on implementing \textbf{Dynamic Pipeline Paradigm} and solving the Algorithm.
    \end{itemize}
  \end{block}
\end{frame}

\begin{frame}[fragile]{Conclusions and Future Work}
  \begin{block}{Conclusions}      
    \begin{itemize}
      \item {\color{light}\textbf{Dynamic Pipeline Paradigm} is a suitable computational model to build an Algorithm for Incrementally Enumerating Bitriangles in large Bipartite Networks. }
      \item {\color{light}\textbf{Haskell} has behaved accordingly and efficiently on implementing \textbf{Dynamic Pipeline Paradigm} and solving the Algorithm.}
      \item In the experimental analysis we have \textbf{empirically} shown that the designed \textbf{algorithm is incrementally generating Bitriangles} from large Networks like Dbpedia. 
    \end{itemize}
  \end{block}
\end{frame}

\begin{frame}[fragile]{Conclusions and Future Work}
  \begin{block}{Future Work}      
    \begin{itemize}
      \item Solve \textbf{memory consumption} for large graphs.
    \end{itemize}
  \end{block}
\end{frame}

\begin{frame}[fragile]{Conclusions and Future Work}
  \begin{block}{Future Work}      
    \begin{itemize}
      \item {\color{light}Solve \textbf{memory consumption} for large graphs.}
      \item Selection of \textbf{more efficient Data Structures} to process query commands faster
    \end{itemize}
  \end{block}
\end{frame}

\begin{frame}[fragile]{Conclusions and Future Work}
  \begin{block}{Future Work}      
    \begin{itemize}
      \item {\color{light}Solve \textbf{memory consumption} for large graphs.}
      \item {\color{light}Selection of \textbf{more efficient Data Structures} to process query commands faster}
      \item Implement a \textbf{distributed model} for filters, to take advantage of a distributed memory model.
    \end{itemize}
  \end{block}
\end{frame}

\begin{frame}[fragile]{Conclusions and Future Work}
  \begin{block}{Future Work}      
    \begin{itemize}
      \item {\color{light}Solve \textbf{memory consumption} for large graphs.}
      \item {\color{light}Selection of \textbf{efficient Data Structures} to process query commands faster}
      \item {\color{light}Implement a \textbf{distributed model} for filters, to take advantage of a distributed memory model.}
      \item \textbf{Improvements on Haskell Framework}: Stream processing and Memory footprint for Boxed Data Types.
    \end{itemize}
  \end{block}
\end{frame}

%
\bibliography{jlamp2022}
%
%\chapter{Appendix}
\section{Experiments}
\subsection{Benchmark Analysis}\label{app:exp:bench}
\begin{longtable}{|l|c|c|}
  \caption{$R^2$ goodness of the fit - Regression model}
  \label{table:app:exp:bench}\\
    \hline
   \textbf{Graph} & \textbf{Experiment} & \textbf{$R^2$}\\
   \hline
   opsahl-ucforum & VL-L & 0.954 \\
   \hline
   opsahl-ucforum & VL-M & 0.812 \\
   \hline
   opsahl-ucforum & VL-H & 0.999 \\
   \hline
   wang-amazon & VL-L & 0.992 \\
   \hline
   wang-amazon & VL-M & 0.933 \\
   \hline
   wang-amazon & VL-H & 0.849 \\
   \hline
   moreno-crime & VL-L & 0.999 \\
   \hline
   moreno-crime & VL-M & 0.997 \\
   \hline
   moreno-crime & VL-H & 0.97 \\
   \hline
   opsahl-ucforum & VU-L & 0.718 \\
   \hline
   opsahl-ucforum & VU-M & 0.972 \\
   \hline
   opsahl-ucforum & VU-H & 0.997 \\
   \hline
   wang-amazon & VU-L & 0.932 \\
   \hline
   wang-amazon & VU-M & 0.896 \\
   \hline
   wang-amazon & VU-H & 0.992 \\
   \hline
   moreno-crime & VU-L & 0.999 \\
   \hline
   moreno-crime & VU-M & 0.998 \\
   \hline
   moreno-crime & VU-H & 0.997 \\
   \hline
   opsahl-ucforum & E-L & 0.996 \\
   \hline
   opsahl-ucforum & E-M & 0.987 \\
   \hline
   opsahl-ucforum & E-H & 0.929 \\
   \hline
   wang-amazon & E-L & 0.954 \\
   \hline
   wang-amazon & E-M & 0.955 \\
   \hline
   wang-amazon & E-H & 0.979 \\
   \hline
   moreno-crime & E-L & 1.00 \\
   \hline
   moreno-crime & E-M & 1.00 \\
   \hline
   moreno-crime & E-H & 0.997 \\
   \hline
 \end{longtable}


\end{document}

