\subsection{The $\dpwcc$ Algorithm- Weakly Connected Components of a Graph in the Dynamic Pipeline Paradigm}\label{prole}
\iffalse
One of the biggest challenges of implementing a Dynamic Pipeline is to find  a programming language with a proper set of tools supporting both of the  primary features of the \acrshort{dp}: \begin{inparaenum}[i\upshape)]
\item  \emph{parallel} processing and 
\item  \emph{strong theoretical} foundations to manage computations as first-class citizens.
 \end{inparaenum}
We present and illustrate with an example, an implementation for the problem of enumerating weakly connected components using the dynamic pipeline paradigm previously described. More details in \cite{prole}. 
\fi
Let us consider the problem of computing/enumerating the weakly connected components (WCC) of a graph $G$ using \acrshort{dp}. 
A weak connected component of a graph is a subgraph in which all vertices are connected to each other by some path. 
Thus, finding connected components of an undirected graph implies obtaining the minimal partition of the set of nodes induced by the relationship \textit{connected}, i.e., there is a path between each pair of nodes. An example of that graph can be seen in \autoref{fig:example_dp_graph}.
The input of the Dynamic Pipeline for computing the WCC of a graph, $\dpwcc$, is a sequence of edges ending with $\eof$. In the source graph, there are neither isolated vertices nor loops. The connected components are output as soon as they are computed, i.e., they are produced incrementally. 
Roughly speaking, the idea of the algorithm is that the weakly connected components are built in two phases. In the first phase, filter instance stages receive the edges of the input graph and create sets of connected vertices. 
During the second phase, each filter instance receives sets of connected nodes. When in a filter the incoming set of nodes intersects with its set of connected nodes, the set of incoming nodes is joined to the latter and no output is produced. Otherwise, the incoming set of connected nodes is passed to the next stage. 
$\dpwcc$ is defined in terms of the behavior of its four kinds of stages: \textit{Source} ($\iwc$),  \textit{Generator} ($\gwc$),  \textit{Sink} ($\owc$), and \textit{Filter}($\fwc$) stages. Additionally,  the channels connecting these stages must be defined. 
In $\dpwcc$, stages are connected linearly and unidirectionally through the channels $\ice$ and  $\csofv$. Channel $\ice$ carries edges, while channel  $\csofv$ conveys sets of connected vertices. Both channels end by the $\eof$ mark. 
The behavior of $\fwc$ is given by a sequence of two actors (scripts). Each actor corresponds to a phase of the algorithm. In what follows, we denote these actors by $\Act$ and $\Actt$, respectively. 
The script $\Act$ keeps a set of connected vertices ($CV$) in the state of the $\fwc$ instance. When an edge $e$ arrives, if an endpoint of $e$ is present in the state, then the other endpoint of $e$ is added to $CV$. 
Edges without incident endpoints are passed to the next stage. When $\eof$ arrives at channel $\ice$, it is passed to the next stage, and the script $\Actt$ starts its execution. 
If the script, $\Actt$ receives a set of connected vertices $CV$ in $\csofv$, it determines if the intersection between $CV$ and the nodes in its state is not empty. If so, it adds the nodes in $CV$  to its state. 
Otherwise, the $CV$ is passed to the next stage.  Whenever $\eof$ is received, $\Actt$ passes--through $\csofv$-- the set of vertices in its state and the $\eof$ mark to the next stage; then, it dies.
The behavior of $\iwc$ corresponds to the identity transformation over the data stream of edges.  As edges arrive, they are passed through  $\ice$ to the next stage. When receiving $\eof$ on $\ice$, this mark is put on both channels. 
Then, $\iwc$ dies. 

%\begin{wrapfigure}{r}{0.4\textwidth}
\begin{figure}
 \begin{center}
\inputtikz{graph_example_wcc}
\end{center}
\caption[{[PoC] Graph WCC Example}]{Example of a graph with two weakly connected components: $\{1,2\}$ and $\{3,4,5,6\}$}
\label{fig:example_dp_graph}
\end{figure}
%\end{wrapfigure}

\begin{figure}[h!]
  \centering
\inputtikz{dp_example_0}
\caption[{[PoC] $\dpwcc$ Initial Setup}]{$\dpwcc$ Initial setup. Stages Source, Generator, and Sink are represented by the squares labeled by $\mathsf{Sr_{WCC}}$, $\mathsf{G_{WCC}}$ and $\mathsf{Sk_{WCC}}$, respectively.  The square $\fwc$ corresponding to the Filter stage template is the parameter of $\gwc$. Arrows $\rightrightarrows$ between represents the connection of stages through two channels, $\ice$, and $\csofv$. The arrow  $\rightarrow$ represents the channel $\csofv$ connecting the stages $\mathsf{G_{WCC}}$ and $\mathsf{Sk_{WCC}}$. The arrow $\Longrightarrow$ stands for I/O data flow. Finally, the input stream comes between the dotted lines on the left and the WCC computed incrementally will be placed between the solid lines on the right.}
\label{fig:dp_example_0}
\end{figure}

Let us describe this behavior with the example of the graph shown in \autoref{fig:example_dp_graph}.
\autoref{fig:dp_example_0} depicts the initial configuration of $\dpwcc$. 
The interaction of $\dpwcc$ with the "external" world is done through the stages $\iwc$ and $\owc$. 
Indeed, once activated the initial $\dpwcc$, the input stream -- consisting of a sequence containing all the edges in the graph in \autoref{fig:example_dp_graph} -- feeds $\iwc$ while  $\owc$ emits incrementally the resulting weakly connected components.  
Figures \ref{fig:dp_example_1_2}, \ref{fig:dp_example_3_4}, \ref{fig:dp_example_5_6}, \ref{fig:dp_example_7_8} and \ref{fig:dp_example_9_10} depict the evolution of the $\dpwcc$.

It is important to highlight that during the states shown in figures \ref{fig:dp_example_1_2a},  \ref{fig:dp_example_1_2b},  \ref{fig:dp_example_3_4a},  \ref{fig:dp_example_3_4b} and  \ref{fig:dp_example_5_6a} the only actor executed in any filter instance is $\Act$ (constructing sets of connected vertices). However, in \autoref{fig:dp_example_5_6a}  to \autoref{fig:dp_example_7_8a} although $\Act$ is still being executed in some filter instances, there are other filters starting the execution of $\Actt$ (either updating the filter's sets of connected vertices, when  its intersection with the incoming set of vertices is not empty or, otherwise,  passing it to next stage.).
 
\begin{figure}[h!]
\centering
\begin{subfigure}[b]{\textwidth}
 \centering
  \inputtikz{dp_example_1}
  \caption{The edge $(1,2)$ is arriving to $\gwc$.}
  \label{fig:dp_example_1_2a}
\end{subfigure}
\vspace{.3cm}

\begin{subfigure}[b]{\textwidth}
 \centering
  \inputtikz{dp_example_2}
  \caption{When the edge $(1,2)$ arrives to $\gwc$, it  spawns a new instance of $\fwc$ before $\gwc$. Filter instance $F_{\{1,2\}}$ is connected to  $\gwc$ through channels $\ice$ and  $\csofv$. The state of the new filter instance $F_{\{1,2\}}$ is initialized with the set of vertices $\{1,2\}$. The edge $(3,6)$ arrives to the new filter instance $F_{\{1,2\}}$.}
  \label{fig:dp_example_1_2b}
\end{subfigure}
\caption[{[PoC] $\dpwcc$ Evolving first state}]{Evolution of the $\dpwcc$: First state}
\label{fig:dp_example_1_2}
\end{figure}
\vspace{.5cm}

\begin{figure}[h!]
\centering
\begin{subfigure}[b]{\textwidth}
 \centering
  \inputtikz{dp_example_3}
  \caption{None of the vertices in the edge $(3,6)$ is in the set of vertices $\{1,2\}$ in the state of $F_{\{1,2\}}$, hence it is passed through $\ice$ to $\gwc$.}
  \label{fig:dp_example_3_4a}
\end{subfigure}
\vspace{.3cm}

\begin{subfigure}[b]{\textwidth}
 \centering
  \inputtikz{dp_example_4}
  \caption{When the edge $(3,6)$ arrives to $\gwc$, it spawns the filter instance $F_{\{3,6\}}$  between $F_{\{1,2\}}$ and $\gwc$. Filter instance $F_{\{1,2\}}$ is connected to the new filter instance $F_{\{3,6\}}$ and this one is connected to  $\gwc$ through channels $\ice$ and  $\csofv$. The state of the new filter instance $F_{\{3,6\}}$ is initialized with the set of vertices $\{3,6\}$. The edge $(3,4)$ arrives to $F_{\{1,2\}}$  and $\mathsf{Sr_{WCC}}$ is fed with the mark $\eof$. Edges $(3,4)$ and $(4,5)$ remain passing through $\ice$.}
  \label{fig:dp_example_3_4b}
\end{subfigure}
\caption[{[PoC] $\dpwcc$ Evolving second state}]{Evolution of the $\dpwcc$: Second state}
\label{fig:dp_example_3_4}
\end{figure}
\vspace{.5cm}

\begin{figure}[h!]
\centering
\begin{subfigure}[b]{\textwidth}
 \centering
  \inputtikz{dp_example_5}
  \caption{$\mathsf{Sr_{WCC}}$  fed both, $\ice$ and $\csofv$, channels with the mark $\eof$ received from the input stream in previous state and then, it died. The edge $(4,5)$ is arriving to $\gwc$ and the edge $(3,4)$ is arriving to $F_{\{3,6\}}$. }
  \label{fig:dp_example_5_6a}
\end{subfigure}
\vspace{.3cm}

\begin{subfigure}[b]{\textwidth}
 \centering
  \inputtikz{dp_example_6}
  \caption{When the edge $(4,5)$ arrives to $\gwc$, it spawns the filter instance $F_{\{4,5\}}$  between $F_{\{3,6\}}$ and $\gwc$. Filter instance $F_{\{3,6\}}$ is connected to the new filter instance $F_{\{4,5\}}$ and this one is connected to  $\gwc$ through channels $\ice$ and  $\csofv$.  Since the edge $(3,4)$ arrived to $F_{\{3,6\}}$ at the same time and  vertex $3$ belongs to the set of connected vertices of the filter $F_{\{3,6\}}$,  the vertex $4$ is added to the state of $F_{\{3,6\}}$. Now, the state of $F_{\{3,6\}}$ is the connected set of vertices $\{3,4,6\}$. When the mark $\eof$ arrives to the first filter instance, $F_{\{1,2\}}$, through  $\csofv$, this stage passes  its partial set of connected vertices,  $\{1,2\}$, through $\csofv$ and dies.  This action will activate $\Actt$ in next  filter instances to start building  maximal connected components. In this example, the state in  $F_{\{3,6\}}$, $\{3,4,6\}$, and the arriving set $\{1,2\}$ do not intersect and, hence, both sets of vertices, $\{1,2\}$ and $\{3,4,6\}$ will be passed  to the next filter instance through $\csofv$.}
  \label{fig:dp_example_5_6b}
\end{subfigure}
\caption[{[PoC] $\dpwcc$ Evolving third state}]{Evolution of the $\dpwcc$: Third state}
\label{fig:dp_example_5_6}
\end{figure}
\vspace{.5cm}

\begin{figure}[h!]
\centering
\begin{subfigure}[b]{\textwidth}
 \centering
  \inputtikz{dp_example_7}
  \caption{The set of connected vertices  $\{3,4,6\}$ is arriving to $F_{\{4,5\}}$. The mark $\eof$ continues passing to next stages through the channel $\ice$.}
  \label{fig:dp_example_7_8a}
\end{subfigure}
\vspace{.3cm}

\begin{subfigure}[b]{\textwidth}
 \centering
  \inputtikz{dp_example_8}
  \caption{Since the intersection of the set of connected vertices $\{3,4,6\}$ arrived to  $F_{\{4,5\}}$ and its state is not empty, this state is enlarged to be $\{3,4,5,6\}$. The set of connected vertices $\{1,2\}$ is arriving to  $F_{\{4,5\}}$}
  \label{fig:dp_example_7_8b}
\end{subfigure}
\caption[{[PoC] $\dpwcc$ Evolving fourth state}]{Evolution of the $\dpwcc$:  Fourth state}
\label{fig:dp_example_7_8}
\end{figure}
\vspace{.5cm}

\begin{figure}[h!]
\centering
\begin{subfigure}[b]{\textwidth}
 \centering
  \inputtikz{dp_example_9}
  \caption{$F_{\{4,5\}}$ has passed the set of connected vertices  $\{1,2\}$ and it is arriving to $\mathsf{Sk_{WCC}}$. The mark $\eof$ is arriving to  $F_{\{4,5\}}$ through $\csofv$.}
  \label{fig:dp_example_9_10a}
\end{subfigure}
\vspace{.3cm}

\begin{subfigure}[b]{\textwidth}
 \centering
  \inputtikz{dp_example_10}
  \caption{Since the mark $\eof$ arrived to $F_{\{4,5\}}$ through $\csofv$, it passes its state, the set $\{3,4,5,6\}$ through $\csofv$ to next stages and died. The set of connected vertices  $\{1,2\}$ arrived to $\mathsf{Sk_{WCC}}$ and this implies  that $\{1,2\}$ is a maximal set of connected vertices, i.e. a connected component of the input graph. Hence,  $\mathsf{Sk_{WCC}}$ output this first weakly connected component.}
 \label{fig:dp_example_9_10b}
\end{subfigure}
\vspace{.5cm}

\begin{subfigure}[b]{\textwidth}
 \centering
  \inputtikz{dp_example_11}
  \caption{Finally, the set of connected vertices  $\{3,4,5,6\}$ arrived to $\mathsf{Sk_{WCC}}$ and was output as a new weakly connected component. Besides, the mark $\eof$ also arrived to $\mathsf{Sk_{WCC}}$ through $\csofv$ and thus, it dies.}
  \label{fig:dp_example_9_10c}
\end{subfigure}
\vspace{.3cm}

\begin{subfigure}[b]{\textwidth}
 \centering
  \inputtikz{dp_example_12}
  \caption{The weakly connected component of in the graph \autoref{fig:example_dp_graph} such as they have been emitted by $\dpwcc$.}
  \label{fig:dp_example_9_10d}
\end{subfigure}
\caption[{[PoC] $\dpwcc$ Evolving last state}]{Last states in the evolution of the $\dpwcc$}
\label{fig:dp_example_9_10}
\end{figure}
%
\clearpage



