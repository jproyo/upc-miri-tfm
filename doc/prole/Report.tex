\documentclass[twoside]{article}
\usepackage[utf8]{inputenc}
\usepackage{amsmath}
\usepackage{amsfonts}
\usepackage{amsthm}
\usepackage{array}
\usepackage{graphicx}
\usepackage{parskip}
\usepackage{fancyhdr}
\usepackage{tikz}
\usepackage{float}
\usepackage{listings}
\usepackage{color}
\usepackage{caption}
\usepackage{longtable}
\usepackage[pdfencoding=auto]{hyperref}
\usepackage[nottoc]{tocbibind}
\usepackage[cache=false]{minted}
\usemintedstyle{default}
\newminted{haskell}{frame=lines,framerule=2pt}
\newminted{R}{frame=lines,framerule=2pt}
\graphicspath{{./images/}}

\tikzstyle{bag} = [align=center]

\title{A Haskell implementation of Dynamic Pipeline Paradigm to find Connected Components of a Graph}
\author{
\begin{tabular}{c c}       
      Juan Pablo Royo Sales & Edelmira Pasarella\\ 
      \small{Universitat Politècnica de Catalunya} & \small{Computer Science Department}\\
      \small{Barcelona - Spain} & \small{Universitat Politècnica de Catalunya}\\ 
      \small{\texttt{juan.pablo.royo@estudiantat.upc.edu}} & \small{Barcelona - Spain}\\ 
      \small{} & \small{\texttt{edelmira@cs.upc.edu}}\\
\end{tabular}
}

\pagestyle{fancy}
\fancyhf{}
\fancyhead[LO]{J. P. Royo Sales}
\fancyhead[RE,RO]{\thepage}
\fancyhead[LE]{Dynamic Pipeline on Haskell}
\fancyfoot[L,C]{}
\renewcommand{\headrulewidth}{0pt}
\renewcommand{\footrulewidth}{0pt}
\newtheorem{hyp}{Hypothesis}
\date{}

\begin{document}

\maketitle
\begin{abstract}
\textit{Dynamic Pipeline Paradigm} has been defined in order to solve problems where the data is heterogeneous 
and in motion. Taking advantage on pipelining stage parallelization, this paradigm allows to dynamically adapt 
stage computations according to the input. It has been shown in the definition of this computational model that any
implementation attempt of the Paradigm requires fast and flexible parallelization techniques as well as tools that 
are suitable with the notion of Computation as a First Class Citizen. In this work we implement \textit{Dynamic Pipeline Paradigm}
using \texttt{Haskell} for \textit{Finding Connected Components of a Graph} as a primary example problem. We show different results 
and how \texttt{Haskell} behaves on that context showing that it is a suitable Language for implementing \textit{Dynamic Pipeline Paradigm} 
not only because its strong theoretical foundations which provides a strong manipulations of Computations as primary entities, but also because it has a powerful 
Set of Tools for writing multithreading and parallel computations with optimal performance.
\end{abstract}
      
\section{Introduction}
lorem ipsum asometakfjdaslfjlkafjlasdfalk
\section{Bi-Triangle Counting}
\section{Dynamic Pipeline}
\section{Haskell Implementation and Techniques}
\section{Implementation}
\section{Experiments and Discussion}
\section{Future Work}
\section{Conclusions}

\nocite{*}
\bibliographystyle{alpha}
\bibliography{Report}

\appendix

\end{document}

