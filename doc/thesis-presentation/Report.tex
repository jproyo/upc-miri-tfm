\documentclass{beamer}
\usetheme{Boadilla}
\usecolortheme{default}
\usepackage[utf8]{inputenc}
\usepackage{amsmath}
\usepackage{amsfonts}
\usepackage{amssymb}
\usepackage{amsthm}
\usepackage{array}
\usepackage{float}
\usepackage{listings}
\usepackage{color}
\usepackage{caption}
\usepackage{multirow}
\usepackage{xcolor}
\usepackage{colortbl}
\usepackage{caption}
\usepackage{subcaption}
\usepackage{multimedia}
\usepackage{tikz}
\usepackage{varwidth}
\usepackage[absolute,overlay]{textpos}
\usetikzlibrary{shapes.misc,shadows}
\usetikzlibrary{quotes,positioning,arrows,decorations.markings}
\usetikzlibrary{positioning} 
\usepackage[cache=true,section]{minted}
\usepackage[algosection,linesnumbered]{algorithm2e}
\usepackage{glossaries}
\usemintedstyle{default}

\def\HiLi{\leavevmode\rlap{\hbox to \hsize{\color{yellow!50}\leaders\hrule height .8\baselineskip depth .5ex\hfill}}}
\definecolor{androidgreen}{rgb}{0.64,0.78,0.22}
\definecolor{titaniumyellow}{rgb}{0.93,0.9,0.0}
\definecolor{light}{rgb}{0.5, 0.5, 0.5}
\newtheorem{complexity}{Complexity}

\makeatletter
\long\def\beamer@author[#1]#2{%
  \def\insertauthor{\def\inst{\beamer@insttitle}\def\and{\beamer@andtitle}%
  \begin{tabular}{rl}#2\end{tabular}}%
  \def\beamer@shortauthor{#1}%
  \ifbeamer@autopdfinfo%
    \def\beamer@andstripped{}%
    \beamer@stripands#1 \and\relax
    {\let\inst=\@gobble\let\thanks=\@gobble\def\and{, }\hypersetup{pdfauthor={\beamer@andstripped}}}
  \fi%
}
\makeatother

\setbeamercolor{block title}{bg=blue!10, fg=black}
\setbeamercolor{block body}{bg=white!25}

\tikzset{
    max width/.style args={#1}{
        execute at begin node={\begin{varwidth}{#1}},
        execute at end node={\end{varwidth}}
    }
}

\newcommand{\itemcheck}{\item[\checkmark]}

\title[Incrementally Enumerating BT in BG]{An Algorithm for incrementally Enumerating Bitriangles in large Bipartite Networks}
\subtitle{Master Thesis\vspace{-0.5cm}}
\author[Juan Pablo Royo Sales (Master Thesis)]{\vspace{-0.5cm}Juan Pablo Royo Sales}
\institute[]{%
  {\small Facultat d’Informàtica de Barcelona (FIB)}\\
  {\small Universitat Politècnica de Catalunya (UPC) – BarcelonaTech}\\
  \vspace{0.2cm}
  Master in Innovation and Research in Informatics\\ 
  Advance Computing\\
  \vspace{0.2cm}
  \tiny{%
  Supervisors: Edelmira Pasarella, Computer Science Department\\
  Maria-Esther Vidal, Leibniz Information Centre for Science and Technology-TIB, and L3S Centre at the Leibniz University of Hannover\\
  Cristina Zoltan, Computer Science Department
  }
}

\date[October 29, 2021]{October 29, 2021}
  
\titlegraphic{%
  \begin{tikzpicture}[overlay,remember picture]
    \node[left=3.5cm] at (current page.30){
      \includegraphics[height=1.5cm]{upc_logo}
    };
  \end{tikzpicture}
}

\usetikzlibrary{shapes.misc,shadows}
\usetikzlibrary{quotes,positioning,arrows,decorations.markings}
\usetikzlibrary{positioning} 
\usemintedstyle{default}
\newminted{haskell}{frame=lines,framerule=2pt}
\newminted{R}{frame=lines,framerule=2pt}
\graphicspath{{./images/}}
\tikzstyle{bag} = [align=center]
\newcommand{\dw}{\mathbb{DW}}
\newcommand{\aw}{\mathbb{AW}}
\newcommand{\bt}{\mathbb{BT}}
\newcommand{\bti}{BT_{(l_1, l_2,l_3)}^{(u_1, u_2, u_3)}}
\newcommand{\at}{\mathbb{AT}}
\newcommand{\st}{ST}
\newcommand{\sw}{\mathtt{spawn}}
\newcommand{\fd}{\mathtt{killFilter}}
\newcommand{\fid}{\mathtt{filterIsDied}}
\newcommand{\us}{\mathtt{updateState}}
\newcommand{\gs}{\mathtt{getState}}
\newcommand{\p}{\mathtt{push}}
\newcommand{\mt}{\mathtt{matchQ}}
\newcommand{\io}{\mathtt{indexOf}}
\newcommand{\la}{\left\langle}
\newcommand{\ra}{\right\rangle}
\newcommand{\DP}{\mathsf{DP}}
\newcommand{\dpwcc}{\mathsf{DP_{WCC}}}
\newcommand{\iwcc}{\mathsf{Sr}}
\newcommand{\iwc}{\mathsf{Sr_{WCC}}}
\newcommand{\owcc}{\mathsf{Sk}}
\newcommand{\owc}{\mathsf{Sk_{WCC}}}
\newcommand{\fwcc}{\mathsf{F}} 
\newcommand{\fwc}{\mathsf{F_{WCC}}} 
\newcommand{\gwcc}{\mathsf{G}}
\newcommand{\gwc}{\mathsf{G_{WCC}}}
\newcommand{\ice}{\mathsf{IC_E}}
\newcommand{\csofv}{\mathsf{IC_{set(V)}}}
\newcommand{\sgen}{\mathsf{S_G}}
\newcommand{\sfilter}{\mathsf{S_F}}
\newcommand{\sinp}{\mathsf{S_I}}
\newcommand{\sout}{\mathsf{S_O}}
\newcommand{\istream}{\mathsf{D}}
\newcommand{\wccout}{\mathsf{R}}
\newcommand{\fmem}{\mathsf{M_F}}
\newcommand{\eof}{\mathsf{eof}}
\newcommand{\Act}{\mathsf{actor_1}}
\newcommand{\Actt}{\mathsf{actor_2}}
\newcommand{\gdsl}{G_{dsl}}

\renewcommand\listingscaption{Source Code}
\providecommand*{\listingautorefname}{Source Code}

\DeclareMathOperator*{\argmax}{arg\,max}
\DeclareMathOperator*{\argmin}{arg\,min}

\newacronym{prole21}{PROLE21}{Jornadas de la Sociedad de Ingeniería de Software y Tecnologías de Desarrollo de Software}
\newacronym{bt}{BT}{Bitriangle}
\newacronym{bg}{BG}{Bipartite Graph}
\newacronym{ug}{UG}{Unipartite Graph}
\newacronym{dp}{DPP}{Dynamic Pipeline Paradigm}
\newacronym{dpf}{DPF}{Dynamic Pipeline Framework}
\newacronym{dpfh}{DPF-Haskell}{Haskell Dynamic Pipeline Framework}
\newacronym{ds}{DS}{Data Streaming}
\newacronym{dap}{DAP}{Data Parallelism}
\newacronym{pip}{PIP}{Pipeline Parallelism}
\newacronym{mr}{MR}{MapReduce}
\newacronym{er}{ER}{Entity Resolution}
\newacronym{hack}{Hackage}{The Haskell Package Repository}
\newacronym{dpbt}{DP-BT-Haskell}{$DP_{BT}$ in Haskell}
\newacronym{dpwcc}{DP-WCC-Haskell}{$DP_{WCC}$ in Haskell}
\newacronym{dpl}{DPL}{dynamic-pipeline}
\newacronym{bfs}{BFS}{Breadth-First Search}
\newacronym{dfs}{DFS}{Depth-First Search}
\newacronym{wcc}{WCC}{Weak Connected Components}
\newacronym{hs}{Haskell}{Haskell Programming Language}
\newacronym{fp}{FP}{Functional Programming}
\newacronym{stm}{STM}{Software Transactional Memory}
\newacronym{rl}{R}{R Language}
\newacronym{os}{OS}{Operative System}
\newacronym{dm}{Dm}{Diefficency Metrics}
\newacronym{tfft}{TFFT}{Time for the first tuple}
\newacronym{et}{ET}{Execution Time}
\newacronym{comp}{Comp}{Completeness}
\newacronym{tt}{T}{Throughput}
\newacronym{dt}{dief$@$t}{Diefficiency first $t$ time units}
\newacronym{snap}{SNAP}{Stanford Network Data Set Collection}
\newacronym{ghc}{GHC}{Glasgow Haskell Compiler}
\newacronym{dsl}{DSL}{Domain-specific Language}
\newacronym{edsl}{EDSL}{Embedded Domain-specific Language}
\newacronym{idl}{IDL}{Interpreter of DSL}
\newacronym{rs}{RS}{Runtime System}
\newacronym{go}{Go}{Go Programming Language}
\newacronym{hl}{HL}{Host Language}

\glsdisablehyper

\newtheorem{hyp}{Hypothesis}


\usetikzlibrary{external,quotes,positioning,calc,arrows,decorations.markings,positioning,fit,shapes.misc,shadows}

\tikzexternalize[shell escape=-shell-escape,mode=graphics if exists,prefix=images/]


\begin{document}

  \begin{frame}
    \vspace{1.3cm}
    \titlepage
  \end{frame}

  \begin{frame}{Agenda}
    \tableofcontents[hidesubsections]
  \end{frame}
  
  \begin{frame}{Agenda}
    \section{Introduction}
    \tableofcontents[currentsection,hideothersubsections]
  \end{frame}

  \subsection{Goal}
  \begin{frame}[fragile]{Goal}
  \begin{center}
    Provide an Algorithm for Incrementally Enumarting Bitriangles in large Bipartite Networks using \textbf{Dynamic Pipeline Paradigm}
  \end{center}    
\end{frame}

\begin{frame}[fragile]{Goal}
  \begin{center}
    {\color{light}Provide an Algorithm for Incrementally Enumarting Bitriangles in large} \underline{{\color{red}\textbf{Bipartite Networks}}} {\color{light}{using \textbf{Dynamic Pipeline Paradigm}}}
  \end{center}    
  \begin{figure}
    \centering
    \resizebox{0.7\textwidth}{!}{\inputtikz{bipartite}}
  \end{figure}
\end{frame}

\begin{frame}[fragile]{Goal}
  \begin{center}
    {\color{light}Provide an Algorithm for Incrementally Enumarting} \underline{{\color{red}\textbf{Bitriangles}}} {\color{light}{in large Bipartite Networks using \textbf{Dynamic Pipeline Paradigm}}}
  \end{center}    
  \begin{figure}
    \centering
    \resizebox{0.7\textwidth}{!}{\inputtikz{bitriangle}}
  \end{figure}
\end{frame}



  \subsection{Motivation}
  % \begin{frame}[fragile]{Motivation}
%   \begin{center}
%     Provide an Algorithm for Incrementally Enumarting Bitriangles in large Bipartite Networks using \textbf{Dynamic Pipeline Paradigm}
%   \end{center}    
%   \vspace{0.5cm}
%   \begin{center}
%   \huge\emph{Why?}
%   \end{center}
% \end{frame}

\begin{frame}[fragile]{Motivation}
  % \begin{center}
  %   Provide an Algorithm for Incrementally Enumarting Bitriangles in large Bipartite Networks using \textbf{Dynamic Pipeline Paradigm}
  % \end{center}    
  \begin{itemize}
    \item \textbf{Bipartite Graphs} (BG) \underline{\color{red}models} interesting \underline{\color{red}relations} between \underline{\color{red}different set of objects}: i.e. \emph{diseasome bipartite networks, drugs-side effects networks, subscribers-TV show network, etc.}
  \end{itemize}
\end{frame}

\begin{frame}[fragile]{Motivation}
    \begin{itemize}
      \setlength\itemsep{2em}
      \item {\color{light}\textbf{Bipartite Graphs} (BG) models interesting relations between different set of objects: i.e. \emph{diseasome bipartite networks, drugs-side effects networks, subscribers-TV show network, etc.}}
      \item \textbf{Metrics} for analyzing BG \underline{\color{red}same as Unipartite Graphs}: \emph{clustering coefficient, communities, social analysis}.
    \end{itemize}
\end{frame}

\begin{frame}[fragile]{Motivation}
    \begin{itemize}
      \setlength\itemsep{2em}
      \item {\color{light}\textbf{Bipartite Graphs} (BG) models interesting relations between different set of objects: i.e. \emph{diseasome bipartite networks, drugs-side effects networks, subscribers-TV show network, etc.}}
      \item {\color{light}\textbf{Metrics} for analyzing BG same as Unipartite Graphs: \emph{clustering coefficient, communities, social analysis}.}
      \item \textbf{Projection method} distorts properties of original BG: i.e. link prediction.
    \end{itemize}
\end{frame}

\begin{frame}[fragile]{Motivation}
  \begin{itemize}
    \setlength\itemsep{1.5em}
    % \item {\color{light}Bipartite Graphs (BG) \textbf{models} interesting relations between \textbf{different set of objects}: i.e. \emph{diseasome bipartite networks, drugs-side effects networks, subscribers-TV show network, etc.}}
    % \item {\color{light}\textbf{Metrics} for \textbf{analyzing} BG same as Unipartite Graphs: \emph{clustering coefficient, communities, social analysis}.}
    % \item {\color{light}\textbf{Projection method} distorts properties of original BG: i.e. link prediction.}
    \item \textbf{Bitriangles} as the \underline{\color{red}small unit of cohesion} in BG. \textbf{Bitriangles counting} to compute \underline{\color{red}BG metrics} -- \emph{clustering coefficient, communities, social analysis} --.
  \end{itemize}
\end{frame}

\begin{frame}[fragile]{Motivation}
  \begin{itemize}
    \setlength\itemsep{1.5em}
    % \item {\color{light}Bipartite Graphs (BG) \textbf{models} interesting relations between \textbf{different set of objects}: i.e. \emph{diseasome bipartite networks, drugs-side effects networks, subscribers-TV show network, etc.}}
    % \item {\color{light}\textbf{Metrics} for \textbf{analyzing} BG same as Unipartite Graphs: \emph{clustering coefficient, communities, social analysis}.}
    % \item {\color{light}\textbf{Projection method} distorts properties of original BG: i.e. link prediction.}
    \item {\color{light}\textbf{Bitriangles} as the small unit of cohesion in BG. \textbf{Bitriangles counting} to compute BG metrics -- \emph{clustering coefficient, communities, social analysis} --.}
    \item Counting is \textbf{not enough}. Knowing \underline{\color{red}structural components of bitriangles} is needed: i.e.: \emph{what gene is involve in two or more diseases in diseasome networks}.
  \end{itemize}
\end{frame}

\begin{frame}[fragile]{Motivation}
    \begin{itemize}
      \setlength\itemsep{1.5em}
      % \item {\color{light}Bipartite Graphs (BG) \textbf{models} interesting relations between \textbf{different set of objects}: i.e. \emph{diseasome bipartite networks, drugs-side effects networks, subscribers-TV show network, etc.}}
      % \item {\color{light}\textbf{Metrics} for \textbf{analyzing} BG same as Unipartite Graphs: \emph{clustering coefficient, communities, social analysis}.}
      % \item {\color{light}\textbf{Projection method} distorts properties of original BG: i.e. link prediction.}
      \item {\color{light}\textbf{Bitriangles} as the small unit of cohesion in BG. \textbf{Bitriangles counting} to compute BG metrics -- \emph{clustering coefficient, communities, social analysis} --.}
      \item {\color{light}Counting is \textbf{not enough}. Knowing structural components of bitriangles is needed: i.e.: \emph{what gene is involve in two or more diseases in diseasome networks}.}
      \item \textbf{\emph{"All-or-Nothing" vs. "Pay-as-you-go"} model}. 
    \end{itemize}
\end{frame}



  \subsection{Methodology}
  
\begin{frame}[fragile]{Working Methodology}
  \begin{itemize}
    \item Conduct a \underline{\color{red}Proof of Concept}: \textbf{DPP Haskell} for \underline{\color{red}WCC}.
  \end{itemize}   
\end{frame}

\begin{frame}[fragile]{Working Methodology}
    \begin{itemize}
      \setlength\itemsep{1.5em}
      \item {\color{light}Conduct a Proof of Concept: \textbf{DPP Haskell} for WCC.}
      \item Build a \textbf{Dynamic Pipeline Framework} in \textbf{Haskell} to implement \underline{\color{red}any algorithm}
  \end{itemize}   
\end{frame}

\begin{frame}[fragile]{Working Methodology}
    \begin{itemize}
      \setlength\itemsep{1.5em}
      \item {\color{light}Conduct a Proof of Concept: \textbf{DPP Haskell} for WCC.}
      \item {\color{light}Build a \textbf{Dynamic Pipeline Framework} in \textbf{Haskell} to implement any algorithm }
      \item Provide a \underline{\color{red}pseudo-code algorithm} definition for \textbf{Incrementally Enumerating Bitriangles in large Bipartite Networks}
  \end{itemize}   
\end{frame}

\begin{frame}[fragile]{Working Methodology}
    \begin{itemize}
      \setlength\itemsep{1.5em}
      \item {\color{light}Conduct a Proof of Concept: \textbf{DPP Haskell} for WCC.}
      \item {\color{light}Build a \textbf{Dynamic Pipeline Framework} in \textbf{Haskell} to implement any algorithm }
      \item {\color{light}Provide an pseudo-code algorithm definition for \textbf{Incrementally Enumerating Bitriangles in large Bipartite Networks}}
      \item Implement \underline{\color{red}that Algorithm} using \textbf{Dynamic Pipeline Framework in Haskell}
  \end{itemize}   
\end{frame}




  \subsection{Dynamic Pipeline Paradigm}
  \inputtikz{genericDP-Fig}
  \begin{frame}[fragile]{Dynamic Pipeline Paradigm}
    \begin{tikzpicture}[overlay, remember picture]
      \node[xshift=-3cm,yshift=-2cm] at (current page.north east) {
        \inputtikz{graph-DP-Fig}
      };
    \end{tikzpicture}
    \begin{figure}
      \centering
      \resizebox{\textwidth}{!}{\inputtikz{dp_example_0}}
    \end{figure}
  \end{frame}

  \begin{frame}[fragile]{Dynamic Pipeline Paradigm}
    \begin{tikzpicture}[overlay, remember picture]
      \node[xshift=-1.5cm,yshift=-0.7cm] at (current page.north east) {
        \resizebox{2cm}{1cm}{\inputtikz{graph-DP-Fig}}
      };
      \node[xshift=3cm,yshift=-2cm,opacity=0.3] at (current page.north west) {
        \resizebox{5cm}{2cm}{\inputtikz{dp_example_0}}
      };
    \end{tikzpicture}
    \begin{figure}
      \centering
      \resizebox{\textwidth}{!}{\inputtikz{dp_example_1}}
    \end{figure}
  \end{frame}

  \begin{frame}[fragile]{Dynamic Pipeline Paradigm}
    \begin{tikzpicture}[overlay, remember picture]
      \node[xshift=-1.5cm,yshift=-0.7cm] at (current page.north east) {
        \resizebox{2cm}{1cm}{\inputtikz{graph-DP-Fig}}
      };
    \node[xshift=3cm,yshift=-2cm,opacity=0.3] at (current page.north west) {
      \resizebox{5cm}{2cm}{\inputtikz{dp_example_0}}
      };
    \node[xshift=9cm,yshift=-2cm,opacity=0.3] at (current page.north west) {
        \resizebox{6cm}{2cm}{\inputtikz{dp_example_1}}
    };
  \end{tikzpicture}
    \begin{figure}
      \centering
      \resizebox{\textwidth}{!}{\inputtikz{dp_example_2}}
    \end{figure}
  \end{frame}   

  \begin{frame}[fragile]{Dynamic Pipeline Paradigm}
    \begin{tikzpicture}[overlay, remember picture]
      \node[xshift=-1.5cm,yshift=-0.7cm] at (current page.north east) {
        \resizebox{2cm}{1cm}{\inputtikz{graph-DP-Fig}}
      };
      \node[xshift=3cm,yshift=-2cm,opacity=0.3] at (current page.north west) {
        \resizebox{5cm}{2cm}{\inputtikz{dp_example_1}}
        };
      \node[xshift=9cm,yshift=-2cm,opacity=0.3] at (current page.north west) {
          \resizebox{6cm}{2cm}{\inputtikz{dp_example_2}}
      };
    \end{tikzpicture}
    \begin{figure}
      \centering
      \resizebox{\textwidth}{!}{\inputtikz{dp_example_3}}
    \end{figure}
  \end{frame}   

  \begin{frame}[fragile]{Dynamic Pipeline Paradigm}
    \begin{tikzpicture}[overlay, remember picture]
      \node[xshift=-1.5cm,yshift=-0.7cm] at (current page.north east) {
        \resizebox{2cm}{1cm}{\inputtikz{graph-DP-Fig}}
      };
      \node[xshift=3cm,yshift=-2cm,opacity=0.3] at (current page.north west) {
        \resizebox{5cm}{2cm}{\inputtikz{dp_example_2}}
        };
      \node[xshift=9cm,yshift=-2cm,opacity=0.3] at (current page.north west) {
          \resizebox{6cm}{2cm}{\inputtikz{dp_example_3}}
      };
    \end{tikzpicture}
    \begin{figure}
      \centering
      \resizebox{\textwidth}{!}{\inputtikz{dp_example_4}}
    \end{figure}
  \end{frame}   

  \begin{frame}[fragile]{Dynamic Pipeline Paradigm}
    \begin{tikzpicture}[overlay, remember picture]
      \node[xshift=-1.5cm,yshift=-0.7cm] at (current page.north east) {
        \resizebox{2cm}{1cm}{\inputtikz{graph-DP-Fig}}
      };
      \node[xshift=3cm,yshift=-2cm,opacity=0.3] at (current page.north west) {
        \resizebox{5cm}{2cm}{\inputtikz{dp_example_3}}
        };
      \node[xshift=9cm,yshift=-2cm,opacity=0.3] at (current page.north west) {
          \resizebox{6cm}{2cm}{\inputtikz{dp_example_4}}
      };
    \end{tikzpicture}
    \begin{figure}
      \centering
      \resizebox{\textwidth}{!}{\inputtikz{dp_example_5}}
    \end{figure}
  \end{frame}   

  \begin{frame}[fragile]{Dynamic Pipeline Paradigm}
    \begin{tikzpicture}[overlay, remember picture]
      \node[xshift=-1.5cm,yshift=-0.7cm] at (current page.north east) {
        \resizebox{2cm}{1cm}{\inputtikz{graph-DP-Fig}}
      };
      \node[xshift=3cm,yshift=-2cm,opacity=0.3] at (current page.north west) {
        \resizebox{5cm}{2cm}{\inputtikz{dp_example_4}}
        };
      \node[xshift=9cm,yshift=-2cm,opacity=0.3] at (current page.north west) {
          \resizebox{6cm}{2cm}{\inputtikz{dp_example_5}}
      };
    \end{tikzpicture}
    \begin{figure}
      \centering
      \resizebox{\textwidth}{!}{\inputtikz{dp_example_6}}
    \end{figure}
  \end{frame}  

  \begin{frame}[fragile]{Dynamic Pipeline Paradigm}
    \begin{tikzpicture}[overlay, remember picture]
      \node[xshift=-1.5cm,yshift=-0.7cm] at (current page.north east) {
        \resizebox{2cm}{1cm}{\inputtikz{graph-DP-Fig}}
      };
      \node[xshift=3cm,yshift=-2cm,opacity=0.3] at (current page.north west) {
        \resizebox{5cm}{2cm}{\inputtikz{dp_example_5}}
        };
      \node[xshift=9cm,yshift=-2cm,opacity=0.3] at (current page.north west) {
          \resizebox{6cm}{2cm}{\inputtikz{dp_example_6}}
      };
    \end{tikzpicture}
    \begin{figure}
      \centering
      \inputtikz{dp_example_7}
    \end{figure}
  \end{frame}  

  \begin{frame}[fragile]{Dynamic Pipeline Paradigm}
    \begin{tikzpicture}[overlay, remember picture]
      \node[xshift=-1.5cm,yshift=-0.7cm] at (current page.north east) {
        \resizebox{2cm}{1cm}{\inputtikz{graph-DP-Fig}}
      };
      \node[xshift=3cm,yshift=-2cm,opacity=0.3] at (current page.north west) {
        \resizebox{5cm}{2cm}{\inputtikz{dp_example_6}}
        };
      \node[xshift=9cm,yshift=-2cm,opacity=0.3] at (current page.north west) {
          \resizebox{6cm}{2cm}{\inputtikz{dp_example_7}}
      };
    \end{tikzpicture}
    \begin{figure}
      \centering
      \inputtikz{dp_example_8}
    \end{figure}
  \end{frame} 
  
  \begin{frame}[fragile]{Dynamic Pipeline Paradigm}
    \begin{tikzpicture}[overlay, remember picture]
      \node[xshift=-1.5cm,yshift=-0.7cm] at (current page.north east) {
        \resizebox{2cm}{1cm}{\inputtikz{graph-DP-Fig}}
      };
      \node[xshift=3cm,yshift=-2cm,opacity=0.3] at (current page.north west) {
        \resizebox{5cm}{2cm}{\inputtikz{dp_example_7}}
        };
      \node[xshift=9cm,yshift=-2cm,opacity=0.3] at (current page.north west) {
          \resizebox{6cm}{2cm}{\inputtikz{dp_example_8}}
      };
    \end{tikzpicture}
    \begin{figure}
      \centering
      \inputtikz{dp_example_9}
    \end{figure}
  \end{frame}  

  \begin{frame}[fragile]{Dynamic Pipeline Paradigm}
    \begin{tikzpicture}[overlay, remember picture]
      \node[xshift=-1.5cm,yshift=-0.7cm] at (current page.north east) {
        \resizebox{2cm}{1cm}{\inputtikz{graph-DP-Fig}}
      };
      \node[xshift=3cm,yshift=-2cm,opacity=0.3] at (current page.north west) {
        \resizebox{5cm}{2cm}{\inputtikz{dp_example_8}}
        };
      \node[xshift=9cm,yshift=-2cm,opacity=0.3] at (current page.north west) {
          \resizebox{6cm}{1cm}{\inputtikz{dp_example_9}}
      };
    \end{tikzpicture}
    \begin{figure}
      \centering
      \inputtikz{dp_example_10}
    \end{figure}
  \end{frame}  

  \begin{frame}[fragile]{Dynamic Pipeline Paradigm}
    \begin{tikzpicture}[overlay, remember picture]
      \node[xshift=-1.5cm,yshift=-0.7cm] at (current page.north east) {
        \resizebox{2cm}{1cm}{\inputtikz{graph-DP-Fig}}
      };
      \node[xshift=3cm,yshift=-2cm,opacity=0.3] at (current page.north west) {
        \resizebox{5cm}{1cm}{\inputtikz{dp_example_9}}
        };
      \node[xshift=9cm,yshift=-2cm,opacity=0.3] at (current page.north west) {
          \resizebox{5cm}{1cm}{\inputtikz{dp_example_10}}
      };
    \end{tikzpicture}
    \begin{figure}
      \centering
      \inputtikz{dp_example_11}
    \end{figure}
  \end{frame}  

  \begin{frame}[fragile]{Dynamic Pipeline Paradigm}
    \begin{tikzpicture}[overlay, remember picture]
      \node[xshift=-1.5cm,yshift=-0.7cm] at (current page.north east) {
        \resizebox{2cm}{1cm}{\inputtikz{graph-DP-Fig}}
      };
      \node[xshift=3cm,yshift=-2cm,opacity=0.3] at (current page.north west) {
        \resizebox{5cm}{1cm}{\inputtikz{dp_example_10}}
        };
      \node[xshift=9cm,yshift=-2cm,opacity=0.3] at (current page.north west) {
          \resizebox{5cm}{1cm}{\inputtikz{dp_example_11}}
      };
    \end{tikzpicture}
    \begin{figure}
      \centering
      \inputtikz{dp_example_12}
    \end{figure}
  \end{frame}  

  
  \begin{frame}{Agenda}
    \section{Work Development}
    \tableofcontents[currentsection,hideothersubsections]
  \end{frame}
  
  \subsection{Proof of Concept: $DP_{WCC}$}
  \begin{frame}[fragile]{Proof of Concept $DP_{WCC}$}
  \begin{itemize}
    \setlength\itemsep{2em}
    \item \underline{\color{red}Robustness and suitability} of the \textbf{DP-Haskell}
    \item Ability to \underline{\color{red}generate incremental results} indicated by higher values of \textbf{$\mathtt{dief@t}$ metric}
    \item \underline{\color{red}Satisfactory performance results} regarding \textbf{Memory allocations and Execution times}. 
    \item Results \underline{\color{red}published} on \textbf{PROLE21 Conference}~\cite{prole:2021:017}
  \end{itemize}
\end{frame}



  \subsection{DP Haskell Framework}
  \begin{frame}[fragile]{DP Haskell Framework}
  \begin{center}
    \includegraphics[width = 0.8\textwidth, height = 0.8\textheight]{dpf_haskell_v3}
  \end{center}
\end{frame}

  \begin{frame}[fragile]{DP Haskell Framework}
    \begin{center}
    \includegraphics[width = 0.8\textwidth, height = 0.8\textheight]{dpf_haskell_v3-1}
  \end{center}
\end{frame}

\lstset{
  basicstyle=\itshape,
  literate={->}{$\rightarrow$}{2}
}

\begin{frame}[fragile]{DP Haskell Framework}
  \frametitle{DSL Grammar}
  \small
  \begin{equation*}
    \boxed{
     \begin{aligned}
    G_{dsl} = (N, \Sigma, DB, P)
    \end{aligned}
    }
\end{equation*}
\tiny
    \begin{equation*}
        \boxed{
         \begin{aligned}
        N &= \{DP,S_r,S_k,G,F_b,CH,CH_s\},\\
        \Sigma &= \{\text{\mintinline{haskell}{Source}},\text{\mintinline{haskell}{Generator}},\text{\mintinline{haskell}{Sink}},\text{\mintinline{haskell}{FeedbackChannel}},\text{\mintinline{haskell}{Type}},\text{\mintinline{haskell}{Eof}},\text{\mintinline{haskell}{:=>}},\text{\mintinline{haskell}{:<+>}}\},
        \end{aligned}
        }
    \end{equation*}
  \small
  \begin{equation*}
    \boxed{
      \begin{aligned}
    P = \{\\
    DP  &\rightarrow S_r\ \text{\mintinline{haskell}{:=>}}\ G\ \text{\mintinline{haskell}{:=>}}\ S_k\ |\ S_r\ \text{\mintinline{haskell}{:=>}}\ G\ \text{\mintinline{haskell}{:=>}}\ F_b\ \text{\mintinline{haskell}{:=>}}\ S_k,\\
    S_r &\rightarrow \text{\mintinline{haskell}{Source}}\ CH_s,\\
    G   &\rightarrow \text{\mintinline{haskell}{Generator}}\ CH_s,\\
    S_k &\rightarrow \text{\mintinline{haskell}{Sink}},\\
    F_b &\rightarrow \text{\mintinline{haskell}{FeedbackChannel}} CH,\\
    CH_s &\rightarrow \text{\mintinline{haskell}{Channel}}\ CH,\\
    CH &\rightarrow \text{\mintinline{haskell}{Type :<+>}}\ CH\ |\ \text{\mintinline{haskell}{Eof}}\}
  \end{aligned}
  }
  \end{equation*}
\end{frame}

\begin{frame}[fragile]{DP Haskell Framework}
  \begin{itemize}
    \item The specification of \textbf{DP} in the Language is compile time checked (\textit{Type-safe})
  \end{itemize}    
  \begin{exampleblock}{Type Level DSL}
    \begin{minted}[fontsize=\small,breaklines,highlightlines={7-17}]{shell}      
ghci> import DynamicPipeline
ghci> type DPExample = Source (Channel (Int :<+> Eof)) :=> Generator (Channel (Int :<+> Eof)) :=> Sink
type DPExample :: *
type DPExample =
Source (Channel (Int :<+> Eof))
:=> (Generator (Channel (Int :<+> Eof)) :=> Sink)
ghci> :t mkDP @DPExample
mkDP @DPExample
:: forall k (st :: k) filterState filterParam.
    Stage (WriteChannel Int -> DP st ())
    -> GeneratorStage DPExample filterState filterParam st
    -> Stage (ReadChannel Int -> DP st ())
    -> DP st ()    
  \end{minted}
  \end{exampleblock}
\end{frame}

\begin{frame}[fragile]{DP Haskell Framework}
  \begin{center}
    \includegraphics[width = 0.8\textwidth, height = 0.8\textheight]{dpf_haskell_v3-3}
  \end{center}
\end{frame}

\begin{frame}[fragile]{DP Haskell Framework}
  \begin{block}{Runtime System}
    \begin{itemize}
      \item \textbf{DP Monad}:
      \begin{itemize} 
      \item Monad with Existential Type to not escape DP Context. (\textit{Rank-2 Polymorphic type})
      \item Associativity Monad law guarantees execution flow (\mintinline{haskell}{ source >>= generator >>= sink }) 
      \end{itemize}
    \end{itemize}
  \end{block}
  \end{frame}

\begin{frame}[fragile]{DP Haskell Framework}
  \begin{block}{Runtime System}
    \begin{itemize}
      \item \textbf{DP Monad}:
      \begin{itemize} 
      \item Monad with Existential Type to not escape DP Context. (\textit{Rank-2 Polymorphic type})
      \item Associativity Monad law guarantees execution flow (\mintinline{haskell}{ source >>= generator >>= sink }) 
      \end{itemize}
      \item \textbf{Filter / Stage}: 
      \begin{itemize}
        \item Use of \mintinline{haskell}{unfold} to generate dynamic filter computations (\textit{Anamorphism})
        \item Use of \mintinline{haskell}{fold} to reduce results to Sink (\textit{Catamorphism})
      \end{itemize}
    \end{itemize}
  \end{block}
  \end{frame}

\begin{frame}[fragile]{DP Haskell Framework}
  \begin{block}{Runtime System}
    \begin{itemize}
      \item \textbf{DP Monad}:
      \begin{itemize} 
      \item Monad with Existential Type to not escape DP Context. (\textit{Rank-2 Polymorphic type})
      \item Associativity Monad law guarantees execution flow (\mintinline{haskell}{ source >>= generator >>= sink }) 
      \end{itemize}
      \item \textbf{Filter / Stage}: 
      \begin{itemize}
        \item Use of \mintinline{haskell}{unfold} to generate dynamic filter computations (\textit{Anamorphism})
        \item Use of \mintinline{haskell}{fold} to reduce results to Sink (\textit{Catamorphism})
      \end{itemize}
      \item \textbf{Multithreading}: \mintinline{shell}{async} library
    \end{itemize}
  \end{block}
  \end{frame}

\begin{frame}[fragile]{DP Haskell Framework}
\begin{block}{Runtime System}
  \begin{itemize}
    \item \textbf{DP Monad}:
    \begin{itemize} 
    \item Monad with Existential Type to not escape DP Context. (\textit{Rank-2 Polymorphic type})
    \item Associativity Monad law guarantees execution flow (\mintinline{haskell}{ source >>= generator >>= sink }) 
    \end{itemize}
  \item \textbf{Filter / Stage}: 
    \begin{itemize}
      \item Use of \mintinline{haskell}{unfold} to generate dynamic filter computations (\textit{Anamorphism})
      \item Use of \mintinline{haskell}{fold} to reduce results to Sink (\textit{Catamorphism})
    \end{itemize}
    \item \textbf{Multithreading}: \mintinline{shell}{async} library
    \item \textbf{Channels}: \mintinline{shell}{unagi-chan} library
  \end{itemize}
\end{block}
\end{frame}

\begin{frame}[fragile]{DP Haskell Framework}
  \begin{center}
    \includegraphics[width = 0.8\textwidth, height = 0.8\textheight]{dpf_haskell_v3-2}
  \end{center}
\end{frame}

\begin{frame}[fragile]{DP Haskell Framework}
  \begin{block}{IDL}
    \begin{minted}[fontsize=\small,breaklines]{shell}      
ghci> type DPExample = Source (Channel (Int :<+> Eof)) :=> Generator (Channel (Int :<+> Eof)) :=> Sink
type DPExample :: *
type DPExample =
Source (Channel (Int :<+> Eof))
:=> (Generator (Channel (Int :<+> Eof)) :=> Sink)
  \end{minted}
\end{block}
\end{frame}

\begin{frame}[fragile]{DP Haskell Framework}
  \begin{block}{IDL}
    \begin{minted}[fontsize=\small,breaklines,highlightlines={7-11}]{shell}      
ghci> type DPExample = Source (Channel (Int :<+> Eof)) :=> Generator (Channel (Int :<+> Eof)) :=> Sink
type DPExample :: *
type DPExample =
Source (Channel (Int :<+> Eof))
:=> (Generator (Channel (Int :<+> Eof)) :=> Sink)
      
ghci> :t withSource @DPExample
withSource @DPExample
:: forall k (st :: k).
    (WriteChannel Int -> DP st ())
    -> Stage (WriteChannel Int -> DP st ())
  \end{minted}
\end{block}
\end{frame}

\begin{frame}[fragile]{DP Haskell Framework}
  \begin{block}{IDL}
    \begin{minted}[fontsize=\small,breaklines,highlightlines={7-9}]{shell}      
ghci> :t withSource @DPExample
withSource @DPExample
:: forall k (st :: k).
    (WriteChannel Int -> DP st ())
    -> Stage (WriteChannel Int -> DP st ())
    
ghci> let source' = withSource @DPExample  $ \wc -> unfoldT ([1..10] <> [1..10]) wc identity
ghci> :t source'
source' :: forall k (st :: k). Stage (WriteChannel Int -> DP st ())
  \end{minted}
\end{block}
\end{frame}

\begin{frame}[fragile]{DP Haskell Framework}
  \begin{block}{IDL}
    \begin{minted}[fontsize=\small,breaklines,highlightlines={2-8}]{shell}      
ghci> :t withGenerator @DPExample
withGenerator @DPExample
:: forall k filter (st :: k).
    (filter -> ReadChannel Int -> WriteChannel Int -> DP st ())
    -> Stage
        (filter -> ReadChannel Int -> WriteChannel Int -> DP st ())    
  \end{minted}
\end{block}
\end{frame}

\begin{frame}[fragile]{DP Haskell Framework}
  \begin{block}{IDL}
    \begin{minted}[fontsize=\small,breaklines,highlightlines={2-8}]{shell}      
ghci> :t withGenerator @DPExample
withGenerator @DPExample
:: forall k filter (st :: k).
    (filter -> ReadChannel Int -> WriteChannel Int -> DP st ())
    -> Stage
        (filter -> ReadChannel Int -> WriteChannel Int -> DP st ())    
  \end{minted}
\end{block}
\begin{block}{Techniques}
  \begin{itemize}
    \item First Class Families
    \item Type-level Defunctionalization 
    \item Defunctionalization
    \item Associated Type Families
  \end{itemize}
\end{block}
\end{frame}

\begin{frame}[fragile]{DP Haskell Framework}
  \begin{block}{}
    Library released on Hackage \\
    https://hackage.haskell.org/package/dynamic-pipeline
    \begin{center}
      \includegraphics[width = 0.9\textwidth, height = 0.6\textheight]{dp-fw-hs}
    \end{center}  
  \end{block}
\end{frame}


  \subsection{Algorithm for Incrementally Enumerating BT in BG}
  \begin{frame}[fragile]{Algorithm for Incrementally Enumerating BT in BG}
  \begin{center}
  \large A \textbf{Bipartite Graph} is an undirected graph $G=(V,E)$  such that $V=(U\cup L)$, $U\cap L=\emptyset$ and $E\subseteq U\times L$.
  \end{center}          
  \begin{figure}
    \centering
    \resizebox{0.7\textwidth}{!}{\inputtikz{bipartite}}
  \end{figure}
\end{frame}

\begin{frame}[fragile]{Algorithm for Incrementally Enumerating BT in BG}
  \begin{center}
  Let the triples $\mu=(u_1, u_2, u_3)$ and $\ell=(l_1, l_2,l_3)$ on $U$ and $L$, respectively, i.e.  $\{u_1, u_2, u_3\} \subseteq U$, $\{l_1, l_2,l_3\} \subseteq L$. 
  The 6-cycle $(u_1,l_1,u_2,l_3,u_3,l_2,u_1)$  is a \textbf{Bitriangle} in $G$, denoted by $BT_{\ell}^{\mu} = \bti$. 
  \end{center}      
  \begin{figure}
    \centering
    \resizebox{0.7\textwidth}{!}{\inputtikz{bitriangle}}
  \end{figure}
\end{frame}

\begin{frame}[fragile]{Algorithm for Incrementally Enumerating BT in BG}
  \begin{center}
    \large We \textbf{introduce} a conceptual framework based on \textbf{compacted structures} that allows to specify the algorithm.
  \end{center}    
\end{frame}

\begin{frame}[fragile]{Algorithm for Incrementally Enumerating BT in BG}
  \begin{center}
    We build first \textbf{Aggregated Wedges}: a pair $\la l, W_l \ra$, where $l \in L$, $W_l \subseteq U$ and for all $u \in W_l$, the edge  $(u,l)\in E$
  \end{center}    
  \begin{figure}
    \centering
    \resizebox{0.8\textwidth}{!}{\inputtikz{bipartite_awg_a}\inputtikz{bipartite_awg_b}\inputtikz{bipartite_awg_c}}
  \end{figure}
\end{frame}

\begin{frame}[fragile]{Algorithm for Incrementally Enumerating BT in BG}
  \begin{center}
    Then, \textbf{Aggregated Double Wedges}: a pair  $\la (l_1, l_2), U_l \ra$, where $\{l_1,l_2\}\subseteq L$ and  for all $u_i \in I, u_j \in J$ and $u_k \in K$, $\{(u_i, l_1), (u_j, l_1), (u_j, l_2), (u_k, l_2)\} \in E$.
    $I \subseteq U, J \subseteq U$ and $K \subseteq U$, where $I, J$ and $K$ are disjoint sets. 
  \end{center}    
  \begin{figure}
    \centering
    \resizebox{0.6\textwidth}{!}{\inputtikz{bipartite_adwg_a}}
  \end{figure}
\end{frame}

\begin{frame}[fragile]{Algorithm for Incrementally Enumerating BT in BG}
  \begin{center}
    \textbf{Finally}, \textbf{Aggregated Bitriangles}: is a pair  $\langle \ell, \hat{U}_l\rangle$, 
    where $\ell=(l_1, l_2, l_3)$ is a triple on $L$, $l_1 < l_2 < l_3$ and for all $\la I, J, K\ra$ and for all $\mu=(u_i, u_j, u_k)$ such that $u_i \in I, u_j \in J, u_k \in K$, $BT_{\ell}^{\mu} \in \mathsf{BT}$.
  \end{center}    
  \begin{figure}
    \centering
    \resizebox{0.6\textwidth}{!}{\inputtikz{bipartite_abt_a}}
  \end{figure}
\end{frame}


\begin{frame}[fragile]{Algorithm for Incrementally Enumerating BT in BG}
  \begin{center}
    \textbf{Dynamic Pipeline} Configuration for Enumerating BT in BG
  \end{center}    
  \begin{figure}
    \centering  
    \resizebox{0.7\textwidth}{!}{\inputtikz{btDP}}
  \end{figure}
  \begin{figure}
    \centering  
    \resizebox{0.7\textwidth}{!}{\inputtikz{btDP_actor1}}
  \end{figure}
\end{frame}

\begin{frame}[fragile]{Algorithm for Incrementally Enumerating BT in BG}
  \begin{center}
    \large Lets see how the Actors' Filter solve this
  \end{center} 
  \begin{center}
  \resizebox{1\textwidth}{!}{  
  \begin{algorithm}[H]
    \SetKwInOut{P}{Filter Parameter}
    \SetKwInOut{FS}{Filter State}
    \SetKwInOut{IC}{Input Channels}
    \SetKwInOut{OC}{Output Channels}
    \SetKwFunction{actora}{actor1}
    \SetKwFunction{actorb}{actor2}
    \SetKwFunction{actorc}{actor3}
    \SetKwFunction{actord}{actor4}
    \SetKwFunction{filter}{filter}
    \SetKwProg{df}{def}{:}{end}
    \SetAlgorithmName{}{fil}{}
    \SetAlgoRefName{[A4]}
    \P{$l \in L$}
    \FS{$\st = \aw + \mathcal{P}(\dw) + \mathcal{P}(\at)$}
    \IC{$ IC = \la IC_E, IC_{W_l1}, IC_{W_l2}, IC_Q, IC_{BT} \ra $}
    \OC{$ OC = \la OC_E, OC_{W_l1}, OC_{W_l2}, OC_Q, OC_{BT} \ra $}
    \df{\filter{}}{
          $\actora()$\\
          $\actorb()$\\
          $\actorc()$\\
          $\actord()$\\
    }
  \end{algorithm}  
  }  
\end{center} 
\end{frame}

\begin{frame}[fragile]{Algorithm for Incrementally Enumerating BT in BG}
  \begin{center}
    \underline{\color{red}\textbf{\texttt{actor1}: Aggregated Wedges}}
  \end{center}
  \begin{center}
  \resizebox{!}{0.4\textheight}{ 
    \begin{algorithm}[H]
      \SetKwFunction{acta}{actor1}
      \SetKwProg{df}{def}{:}{end}
      \df{\acta{}}{
      $\la l, W_l \ra \leftarrow \gs$\\
      \ForAll{$(u',l') \in IC_E$}
      {\HiLi\uIf{$l = l'$}{
        \HiLi$W_l \leftarrow W_l \cup \{u'\}$
      }\Else{\HiLi$\p((u',l'),OC_E)$}
      }
      \If{$|W_l| > 1$}{
          \HiLi$\us(\la l, W_l \ra)$\\ 
          \HiLi$\p(\la l, W_l \ra, OC_{W_l1})$\\
      }
      }
      \end{algorithm}
  }  
\end{center}
\end{frame}

\begin{frame}[fragile]{Algorithm for Incrementally Enumerating BT in BG}
  \begin{center}
    \underline{\color{red}\textbf{\texttt{actor2}: Aggregated Double Wedges}}
  \end{center}
  \begin{center}
  \resizebox{!}{0.4\textheight}{  
    \begin{algorithm}[H]
      \SetKwFunction{actb}{actor2}
      \SetKwProg{df}{def}{:}{end}
      \BlankLine
      \df{\actb{}}{
      $\la l, W_l \ra \leftarrow \gs$\\
      \HiLi\ForAll{$\la l', W_l' \ra \in IC_{W_l1}$}{
        \HiLi$\p(\la l', W_l \ra, OC_{W_l1})$\\
            $\dwi \leftarrow \emptyset$\\
            \If{$W_l' \cap W_l \neq \emptyset$}{ 
                  $(l_l, l_u) \leftarrow (\argmin_{l,l'}, \argmax_{l,l'})$\\
                  \uIf{$l < l'$}{
                        $(W_{l_l}, W_{l_u}) \leftarrow (W_l, W_l')$
                  }\Else{
                        $(W_{l_l}, W_{l_u}) \leftarrow (W_l', W_l)$
                  }
                  $I \leftarrow W_{l_l} \setminus W_{l_u}$\\ 
                  $J \leftarrow W_{l_l} \cap W_{l_u}$\\
                  $K \leftarrow W_{l_u} \setminus W_{l_l}$\\
                  $U_l \leftarrow \la I, J, K\ra$\\
                  \HiLi$\dwi \leftarrow dw \cup \{\la (l_l, l_u), U_l \ra\}$
            }
      }
      \HiLi$\us(\dwi)$
      }
      \end{algorithm}
  }  
\end{center}
\end{frame}

\begin{frame}[fragile]{Algorithm for Incrementally Enumerating BT in BG}
  \begin{center}
    \underline{\color{red}\textbf{\texttt{actor3}: Aggregated Bitriangles}}
  \end{center}
  \begin{center}
  \resizebox{!}{0.4\textheight}{  
    \begin{algorithm}[H]
      \SetKwFunction{actc}{actor3}
      \SetKwProg{df}{def}{:}{end}
      \BlankLine
      \df{\actc{}}{
        $\dwi \leftarrow \gs$\\
        $\ati \leftarrow \emptyset$\\
        \HiLi\ForAll{$\la l', W_l \ra \in IC_{W_l2}$}{
          \HiLi$\p(\la l', W_l \ra, OC_{W_l2})$\\ 
          \HiLi\ForEach{$\la (l_l, l_u), \la I, J, K \ra \ra \in \dwi, l_l < l' \land l_u > l'$}{
                    $I' \leftarrow I \cup J$\\
                    $K' \leftarrow K \cup J$\\
                    \If{$W_l \cap I' \neq \emptyset \land W_l \cap K' \neq \emptyset$}{
                          $I' \leftarrow I' \cap W_l$\\
                          $K' \leftarrow K' \cap W_l$\\
                          $\hat{U}_l  \leftarrow \la I', J, K' \ra$\\
                          \HiLi$\ati \leftarrow \ati \cup \big\{\la (l_l, l', l_u), \hat{U}_l \ra\big\}$
                    }
              }
        }      
        \HiLi$\us(\ati)$\\
        }
        \end{algorithm}
  }  
\end{center}
\end{frame}

\begin{frame}[fragile]{Algorithm for Incrementally Enumerating BT in BG}
  \begin{center}
    \underline{\color{red}\textbf{\texttt{actor4}: Process Query to Incrementally retrieve Bitriangles}}
  \end{center}
  \begin{center}
  \resizebox{!}{0.4\textheight}{  
    \begin{algorithm}[H]
      \SetKwFunction{butV}{buildBtVertex}
      \SetKwFunction{butE}{buildBtEdge}      
      \SetKwFunction{actd}{actor4}
      \SetKwProg{df}{def}{:}{end}
      \BlankLine
      \df{\actd{}}{
        $\ati \leftarrow \gs$\\
        \ForAll{$Q \in IC_Q$}{
          \HiLi\ForEach{$\la (l_l, l_m, l_u), \la I,J,K \ra \ra \in \ati$}{
                    \Switch{$Q$}{
                          \Case{$\mathcal{P}(U + L)$}{
                                \If{$\mathcal{P}(U + L) \cap \{l_l, l_m, l_u\} \neq \emptyset \lor \mathcal{P}(U + L) \cap (I \cup J \cup K) \neq \emptyset$}{
                                      \HiLi$\btii \leftarrow \butV(\la (l_l, l_m, l_u), \la I,J,K \ra \ra), \mathcal{P}(U + L))$\\
                                      \HiLi\ForAll{$\bti \in \btii$}{
                                        \HiLi$\p(\bti, OC_{BT})$
                                      }
                                }
                          }
                          \Case{$\mathcal{P}(E)$}{
                              \tcp*[h]{SAME FOR EDGES......}\\
                          }
                    }
              }
        }
        }
      \end{algorithm}
  }  
\end{center}
\end{frame}


\begin{frame}[fragile]{Algorithm for Incrementally Enumerating BT in BG}
  \begin{block}{Correctness of the Algorithm}
    We need to prove that the algorithm \textbf{can enumerate all bitriangles} in the graph and also that there are \textbf{no duplicates in the enumeration}.
  \end{block}
\end{frame}


\begin{frame}[fragile]{Algorithm for Incrementally Enumerating BT in BG}
  \begin{theorem}[Uniqueness] 
    Given a bipartite graph $G = ((U\cup L),E)$, $\forall \btii\in\bt$  \acrshort{iebt} stores $\btii$ in an $\ati\in \at$ only once.
  \end{theorem}
  \vspace{1.5cm}
  \begin{theorem}[All Bitriangles can be enumerated]
    Given a bipartite graph $G$, if the bitriangle $\btii = (u_1,l_1,u_2,l_3,$ $u_3,l_2,u_1)\in \bt$, then $\btii$  can be enumerated.
  \end{theorem}  
\end{frame} 

\iffalse
\begin{frame}[fragile]{Algorithm for Incrementally Enumerating BT in BG}
  \begin{proof}[Proof. No Bitriangles are duplicated]   
  Let $\bti =$  $(u_1,l_1,u_2,l_3,u_3,l_2,u_1)$, such that $l_1 < l_2 <l_3$. For every feasible permutation  starting with a node in $U$ we are going to proof that only one will be accepted by $\ab$, or $\ac$.  
  \begin{itemize}
        \item According to lines 7-12 of $\ab$ $(u_1,l_2,u_3,l_3,u_2,l_1,u_1)$ is  not accepted  when constructing elements of $\dw$ to be used by $\ac$  because $l_2 > l_1$ 
        \item According to lines 7 of $\ac$ $(u_3,l_2,u_1,l_1,u_2,l_3,u_3)$ is not accepted because $l_2 > l_1$
        \item According to lines 7-12 of $\ab$ $(u_3,l_3,u_2,l_1,u_1,l_2,u_3)$ is  not accepted  when constructing elements of $\dw$ to be used by $\ac$  because  $l_3 > l_2$
        \item According to lines 7-12 of $\ab$ $(u_2,l_3,u_3,l_2,u_1,l_1,u_2)$ not accepted because $l_3 > l_1 $
  \end{itemize}
  \end{proof}  
\end{frame}

\begin{frame}[fragile]{Algorithm for Incrementally Enumerating BT in BG}
  \begin{proof}[Proof. (Cont.) No Bitriangles are duplicated]   
  Let $\bti =$  $(u_1,l_1,u_2,l_3,u_3,l_2,u_1)$, such that $l_1 < l_2 <l_3$. For every feasible permutation  starting with a node in $U$ we are going to proof that only one will be accepted by $\ab$, or $\ac$.  
  \begin{itemize}
        \item According to lines 7-12 of $\ab$ and line 7 in $\ac$ $(u_2,l_1,u_1,l_2,u_3,l_3,u_2)$ is accepted because $l_1 < l_3 $ and $l_1 < l_2 <l_3$ 
        \item According to  line 7 in $\ac$ $(u_1, l_1,u_2,l_3,u_3,l_2,u_1)$ is not accepted because $l_3 > l_2$
  \end{itemize}
  Therefore the  only bitriangle  constructed by the algorithm is the one that satisfies $l_1 < l_2 < l_3$ where $u_1,u_2,u_3$  are distinct and $u_1$ is incident to $l_1$ and $l_2$, $u_2$ is incident to $l_1$ and $l_3$ and $u_3$ is incident to $l_2$ and $u_3$.
  \end{proof}  
\end{frame}


\begin{frame}[fragile]{Algorithm for Incrementally Enumerating BT in BG}
  \begin{proof}[All Bitriangles can be enumerated] 
    \begin{itemize}
      \item Let $\bti =$  $(u_1,l_1,u_2,l_3,u_3,l_2,u_1)$, such that $l_1 < l_2 <l_3$ is present in the graph. 
      \item When $\aaa$ acting in filter $F_{l_1}$ ends reading all the edges, $\{u_1,u_2\} \subseteq W_{l_1}$. Also when $\aaa$ in filter $F_{l_3}$ ends reading all the edges, $\{u_2,u_3\} \subseteq W_{l_3}$. 
      \item As $l_1 < l_3$ in $\ab$, in lines 7-12, in filter $F_{l_1}$ or in filter $F_{l_3}$, the pair $(l_1,l_3)$  will be added to $\mathsf{dw}$.
      \item When $\ac$ in lines 7-16, in filter $F_{l_1}$ or in filter $F_{l_3}$ receives $(l_2, W_{l_2})$  will construct the corresponding  $BT$ because of the condition in line 10 is satisfied (non empty intersection). 
    \end{itemize}    
    Therefore, the bitriangle is stored in $\mathsf{bt}$ an thus $(u_1,l_1,u_2,l_3,u_3,l_2,u_1)$ can be listed by $\ad$.
  \end{proof}
  
\end{frame} 
\fi


  \begin{frame}{Agenda}
    \section{Empirical Evaluation}
    \tableofcontents[currentsection,hideothersubsections]
  \end{frame}

  \begin{frame}[fragile]{Empirical Evaluation}
  \begin{block}{Research Questions}
    \begin{itemize}
        \item Does \acrshort{dpbt} generate incremental results regardless of the size of the graph?
        \item Does the type of query $Q$ impact on the execution of \acrshort{dpbt}?
        \item How effectively \acrshort{dpbt} implements a \emph{pay-as-you-go} model?
        \item Does \acrshort{dpbt} handle memory and threads efficiently?
    \end{itemize}        
  \end{block}
\end{frame}

\begin{frame}[fragile]{Empirical Evaluation}
  \begin{block}{Research Questions}
    \begin{itemize}
        \item Does \acrshort{dpbt} generate incremental results regardless of the size of the graph?
        \item Does the type of query $Q$ impact on the execution of \acrshort{dpbt}?
        \item How effectively \acrshort{dpbt} implements a \emph{pay-as-you-go} model?
        \item Does \acrshort{dpbt} handle memory and threads efficiently?
    \end{itemize}        
  \end{block}
  \begin{block}{Experiments}
    \begin{itemize}
      \item \textbf{Continuous behavior Analysis}: using \acrshort{dt} and \acrshort{dk} to assess the continuous behavior capabilities.
      \item \textbf{Benchmark Analysis}: to identify how the behavior of \acrshort{dpbt} varies depending on the type of query command.
      \item \textbf{Performance Analysis} \acrfull{ghc} Profiling for one of the graphs to measure multithreading and memory allocation. 
    \end{itemize}
  \end{block}
\end{frame}

\begin{frame}[fragile]{Experiment Configuration}
  \begin{block}{Graphs Tested}
    \begin{table}[H]
      \centering
      \resizebox{1\textwidth}{!}{
      \begin{tabular}{|p{0.25\linewidth}|c|c|c|c|c|}
        \hline
       \textbf{Network} & \textbf{$|U|$} & \textbf{$|L|$} & \textbf{$|E|$} & \textbf{Wedges} & \textbf{\#\acrshort{bt}} \\
       \hline
       Dbpedia & 18422 & 168338 & 233286 & $1.45 \times 10^8$ & $3.62 \times 10^8$\\
       \hline
       Moreno Crime & 829 & 551 & 1476 & 4816 & 211\\
       \hline
       Opsahl UC Forum  & 899 & 522 & 33720 & 174069 & $2.2 \times 10^7$ \\
       \hline
       Wang Amazon & 26112 & 799 & 29062 & $3.4 \times 10^6$ & 110269\\
       \hline
      \end{tabular}
      }
     \end{table}
  \end{block}
  \begin{block}{Hardware Environment}
    \begin{itemize}
          \item \emph{HPC Cluster at UPC}
          \item $x86$ $64$ bits
          \item $24$-Core Intel(R) Xeon(R) CPU X5650 processor of $2.67$ GHz
          \item \emph{Hyper-threading} enable
          \item $40 GB$ up to $120 GB$ of RAM for the biggest \acrfull{dbpedia} graph
      \end{itemize}        
  \end{block}
\end{frame}

\begin{frame}[fragile]{Experiment Configuration}
  \begin{block}{Test Case Scenarios}
    \begin{table}[H]
      \centering
      \resizebox{1\textwidth}{!}{
        \begin{tabular}{|l|c|c|}
          \hline
          \textbf{Scenario ID} & \textbf{Name} & \textbf{Search by}\\
          \hline
          E-H & Edge High & edge with high incidence \\
          \hline
          E-L & Edge Low & edge with low incidence \\
          \hline
          E-M & Edge Medium & edge with medium incidence \\
          \hline
          VL-H & $l \in L$ High & vertex in lower layer with high incidence \\
          \hline
          VL-L & $l \in L$ Low & vertex in lower layer with low incidence \\
          \hline
          VL-M & $l \in L$ Medium & vertex in lower layer with medium incidence \\
          \hline
          VU-H & $u \in U$ High & vertex in upper layer with high incidence \\
          \hline
          VU-L & $u \in U$ Low & vertex in upper layer with low incidence \\
          \hline
          VU-M & $u \in U$ Medium & vertex in upper layer with medium incidence \\
          \hline
        \end{tabular}
    }
     \end{table}
  \end{block}
\end{frame}

\begin{frame}[fragile]{Experiment Configuration}
  \begin{block}{Selection of Query $Q$ values}
    \begin{itemize}
      \item Sort the vertices by its degree.
      \item Randomly select following a uniform distribution,  a vertex or edge depending on the scenario, from the subset of vertices or edges with the required incidence.
      \item Execution of a sample of experiments to check if that selections provides results or not. If not, the test case is eliminated.
    \end{itemize}    
  \end{block}
\end{frame}


\begin{frame}[fragile]{E1: Continuous Behavior: \acrshort{dt} and \acrshort{dk}}
  \begin{figure}[!htp]
    \centering
    \begin{subfigure}[t]{0.45\textwidth}
     \includegraphics[width=1\linewidth, height=0.4\textheight]{experiments/diepfy/dbpedia.png}
    \end{subfigure}\hfill
    \begin{subfigure}[t]{0.45\textwidth}
     \includegraphics[width=1\linewidth, height=0.4\textheight]{experiments/diepfy/moreno_crime.png}
    \end{subfigure}
    \vspace{0.5cm}
  
    \begin{subfigure}[t]{0.45\textwidth}
     \includegraphics[width=1\linewidth, height=0.4\textheight]{experiments/diepfy/opsahl-ucforum.png}
    \end{subfigure}\hfill
    \begin{subfigure}[t]{0.45\textwidth}
      \includegraphics[width=1\linewidth, height=0.4\textheight]{experiments/diepfy/wang-amazon.png}
     \end{subfigure}
   \end{figure}
  \end{frame}

  \begin{frame}[fragile]{E1: Continuous Behavior: \acrshort{dt} and \acrshort{dk}}
    \begin{figure}[!htp]
      \centering
      \begin{subfigure}[t]{0.45\textwidth}
      \includegraphics[width=1\linewidth, height=0.4\textheight]{experiments/diepfy/dbpedia_radial.png}
      \end{subfigure}\hfill
      \begin{subfigure}[t]{0.45\textwidth}
      \includegraphics[width=1\linewidth, height=0.4\textheight]{experiments/diepfy/moreno_crime_radial.png}
      \end{subfigure}
      \vspace{0.5cm}
      %
      \begin{subfigure}[t]{0.45\textwidth}
      \includegraphics[width=1\linewidth, height=0.4\textheight]{experiments/diepfy/opsahl-ucforum_radial.png}
      \end{subfigure}\hfill
      \begin{subfigure}[t]{0.45\textwidth}
        \includegraphics[width=1\linewidth, height=0.4\textheight]{experiments/diepfy/wang-amazon_radial.png}
      \end{subfigure}
    \end{figure}
    \end{frame}

  \begin{frame}[fragile]{E1: Continuous Behavior: \acrshort{dt} and \acrshort{dk}}
    \begin{table}[H]
      \centering
      \resizebox{1\textwidth}{!}{
      \begin{tabular}{|p{0.25\linewidth}|c|c|c|}
        \hline
       \textbf{Network} & \textbf{Scenario ID} & \textbf{\acrshort{dt} Metric}  & \textbf{\acrshort{dk} Metric}\\
       \hline
       \multirow{3}{*}{Moreno Crime}
        & \cellcolor{yellow!35}VU-H & \cellcolor{yellow!35}$6.05 \times 10^2$ & $0.00$\\
        & \cellcolor{blue!25}VL-H & \cellcolor{blue!25}$7.95 \times 10^3$ & $0.00$\\
        & \cellcolor{blue!25}E-H & \cellcolor{blue!25}$9.85 \times 10^3$ & $0.00$\\
        \hline
        \multirow{3}{*}{Dbpedia}
        & \cellcolor{yellow!35}VU-H & \cellcolor{yellow!35}$3.32 \times 10^{13}$ & $1.97 \times 10^5$\\
        & \cellcolor{blue!25}VL-H &  \cellcolor{blue!25}$1.81 \times 10^{14}$ & $2.34 \times 10^4$\\
        & \cellcolor{blue!25}E-H &  \cellcolor{blue!25}$1.75 \times 10^{13}$ & $3.28 \times 10^5$\\
        \hline
        \multirow{3}{*}{Opsahl UC Forum}
        & \cellcolor{blue!25}VU-H &  \cellcolor{blue!25}$1.99 \times 10^{12}$ & $1.27 \times 10^5$\\
        & \cellcolor{blue!25}VL-H &  \cellcolor{blue!25}$6.44 \times 10^{11}$ & $1.90 \times 10^5$\\
        & \cellcolor{yellow!35}E-H &  \cellcolor{yellow!35}$1.02 \times 10^{11}$ & $2.93 \times 10^5$\\
        \hline
        \multirow{3}{*}{Wang Amazon}
        & \cellcolor{yellow!35}VU-H & \cellcolor{yellow!35}$1.50 \times 10^7$ & $43.6$\\
        & \cellcolor{blue!25}VL-H & \cellcolor{blue!25}$2.24 \times 10^7$ & $63.1$\\
        & \cellcolor{blue!25}E-H & \cellcolor{blue!25}$8.06 \times 10^6$  & $42.3$\\
        \hline
      \end{tabular}
      }
     \end{table}
\end{frame}

\begin{frame}[fragile]{E2: Average Excution Time}
\begin{figure}[H]
  \begin{center}
     \includegraphics[width=1\textwidth]{experiments/bench_1}
  \end{center}
 \end{figure}
\end{frame}

\begin{frame}[fragile]{E2: Total Excution Time}
  \begin{figure}[H]
    \begin{center}
       \includegraphics[width=1\textwidth] {experiments/execution_time_by_experiments}
    \end{center}
  \end{figure}
\end{frame}

\begin{frame}[fragile]{E3: Performance Analysis - Multithreading}
  \begin{figure}[H]
    \begin{center}
      \includegraphics[width=1\textwidth]{experiments/thread/general_overview}
    \end{center}
  \end{figure}
\end{frame}

\begin{frame}[fragile]{E3: Performance Analysis - Memory Allocation}
  \begin{figure}[H]
    \begin{center}
      \includegraphics[width=0.8\textwidth] {experiments/mem/overview}
    \end{center}
  \end{figure}
\end{frame}


\begin{frame}[fragile]{Empirical Evaluation}
  \begin{block}{Conclusions}
    \begin{itemize}
      \item High values of the metric dief\@t indicates continuous behavior.
      \item Lower values of the metric dief\@k indicates continuous behavior.
      \item High values of the metrics \emph{Average Running Time} and \emph{Total Running Time} for scenarios which enumerates more bitriangles, are suggesting an effective implementation of a \emph{pay-as-you-go} model of \acrshort{dpbt}.
      \item Results captured by the \texttt{ThreadScope} tool indicating an even distribution of the threads among processors, showing efficient use of the parallel model.
      \item Results gathered by the \texttt{eventlog2html} tool suggesting that memory consumption is efficiently handled in the intermediate objects that \acrshort{dpfh}.
    \end{itemize} 
  \end{block}
\end{frame}


  \begin{frame}{Agenda}
    \section{Conclusions and Future Work}
    \tableofcontents[currentsection,hideothersubsections]
  \end{frame}

  \begin{frame}[fragile]{Conclusions and Future Work}
  \begin{block}{Conclusions}      
    \begin{itemize}
      \item \textbf{Dynamic Pipeline Paradigm} is a suitable computational model to build an Algorithm for Incrementally Enumerating Bitriangles in large Bipartite Networks. 
    \end{itemize}
  \end{block}
\end{frame}

\begin{frame}[fragile]{Conclusions and Future Work}
  \begin{block}{Conclusions}      
    \begin{itemize}
      \item {\color{light}\textbf{Dynamic Pipeline Paradigm} is a suitable computational model to build an Algorithm for Incrementally Enumerating Bitriangles in large Bipartite Networks. }
      \item \textbf{Haskell} has behaved accordingly and efficiently on implementing \textbf{Dynamic Pipeline Paradigm} and solving the Algorithm.
    \end{itemize}
  \end{block}
\end{frame}

\begin{frame}[fragile]{Conclusions and Future Work}
  \begin{block}{Conclusions}      
    \begin{itemize}
      \item {\color{light}\textbf{Dynamic Pipeline Paradigm} is a suitable computational model to build an Algorithm for Incrementally Enumerating Bitriangles in large Bipartite Networks. }
      \item {\color{light}\textbf{Haskell} has behaved accordingly and efficiently on implementing \textbf{Dynamic Pipeline Paradigm} and solving the Algorithm.}
      \item In the experimental analysis we have \textbf{empirically} shown that the designed \textbf{algorithm is incrementally generating Bitriangles} from large Networks like Dbpedia. 
    \end{itemize}
  \end{block}
\end{frame}

\begin{frame}[fragile]{Conclusions and Future Work}
  \begin{block}{Future Work}      
    \begin{itemize}
      \item Solve \textbf{memory consumption} for large graphs.
    \end{itemize}
  \end{block}
\end{frame}

\begin{frame}[fragile]{Conclusions and Future Work}
  \begin{block}{Future Work}      
    \begin{itemize}
      \item {\color{light}Solve \textbf{memory consumption} for large graphs.}
      \item Selection of \textbf{more efficient Data Structures} to process query commands faster
    \end{itemize}
  \end{block}
\end{frame}

\begin{frame}[fragile]{Conclusions and Future Work}
  \begin{block}{Future Work}      
    \begin{itemize}
      \item {\color{light}Solve \textbf{memory consumption} for large graphs.}
      \item {\color{light}Selection of \textbf{more efficient Data Structures} to process query commands faster}
      \item Implement a \textbf{distributed model} for filters, to take advantage of a distributed memory model.
    \end{itemize}
  \end{block}
\end{frame}

\begin{frame}[fragile]{Conclusions and Future Work}
  \begin{block}{Future Work}      
    \begin{itemize}
      \item {\color{light}Solve \textbf{memory consumption} for large graphs.}
      \item {\color{light}Selection of \textbf{efficient Data Structures} to process query commands faster}
      \item {\color{light}Implement a \textbf{distributed model} for filters, to take advantage of a distributed memory model.}
      \item \textbf{Improvements on Haskell Framework}: Stream processing and Memory footprint for Boxed Data Types.
    \end{itemize}
  \end{block}
\end{frame}


  \begin{frame}[allowframebreaks]
    \frametitle{References}
    \bibliographystyle{amsalpha}
    \bibliography{Report.bib}
  \end{frame}

  \begin{frame}
    \centering \Huge
    \emph{Questions?}
  \end{frame}  
  
  \iffalse
  \chapter{Appendix}
\section{Experiments}
\subsection{Benchmark Analysis}\label{app:exp:bench}
\begin{longtable}{|l|c|c|}
  \caption{$R^2$ goodness of the fit - Regression model}
  \label{table:app:exp:bench}\\
    \hline
   \textbf{Graph} & \textbf{Experiment} & \textbf{$R^2$}\\
   \hline
   opsahl-ucforum & VL-L & 0.954 \\
   \hline
   opsahl-ucforum & VL-M & 0.812 \\
   \hline
   opsahl-ucforum & VL-H & 0.999 \\
   \hline
   wang-amazon & VL-L & 0.992 \\
   \hline
   wang-amazon & VL-M & 0.933 \\
   \hline
   wang-amazon & VL-H & 0.849 \\
   \hline
   moreno-crime & VL-L & 0.999 \\
   \hline
   moreno-crime & VL-M & 0.997 \\
   \hline
   moreno-crime & VL-H & 0.97 \\
   \hline
   opsahl-ucforum & VU-L & 0.718 \\
   \hline
   opsahl-ucforum & VU-M & 0.972 \\
   \hline
   opsahl-ucforum & VU-H & 0.997 \\
   \hline
   wang-amazon & VU-L & 0.932 \\
   \hline
   wang-amazon & VU-M & 0.896 \\
   \hline
   wang-amazon & VU-H & 0.992 \\
   \hline
   moreno-crime & VU-L & 0.999 \\
   \hline
   moreno-crime & VU-M & 0.998 \\
   \hline
   moreno-crime & VU-H & 0.997 \\
   \hline
   opsahl-ucforum & E-L & 0.996 \\
   \hline
   opsahl-ucforum & E-M & 0.987 \\
   \hline
   opsahl-ucforum & E-H & 0.929 \\
   \hline
   wang-amazon & E-L & 0.954 \\
   \hline
   wang-amazon & E-M & 0.955 \\
   \hline
   wang-amazon & E-H & 0.979 \\
   \hline
   moreno-crime & E-L & 1.00 \\
   \hline
   moreno-crime & E-M & 1.00 \\
   \hline
   moreno-crime & E-H & 0.997 \\
   \hline
 \end{longtable}


  \fi

  \end{document}