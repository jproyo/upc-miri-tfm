\begin{frame}[fragile]{Conclusions and Future Work}
  \begin{block}{Conclusions}      
    \begin{itemize}
      \item \textbf{Dynamic Pipeline Paradigm} is a suitable computational model to build an Algorithm for Incrementally Enumerating Bitriangles in large Bipartite Networks. 
    \end{itemize}
  \end{block}
\end{frame}

\begin{frame}[fragile]{Conclusions and Future Work}
  \begin{block}{Conclusions}      
    \begin{itemize}
      \item {\color{light}\textbf{Dynamic Pipeline Paradigm} is a suitable computational model to build an Algorithm for Incrementally Enumerating Bitriangles in large Bipartite Networks. }
      \item \textbf{Haskell} has accordingly and efficiently implemented \textbf{Dynamic Pipeline Paradigm} for solving the Algorithm.
    \end{itemize}
  \end{block}
\end{frame}

\begin{frame}[fragile]{Conclusions and Future Work}
  \begin{block}{Conclusions}      
    \begin{itemize}
      \item {\color{light}\textbf{Dynamic Pipeline Paradigm} is a suitable computational model to build an Algorithm for Incrementally Enumerating Bitriangles in large Bipartite Networks. }
      \item {\color{light}\textbf{Haskell} has accordingly and efficiently implemented \textbf{Dynamic Pipeline Paradigm} for solving the Algorithm.}
      \item In the experimental analysis we have empirically shown that the designed algorithm is incrementally generating Bitriangles from large Networks like Dbpedia. 
    \end{itemize}
  \end{block}
\end{frame}

\begin{frame}[fragile]{Conclusions and Future Work}
  \begin{block}{Future Work}      
    \begin{itemize}
      \item Solve memory consumption for large graphs.
      \item Selection of efficient Data Structures to process query commands faster
      \item Distribution of Filters instance of Parallelization to take advantage of distributed memory model
      \item Improvement in Haskell Framework related to stream processing and memory footprint for Boxed Data Types.
    \end{itemize}
  \end{block}
\end{frame}
