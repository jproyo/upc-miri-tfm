\begin{figure}[h]
% \centering
\begin{subfigure}[b]{0.5\textwidth}
 \centering
 \begin{tikzpicture}
%Nodes
\node[ionode]      (in)                              {$\iwcc$};
\node[gennode]   (gen)                [right=of in] {$\gwcc$};
\node[ionode]      (out)                 [right=of gen] {$\owcc$};
\node[paramnode]  (filter)                [above= 0.2mm of gen] {$\fwcc$};
%Lines
\draw[ultra thick,->] (in)to (gen);
\draw[ultra thick,->] (gen) to (out);
\end{tikzpicture}
\caption{Initial configuration of a Dynamic Pipeline.  An initial DP consists of three stages: $\iwcc$, $\gwcc$ together its filter parameter $\fwcc$, and $\owcc$. These stages are connected through its channels --represented by right arrows-- as shown in this figure.}
\label{fig:initialDP}
\end{subfigure}
\hspace{0.3cm}
\begin{subfigure}[b]{0.5\textwidth}
 \centering
\begin{tikzpicture}
%Nodes
\node[ionode]      (in)                              {$\iwcc$};
\node[filternode]  (filter1)               [right=of in] {$\fwcc$};
\node[filternode]  (filter2)               [right=of filter1] {$\fwcc$};
\node[gennode]   (gen)                 [right=of filter2] {$\gwcc$};
\node[paramnode]  (param)          [above= 0.2mm of gen] {$\fwcc$};
\node[ionode]      (out)                  [right=of gen] {$\owcc$};
%Lines
\draw[ultra thick,->] (in) to (filter1);
\draw[ultra thick,->] (filter1) to (filter2);
\draw[ultra thick,->] (filter2) to (gen);
\draw[ultra thick,->] (gen) to (out);
\end{tikzpicture}
\caption{Evolution of a DP. After creating some filter instances (shadow Filter squares) of the filter parameter (light Filter square) in the Generator, the DP has stretched.}
\label{fig:activeDP}
\end{subfigure}
\caption{Dynamic Pipeline configuration}
\end{figure}
%%%
