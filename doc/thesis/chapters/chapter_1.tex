\chapter{Introduction}\label{intro}
This chapter presents the main motivation of our research, the problem statement that we gather from that motivation
and what is the proposed solution to that.

\section{Motivation}\label{sec:motivation}
A \acrfull{bg} is a graph with two disjoint vertex sets where its edges only connect vertices from different sets. 
For instance, two different sets of objects can be modeled with a \acrshort{bg}, establishing the relationships between them.
There are different use cases in which we can take advantage of a \acrshort{bg} representation and detect how the elements between the sets
are related. For example, coding theory~\cite{DBLP:journals/corr/WangL13} to represent the relationships between words in code or graphs in hypergraph theory~\cite{hypergraph}. 
For instance, an important field that can be modeled with \acrshort{bg} is pharmacology research, for establishing the relation between the drugs and side effects~\cite{drugs}.

For detecting underlying relations of the components of the different sets in a \acrshort{bg} there exist the same metrics as we have in \acrfull{ug} like clustering coefficient, social analysis or triangle-based community computation~\cite{ccoef,detect_graph,Newman_2003}.
The majority of those metrics in \acrshort{ug} are based on computing the number of triangles in the network. Obtaining those metrics on \acrshort{bg} requires to count \acrfull{bt}~\cite{opsahl} on \acrshort{bg}.
In that matter, there has been a novel work that provides efficient algorithms to count \acrshort{bt} in $P$~\cite{btcount}.

There exists a relationship in \acrshort{bg} that cannot be obtained only by counting \acrshort{bt} and requires knowing specific
structural details. One of those relations are motif-paths~\cite{Li2019MotifPA}, in particular, a \acrshort{bt} can be seen as a motif. To detect a path between different motifs, we could potentially enumerate \acrshort{bt} and check how they are connected.
Moreover a \acrshort{bt} enumeration algorithm could lead to a timeout if it is required to enumerate all the \acrshort{bt} in a \acrshort{bg} with millions of them. For overcoming this problem, users might want to query or request some \acrshort{bt} that follows certain criteria, reducing the search space. 
This leads us to the main motivation of this work which is to provide an \acrfull{iebt}. It should be \emph{incremental} because we want the user to be able to obtain \acrshort{bt} as long as the algorithm is capable of generate them. 
And, it should \emph{enumerate}, because it is suitable for a wide range of problems like motif-paths, in which the user needs to know the specific structure of the \acrshort{bt} as we stated before.

The main question that arise is \emph{how to solve} the \acrshort{iebt}. The answer and the propose of this work is to implement this using a \emph{Streaming Processing} model that can fullfil this requirements.
Streaming processing has given rise to new computation paradigms to provide effective and efficient data stream processing.
The most important features of these new paradigms are the exploitation of parallelism, the capacity to adapt execution schedulers, 
reconfigure computational structures, adjust the use of resources according to the characteristics of the input stream and produce incremental results. 
The \acrfull{dp} is a naturally functional approach to deal with stream processing. 
Now that we choose the paradigm model to implement a solution to the \acrshort{iebt} we finally need a proper tool for doing that. 
This fact encourages us to use \acrlong{hs}, a purely functional programming language, for \acrshort{dp}.

In conclusion, taking advantage of this computational model (\acrshort{dp}) we can design an \acrshort{iebt} using \acrshort{hs}.

\section{Problem Statement}
A \acrlong{bt} in a \acrlong{bg} is defined as a cycle with three vertices from one vertex set and three for the other. We can also say that it is formed by two connected wedges closed by an additional wedge~\cite{btcount}.
An example of what it is a \acrshort{bt} and what it is not, can be seen in \autoref{fig:bitriangle-example} and \autoref{fig:bitriangle-not}.

% \begin{figure}[ht]
% \centering	
% \inputtikz{bipartite}
% \caption[Example of \acrshort{bg}]{Example of \acrshort{bg}}
% \label{fig:bipartite-graph-example}
% \end{figure}

\begin{figure}[htp!]
\begin{subfigure}[b]{0.5\textwidth}
\centering
\inputtikz{bitriangle}
\caption{Example of \acrshort{bt} in \acrshort{bg}}
\label{fig:bitriangle-example}
\end{subfigure}
\begin{subfigure}[b]{0.5\textwidth}
\centering
\inputtikz{not-bitriangle}
\caption{Not a \acrshort{bt} in \acrshort{bg}}
\label{fig:bitriangle-not}
\end{subfigure}
\caption[Example of \acrshort{bt} in \acrshort{bg}]{Example of \acrshort{bt} in \acrshort{bg}}
\end{figure}

As we have stated in the previous \autoref{sec:motivation}, enumerating all possible \acrshort{bt} it is not only computationally hard but, in general, only partial results are needed.
In that sense, we can provide a query-based command to search all \acrshort{bt} that matches the query criteria and incrementally deliver results to the user.
In this work, we are going to provide an \acrfull{iebt}, in which given a query command and a \acrlong{bg} incrementally returns all the \acrlong{bt} that matches the search criteria.

\section{Proposed Solution}
The solution proposed in this work is to implement an \acrfull{iebt} using \acrfull{dp} implemented in \acrfull{hs}.
In order to achieve that goal, we first conduct a proof of concept to assess the feasibility of using \acrshort{hs} for implementing an algorithm with the \acrshort{dp}.
In that assessment, we work on solving the problem of \acrfull{wcc} of a graph. Then, we develop a \acrlong{dpf} written in \acrlong{hs} that could help to implement any algorithm using \acrshort{dp}.
Following that, we provide the formal definition of \acrshort{iebt} using \acrshort{dp}, in a pseudo-code format. 
Finally, we provide correctness proof of the algorithm plus the implementation using the \acrshort{dpf} in \acrshort{hs}, that we call \acrfull{dpbt}.

\section{Contribution}\label{sec:contrib}
The main contribution of this work was to implement, formalize and empirically evaluate a novelty \acrlong{iebt} using \acrlong{dp} written in \acrshort{hs}.
We believe that this implementation is a step forward in the field, and we think our approach opens new research lines and improvements to be addressed in the future. 

In order to assess the feasibility of \acrshort{dp} implemented in \acrshort{hs}, we made a proof of concept solving \acrfull{wcc} problem of a Graph 
using \acrshort{dp} with \acrshort{hs}. We have also empirically evaluated this implementation with interesting results that we are going to cover in \autoref{prole}. 
This proof of concept and its results has been presented and accepted in the \acrfull{prole21} Conference~\cite{prole21}. 

Finally, and as a result of the proof of concept work, was the development and publication of a \acrshort{hs} framework called \mintinline{shell}{dynamic-pipeline}~\cite{dynamic-pipeline}. 
The framework was published on \acrfull{hack}~\cite{hackage} on 2021 June 17th in its first version,
providing to \acrshort{hs} community the ability to build algorithms using \acrshort{dp}. This is a novel contribution since it is the first library published on \acrshort{hs}
that implements \acrshort{dp}.

\section{Document Overview}
This document is organized as follows. In \autoref{prelim} we present and describe all the previous work that has been done on the related fields of Streaming Processing, Dynamic Pipeline Paradigm, and Streaming processing related to \acrshort{hs}. 
Following that, in \autoref{relate-work} we describe the state of the art related \acrshort{bt} in \acrshort{bg} and incremental processing algorithms in general.
In \autoref{prole} we show and describe with some level of details, an overview to the \acrshort{prole21} as well as the details of the results obtained. 
Continuing with that, in \autoref{dp-hs} we deeply describe the \acrshort{dpf} written in \acrshort{hs}, the design of the framework, and all the used techniques for the implementation.
After that chapter, we finally arrive at \autoref{incr-algo-bt-dp} where we focus on the main problem of this work. There we provide all the details related to the incremental algorithm for enumerating \acrshort{bt} in \acrshort{bg}, its pseudo-code,
correctness proof and \acrshort{hs} implementation using the framework.
In \autoref{experiments} we describe the experimental analysis conducted to assert our assumptions and answer our research questions.
Finally, at the end of the document in \autoref{conclusions}, we present the future work and the conclusions obtained.

\section{Chapter Summary}
In this chapter, we have shown the motivation of this work, as well as the problem statement related to that motivation and the proposed solution to that problem.
We have also described how is going to be the organization of all the documents to facilitate the reader review.
