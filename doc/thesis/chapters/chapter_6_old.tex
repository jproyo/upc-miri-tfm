\chapter[An \acrshort{iebt} using DP]{An algorithm for incrementally enumerating Bitriangles using DP}\label{incr-algo-bt-dp}
In this chapter, we first describe the overall idea of the algorithm for incrementally enumerate 
\acrfull{bt}. After that, we present the algorithms in pseudo-code format and then a complexity analysis of the principal functions of the algorithm.
In the end, we provide the proof of correctness and the implementation using the \acrshort{dpfh}.  

\section{Preliminaries Definitions}\label{sec:prem:def}
In order to understand how the algorithm works, we need to provide some basic definitions that are already exposed in previous work~\cite{btcount}, and
at the same time, we define other structures that we use for giving a solution to the problem.
Let's enumerate all those definitions in the following paragraphs.

\begin{definition}[\acrlong{bg}] 
A \acrfull{bg} is an undirected graph $G=(V,E)$  such that $V=(U\cup L)$, $U\cap L=\emptyset$ and $E\subseteq U\times L$.\cite{Bondy1976}
\end{definition}

Additionally, without loss of generality, we assume that  $U\subseteq \mathbb{N}$ and $L\subseteq \mathbb{N}$. Consequently, $(U,<)$  and $(L,<)$ are strict total orders. We can see an example of a \acrshort{bg}
in \autoref{fig:bipartite-graph-example}.

\begin{figure}[ht]
\centering	
\inputtikz{bipartite}
\caption[{[\acrshort{iebt}] Example of \acrlong{bg}}]{Example of \acrlong{bg} which has $5$ (five) \acrshort{bt}. We are going to use this example to show how to build the different objects definitions presented in \autoref{sec:prem:def}}
\label{fig:bipartite-graph-example}
\end{figure}

As we can see in \autoref{fig:bipartite-graph-example}, trying to identify the different \acrshort{bt} in this graph, it can be done manually, but in larger graphs can be more challenging.
The minimal structure that we can extract from this \acrshort{bt} is \acrlong{awg}, but for that, we need to define what is a \acrfull{wg}.

\begin{definition}[\acrfull{wg}]\label{def:wg}
A \textit{\acrfull{wg}} in $G$ is a triple $(u_1,l,u_2), \{u_1,u_2\}\subseteq U$, $l \in L$ and $\{(u_1,l),$ $(l,u_2)\} \subseteq E$. The vertex $l$ is the middle vertex of the wedge. 
\end{definition}
With this \dref{def:wg} in place, we can now define a \acrshort{awg}, which is an aggregated form of all the wedges of one $l \in L$.
      
\begin{definition}[\acrfull{awg}]\label{def:awg}
An \textit{\acrfull{awg}} is a pair $\la l, W_l \ra$, where $l \in L$, $W_l \subseteq U$ and for all $u\in W_l$, the edge  $(u,l)\in E$. 
\end{definition}

\begin{figure}[htp!]
\begin{subfigure}[b]{0.3\textwidth}
\centering
\inputtikz{bipartite_awg_a}
\caption{\acrshort{awg} on $a$}
\label{fig:awedge-example-a}
\end{subfigure}
\begin{subfigure}[b]{0.3\textwidth}
\centering
\inputtikz{bipartite_awg_b}
\caption{\acrshort{awg} on $b$}
\label{fig:awedge-example-b}
\end{subfigure}
\begin{subfigure}[b]{0.3\textwidth}
\centering
\inputtikz{bipartite_awg_c}
\caption{\acrshort{awg} on $c$}
\label{fig:awedge-example-c}
\end{subfigure}
\caption[{[\acrshort{iebt}] Definitions Examples \acrlong{awg}}]{ \acrlong{awg} of nodes $a,b,c$ of \autoref{fig:bipartite-graph-example}}
\label{fig:awedge-example}
\end{figure}

As we can see on \autoref{fig:awedge-example}, we are only enumerating the first three lower layer vertices \acrshort{awg}.
Following that, now we can define a more refined structure that will allow us to enrich the \acrshort{awg} in order to lead us to a \acrshort{bt}. 
That intermediate structure is an \acrlong{adwg}. For defining \acrshort{adwg} we need also another structure called \acrlong{dwg}. Intuitively is similar to \acrshort{wg} and \acrshort{awg} but relating two vertices in the Lower Layer $L$.

\begin{definition}[\acrfull{dwg}]
A \textit{\acrfull{dwg}} in $G$ is a path of length 4 $(u_1,l_1,u_2,l_2,u_3)$ where  $\{u_1,$ $u_2,u_3\}\subseteq U$ and $\{l_1,l_2\}\subseteq L$. Vertices $l_1$ and $l_2$ are the middle vertices of \acrshort{dwg}. 
\end{definition}
      
\begin{definition}[\acrfull{adwg}]\label{def:adwg}
Let $U_l = \la I, J, K\ra$ be a triplet such that $I \subseteq U, J \subseteq U$ and $K \subseteq U$, where $I, J$ and $K$ are disjoint sets. 
An \textit{\acrfull{adwg}}  is a pair  $\la (l_1, l_2), U_l \ra$, where $\{l_1,l_2\}\subseteq L$ and  for all $u_i \in I, u_j \in J$ and $u_k \in K$, $\{(u_i, l_1), (u_j, l_1), (u_j, l_2), (u_k, l_2)\} \in E$.
\end{definition}
      
\begin{figure}[htp!]
\centering
\inputtikz{bipartite_adwg_a}
\caption[{[\acrshort{iebt}] Example Aggregated double-wedge}]{\acrlong{adwg} of the pair of nodes $(a,c)$ of \autoref{fig:bipartite-graph-example}. Upper layer nodes $1,3,7$ and $5,9$ are enclosed in a Square indicating the set that they belong to. Remember in \dref{def:adwg} we are forming three sets $U_l = \la I, J, K \ra$. $I = \emptyset$ in ths case because they should be disjointed sets and $a$ and $c$ share the same upper layer nodes in $J$.}
\label{fig:agg-double-wedge-example}
\end{figure}

In \autoref{fig:agg-double-wedge-example} we can see the \acrshort{adwg} built by $a$ and $c$ where according to \dref{def:adwg} $a = l_1$ and $c = l_2$.
The last aggregated structure that will finally enable the algorithm to build a \acrshort{bt} is an \acrlong{abt}. In \acrshort{abt} we need some other structure definitions before to understand how it is built, but intuitively it is an aggregation of a \acrshort{adwg} with another new \acrshort{awg} that is related with the former structure.

\begin{definition}[\acrfull{awgc}]\label{def:awgc}
Let $\la (l_1, l_2), U_l \ra$ be an \acrshort{adwg}, where $\{l_1,l_2\}\subseteq L$ and $U_l = \la I, J, K\ra$.
An \acrshort{awg} $\la l, W_l \ra$ is an \textit{\acrfull{awgc}} \emph{if and only if} $l > l_1$ and $l < l_2$ and $W_l \cap (I \cup J) \neq \emptyset$ and $W_l \cap (K \cup J) \neq \emptyset$
\end{definition}
      
\begin{definition}[\acrfull{abt}]\label{def:abt}
Let $\hat{U}_l=\la I, J, K\ra$, such that $I \subseteq U, J \subseteq U, K \subseteq U$. An \textit{\acrfull{abt}}  is a pair  $\langle \ell, \hat{U}_l\rangle$, 
where $\ell=(l_1, l_2, l_3)$ is a triple on $L$, $l_1 < l_2 < l_3$ and for all $\la I, J, K\ra$ and for all $\mu=(u_i, u_j, u_k)$ such that $u_i \in I, u_j \in J, u_k \in K$, $BT_{\ell}^{\mu}\in \bt$.
\end{definition}

\begin{figure}[h!]
\centering      
\inputtikz{bipartite_abt_a}
\caption[{[\acrshort{iebt}] Example Aggregated bi-triangle}]{Aggregated bi-triangle of the triple $(a,b,c)$ of \autoref{fig:bipartite-graph-example}. In this case as the \dref{def:abt} states, $\hat{U}_l$ does not contains disjoint sets, because we are using a compressing mechanism similar as described in \cite{Lai}. We can appreciate in the image that we can built an \acrshort{abt} with $b$ which has incidents edges in $1,5$. Note that not all of combinations are going to form a \acrshort{bt}.}
\label{fig:agg-bt-example}
\end{figure}
      
As we can see in \autoref{fig:agg-bt-example}, we have all the structures at our disposal to build a \acrshort{bt}. Now we are in condition to define a \acrshort{bt}.

\begin{definition}[\acrlong{bt}]\label{def:bt}
Let the triples $\mu=(u_1, u_2, u_3)$ and $\ell=(l_1, l_2,l_3)$ on $U$ and $L$, respectively, i.e.  $\{u_1, u_2, u_3\} \subseteq U$, $\{l_1, l_2,l_3\} \subseteq L$. 
The 6-cycle $(u_1,l_1,u_2,l_3,u_3,l_2,u_1)$  is a \textit{\acrfull{bt}} in $G$, denoted by $BT_{\ell}^{\mu} = \bti$. 
\end{definition}      

\begin{figure}[h!]
\centering      
\inputtikz{bipartite_bt_a}
\caption[{[\acrshort{iebt}] Example of bi-triangle}]{Bi-triangle of the triple $(a,b,c)$ of \autoref{fig:bipartite-graph-example}. In this case the bi-triangle has been built from the \autoref{fig:agg-bt-example} which is the only possible bi-triangle that can be extracted from there.}
\label{fig:bt-a-example}
\end{figure}

As we can see in \autoref{fig:bt-a-example}, the \acrshort{bt} $(1,a,3,c,5,b,1)$ is the only possible one that can be extracted from the \acrshort{abt} presented in \autoref{fig:agg-bt-example}.

Because \acrshort{iebt} will allow to enumerate \acrshort{bt} using a \emph{pay-as-you-go} model, we need to allow the user enumerates specific \acrshort{bt} according to some criteria.
In regards to that, we are going to define a Query Command that is going to be used in the pseudo-algorithm for that purpose.

\begin{definition}[\acrfull{qo}]\label{def:query:match} 
$Q$ is a \emph{\acrfull{qo}} is a value from the  \textit{sum type}
$\mathbb{P}(U + L) + \mathbb{P}(E)$ producing as a result a possible set of \acrshort{bt} that include any of the vertices or edges given in $Q$.
\end{definition}

Finally, we can elaborate a summary of all the definitions and terms in the \autoref{table:notation}.

\begin{table}[!ht]
\centering
\begin{tabular}{|c|l|} \hline
\textbf{Notation} & \textbf{Meaning}\\ \hline
$G=((U\cup L),E)$ & a bipartite graph\\  \hline
$n,m$ & the number of vertices and edges in $G$, resp.\\  \hline
$(u,l)$ & an edge between vertices $u$ and $l$\\  \hline
$(u_1,l,u_2)$ & a wedge with  middle vertex $l$\\  \hline
$\la l, W_l \ra$ & an \acrshort{awg}\\  \hline
$\aw$ & the set of all the possible \acrshort{awg} in $G$\\  \hline
$(u_1,l_1,u_2,l_2,u_3)$ & a \acrshort{dwg} with middle vertices $l_1$ and $l_2$\\  \hline 
$\la (l_1, l_2), U_l \ra$ & an \acrshort{adwg}\\  \hline
$\dw$ & the set of all the possible \acrshort{adwg} in $G$\\  \hline
$\dwi$ & Subset of $\dw$, such that $\dwi \subseteq \dw$ \\  \hline
$\la (l_1, l_2,l_3), \hat{U}_l \ra$ & an \acrshort{abt}\\  \hline
$\at$ & the set of all the possible \acrshort{abt} in $G$ \\  \hline
$\ati$ & Subset of $\at$, such that $\ati \subseteq \at$ \\  \hline 
$\bti$ & the \acrshort{bt} $u_1,l_1,u_2,l_3,u_3,l_2,u_1$\\  \hline
$\bt$ & the set of all the possible \acrshort{bt} in $G$ \\  \hline
$bt$ & Subset of $\bt$, such that $bt \subseteq \bt$ \\  \hline
\end{tabular}
\caption[{[\acrshort{iebt}] Summary of notations and their meanings}]{This table summarizes all the different object definitions that we have detailed in \autoref{sec:prem:def}. The first column describe the formal term used in the definitions on \autoref{sec:prem:def}. The second column summarize the meaning of each term}
\label{table:notation}
\end{table}
      
\subsection{Algorithm Sketch}\label{sub:sec:algo-sketch}
Dynamic Pipeline Algorithm for Enumerating Bi-triangles ($\dpbt$) is defined in terms of the behavior of its four kinds stages: \textit{Source} ($\ibt$),  
\textit{Generator} ($\gbt$),  \textit{Sink} ($\obt$), and \textit{Filter}($\fbt$) stages. 
The algorithm considers a \acrlong{bt} as a convenient composition of three wedges as we can see the example in \autoref{fig:bitriangle-example}.
In order to reduce memory footprint, the algorithm aggregates results, i.e. the set of wedges having the same middle vertex is represented as a pair $\la l, W_l \ra$ where $l$ is the middle vertex and $W_l$ is the set of adjacent vertices of $l$ called \acrfull{awg} (see \dref{def:awg}).
The algorithm first collects \acrshort{awg} for every vertex in the $L$ set of the graph. Afterward it constructs \acrfull{adwg} for every pair of distinct vertices. Finally constructs \acrshort{abt}  for selected triples of vertices. 
The following \autoref{table:channels} describes the different channels that are connecting the stages in $\dpbt$.

\begin{table}[ht!]
\centering
\begin{tabular}{|p{0.2\linewidth}|p{0.8\linewidth}|} \hline
\textbf{Channel} & \textbf{Meaning}\\ \hline
$C = \la IC, OC \ra$ & A Channel pair that connects input and output channel\\ \hline
$OC$ & Set of Output Channels \\ \hline
$C_E$ & Channel of $e \in E$ \\ \hline
$IC_E$ & Input Channels carrying $e \in E$ \\ \hline
$OC_E$ & Output Channels carrying $e \in E$ \\ \hline
$C_{W_l1}$ & Channel of \acrshort{awg} \\ \hline
$IC_{W_l1}$ & Input Channels carrying \acrshort{awg} \\ \hline
$OC_{W_l1}$ & Output Channels carrying \acrshort{awg} \\ \hline
$C_{W_l2}$ & Channel of \acrshort{awg} \\ \hline
$IC_{W_l2}$ & Input Channels carrying \acrshort{awg} \\ \hline
$OC_{W_l2}$ & Output Channels carrying \acrshort{awg} \\ \hline
$C_Q$ & Channel of \acrshort{qo} \\ \hline
$IC_Q$ & Input Channels carrying \acrshort{qo} \\ \hline
$OC_Q$ & Output Channels carrying \acrshort{qo} \\ \hline
$C_{BT}$ & Channel of $\bt$ \\ \hline
$IC_{BT}$ & Input Channels carrying $\bt$ \\ \hline
$OC_{BT}$ & Output Channels carrying $\bt$ \\ \hline
\end{tabular}
\caption[{[\acrshort{iebt}] Summary of Channels used in \acrshort{dpbt}}]{Summary of Channels used in \acrshort{dpbt}. The subindex on the Channel name indicates the element type this channel is carrying, either producing or consuming. Channels prefixed with $C$ are a generic form of denominating a channel independently of it is a producing or consuming. Channels prefixed with $IC$ are Input Channels or Consumers. Channels prefixed with $OC$ are Output Channels or Producers.}
\label{table:channels}
\end{table}

\begin{definition}[Filter State]
Let $\aw$ be a set of all possibles \acrlong{awg} in $G$.
Let $\dw$ be a set of all possibles \acrlong{adwg} in $G$.
Let $\at$ be a set of all possibles \acrlong{abt} in $G$.
An \textit{filter state} of \acrshort{dp} Filter with parameter $l$ has a sum type $\st = \la l, W_l \ra + \dwi + \ati$, such that $\la l, W_l \ra \in \aw, \dwi \subseteq \dw, \ati \subseteq \at$.
\end{definition}
 
\begin{figure}[h]
\centering  
\resizebox{1\textwidth}{!}{%
\inputtikz{btDP}
}
\caption[{[\acrshort{iebt}] $\dpbt$ Initial setup}]{Example of an initial setup of $\dpbt$. The text between dotted lines indicates the incoming data from an external source such as a file, socket, or any other. There are two incoming sources because one of them is carrying the edges and the other the commands \acrshort{qo}. Thick black double lines at the most right indicate the output targe that can be file, socket, screen, or any other. There is no data there because it is the initial step of \acrshort{dpbt}. At this initial step only $\ibt$, $\gbt$, and $\obt$ are set up with initial channels.}
\label{fig:btDP}
\end{figure}

\paragraph{Sketch} The setup of \acrshort{dp} can be appreciated in \autoref{fig:btDP}. $\ibt$ reads from input stream all $(u,l) \in E$ in an incremental non-blocking manner and transfers to the following stage each $(u,l)$ using $C_E$.
For every $(u,l)$ that arrives to $\gbt$, a new $\fbt$ instance with parameter $l \in L$ is spawn. $\fbt$ contains four actors. 
First, $\aaa$ receives from $IC_E$ the edges and builds \acrshort{awg}, when there are no more edges it downstreams it to it's neighbour $\fbt$ using $OC_{W1}$. 
Then $\ab$ receives from $IC_{W1}$ aggregated wedges from previous $\fbt$, downstream to next $\fbt$ and $\gbt$ using $OC_{W1}$ and at the same time use its information to see 
if it can build an \acrshort{adwg}. If it can, it will store in its $\st$ those $dw \in \dw$.

$\ab$ downstream the \acrshort{awg} to  it's neighbour $\fbt$ or to $\gbt$ if it is the last $\fbt$. This is because we need to retro feed all the pipeline with all the \acrshort{awg}, in order to find an \acrfull{awgc} as we defined in \dref{def:awgc}.
From retro feeding channel $IC_{W2}$, $\ac$ receives all the \acrshort{awg}  and builds \acrshort{abt} storing them in $\st$. 

$Q$ Commands are downstream from $\ibt$ to $\ad$ using channel $IC_Q$. $\ad$ receives all the commands, and for each of them, if there is match according to \dref{def:query:match}, it enumerates those \acrshort{bt} extracted from $\st$ and downstream to $\obt$ through $OC_{BT}$.

\subsection{Dynamic Pipeline for Enumerating Bi-triangles}
In this section, we present all the pseudo-code definitions of each of the stages described in ~\autoref{sub:sec:algo-sketch}. 
In spite of this work has been implemented using \acrshort{hs}, and in particular \acrshort{dpfh}, the pseudo-code algorithms presented here are language independent.
Before starting with the details, we introduce in \autoref{table:aux:fn} auxiliary functions that we use in the pseudo-code to help understanding better the desired behavior of the algorithm.

\begin{table}[!htp]
\centering
\begin{tabular}{|p{0.3\linewidth}|p{0.7\linewidth}|} \hline
\textbf{Function} & \textbf{Meaning}\\ \hline
$\sw(F,l,\st)$ & Spawn new filter instance with parameters $F$ (Filter template), $l \in L$ and $\st$ as the State of the Filter\\\hline
$\fd$ & Kill this filter instance because PostCondition is not satisfied\\ \hline
$\fid$ & State after calling $\fd$ on filter. Indicates if Filter is die or not. If it is dead, this filter instance does not participate anymore in the pipeline streaming processor\\ \hline
$\gs$ & Get Current State $\st$ for Filter Instance \\ \hline
$\us(\st)$ & Update Current State $\st$ for Filter Instance \\ \hline
$\p(\mathtt{v}, OC_x)$ & push some value $\mathtt{v}$ to some Output Channel $OC_x$ \\\hline
$\mt(Q, BT)$ & Check if a \acrshort{qo} $Q$ produces $\bt$ \\ \hline
\end{tabular}
\caption[{[\acrshort{iebt}] Summary of auxiliary functions for handling \acrshort{dpbt} internals}]{This table shows some auxiliary functions that are used to help understanding the pseudo-code general behavior. Depending on the language implementation chosen these functions might not exists at all, but we generalize here in order to describe a pseudo-code language-independent. For example in our \acrshort{hs} implementation there is not $\fd$ because functions in \acrshort{hs} clean after execution finish automatically by \acrshort{ghc}}
\label{table:aux:fn}
\end{table}
      
\begin{algorithm}
\SetKwInOut{P}{Input Data}
\SetKwInOut{Q}{Input Commands}
\SetKwInOut{IC}{Input Channels}
\SetKwInOut{OC}{Output Channels}
\SetAlgorithmName{}{src}{}
\SetAlgoRefName{[A1]}
\P{$IO_E$: File or Input Stream with Set of Edges $E$}
\Q{$IO_Q$: File or Input Stream with \acrlong{qo} $Q$}
\IC{$IC = \la IC_{W_l2} \ra$}
\OC{$ OC = \la OC_E, OC_{W_l1}, OC_{W_l2}, OC_Q, OC_{BT} \ra $}
\ForAll(\tcp*[f]{Edges to Generator/Filter}){$(u,l) \in IO_E$}
{$\p((u,l), OC_E)$ \label{algo:source:1}
}
\ForAll(\tcp*[f]{Feedback from Generator to Filter}){$\la l, W_l \ra \in IC_{W_l2}$}
{$\p(\la l, W_l \ra, OC_{W_l2})$ \label{algo:source:2}
}
\ForAll(\tcp*[f]{Send Query Commands}){$Q \in IO_Q$}
{$\p(Q, OC_Q)$ \label{algo:source:3}
}
\caption{Source ($\ibt$)}
\label{algo:source}
\end{algorithm}

\paragraph{Source $\ibt$} In \autoref{algo:source} we can see in \autoref{algo:source:1} how the edges arriving from the input stream of the graph are downstream to the pipeline. 
Another important part as well is \autoref{algo:source:2}, the $\ibt$ is retro feed with \acrshort{awg} stream that is generated during pipeline execution.
This is important to finally build the \acrshort{bt} as we have describe in \autoref{sub:sec:algo-sketch}. Finally, \autoref{algo:source:3} shows how all the queries are downstream as well.

\begin{algorithm}
\SetKwInOut{P}{Filter Template}
\SetKwInOut{IC}{Input Channels}
\SetKwInOut{OC}{Output Channels}
\SetAlgorithmName{}{gen}{}
\SetAlgoRefName{[A2]}
\P{$F$}
\IC{$ IC = \la IC_E, IC_{W_l1}, IC_{W_l2}, IC_Q, IC_{BT} \ra $}
\OC{$ OC = \la OC_{W_l2}, OC_{BT} \ra$}
\ForAll{$(u,l) \in IC_E$}
{$\sw(F, l, \la l, \{u\} \ra)$ \label{algo:gen:1}
}
\ForAll(\tcp*[f]{Feedback channel to retrofit Source}){$\la l, W_l \ra \in IC_{W_l1}$}
{$\p(\la l, W_l \ra, OC_{W_l2})$ \label{algo:gen:2}
}
\ForAll{$\bti \in IC_{BT}$}
{$\p(\bti, OC_{BT})$ \label{algo:gen:3}
}
\caption{Generator ($\gbt$)}
\label{algo:gen}
\end{algorithm}

\begin{figure}[h]
\centering  
\resizebox{1\textwidth}{!}{%
\inputtikz{btDP_actor1}
}
\caption[{[\acrshort{iebt}] ]$\dpbt$ With Filter instances}]{This is the evolving state of the $\dpbt$ shows in \autoref{fig:btDP}. This image is showing what happen when a $\fbt$ is spawned. We can see how all the channels are set up in the middle of the spawned $\fbt$ and also between $\fbt$ and $\gbt$.}
\label{fig:btDP_actor1}
\end{figure}

\paragraph{Generator $\gbt$} $\gbt$ also have three main loops. In \autoref{algo:gen:1} it receives each edge $(u,l)$ not consumed by any already spawn  $\fbt$, and 
spawns a new $\fbt$ using $l \in L$ as filter parameter and initializing $\st = \la l, \{u\} \ra$ as it can be seen in \autoref{fig:btDP_actor1}. 
Spawn assumes that, the implementation will connect the channels to keep the downstream correct. Defines all the input channels of $\gbt$ as input channels of the newly spawn $\fbt$ and set $\fbt$'s output channels as $\gbt$ input channels.
In \autoref{algo:gen:2} we can see how the algorithm is receiving and retro feeding $\ibt$ with all the \acrshort{awg} produced by the $\fbt$ . This also assumes that $OC_{W2}$ in $\gbt$ 
is connected with $IC_{W2}$ in $\ibt$.
Finally, \autoref{algo:gen:3} sends all the results that $Q$ matches in the different filters to $\obt$.

\begin{algorithm}
\SetKwInOut{IC}{Input Channels}
\SetKwInOut{O}{Output}
\SetAlgorithmName{}{sink}{}
\SetAlgoRefName{[A3]}
\O{$IO_{BT}$: File or Output Stream with $BT$}
\IC{$IC = \la IC_{BT} \ra$}
\ForAll(\tcp*[f]{Read Results and put in $IO_{BT}$}){$\bti \in IC_{BT}$}
{$put(\bti, IO_{BT})$
}
\caption{Sink ($\obt$)}
\label{algo:sink}
\end{algorithm}

\paragraph{Sink $\obt$} In \autoref{algo:sink} shows a simple stage that receives all results and sends them to some output handler (file, standard output, etc.). 

\begin{algorithm}
\SetKwInOut{P}{Filter Template}
\SetKwInOut{IC}{Input Channels}
\SetKwInOut{OC}{Output Channels}
\SetKwFunction{actora}{actor1}
\SetKwFunction{actorb}{actor2}
\SetKwFunction{actorc}{actor3}
\SetKwFunction{actord}{actor4}
\SetKwFunction{filter}{filter}
\SetKwProg{df}{def}{:}{end}
\SetAlgorithmName{}{fil}{}
\SetAlgoRefName{[A4]}
\P{$l \in L$}
\IC{$ IC = \la IC_E, IC_{W_l1}, IC_{W_l2}, IC_Q, IC_{BT} \ra $}
\OC{$ OC = \la OC_E, OC_{W_l1}, OC_{W_l2}, OC_Q, OC_{BT} \ra $}
\df{\filter{}}{
      $\actora()$\\
      $\actorb()$\\
      $\actorc()$\\
      $\actord()$\\
}
\caption{Filter ($\fbt$)}
\label{algo:fil}
\end{algorithm}

\paragraph{Filter $\fbt$} In \autoref{algo:fil} we can see the simple call sequence over all the actor functions of the filter template. 

\begin{algorithm}
\SetKwInOut{P}{Parameter}
\SetKwInOut{IC}{Input Channels}
\SetKwInOut{OC}{Output Channels}
\SetKwInOut{FS}{$\st$}
\SetKwInOut{PC}{Post-Cond}
\SetKwInOut{PrC}{Pre-Cond}
\SetKwFunction{acta}{actor1}
\SetKwProg{df}{def}{:}{end}
\SetAlgorithmName{}{fa1}{}
\SetAlgoRefName{[A5]}
\P{$l \in L$}
\FS{$\la l, W_l \ra$}
\IC{$ IC = \la IC_E, IC_{W_l1}, IC_{W_l2}, IC_Q, IC_{BT} \ra $}
\OC{$ OC = \la OC_E, OC_{W_l1}, OC_{W_l2}, OC_Q, OC_{BT} \ra $}
\PC{$|W_l| > 1 \lor \fid$}
\BlankLine
\df{\acta{}}{
$\la l, W_l \ra \leftarrow \gs$\\
\ForAll{$(u',l') \in IC_E$}
{\uIf{$l = l'$}{
      $W_l \leftarrow W_l \cup \{u'\}$\label{algo:act-1:1}
}\Else{$\p((u',l'),OC_E)$}
}
\uIf{$|W_l| > 1$}{
      $\us(\la l, W_l \ra)$\\ \label{algo:act-1:2}
      $\p(\la l, W_l \ra, OC_{W_l1})$\\
}\Else{$\fd$}
}
\caption{Actor1 ($actor_1$)}
\label{algo:act-1}
\end{algorithm}

\paragraph{Filter $\aaa$} %As we have described in \autoref{sub:sec:algo-sketch} and it according to \autoref{algo:act-1:1}, 
$\aaa$ receives all edges not consumed by previous $\fbt$.
If the received edge $(u',l')$ is incident to l i.e. $l = l'$,  then $u'$ will be added to the list of \acrshort{awg}. Otherwise the edge is downstream to the next stage.
In \autoref{algo:act-1:2} the state is updated and \acrshort{awg} downstream, if and only if at least 1 wedge was collected. Otherwise, the filter is marked as dead using the  $\fd$ function.

\begin{algorithm}
\SetKwInOut{P}{Parameter}
\SetKwInOut{IC}{Input Channels}
\SetKwInOut{OC}{Output Channels}
\SetKwInOut{FS}{$\st$}
\SetKwInOut{PC}{Post-Cond}
\SetKwInOut{PrC}{Pre-Cond}
\SetKwFunction{actb}{actor2}
\SetKwProg{df}{def}{:}{end}
\SetAlgorithmName{}{fa2}{}
\SetAlgoRefName{[A6]}
\P{$l \in L$}
\FS{$\la l, W_l \ra$}
\IC{$ IC = \la IC_E, IC_{W_l1}, IC_{W_l2}, IC_Q, IC_{BT} \ra $}
\OC{$ OC = \la OC_E, OC_{W_l1}, OC_{W_l2}, OC_Q, OC_{BT} \ra $}
\BlankLine
\PrC{$W_l \subseteq U, |W_l| > 1$}
\PC{$|\dwi| \geq 1 \lor \fid$}
\df{\actb{}}{
$\la l, W_l \ra \leftarrow \gs$\\
\ForAll{$\la l', W_l' \ra \in IC_{W_l1}$}{
      \tcp*[h]{Send Wedge from previous filters to next one}
      $\p(\la l', W_l \ra, OC_{W_l1})$\\
      $\dwi \leftarrow \emptyset$\\
      \If{$W_l' \cap W_l \neq \emptyset$}{ \label{algo:act-2:1}
            $(l_l, l_u) \leftarrow (\argmin_{l,l'}, \argmax_{l,l'})$\\
            \uIf{$l < l'$}{
                  $(W_{l_l}, W_{l_u}) \leftarrow (W_l, W_l')$
            }\Else{
                  $(W_{l_l}, W_{l_u}) \leftarrow (W_l', W_l)$
            }
            $I \leftarrow W_{l_l} \setminus W_{l_u}$\\ \label{algo:act-2:2}
            $J \leftarrow W_{l_l} \cap W_{l_u}$\\
            $K \leftarrow W_{l_u} \setminus W_{l_l}$\\ \label{algo:act-2:3}
            $U_l \leftarrow \la I, J, K\ra$\\
            $\dwi \leftarrow dw \cup \{\la (l_l, l_u), U_l \ra\}$
      }
}
\uIf{$\dwi = \emptyset$}{$\fd$}
\Else{$\us(\dwi)$}
}
\caption{Actor2 ($actor_2$)}
\label{algo:act-2}
\end{algorithm}

\paragraph{Filter $\ab$} In \autoref{algo:act-2} \acrshort{adwg} is built. Following that idea and according to the \dref{def:adwg}, this algorithm will collect all the \acrshort{awg} 
from previous filters if an only if the condition in \autoref{algo:act-2:1} is met. Note that the two \acrshort{awg}, $W_l$ and $W_l'$ have different vertices from the set $L$. $W_l$ intersection check is mandatory since if \acrshort{awg} are disjoint, we cannot aggregate them.
After this checking from \autoref{algo:act-2:2} to \autoref{algo:act-2:3}, the algorithm builds three disjoint Sets to separate upper edges in three subsets; those which are incident only of 
$l$ and $l'$ which are $I$ and $K$ and those that are shared by both lower layer vertex. This can be appreciated in \autoref{fig:agg-double-wedge-example}.
Once \acrshort{adwg} is built $\st = \dwi$ updating the state for the next $\ac$.


\begin{algorithm}
\DontPrintSemicolon
\SetKwInOut{P}{Parameter}
\SetKwInOut{IC}{Input Channels}
\SetKwInOut{OC}{Output Channels}
\SetKwInOut{FS}{$\st$}
\SetKwInOut{PC}{Post-Cond}
\SetKwInOut{PrC}{Pre-Cond}
\SetKwFunction{actc}{actor3}
\SetKwProg{df}{def}{:}{end}
\SetAlgorithmName{}{fa3}{}
\SetAlgoRefName{[A7]}
\P{$l \in L$}
\FS{$\dwi \subseteq \dw$}
\IC{$ IC = \la IC_E, IC_{W_l1}, IC_{W_l2}, IC_Q, IC_{BT} \ra $}
\OC{$ OC = \la OC_E, OC_{W_l1}, OC_{W_l2}, OC_Q, OC_{BT} \ra $}  
\BlankLine
\PrC{$|\dwi| \geq 1$}
\PC{$|\ati| \geq 1 \lor \fid$}
\df{\actc{}}{
      $\dwi \leftarrow \gs$\\
      $\ati \leftarrow \emptyset$\\
      \ForAll{$\la l', W_l \ra \in IC_{W_l2}$}{
            \tcp*[h]{By pass to be used by following Filters}
            $\p(\la l', W_l \ra, OC_{W_l2})$\\ 
            \tcp*[h]{For each double wedge in State}\\
            \ForEach{$\la (l_l, l_u), \la I, J, K \ra \ra \in \dwi, l_l < l' \land l_u > l'$}{
                  $I' \leftarrow I \cup J$\\
                  $K' \leftarrow K \cup J$\\
                  \If{$W_l \cap I' \neq \emptyset \land W_l \cap K' \neq \emptyset$}{
                        $I' \leftarrow I' \cap W_l$\\
                        $K' \leftarrow K' \cap W_l$\\
                        $\hat{U}_l  \leftarrow \la I', J, K' \ra$\\
                        $\ati \leftarrow \ati \cup \big\{\la (l_l, l', l_u), \hat{U}_l \ra\big\}$
                  }
            }
      }
      \uIf{$\ati = \emptyset$}{$\fd$}
      \Else{$\us(\ati)$\\}
}
\caption{Actor3 ($actor_3$)}
\label{algo:act-3}
\end{algorithm}

\paragraph{Filter $\ac$} $\ac$ focuses on treating elements from feedback channel $IC_{W2}$ which is going to downstream all the \acrshort{awg} of all filters.
This is because in order to build \acrshort{abt} finding all the possibles \acrshort{awgc}. This is what is doing \autoref{algo:act-3} according to definition 
\dref{def:awgc} and \dref{def:abt}. If that can be achieve, algorithm sets $\st = \ati$ and $\ad$ can be executed.

\begin{algorithm}
\SetKwInOut{P}{Parameter}
\SetKwInOut{IC}{Input Channels}
\SetKwInOut{OC}{Output Channels}
\SetKwInOut{FS}{$\st$}
\SetKwInOut{PC}{Post-Cond}
\SetKwInOut{PrC}{Pre-Cond}
\SetKwFunction{actc}{actor4}
\SetKwFunction{butV}{buildBtVertex}
\SetKwFunction{butE}{buildBtEdge}
\SetKwProg{df}{def}{:}{end}
\SetAlgorithmName{}{fa4}{}
\SetAlgoRefName{[A8]}
\P{$l \in L$}
\FS{$\ati \subseteq \at$}
\IC{$ IC = \la IC_E, IC_{W_l1}, IC_{W_l2}, IC_Q, IC_{BT} \ra $}
\OC{$ OC = \la OC_E, OC_{W_l1}, OC_{W_l2}, OC_Q, OC_{BT} \ra $}  
\BlankLine
\PrC{$|\ati| \geq 1$}
\df{\actc{}}{
$\ati \leftarrow \gs$\\
\ForAll{$Q \in IC_Q$}{
      \ForEach{$\la (l_l, l_m, l_u), \la I,J,K \ra \ra \in \ati$}{
            \Switch{$Q$}{
                  \Case{$\mathbb{P}(U + L)$}{
                        \If{$\mathbb{P}(U + L) \cap \{l_l, l_m, l_u\} \neq \emptyset \lor \mathbb{P}(U + L) \cap (I \cup J \cup K) \neq \emptyset$}{
                              $\btii \leftarrow \butV(\la (l_l, l_m, l_u), \la I,J,K \ra \ra), \mathbb{P}(U + L))$\\
                              \ForAll{$\bti \in \btii$}{
                                    $\p(\bti, OC_{BT})$
                              }
                        }
                  }
                  \Case{$\mathbb{P}(E)$}{
                        \ForEach{$(u,l) \in \mathbb{P}(E)$}{
                              \If{$(l = l_l \lor l = l_m \lor l = l_u) \land u \in (I \cup J \cup K)$}{
                                    $\btii \leftarrow \butE(\la (l_l, l_m, l_u), \la I,J,K \ra \ra), (u,l))$\\
                                    \ForAll{$\bti \in \btii$}{
                                          $\p(\bti, OC_{BT})$
                                    }
                              }
                        }
                  }
            }
      }
}
}
\caption{Actor4 ($actor_4$)}
\label{algo:act-4}
\end{algorithm}

\paragraph{Filter $\ad$} Once the execution reaches $\ad$, it is ready for processing $Q$ \acrlong{qo}. 
Since we have a compress representation of \acrshort{bt}, which is a similar idea exposed here~\cite{Lai}, we need to enumerate incrementally all the \acrshort{bt} present in the compressed format and filter the ones that match the command.
The matches proceeds in the following manner. Given a \acrshort{abt} with the form $\la \ell, \hat{U}_l \ra$, such that $\ell = (l_1,l_2,l_3)$ and $\hat{U}_l = \la I,J,K \ra$, where $I \subseteq U, J \subseteq U, K \subseteq U$, if the \acrshort{qo} contains a command with vertices, those vertices are search on $\ell$ and $\hat{U}_l$. 
If at least one of them matches, forall $u_i \in I, u_j \in J, u_k \in K$ where $ u_i \neq u_j \neq u_k$, a \acrshort{bt} is built using a 6-cycle $(u_i, l_1, u_j, l_3, u_k, l_2, u_i)$ path.
Using the same construction process a \acrshort{bt} is built if \acrshort{qo} contains a command with edges and any of those edges matches with any of the possible \acrshort{bt} 6-cycle path.

The following pseudo-code definitions on \autoref{algo:buildBtVertex} and \autoref{algo:buildBtEdge} are auxiliary functions that are called
from $\ad$ if $Q$ pattern match either with $\mathbb{P}(U + L)$ or $\mathbb{P}(E)$.

\begin{algorithm}
\SetKwInOut{I}{Input}
\SetKwInOut{O}{Output}
\SetKwFunction{but}{buildBtVertex}
\SetKwProg{df}{def}{:}{end}
\SetAlgorithmName{}{a1}{}
\SetAlgoRefName{[A9]}
\I{$\la (l_l, l_m, l_u), \la I,J,K \ra \ra \in \ati, \ati \subseteq \at$}
\I{$\mathbb{P}(U + L)$}
\O{$\btii \subseteq \bt$ or $\emptyset$ if cannot build any $\btii$}
\df{\but{$\la (l_l, l_m, l_u), \la I,J,K \ra \ra$, $\mathbb{P}(U + L)$}}{
      $\btii \leftarrow \emptyset$\\
      \uIf{$\mathbb{P}(U + L) \cap \{l_l, l_m, l_u\}$}{
            \tcp*[h]{If it is in lower i need to build all for this lower triplet}\\
            \ForEach{$i \in I$}{
                  \ForEach{$j \in J, j \neq i$}{
                        \ForEach{$k \in K, k \neq i \land k \neq j$}{
                              $\btii \leftarrow \btii \cup \{BT_{(l_l, l_m,l_u)}^{(i, j, k)}\}$
                        }
                  }
            }
      }\Else{
            \tcp*[h]{Otherwise just build those that are in the this upper $v$}\\
            \ForEach{$i \in I$}{
                  \ForEach{$j \in J, j \neq i$}{
                        \ForEach{$k \in K, k \neq i \land k \neq j \land (\mathbb{P}(U + L) \cap \{i,j,k\} \neq \emptyset$}{
                              $\btii \leftarrow \btii \cup \{BT_{(l_l, l_m,l_u)}^{(i, j, k)}\}$
                        }
                  }
            }
      }
      \Return{$\btii$}
}
\caption{Function \mintinline{shell}{buildBtVertex}}
\label{algo:buildBtVertex}
\end{algorithm}

\begin{algorithm}
\SetKwInOut{I}{Input}
\SetKwInOut{O}{Output}
\SetKwFunction{but}{buildBtEdge}
\SetKwFunction{he}{hasEdge}
\SetKwProg{df}{def}{:}{end}
\SetAlgorithmName{}{a2}{}
\SetAlgoRefName{[A10]}
\I{$\la (l_l, l_m, l_u), \la I,J,K \ra \ra \in \ati, \ati \subseteq \at$}
\I{$(u,l)$}
\O{$\btii \subseteq \bt$ or $\emptyset$ if cannot build any $\btii$}
\df{\but{$\la (l_l, l_m, l_u), \la I,J,K \ra \ra$, $\mathbb{P}(U + L)$}}{
      $\btii \leftarrow \emptyset$\\
      \ForEach{$i \in I$}{
            \ForEach{$j \in J, j \neq i$}{
                  \ForEach{$k \in K, k \neq i \land k \neq j \land \he((u,l), (l_l,l_m,l_u),(i,j,k))$}{
                        $\btii \leftarrow \btii \cup \{BT_{(l_l, l_m,l_u)}^{(i, j, k)}\}$
                  }
            }
      }
      \Return{$\btii$}
}
\df{\he{$(u,l), (l_l,l_m,l_u), (i,j,k)$}}{
      \If{$(u,l) = (l_l, i) \lor (u,l) = (l_l, j) \lor (u,l) = (l_u, j) \lor (u,l) = (l_u, k) \lor (u,l) = (l_m, i) \lor (u,l) = (l_m, k)$}{
            \Return{True}
      }\Else{
            \Return{False}
      }
}
\caption{Function \mintinline{shell}{buildBtEdge}}
\label{algo:buildBtEdge}
\end{algorithm}

\clearpage
\section{Complexity Analysis of Algorithm}
This complexity analysis does not provide an analysis of the algorithm as a whole with all the details of \acrshort{dp} model included.
Just the opposite, we are only limiting the analysis to the specific functions without taking into consideration neither the context of the parallelization model nor \acrshort{dp}.

Having said that, we are going to conduct the complexity analysis of the algorithm focusing on each stage. 
$\fbt$ stage is going to be analyzed with all its actors as a single stage.

\begin{theorem}[$|\aw| \leq |L|$]\label{theorem:awg}
The size of all possible aggregated wedges is upper bound by $|L|$.
\end{theorem}
\begin{proof}
This can be deduced by \dref{def:awg}, because exactly each $l \in L$ is participating on a wedge.
\end{proof}

\begin{proposition}[Complexity on Source $\ibt$]\label{prop:comp-src}
The total complexity of $\ibt$ algorithm in worst-case is $O(|E|)$.
\end{proposition}
\begin{proof}
In the case of this stage we have three loops. The first one is the number of edges of the graph $|E|$, the second in the number 
of possible \acrshort{awg} which is $|L|$ because of \autoref{theorem:awg}.
And the last is the amount of $Q$, then worst-case complexity is $O(|E| + |L| + |Q|)$.
Since $|L| \leq |E|$ by \dref{def:bt} and $|Q| << |E|$ because otherwise the user will be able to manually inspect the whole graph and we are in a \emph{pay-as-you-go} model
as we have described in \autoref{relate-work}.
Therefore, $O(|E|)$ 
\end{proof}

% TODO: Recibe potencialmente todos los l \in L en el peor caso. 
\begin{proposition}[Complexity on Generator $\gbt$]
The total complexity of $\gbt$ algorithm in worst-case is $O(|E| + |\bt|)$.
\end{proposition}
\begin{proof}
$\gbt$ could receive potentially all edges, then there is a worst case cost of $|E|$.
After that it will receive all $|L|$ worst case, as we have seen in \autoref{theorem:awg}.
Finally, it will receive all possible $\bti \subseteq \bt$. Then worst-case complexity is $O(|E| + |\bt| + |L|)$.
For same reason in \pref{prop:comp-src} $|L|$ can be discarded, therefore complexity is $O(|E| + |\bt|)$.
\end{proof}

\begin{proposition}[Complexity on Sink $\obt$]
The total complexity of $\obt$ algorithm in worst-case is $O(|\bt|)$.
\end{proposition}
\begin{proof}
Obvious by definition of \autoref{algo:sink}.
\end{proof}

% TODO: actor1 |E| worst case no es correcto porque eso ocurriria solo en el primer filter y no en el resto.
% TODO: actor2 La interseccion es 3 veces entre el U y los W_l, y a su vez no es cierto que m = n = |U|
% TODO: actor3 Los double wedges que se intentan construir no se estan combinando en el mismo filtro, sino que el filtro
% recibe wedges y construye los que puede, con lo cual no es una combinatoria.
% TODO: actor4 
\begin{proposition}[Complexity of Filter $\fbt$]
The total complexity of $\fbt$ algorithm in worst-case is $O(\binom{|L|}{2} \times 3|U| + \binom{|V|}{6})$.
\end{proposition}
\begin{proof}
Since actors in filters are executed sequentially, the complexity of $\aaa$ is $O(|E|)$ by \dref{algo:act-1}.
Then for $\ab$ complexity is $O(3|U|)$ because of intersection and difference operation on Set inside the loop is $O(m+n)$ each. 
Since $|U| = m = n$ worst case, then the cost of each loop is $O(3|U|)$.  
The complexity of $\ac$ is $O(\binom{|L|}{2} \times 3|U|)$. For building all possible $\dw$ we need all possible combinations of $|L|$ taken by $2$ worst case, times the complexity of building for each case the three sets of incident edges.
Finally for building all possible $\bt$ in $\ad$ worst case is $O(\binom{|V|}{6})$.
\end{proof}
      
      
\clearpage
\section{Correctness of the Algorithm}
Given a \acrshort{bt} $BT_{\ell}^{\mu} = \bti$ as we defined in \dref{def:bt}, where the 6-cycle is $(u_1,l_1,u_2,l_3,u_3,l_2,u_1)$.

%A bi-triangle can be described as a sequence $(C,a,A,b,B,c,C)$  where $\{ a,b,c\} \in L$  and $\{ A,B,C\} \in U\footnote{In the enumeration, L and $U$ be interchanged}$

We need to prove that the algorithm can enumerate all bi-triangle in the graph and also that there are no duplicates in the enumeration.

Very recently, Qiao et al. proposed CrystalJoin~\cite{Lai} that aims at resolving the output crisis by compressing the intermediate results.

We use the same technique, compressing a group of bi-triangles as we define in \dref{def:abt} $\langle \ell, \hat{U}_l\rangle$, where $\ell = (l_1,l_2,l_3,)$ and $\hat{U}_l= \la I, J, K\ra$, such that $I \subseteq U, J \subseteq U, K \subseteq U$ ($\at$).
 

First we are going to prove that if we take all the triples $(l_1,l_2,l_3)$  such $l_1<l_2$ and $l_2<l_3$  and  $\la (l_1,l_2,l_3,), \hat{U}_l \ra$ is an aggregate  bi-triangle, then it is not stored twice (theorem \ref{TH-unicity}). 
We will show that we store all the aggregate bi-triangles where $u_1,u_2,u_3$  are distinct and $u_1$ is incident to $l_1$ and $l_2$, $u_2$ is incident to $l_1$ and $l_3$ and $u_3$ is incident to $l_2$ and $l_3$ (theorem \ref{TH-all}). 
\begin{theorem} \label{TH-unicity}Every aggregated bi-triangle is stored  at most once 

\end{theorem}
\begin{proof}
As a bi-triangle is a 6-closed cycle it can be described as a sequence of nodes  $(u_1,l_1,u_2,l_3,u_3,l_2,u_1)$, each one adjacent to its neighbour. For every feasible permutation  starting with a node in $U$ we are going to proof that only one will be accepted by $\ab$ .  
\begin{itemize}
      \item $(u_1,l_2,u_3,l_3,u_2,l_1,u_1)$ not accepted because $l_3 > l_1$
      \item $(u_3,l_2,u_1,l_1,u_2,l_3,u_3)$ not accepted because $l_2 > l_1$
      \item $(u_3,l_3,u_2,l_1,u_1,l_2,u_3)$ not accepted because $l_3 > l_1$ and $l_3 > l_2$
      \item $(u_2,l_3,u_3,l_2,u_1,l_1,u_2)$ not accepted because $l_3 > l_1 $
      \item $(u_2,l_1,u_1,l_2,u_3,l_3,u_2)$ is accepted because $l_1 < l_3 $ in actor 2 of filter $F_{l_1}$ or $F_{l_3}$ and $l_1 < l_2 <l_3$  in actor 3
      \item $(u_1, l_1,u_2,l_3,u_3,l_2,u_1)$ not accepted because $l_3 > l_1$
\end{itemize}


Therefore the  only sequence constructed by the algorithm is the one that satisfies $l_1 < l_2 < l_3$ where $u_1,u_2,u_3$  are distinct and $u_1$ is incident to $l_1$ and $l_2$, $u_2$ is incident to $l_1$ and $l_3$ and $u_3$ is incident to $l_2$ and $u_3$.
\end{proof}


\begin{theorem}\label{TH-all}Every bi-triangle present in the graph can be  listed.

\end{theorem}

\begin{proof}
\iffalse
Obvious from the definition of $I,J,K$  and the algorithm of obtaining bi-triangles. 
\fi
Lets assume that a bi-triangle $(C,a,A,b,B,c,C)$ is present in the graph. We will assume that $a$ is the smallest element in the set $\{a,b,c\}$ and $c$ is the bigger.

When $\aaa$ acting in filter $F_a$ ends reading all the edges, $\{A,B\} \subseteq W_a$. Also when $\aaa$ in filter $F_c$ ends reading all the edges, $\{B,C\} \subseteq W_c$. 

As $a < c$  in filter $F_a$ the pair $(a,c)$  will be added to 
$DW$. When $\ac$ in filter $F_a$ receives $(b, W_b)$ will add to $BT$ because of the non empty intersection. Therefore the bi-triangle $(a,A,b,B,c,C)$ can be recognized
\end{proof}
\iffalse
So we are going to talk from now on about the bi-triangles stored in filter $F_a$.

At the end of the execution of $\ac$ $F_a$ constructs the sets where to choose $A,B,C$ $\at$

Actor 1 collects in $F_a$ all the nodes in $U$ adjacent to $a$ i.e. all the candidates to choose $A$. Afterward this set will be reduced. 

Actor 2 receives pairs $(c,V)$ where $V$ are all the nodes adjacent to $c$ in order to test the possibility of constructing a new double wedge.  If the conditions are fulfilled, it adds the double wedge $\la (a,c), U_{t_1} \ra$,  or $\la (c, a), U_{t_2} \ra$. 

In each filter, the pair of lower vertices in the double edges are pairs constructed by $a$ and vertices that are parameters of filters that are before the filter $F_a$. Therefore, the double edges recorded are present in at most one filter.

Actor 3 receives pairs $(b,M) $ where $M$ are all the nodes adjacent to $b$ and has in memory a set of double wedges $\la (a,c), l_2), U_{t_1} \ra$

$\ac$ accepts b as a candidate to construct an element of the set of aggregate bi-triangles if $a < b < c$. Therefore the aggregated bi-triangles so constructed are recorded in a single filter due to the fact that that the pair $(a,c)$ is present in a single filter.
\fi

        
 
\section{\acrshort{dpbt} implementation}
As we have seen in the previous \autoref{dp-hs}, first we need to define the \acrshort{dp} using the \acrshort{dsl}.

\begin{listing}[H]
\begin{minted}[fontsize=\small,numbers=left,breaklines,frame=lines,framerule=2pt,framesep=2mm,baselinestretch=1.2,highlightlines={3}]{haskell}

type DPBT = Source (Channel (Edge :<+> W :<+> Q :<+> BT :<+> BTResult :<+> W :<+> Eof))
      :=> Generator (Channel (Edge :<+> W :<+> Q :<+> BT :<+> BTResult :<+> Eof))
      :=> FeedbackChannel (W :<+> Eof)
      :=> Sink

\end{minted}
\caption{[\mintinline{shell}{BTriangle.hs}] Enconding of \acrshort{dpbt}}
\label{src:dpbt:1}
\end{listing}

In \autoref{src:dpbt:1} can be appreciated the use of retro feed channel in the highlighted line. 
Automatically \acrshort{dpbt} is going to connect this with the $\ibt$.

$\ibt, \obt, gbt$ are not going to be covered because they are straightforward to follow from the code. 
The main part of the algorithmic complexity, as we have seen in the pseudo-code, relies on actor filter.

\begin{listing}[H]
\begin{minted}[fontsize=\small,numbers=left,breaklines,frame=lines,framerule=2pt,framesep=2mm,baselinestretch=1.2,highlightlines={16,21}]{haskell}

actor1 :: Edge
      -> ReadChannel (UpperVertex, LowerVertex)
      -> ReadChannel W
      -> ReadChannel Q
      -> ReadChannel BT
      -> ReadChannel BTResult
      -> ReadChannel W
      -> WriteChannel (UpperVertex, LowerVertex)
      -> WriteChannel W
      -> WriteChannel Q
      -> WriteChannel BT
      -> WriteChannel BTResult
      -> WriteChannel W
      -> StateT FilterState (DP st) ()
actor1 (_, l) redges _ _ _ _ _ we ww1 _ _ _ _ = do
 foldM_ redges $ \e@(u', l') -> do
   e `seq` if l' == l then modify $ flip modifyWState u' else push e we
 finish we
 state' <- get
 case state' of
   Adj w@(W _ ws) -> when (IS.size ws > 1) $ push w ww1
   _              -> pure ()

\end{minted}
\caption{[\mintinline{shell}{BTriangle.hs}] $\aaa$}
\label{src:dpbt:2}
\end{listing}

In \autoref{src:dpbt:2} $\aaa$ source code can be appreciated. Since all the actors are inside the same filter context, all of them has access to read and write
channels of all the filters. That explains the number of parameters which is generated by the \acrshort{idl}.
The first highlighted line is the catamorphism (\mintinline{haskell}{foldM_}) of the $IC_E$ channel. 

\begin{listing}[H]
\begin{minted}[fontsize=\small,numbers=left,breaklines,frame=lines,framerule=2pt,framesep=2mm,baselinestretch=1.2,highlightlines={22,30,36}]{haskell}

actor2 :: Edge
      -> ReadChannel (UpperVertex, LowerVertex)
      -> ReadChannel W
      -> ReadChannel Q
      -> ReadChannel BT
      -> ReadChannel BTResult
      -> ReadChannel W
      -> WriteChannel (UpperVertex, LowerVertex)
      -> WriteChannel W
      -> WriteChannel Q
      -> WriteChannel BT
      -> WriteChannel BTResult
      -> WriteChannel W
      -> StateT FilterState (DP st) ()
actor2 (_, l) _ rw1 _ _ _ _ _ ww1 _ _ _ _ = do
 state' <- get
 case state' of
   Adj (W _ w_t) -> do
     modify $ const $ DoubleWedges mempty
     foldM_ rw1 $ \w@(W l' w_t') -> do
       push w ww1
       buildDW w_t w_t' l l'
     finish ww1
   _ -> pure ()

buildDW :: IntSet -> IntSet -> LowerVertex -> LowerVertex -> StateT FilterState (DP st) ()
buildDW w_t w_t' l l' =
let pair       = Pair (min l l') (max l l')
      paramBuild = if l < l' then (w_t, w_t') else (w_t', w_t)
      ut         = uncurry buildDW' paramBuild
in  if (IS.size w_t > 1) && l /= l' && not (IS.null (IS.intersection w_t w_t')) && not (nullUT ut)
      then modify $ flip modifyDWState (DW pair ut)
      else pure ()

buildDW' :: IntSet -> IntSet -> UT
buildDW' !w_t !w_t' = (w_t IS.\\ w_t', IS.intersection w_t w_t', w_t' IS.\\ w_t)

\end{minted}
\caption{[\mintinline{shell}{BTriangle.hs}] $\ab$}
\label{src:dpbt:3}
\end{listing}

In \autoref{src:dpbt:3}, the interesting part of $\ab$ is the construction of $\dwi$. That is done by the function \mintinline{haskell}{buildDW}
and \mintinline{haskell}{buildDW'} which are building the three subsets required to build $\dwi$.

\begin{listing}[H]
\begin{minted}[fontsize=\small,numbers=left,breaklines,frame=lines,framerule=2pt,framesep=2mm,baselinestretch=1.2,highlightlines={27,34-40}]{haskell}      
actor3 :: Edge
       -> ReadChannel (UpperVertex, LowerVertex)
       -> ReadChannel W
       -> ReadChannel Q
       -> ReadChannel BT
       -> ReadChannel BTResult
       -> ReadChannel W
       -> WriteChannel (UpperVertex, LowerVertex)
       -> WriteChannel W
       -> WriteChannel Q
       -> WriteChannel BT
       -> WriteChannel BTResult
       -> WriteChannel W
       -> StateT FilterState (DP st) ()
actor3 (_, l) _ _ _ _ _ rfb _ _ _ _ _ wfb = do
  state' <- get
  case state' of
    DoubleWedges dwtt -> do
      modify $ const $ BiTriangles mempty
      foldM_ rfb $ \w@(W l' w_t') -> do
        push w wfb
        when (hasDW dwtt) $ do
          let (DWTT dtlist) = dwtt
          forM_ dtlist $ \(DW (Pair l_l l_u) ut) ->
            let triple = Triplet l_l l' l_u
                result =
                  if l' < l_u && l' > l_l then filterUt w_t' ut else Nothing
            in  maybe (pure ()) (modify . flip modifyBTState . BT triple) result
      finish wfb
    _ -> pure ()

filterUt :: IntSet -> UT -> Maybe UT
filterUt wt (si, sj, sk) =
  let si' = IS.filter (`IS.member` wt) si 
      sj' = IS.filter (`IS.member` wt) sj
      sk' = IS.filter (`IS.member` wt) sk
      sij' = si' `IS.union` sj'
      sjk' = sk' `IS.union` sj'
      wtInSome = not (IS.null sij' || IS.null sjk')
  in  if wtInSome then Just (sij', sj, sjk') else Nothing
\end{minted}
\caption{[\mintinline{shell}{BTriangle.hs}] $\ac$}
\label{src:dpbt:4}
\end{listing}

$\ac$ in \autoref{src:dpbt:4}, shows the build $\ati$ and in highlighted lines we can see the algorithm to detect when the \acrshort{awg}
received in the feedback channel are \acrshort{awgc} according to \dref{def:awgc}.

\begin{listing}[H]
\begin{minted}[fontsize=\small,numbers=left,breaklines,frame=lines,framerule=2pt,framesep=2mm,baselinestretch=1.2,highlightlines={21-22,27-31}]{haskell}            
actor4 :: Edge
       -> ReadChannel (UpperVertex, LowerVertex)
       -> ReadChannel W
       -> ReadChannel Q
       -> ReadChannel BT
       -> ReadChannel BTResult
       -> ReadChannel W
       -> WriteChannel (UpperVertex, LowerVertex)
       -> WriteChannel W
       -> WriteChannel Q
       -> WriteChannel BT
       -> WriteChannel BTResult
       -> WriteChannel W
       -> StateT FilterState (DP st) ()
actor4 (_, l) _ _ query _ rbtr _ _ _ wq _ wbtr _ = do
  state' <- get
  case state' of
    BiTriangles bttt -> do
      rbtr |=> wbtr
      foldM_ query $ \e -> do
        push e wq
        unless (hasNotBT bttt) $ sendBts bttt e wbtr
    _ -> pure ()

sendBts :: MonadIO m => BTTT -> Q -> WriteChannel BTResult -> m ()
sendBts (BTTT bttt) q@(Q c _ _) wbtr = case c of
  ByVertex vx    -> forM_ bttt (\bt -> filterBTByVertex bt vx (flip push wbtr . RBT q))
  ByEdge   edges -> forM_ bttt (\bt -> filterBTByEdge bt edges (flip push wbtr . RBT q))
  AllBT          -> forM_ bttt (R.mapM_ (flip push wbtr . RBT q) . buildBT)
  Count          -> forM_ bttt (flip push wbtr . RC q . R.length . buildBT)
  _              -> pure ()
\end{minted}
\caption{[\mintinline{shell}{BTriangle.hs}] $\ad$}
\label{src:dpbt:5}
\end{listing}

Finally $\ad$ in \autoref{src:dpbt:5} shows the pattern match done over the query $Q$ sum type and the construction
of the final \acrshort{bt} to be downstream to the $\obt$. We are ommiting here the function definition of \mintinline{haskell}{filterBTByEdge}
and \mintinline{haskell}{filterBTByVertex}.

\section{Chapter Summary}
This chapter presents the details of the \acrshort{dpbt} algorithm. In the first part, we have introduced the sketch of the algorithm.
Then, we have described each of the pseudo-code definitions of each component: $\ibt, \gbt, \fbt, \obt$ and actors. 
After that, we have presented a complexity analysis and correctness proof of the pseudo-code algorithm.
At the end of the chapter, we have shown how we implemented this using \acrshort{dpfh} is described in \autoref{dp-hs}.

