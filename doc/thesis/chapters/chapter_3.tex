\chapter{Related Work}\label{relate-work}

\section{\acrlong{bt} Counting}
As we have stated on the \autoref{intro}, there is also no previous work done on \acrshort{bt} enumeration and it is, as well as \acrshort{hs} Framework, a novelty introduced by this work.
A \acrshort{bt} is a Hamiltonian circuit as well, and finding Hamiltonian circuits in \acrshort{bg} is $NP$-hard~\cite{hamilbipartite-np}. On the other hand counting \acrshort{bt} is not in $NP$ 
but in $P$ and there is are novel algorithms in $P$ for counting \acrshort{bt} using combinatorial algorithms presented on this work~\cite{btcount}.

In that paper the authors present different concepts to calculate in a combinatorial algorithmic approach the counting problem of \acrshort{bt} in \acrshort{bg}. They present 3 global approaches to count
all the \acrshort{bt} in the graph and 2 local algorithms: one that counts all the \acrshort{bt} given a particular vertex of the Graph and other for counting all the \acrshort{bt} given a particular edge.

Although the authors achieve not only empirically remarkable execution times for counting \acrshort{bt} in huge \acrshort{bg}, and quadratic worst case complexity they are counting and the scope of our work
is enumerating. In that sense they use some specific gadgets (\emph{super-wedges}, \emph{swj-unit} and \emph{wj-unit}) that feed the final combinatorial counting algorithm. We are not going to use those because our 
aim is to identify specific \acrshort{bt} with their component and not the global number.

\section{Incremental Computation}

\section{Chapter Summary}
In this chapter we have summarized all the \emph{State of the Art} of the aspects that our research work aims to solve.
First we have shown the different Streaming processing computational models, then we have describe the \acrshort{dp} as we know it now and that is used 
as the core paradigm to solve our problem. Then we have explored what is the related work on that aspect at this moment in \acrshort{hs}. Finally we have shown 
a related novel work on \acrlong{bg} for counting \acrlong{bt} which droves our motivation on first place.
