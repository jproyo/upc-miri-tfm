\documentclass[thesis]{Thesis}
\usepackage[utf8]{inputenc}
\usepackage{amsmath}
\usepackage{amsfonts}
\usepackage{amssymb}
\usepackage{amsthm}
\usepackage{array}
\usepackage{graphicx}
\usepackage[pdfencoding=auto]{hyperref}
\usepackage{fancyvrb}
\usepackage{fancyhdr}
\usepackage{lastpage}
\usepackage{tikz}
\usepackage{float}
\usepackage{listing}
\usepackage{color}
\usepackage{caption}
\usepackage{authblk}
\usepackage{longtable}
\usepackage{paralist}
\usepackage{wrapfig}
\usepackage{subcaption}
\usepackage[square, numbers, comma, sort&compress]{natbib}
\usepackage[ruled,linesnumbered,lined,boxed,commentsnumbered]{algorithm2e}
\usepackage[acronym]{glossaries}
\usepackage[nottoc]{tocbibind}
\usepackage[cache=true]{minted}

\title{An incremental algorithm for enumerating bi-triangles in large bi-partite networks}
\setthesistype{Master}
\author{Juan Pablo Royo Sales}
\degree{%
  Master in Innovation and Research in Informatics\\ 
  Advance Computing
}
\supervisor{%
  Edelmira Pasarella, Computer Science Department\\
  Maria-Esther Vidal, TIB-Leibniz Infomation Centre for Science and Technology, and Leibniz University of Hannover, Germany\\
  Cristina Zoltan
}

\date{Octuber 25th, 2021}

\usetikzlibrary{shapes.misc,shadows}
\usetikzlibrary{quotes,positioning,arrows,decorations.markings}
\usetikzlibrary{positioning} 
\usemintedstyle{default}
\newminted{haskell}{frame=lines,framerule=2pt}
\newminted{R}{frame=lines,framerule=2pt}
\graphicspath{{./images/}}
\tikzstyle{bag} = [align=center]
\newcommand{\dw}{\mathbb{DW}}
\newcommand{\aw}{\mathbb{AW}}
\newcommand{\bt}{\mathbb{BT}}
\newcommand{\bti}{BT_{(l_1, l_2,l_3)}^{(u_1, u_2, u_3)}}
\newcommand{\at}{\mathbb{AT}}
\newcommand{\st}{ST}
\newcommand{\sw}{\mathtt{spawn}}
\newcommand{\fd}{\mathtt{killFilter}}
\newcommand{\fid}{\mathtt{filterIsDied}}
\newcommand{\us}{\mathtt{updateState}}
\newcommand{\gs}{\mathtt{getState}}
\newcommand{\p}{\mathtt{push}}
\newcommand{\mt}{\mathtt{matchQ}}
\newcommand{\io}{\mathtt{indexOf}}
\newcommand{\la}{\left\langle}
\newcommand{\ra}{\right\rangle}
\newcommand{\DP}{\mathsf{DP}}
\newcommand{\dpwcc}{\mathsf{DP_{WCC}}}
\newcommand{\iwcc}{\mathsf{Sr}}
\newcommand{\iwc}{\mathsf{Sr_{WCC}}}
\newcommand{\owcc}{\mathsf{Sk}}
\newcommand{\owc}{\mathsf{Sk_{WCC}}}
\newcommand{\fwcc}{\mathsf{F}} 
\newcommand{\fwc}{\mathsf{F_{WCC}}} 
\newcommand{\gwcc}{\mathsf{G}}
\newcommand{\gwc}{\mathsf{G_{WCC}}}
\newcommand{\ice}{\mathsf{IC_E}}
\newcommand{\csofv}{\mathsf{IC_{set(V)}}}
\newcommand{\sgen}{\mathsf{S_G}}
\newcommand{\sfilter}{\mathsf{S_F}}
\newcommand{\sinp}{\mathsf{S_I}}
\newcommand{\sout}{\mathsf{S_O}}
\newcommand{\istream}{\mathsf{D}}
\newcommand{\wccout}{\mathsf{R}}
\newcommand{\fmem}{\mathsf{M_F}}
\newcommand{\eof}{\mathsf{eof}}
\newcommand{\Act}{\mathsf{actor_1}}
\newcommand{\Actt}{\mathsf{actor_2}}
\newcommand{\gdsl}{G_{dsl}}

\renewcommand\listingscaption{Source Code}
\providecommand*{\listingautorefname}{Source Code}

\DeclareMathOperator*{\argmax}{arg\,max}
\DeclareMathOperator*{\argmin}{arg\,min}

\newacronym{prole21}{PROLE21}{Jornadas de la Sociedad de Ingeniería de Software y Tecnologías de Desarrollo de Software}
\newacronym{bt}{BT}{Bitriangle}
\newacronym{bg}{BG}{Bipartite Graph}
\newacronym{ug}{UG}{Unipartite Graph}
\newacronym{dp}{DPP}{Dynamic Pipeline Paradigm}
\newacronym{dpf}{DPF}{Dynamic Pipeline Framework}
\newacronym{dpfh}{DPF-Haskell}{Haskell Dynamic Pipeline Framework}
\newacronym{ds}{DS}{Data Streaming}
\newacronym{dap}{DAP}{Data Parallelism}
\newacronym{pip}{PIP}{Pipeline Parallelism}
\newacronym{mr}{MR}{MapReduce}
\newacronym{er}{ER}{Entity Resolution}
\newacronym{hack}{Hackage}{The Haskell Package Repository}
\newacronym{dpbt}{DP-BT-Haskell}{$DP_{BT}$ in Haskell}
\newacronym{dpwcc}{DP-WCC-Haskell}{$DP_{WCC}$ in Haskell}
\newacronym{dpl}{DPL}{dynamic-pipeline}
\newacronym{bfs}{BFS}{Breadth-First Search}
\newacronym{dfs}{DFS}{Depth-First Search}
\newacronym{wcc}{WCC}{Weak Connected Components}
\newacronym{hs}{Haskell}{Haskell Programming Language}
\newacronym{fp}{FP}{Functional Programming}
\newacronym{stm}{STM}{Software Transactional Memory}
\newacronym{rl}{R}{R Language}
\newacronym{os}{OS}{Operative System}
\newacronym{dm}{Dm}{Diefficency Metrics}
\newacronym{tfft}{TFFT}{Time for the first tuple}
\newacronym{et}{ET}{Execution Time}
\newacronym{comp}{Comp}{Completeness}
\newacronym{tt}{T}{Throughput}
\newacronym{dt}{dief$@$t}{Diefficiency first $t$ time units}
\newacronym{snap}{SNAP}{Stanford Network Data Set Collection}
\newacronym{ghc}{GHC}{Glasgow Haskell Compiler}
\newacronym{dsl}{DSL}{Domain-specific Language}
\newacronym{edsl}{EDSL}{Embedded Domain-specific Language}
\newacronym{idl}{IDL}{Interpreter of DSL}
\newacronym{rs}{RS}{Runtime System}
\newacronym{go}{Go}{Go Programming Language}
\newacronym{hl}{HL}{Host Language}

\glsdisablehyper

\newtheorem{hyp}{Hypothesis}


\tikzexternalize[shell escape=-shell-escape,mode=graphics if exists,prefix=images/]
\begin{document}

\maketitle

\cleardoublepage
% The "Quote Page"
%%%\pagestyle{empty}  % No headers or footers for the following pages
\null\vfill
\textit{".. when a Mathematical Reasoning can be had it's as great a folly to make use of any other, as to grope for a thing in the dark, when you have a Candle standing by you."}
\begin{flushright}
	Of the Laws of Chance, John Arbuthnot (1662)\\
\end{flushright}
\vfill\vfill\vfill\vfill\vfill\vfill\null
\cleardoublepage 

\rhead{\thepage}
\pagestyle{fancy}
\tableofcontents
\listoffigures
\listoftables
\listofalgorithms
\addcontentsline{toc}{chapter}{List of Algorithm}
\listoflistings
\addcontentsline{toc}{chapter}{List of Haskell Code}
\mainmatter

\acknowledgements{
  Thanks to ...... % TODO
}

\abstract{
A bipartite network is a graph with two disjoint vertex sets where its edges only connect vertices from different sets. 
For instance, two different object sets can be modeled with a bipartite graph, establishing the relationships between them (e.g., a network of drugs and side effects of those drugs). 
One of the metrics that reflect properly if the relationships are strong on particular elements of sets are bitriangles of bipartite graphs. 
Although it is known that enumerating those bi-triangles is an NP-complete problem, having an algorithm that enumerates bi-triangles incrementally that matches specific criteria, 
is extremely useful for real networks and scenarios like expressed before. 
On the other hand, streaming processing has given rise to new computation paradigms processing data effectively.
The most important features of these paradigms are the exploitation of parallelism, the capacity to adapt execution schedulers, 
reconfigure computational structures, adjust the use of resources according to the characteristics of the input stream and produce incremental results. 
The Dynamic Pipeline Paradigm (DPP) is a naturally functional approach to deal with stream processing. 
This fact encourages us to use Haskell, a purely functional programming language, for DPP.  
In this work, we validate the suitability of (parallel) Haskell to implement a Dynamic Pipeline Framework (DPF). 
After that, we show the abstraction and implementation details of the DP Framework written in Haskell and then, 
we provide an algorithm for enumerating bi-triangles in a bipartite graph given particular criteria. 
At the end of the work, we empirically evaluate the implementation against large bipartite networks through a series of experiments and metrics.
We also discuses the conclusions that we arrive at based on these experiments' results.
All these results are novel and cannot be compared with related works since they focus on counting problems and not enumeration.
}

\chapter{Introduction and Preliminaries}
\section{Introduction}
\section{Preliminaries}
\section{Dynamic Pipeline Paradigm}
\section{Proof of Concept - $\dpwcc$}


\chapter{Contribution}
\section{Motivation}
\section{Preliminaries}
\section{Implementation of $\dpwcc$ in \acrshort{hs}}
\section{Empirical Evaluation}
\section{Conclusions}
\section{Chapter Summary}



\chapter{Dynamic Pipeline Framework in Haskell}
\section{Framework Model}
\section{Implementation}
\section{Chapter Summary}


\chapter{\acrshort{wcc} - Proof of concept}\label{prole}
In the context of this research, we have started approaching the problem first evaluating the suitability of \acrlong{hs}
to implement the \acrlong{dp}. On the other hand and for test empirically this implementation, we develop a Proof of Concept solving the \acrlong{wcc} of a Graph using \acrlong{dp} with \acrshort{hs}.
Since we have obtained positive results on this initial research, we decided to write a contribution that was accepted, and we presented in \acrshort{prole21}~\cite{prole21} conference in September of 2021.
In this chapter, we are going to describe the details and results obtained in that contribution that plants the seed for the present work.

\section{Motivation}
Effective streaming processing of large amounts of data has been studied for several years~\cite{enumeratingsg, exploiting, onthefly} as a key factor providing fast and incremental results in big data algorithmic problems. 
One of the most explored techniques, regardless of the approach, is the exploitation of parallel techniques to take advantage of the available computational power as much as possible. 
In that regard, the \acrfull{dp} \cite{dpdef} has lately emerged as one of the models that exploit data streaming processing using a dynamic pipeline parallelism approach \cite{onthefly}. 
This computational model has been designed with a functional focus, where the main components of the paradigm are functional stages or pipes which dynamically enlarge and shrink depending on incoming data.  

One of the biggest challenges of implementing a \acrfull{dpf} is to find a proper set of tools and programming language which can take advantage of both of its primary aspects: \begin{inparaenum}[i\upshape)]
\item  \emph{fast parallel} processing and 
\item  \emph{strong theoretical} foundations that manage computations as first-class citizens.
 \end{inparaenum}
\acrfull{hs} is a statically typed pure functional language which has been designed and evolved from its birth in 1987, on strong theoretical foundations where computations are primary entities, 
and at the same time has been providing a powerful set of tools for writing multithreading and parallel programs with optimal performance \cite{parallelbook, monadpar}.

The main objective of this contribution was to explore the feasibility of using a  \acrfull{fp} language to implement a \acrshort{dpf}. In particular, 
we tackle the problem of establishing the basis of an implementation of a \acrshort{dpf}  in \acrshort{hs}, a pure functional language. This is,  our aim is to determine the particular features (i.e., versions and libraries) 
of this language that will allow for an efficient implementation of the \acrshort{dpf}. To be concrete, through a particular and very relevant problem as the computation of the \acrfull{wcc} of a graph,  we study the critical features required in \acrshort{hs} for a \acrshort{dpf} implementation.

\section{Implementation}
Let us consider the problem of computing the (weak) connected components of a graph $G$ using \acrshort{dp}. A connected component of a graph is a subgraph in which any two vertices are connected by paths. Thus, finding connected components of a directed graph implies obtaining the minimal partition of the set of nodes induced by the relationship \textit{connected}, i.e., there is a path between each pair of nodes. The input of the Dynamic Pipeline for computing the WCC of a graph, $\dpwcc$, is a sequence of edges ending with $\eof$%$G=\{(v,w) | v,w\in V, v\neq w\}$, this is a graph given in terms of a stream of edges
\footnote{Note that there are neither isolated vertices nor loops in the source graph $G$.}. The connected components are output as soon as they are computed, i.e., they are produced incrementally. 
%
$\dpwcc$ is defined in terms of the behavior of its four kinds stages: \textit{Source} ($\iwc$),  \textit{Generator} ($\gwc$),  \textit{Sink} ($\owc$), and \textit{Filter}($\fwc$) stages. Additionally,  the channels connecting these stages must be defined. In $\dpwcc$, stages are connected linearly and unidirectionally through the channels $\ice$ and  $\csofv$. Channel $\ice$ carries edges while channel  $\csofv$ conveys sets of connected vertices. Both channels end by the $\eof$ mark. The initial configuration of $\dpwcc$ is $\iwc \:\rightrightarrows\:\gwc \:\rightarrow \: \owc$, where $\rightrightarrows$ represents  channels $\ice$ and $\csofv$ while $\rightarrow$ represents the channel $\csofv$.
 
 Once activated the initial $\dpwcc$, the stream of edges is fed into $\iwc$ and $\owc$ produces the resulting connected components. 
 $\gwc$ has as parameter the template of the stage  $\fwc$. When an edge $(v,w)$ arrives to $\gwc$, it  spawns a new instance of $\fwc$ before $\gwc$. 
 For example, the first time a filter instance is spawned, the $\dpwcc$ evolves to this one: $\iwc \:\rightrightarrows\: \fwc \:\rightrightarrows\: \gwc \:\rightarrow \: \owc$. 
 The state of this new filter instance is initialized with the vertices $\{v,w\}$. When $\eof$ arrives to $\gwc$, it connects previous filter instance to $\owc$ through $\csofv$; then, $\gwc$ dies and the $\dpwcc$ evolves as follows: $\iwc \:\rightrightarrows\: \fwc \:\rightrightarrows\: \cdots \fwc \:\rightrightarrows \: \owc$. 
 The behavior of $\fwc$ is given by a sequence of two actors (scripts). In what follows, we denote these actors by $\Act$ and $\Actt$, respectively. 
 The script $\Act$ keeps a set of connected vertices ($CV$) in the state of the $\fwc$ instance. When an edge $e$ arrives, if an endpoint of $e$ is present in the state, then the other endpoint of $e$ is added to $CV$. 
 Edges without incident endpoints are passed to the next stage. When $\eof$ arrives at channel $\ice$, it is passed to the next stage, and the script $\Actt$ starts its execution. 
 If script $\Actt$ receives a set of connected vertices $CV$ in $\csofv$, it determines if the intersection between $CV$ and the nodes in its state is not empty. If so, it adds the nodes in $CV$  to its state. 
 Otherwise, the $CV$ is passed to the next stage.  Whenever $\eof$ is received, $\Actt$ passes--through $\csofv$-- the set of vertices in its state and the $\eof$ mark to the next stage; then, it dies.
 The behavior of $\iwc$ corresponds to the identity transformation over the data stream of edges.  As edges arrive, they are passed through  $\ice$ to the next stage. When receiving $\eof$ on $\ice$, this mark is put on both channels. 
 Then, $\iwc$ dies. 


\section{Empirical Evaluation}
The empirical study aims at evaluating the performance of $\dpwcc$ when implemented in \acrshort{hs}. 
Our goal is to answer the following research questions: 

\begin{inparaenum}[\bf {\bf RQ}1\upshape)]
\label{res:question}
    \item Does $\dpwcc$ in \acrshort{hs} support the dynamic parallelization level that $\dpwcc$ requires?
    \item Is $\dpwcc$ in \acrshort{hs} competitive compared with default implementations on base libraries for the same problem?
    \item Does $\dpwcc$ in \acrshort{hs} handle memory efficiently?
\end{inparaenum}

We have conducted different kinds of experiments to test our assumptions and verify the correctness of the implementation.
First, we have performed an \emph{Implementation Analysis} in which we have selected some graphs from \acrfull{snap} \cite{stanford} 
and analyze how the implementation behaves under real-world graphs if it timeouts or not and if it is producing correct results in terms of the amount of \acrshort{wcc} that we know beforehand.
We have also tested the implementation doing a \emph{Benchmark Analysis} where we focus on two different types of benchmarks. On the one hand, 
using \texttt{criterion} library \cite{criterion}, we have evaluated a benchmark between our solution and \acrshort{wcc} algorithm implemented in \texttt{containers} \acrshort{hs} library \cite{containers} 
using \mintinline{haskell}{Data.Graph}. On the other hand, we have compared if the results are being generated incrementally in both cases and how that is done during the pipeline execution time. 
This last analysis has been conducted using \texttt{diefpy} tool \cite{diefpaper,diefpy}.
Finally, we have executed a \textit{Performance Analysis} in which we have to gather profiling data from \acrfull{ghc} for one of the real-world graphs to measure how the program performs regarding multithreading and memory allocation.

\paragraph{Implementation analysis} The following represents the execution for running these graphs on our \acrshort{dp} implementation.

\begin{table}[H]
  \centering
  \begin{tabular}{|l|r|r|r|r|}
   \hline
   \textbf{Network} & \textbf{Exec Param} & \textbf{MUT Time} & \textbf{GC Time} & \textbf{Total Time}\\
   \hline
   Enron Emails & \mintinline{bash}{+RTS -N4 -s} & 2.797s & 0.942s & 3.746s \\
   \hline
   Astro Physics Coll Net & \mintinline{bash}{+RTS -N4 -s} & 2.607s & 1.392s & 4.014s \\
   \hline
   Google Web Graph & \mintinline{bash}{+RTS -N8 -s} & 137.127s & 218.913s & \textbf{\textcolor{red}{356.058s}} \\
   \hline
  \end{tabular}
 \caption{Execution times}
 \label{table:5}
 \end{table}

It is important to point out that since the first two networks are smaller in the number of edges compared with \emph{web-Google}, 
executing those with $8$ cores as the \mintinline{bash}{-N} parameters indicates does not affect the final speed-up since \acrshort{ghc} 
is not distributing threads on extra cores because it handles the load with $4$ cores only.
As we can see in \autoref{table:5}, we are obtaining remarkable execution times for the first two graphs, and it seems not to be the case 
for \textit{web-Google} due to the topology of the graph; it is denser in terms of connected components than the others.

\paragraph{Benchmark Analysis} Regarding mean execution times for each implementation on each case measure by \texttt{criterion} library \cite{criterion}, we can display the following results:

\begin{table}[H]
  \centering
  \begin{tabular}{|l|l|l|l|}
   \hline
   \textbf{Network} & \textbf{\acrshort{dpwcc}} & \textbf{\acrshort{hs} \texttt{containers}} & \textbf{Speed-up}\\
   \hline
   Enron Emails & 4.68s &  6.46s & 1.38\\
   \hline
   Astro Physics Coll Net & 4.98s & 6.95s  & 1.39\\
   \hline
   Google Web Graph & 386s & 106s & 0.27\\
   \hline
  \end{tabular}
 \caption{Mean Execution times}
 \label{table:6}
 \end{table}

These results allow for answering Question [Q2], where we have seen that the graph topology is affecting the performance and the parallelization, penalizing \acrshort{dpwcc} for this particular case. In this benchmark, 
the solution against a non-parallel \texttt{containers} \mintinline{haskell}{Data.Graph} confirms the hypothesis. 

\paragraph{Diefficency metrics} Some considerations are needed before starting to analyze the data gathered with \acrfull{dm} tool. Firstly, the tool is plotting the results according to the traces generated by the implementation, 
both \acrshort{dpwcc} and \acrshort{hs} \emph{containers}. By the nature of \acrshort{dp} model, we can gather or register that timestamps as long as the model is generating results. In the case of \acrshort{hs} \texttt{containers}, this is not possible since it 
calculates \acrshort{wcc} at once. This is not an issue and we still can check at what point in time all \acrshort{wcc} in \acrshort{hs} \texttt{containers} are generated. In those cases, we are going to see a straight vertical line. 

It is important to remark that we needed to scale the timestamps because we have taken the time in nanoseconds. After all, the incremental generation between one \acrshort{wcc} and the other is very small but significant enough to be taken into consideration. 
Thus, if we left the time scale in integer milliseconds, microseconds, or nanoseconds integer part, it cannot be appreciated. In case of escalation, we are discounting the nanosecond integer of the first generated results resulting in a time scale that starts close to $0$. 
This does not mean that the first result is generated at $0$ time, but we are discarding the previous time to focus on how the results are incrementally generated.

Having said that, we can see the results of \acrshort{dm} which are presented in two types of plots. The first one is regular line graphs in where the $x$ axis shows the time that escalated when the result was generated, and the $y$ axis shows the component number that was generated at that time. 
The second type of plot is a radar plot in which shows how the solution is behaving 
on the dimensions of  \acrfull{tfft}, \acrfull{et}, \acrfull{tt}, \acrfull{comp} and \acrfull{dt} and how are the tension between them; all these metrics are higher is better. 
All the details about these metrics are explained here \cite{diefpaper}.

\begin{figure}[!htb]
    \centering
    \begin{minipage}{0.33\textwidth}
     \includegraphics[width=1\linewidth, height=0.2\textheight]{email_enron}
      \caption{email-Enron \acrshort{dm}}
      \label{fig:dief:1}
    \end{minipage}%
    \begin{minipage}{0.33\textwidth}
     \includegraphics[width=1\linewidth, height=0.2\textheight]{ca_astroph}
      \caption{ca-AstroPh \acrshort{dm}}
      \label{fig:dief:2}
    \end{minipage}%
    \begin{minipage}{0.33\textwidth}
     \includegraphics[width=1\linewidth, height=0.2\textheight]{web_google}
      \caption{web-Google \acrshort{dm}}
      \label{fig:dief:3}
    \end{minipage}
\end{figure}

Based on the results shown in all the figures above, all the solutions in \acrshort{dpwcc} are being generated incrementally, 
but there is some difference that we would like to remark. In the case of \emph{email-Enron} and \emph{ca-AstroPh} graphs 
as we can see in \autoref{fig:dief:1} and \autoref{fig:dief:2}, there seems to be a more incremental generation of results. 
This behavior is measured with the values of \acrfull{dt}. \emph{ca-AstroPh} as it can be seen in \autoref{fig:dief:2}, is even more incremental, and it is showing a clear separation between some results and others. 
The \emph{web-Google} network, which is shown in \autoref{fig:dief:3}, is a little more linear, and that is because all the results are being generated with very little difference in time between them. 
Having the biggest \acrshort{wcc} at the end of \emph{web-Google} \acrshort{dp} algorithm 
it is retaining results until the biggest \acrshort{wcc} can be solved, which takes longer. 


\paragraph{Multithreading} For analyzing parallelization and multithreading we have used \textit{ThreadScope} \cite{threadscope} which allows us to see how the parallelization is taking place on \acrshort{ghc} at a fine grained level and how the threads are distributed throughout the different cores requested with the \mintinline{bash}{-N} execution \texttt{ghc-option} flag.
The distribution of the load is more intensive at the end of the execution, where \mintinline{haskell}{actor2} filter stage 
of the algorithm is taking place and different filters are reaching execution of that second actor.
\begin{wrapfigure}{r}{0.5\textwidth}
  \begin{center}
     \includegraphics[width=0.48\textwidth, height=0.2\textheight] {screen_2}
    \end{center}
    \captionsetup{type=figure}
    \captionof{figure}{Threadscope Image of Zoomed Fraction}
    \label{fig:4}
 \end{wrapfigure}
~\autoref{fig:4} zooms in on \textit{ThreadScope} output in a particular moment, approximately in the middle of the execution. 
We can appreciate how many threads are being spawned and by the tool and if they are evenly distributed among cores. 
The numbers inside green bars represent the number of threads that are being executed on that particular core (horizontal line) at that execution slot. 
Thus, the number of threads varies among slot execution times, because as it is already known, \acrshort{ghc} implements \emph{Preemptive Scheduling} \cite{lightweightghc}.
It can be appreciated in \autoref{fig:4} our first assumption that the load is evenly distributed because the mean number of executing threads per core is $571$.

\paragraph{Memory allocation} Another important aspect in our case is how the memory is being managed to avoid memory leaks or other non-desired behavior that increases memory allocation during the execution time. This is even more important in the particular implementation of \acrshort{wcc} using \acrshort{dp} model because it requires to maintain the set of connected components in memory throughout the execution of the program or at least until we can output the calculated \acrshort{wcc} if we reach to the last \textit{Filter} and we know that this \acrshort{wcc} cannot be enlarged anymore.
In order to verify this, we measure memory allocation with \textit{eventlog2html} \cite{eventlog2html} which converts generated profiling memory eventlog files into graphical HTML representation. 
\begin{wrapfigure}{rt!}{0.5\textwidth}
  \begin{center}
     \includegraphics[width=0.48\textwidth, height=0.2\textheight] {visualization}
       \end{center}
     \caption{Memory Allocation}
     \label{fig:5}
 \end{wrapfigure}

As we can see in \autoref{fig:5}, \acrshort{dpwcc} does efficient work on allocating memory since we are not using more than $57$ MB of memory during the whole execution of the program.
On the other hand, if we analyze how the memory is allocated during the execution of the program, it can also be appreciated that most of the memory is allocated at the beginning of the program and steadily decrease over time, with a small peak at the end that does not overpass even half of the initial peak of $57$ MB. 
The explanation for this behavior is quite straightforward because, in the beginning, we are reading from the file and transforming a \mintinline{haskell}{ByteString} buffer to \mintinline{haskell}{(Int, Int)} edges. 
This is seen in the image in which the dark blue that is on top of the area is \mintinline{haskell}{ByteString} allocation. 
Light blue is allocation of \mintinline{haskell}{Maybe a} type which is the type that is returned by the \textit{Channels} because it can contain a value or not. 
Data value \mintinline{haskell}{Nothing} is indicating end of the \textit{Channel}. 
Another important aspect is the green area which represents \mintinline{haskell}{IntSet} allocation, which in the case of our program is the data structure that we use to gather the set of vertices that represents a \acrshort{wcc}. 
This means that the amount of memory used for gathering the \acrshort{wcc} itself is minimum, and it is decreasing over time, which is another empirical indication that we are incrementally releasing results to the user. 
It can be seen as well that as long the green area reduces the lighter blue (\mintinline{haskell}{MUT_ARR_PTRS_CLEAN} \cite{ghcheap}) increases at the same time indicating that the computations for the output (releasing results) is taking place. 
Finally, according to what we have stated above, we can answer the question [Q3], showing that not only the memory management was efficient, but at the same time, the memory was not leaking or increasing across the running execution program.

\section{Conclusions}
The empirical evaluation of the \acrshort{dpwcc} implementation to compute weakly connected components of a graph, evidence suitability, 
and robustness to provide a Dynamic Pipeline Framework in that language. Measuring using \acrshort{dt} metrics reveals some advantageous capability of $\dpwcc$ implementation to deliver incremental results compared with default containers library implementation. 
Regarding the main aspects where DPP is strong, i.e., pipeline parallelism and time processing, the $\dpwcc$ performance shows that Haskell 
can deal with the requirements for the \acrshort{wcc} problem without penalizing neither execution time nor memory allocation. 
In particular, the $\dpwcc$ implementation outperforms in those cases where the topology of the graph is sparse and where the number of vertices in the largest \acrshort{wcc} is not big enough. We think this work has gathered enough evidence to show that the implementation of Dynamic Pipeline in Haskell Programming Language is feasible. 
This fact opens a wide range of algorithms to be explored using the Dynamic Pipeline Paradigm, supported by purely functional programming language.

\section{Chapter Summary}
In this chapter we have presented an overview of the previous contribution done for \acrshort{prole21}~\cite{prole21} conference, in order to assess the 
feasibility of implementing \acrshort{dp} using \acrshort{hs}. In that sense, we presented an overview of the work, the experiments, and the conclusions.


\chapter{Dynamic Pipeline Framework in Haskell}\label{dp-hs}
\section{Introduction}
One of the fundamental piece of the present work as we have described in ~\autoref{intro} and ~\autoref{prelim},
is the design and implementation of \acrshort{dp} in a Framework written in \acrshort{hs} which allow \acrshort{hs} users
to implement any suitable algorithm in \acrshort{dp}, providing the correct abstractions that helps on that matter.

During the process of conducting this research, we implemented \acrfull{dpl}~\cite{dynamic-pipeline}: 
a \acrfull{dpf} written in \acrshort{hs} and published into Hackage (The public Haskell Package Repository of Libraries).
In this chapter we are going to describe the design and implementation details of \acrshort{dpl}.

\section{Framework Model}
\section{Implementation}
\section{Chapter Summary}


\chapter[An Algorithm for Incrementally Enumerating Bitriangles]{An algorithm for incrementally enumerating Bitriangles using DP}\label{incr-algo-bt-dp}
In this chapter, we first give the foundation to define and implement an algorithm for incrementally enumerating bitriangles in a large bipartite graph. 
Then, we introduce an algorithm based on the \acrshort{dp}. In particular,  we present the pseudo-code of the stages of the proposed $\dpbt$. 
Finally, we provide a proof of correctness of our proposal and the main details of the implementation of the $\dpbt$ using the \acrshort{dpbt}.

\section{Preliminaries Definitions}\label{sec:prem:def}
In order to understand how the algorithm works, we need to provide some basic definitions, and define other small structural units that we use for giving a solution to the problem.
Let's enumerate all those definitions in the following paragraphs.

\begin{definition}[\acrlong{bg}] 
A \acrfull{bg} is an undirected graph $G=(V,E)$  such that $V=(U\cup L)$, $U\cap L=\emptyset$ and $E\subseteq U\times L$.
\end{definition}

Additionally, without loss of generality, we assume that  $U\subseteq \mathbb{N}$ and $L\subseteq \mathbb{N}$. Consequently, $(U,<)$  and $(L,<)$ are strict total orders. We can see an example of a \acrshort{bg}
in \autoref{fig:bipartite-graph-example}.

\begin{figure}[ht]
\centering	
\inputtikz{bipartite}
\caption[{[\acrshort{iebt}] Example of \acrlong{bg}}]{A bipartite graph in which five bitriangles can be enumerated}
\label{fig:bipartite-graph-example}
\end{figure}

Identifying the different bitriangles in the graph in \autoref{fig:bipartite-graph-example} can be done manually. 
In what follows, we use this graph to illustrate the new definitions. 
This small example allows us to realize that identifying and listing bitriangles in large graphs is a challenging task.

\begin{definition}[\acrlong{bt}]\label{def:bt}
Let the triples $\mu=(u_1, u_2, u_3)$ and $\ell=(l_1, l_2,l_3)$ on $U$ and $L$, respectively, i.e.  $\{u_1, u_2, u_3\} \subseteq U$, $\{l_1, l_2,l_3\} \subseteq L$. 
The 6-cycle $(u_1,l_1,u_2,l_3,u_3,l_2,u_1)$  is a \textit{\acrfull{bt}} in $G$, denoted by $BT_{\ell}^{\mu} = \bti$. 
\end{definition}      

\begin{figure}[h!]
\begin{subfigure}[b]{0.5\textwidth}
\centering
\inputtikz{bipartite_bt_a}
\caption{\acrshort{bt} $(1,a,3,c,5,b,1)$}
\end{subfigure}
\begin{subfigure}[b]{0.5\textwidth}
\centering
\inputtikz{bipartite_bt_b}
\caption{\acrshort{bt} $(1,a,7,c,5,b,1)$}
\end{subfigure}
\caption[{[\acrshort{iebt}] Example of bitriangles}]{Bitriangles of the triple $(a,b,c)$ combined with vertices from $U$ $\{1,3,5,7\}$ of \autoref{fig:bipartite-graph-example}. In this case two bitriangles have been built from the \autoref{fig:agg-bt-example}}
\label{fig:bt-a-example}
\end{figure}

A bitriangle $(u_1,l_1,u_2,l_3,u_3,l_2,u_1)$ admits different ways of traversing it depending on the vertex where the traversal starts. This fact gives rise to different feasible permutations of its representation as a 6-cycle. 
For example in the \autoref{fig:bt-a-example} the bitriangle $(1,a,3,c,5,b,1)$, can also be traverse using the permutations $(a,1,b,5,c,3,a)$, $(5,c,3,a,1,b,5)$, $\dots$, and so on.
Notice that all these feasible permutations guarantee the bitriangle condition of vertices intercalation from $U$ and $L$.

\begin{definition}[\acrfull{wg}]\label{def:wg}
A \textit{\acrfull{wg}} in $G$ is a triple $(u_1,l,u_2)$, $\{u_1,u_2\}\subseteq U$, $l \in L$ and $\{(u_1,l),$ $(u_2,l)\} \subseteq E$. The vertex $l$ is the middle vertex of the wedge. 
\end{definition}

With this definition in place, we can now define an aggregated wedge, which is a compressed form of all the wedges having $l \in L$ as a middle vertex.
      
\begin{definition}[\acrfull{awg}]\label{def:awg}
An \textit{\acrfull{awg}} is a pair $\la l, W_l \ra$, where $l \in L$, $W_l \subseteq U$ and for all $u\in W_l$, the edge  $(u,l)\in E$. 
\end{definition}

\begin{figure}[htp!]
\begin{subfigure}[t]{0.3\textwidth}
\centering
\inputtikz{bipartite_awg_a}
\caption[\acrshort{awg} on $a$]{This is the \acrshort{awg} $\la a, \{1,3,7\}\ra$, the compact representation of wedges}
\label{fig:awedge-example-a}
\end{subfigure}\hfill
\begin{subfigure}[t]{0.3\textwidth}
\centering
\inputtikz{bipartite_awg_b}
\caption[\acrshort{awg} on $b$]{This is the \acrshort{awg} $\la b, \{1,5\}\ra$, the compact representation of wedges}
\label{fig:awedge-example-b}
\end{subfigure}\hfill
\begin{subfigure}[t]{0.3\textwidth}
\centering
\inputtikz{bipartite_awg_c}
\caption[\acrshort{awg} on $c$]{This is the \acrshort{awg} $\la c, \{1,3,5,7,9\}\ra$, the compact representation of wedges}
\label{fig:awedge-example-c}
\end{subfigure}
\caption[{[\acrshort{iebt}] Definitions Examples \acrlong{awg}}]{\acrlong{awg} of nodes $a,b,c$ of \autoref{fig:bipartite-graph-example}}
\label{fig:awedge-example}
\end{figure}

In \autoref{fig:awedge-example}, we are only enumerating the first three lower layer vertices \acrshort{awg}.
Next, we define a more refined structure that will allow us to enrich the \acrshort{awg} in order to lead us to a \acrshort{bt}. 
That intermediate structure is an \emph{aggregated double-wedge}. For defining \acrshort{adwg} we first need another structure called \emph{double-wedge}. Intuitively, a \emph{double-wedge}, is similar to a \acrshort{wg} and \acrshort{awg} but relating two vertices in the lower layer $L$.

\begin{definition}[\acrfull{dwg}]
A \textit{\acrfull{dwg}} in $G$ is a path of length 4 $(u_1,l_1,u_2,l_2,u_3)$ where  $\{u_1,$ $u_2,u_3\}\subseteq U$ and $\{l_1,l_2\}\subseteq L$. Vertices $l_1$ and $l_2$ are the middle vertices of \acrshort{dwg}. 
\end{definition}

\begin{figure}[htp!]
\begin{subfigure}[t]{0.5\textwidth}
\centering
\inputtikz{bipartite_dw_a}
\caption[\acrshort{dwg} $(u_1,l_l,u_2,l_u,u_3)$]{A \emph{double-wedge} $(u_1,l_l,u_2,l_u,u_3)$}
\end{subfigure}\hfill
\begin{subfigure}[t]{0.5\textwidth}
\centering
\inputtikz{bipartite_wg_a}
\caption[\acrshort{wg} $(u_1,l_m,u_3)$]{A connector \emph{wedge} $(u_1,l_m,u_3)$ with middle vertex $l_m$}
\end{subfigure}\hfill
%
\begin{subfigure}[t]{1\textwidth}
\centering
\inputtikz{bipartite_bt_dw_wg_a}
\caption[\acrshort{bt} with connector]{A \emph{bitriangle} formed by \emph{double-wedge} $(u_1,l_l,u_2,l_u,u_3)$ and connector \emph{wedge} $(u_1,l_m,u_3)$}
\end{subfigure}
\caption[{[\acrshort{iebt}] Example of connector wedge}]{Example on how to build a bitriangle from a \emph{double-wege} plus a \emph{wedge} connector}
\label{fig:wg-dw-example}
\end{figure}
      
In \autoref{fig:wg-dw-example} the \emph{wedge} $(u_1,l_m,u_3)$, is called the \textit{connector wedge}. 
This connector wedge together with the \emph{double-wedge} $(u_1,l_l,u_2,l_u,u_3)$ form a  a bitriangle.  
      
\begin{definition}[\acrfull{adwg}]\label{def:adwg}
Let $U_l = \la I, J, K\ra$ be a triplet such that $I \subseteq U, J \subseteq U$ and $K \subseteq U$, where $I, J$ and $K$ are disjoint sets. 
An \textit{\acrfull{adwg}}  is a pair  $\la (l_1, l_2), U_l \ra$, where $\{l_1,l_2\}\subseteq L$ and  for all $u_i \in I, u_j \in J$ and $u_k \in K$, $\{(u_i, l_1), (u_j, l_1), (u_j, l_2), (u_k, l_2)\} \in E$.
\end{definition}
      
\begin{figure}[htp!]
\centering
\inputtikz{bipartite_adwg_a}
\caption[{[\acrshort{iebt}] Example Aggregated double-wedge}]{\acrlong{adwg} $\la (a,c), \la \emptyset, \{1,3,7\}, \{5,9\}\ra \ra$ of \autoref{fig:bipartite-graph-example}. Upper layer nodes $1,3,7$ and $5,9$ are enclosed in a square indicating the set that they belong to. Remember in \dref{def:adwg} we are forming three sets $U_l = \la I, J, K \ra$. $I = \emptyset$ in this case because they should be disjoint sets and $a$ and $c$ share the same upper layer nodes in $J$.}
\label{fig:agg-double-wedge-example}
\end{figure}

In \autoref{fig:agg-double-wedge-example} we can see the \acrshort{adwg} built by $a$ and $c$ where, according to \dref{def:adwg}, $a = l_1$ and $c = l_2$.
The last aggregated structure that enable the algorithm to build a \acrshort{bt} is the \emph{\acrlong{abt}}. 
Intuitively, this intermediate structure is an aggregation of a \acrshort{adwg} with another new \acrshort{awg} related with the former structure.

\begin{definition}[\acrfull{awgc}]\label{def:awgc}
Let $\la (l_1, l_2), U_l \ra$ be an \acrshort{adwg}, where $\{l_1,l_2\}\subseteq L$ and $U_l = \la I, J, K\ra$.
An \acrshort{awg} $\la l, W_l \ra$ is an \textit{\acrfull{awgc}} \emph{if and only if} $l > l_1$ and $l < l_2$ and $W_l \cap (I \cup J) \neq \emptyset$ and $W_l \cap (K \cup J) \neq \emptyset$
\end{definition}
      
\begin{definition}[\acrfull{abt}]\label{def:abt}
Let $\hat{U}_l=\la I, J, K\ra$, such that $I \subseteq U, J \subseteq U, K \subseteq U$. An \textit{\acrfull{abt}}  is a pair  $\langle \ell, \hat{U}_l\rangle$, 
where $\ell=(l_1, l_2, l_3)$ is a triple on $L$, $l_1 < l_2 < l_3$ and for all $\la I, J, K\ra$ and for all $\mu=(u_i, u_j, u_k)$ such that $u_i \in I, u_j \in J, u_k \in K$, $BT_{\ell}^{\mu}\in \bt$.
\end{definition}
      
\begin{figure}[h!]
\centering      
\inputtikz{bipartite_abt_a}
\caption[{[\acrshort{iebt}] Example Aggregated bitriangle}]{%
\acrlong{awgc} $\la b, \{1,5\}\ra$ connects to \acrlong{adwg} $\la (a,c), \la \emptyset, \{1,3,7\}, \{5,9\}\ra \ra$ of \autoref{fig:bipartite-graph-example}, constructing the \acrlong{abt} $\la (a,b,c), \la \{1\},\{1,3,7\},\{1,5\} \ra \ra$.
Note that $\hat{U}_{(a,b,c)}=\{I,J,K\}$ does not contains disjoint sets according to \dref{def:abt}. 
}
\label{fig:agg-bt-example}
\end{figure}
      
As we can see in \autoref{fig:agg-bt-example}, we have all the structures at our disposal to build a \acrshort{bt}. 
In \autoref{fig:bt-a-example}, the two possible \acrshort{bt} can be extracted from the \acrshort{abt} presented in \autoref{fig:agg-bt-example} are $(1,a,3,c,5,b,1)$ and $(1,a,7,c,5,b,1)$.

\acrshort{iebt} will allow to enumerate \acrshort{bt} according to some locality criteria.
In this regards, we next define the \emph{query operators} to be used for that purpose.

\begin{definition}[\acrfull{qo}]\label{def:query:match} 
a \emph{\acrlong{qo}} $Q$ is a value from the  \textit{sum type}
$\mathcal{P}(U + L) + \mathcal{P}(E)$ producing as a result a possible set of \acrshort{bt} that include any of the vertices or edges given in $Q$.
\end{definition}

In \autoref{table:notation}, we present a summary of the notation used in this chapter.

\begin{table}[!ht]
\centering
\begin{tabular}{|c|l|} \hline
\textbf{Notation} & \textbf{Meaning}\\ \hline
$G=((U\cup L),E)$ & a bipartite graph\\  \hline
$n,m$ & the number of vertices and edges in $G$, resp.\\  \hline
$(u,l)$ & an edge between vertices $u$ and $l$\\  \hline
$(u_1,l,u_2)$ & a wedge with  middle vertex $l$\\  \hline
$\la l, W_l \ra$ & an \acrshort{awg}\\  \hline
$\aw$ & the set of all the possible \acrshort{awg} in $G$\\  \hline
$(u_1,l_1,u_2,l_2,u_3)$ & a \acrshort{dwg} with middle vertices $l_1$ and $l_2$\\  \hline 
$\la (l_1, l_2), U_l \ra$ & an \acrshort{adwg}\\  \hline
$\dw$ & the set of all the possible \acrshort{adwg} in $G$\\  \hline
$\dwi$ & Subset of $\dw$, such that $\dwi \subseteq \dw$ \\  \hline
$\la (l_1, l_2,l_3), \hat{U}_l \ra$ & an \acrshort{abt}\\  \hline
$\at$ & the set of all the possible \acrshort{abt} in $G$ \\  \hline
$\ati$ & Subset of $\at$, such that $\ati \subseteq \at$ \\  \hline 
$\bti$ & the \acrshort{bt} $u_1,l_1,u_2,l_3,u_3,l_2,u_1$\\  \hline
$\bt$ & the set of all the possible \acrshort{bt} in $G$ \\  \hline
$bt$ & Subset of $\bt$, such that $bt \subseteq \bt$ \\  \hline
$\mathcal{P}$ & Parts of set \\  \hline
\end{tabular}
\caption[{[\acrshort{iebt}] Summary of notations and their meanings}]{This table summarizes all the different object definitions that we have detailed in \autoref{sec:prem:def}. The first column describe the formal term used in the definitions on \autoref{sec:prem:def}. The second column summarize the meaning of each term}
\label{table:notation}
\end{table}
      
\section{Algorithm Sketch}\label{sub:sec:algo-sketch}
The algorithm for enumerating incrementally bitriangles in a large bipartite graph consists in two main phases. 
During the first phase, the bipartite network is received by the $\dpbt$ as a stream of edges and a graph index structure is created. 
This graph index is represented by the different structures stored along the filters stages of the $\dpbt$. 
This first phase is in charge of constructing the compressed structures \emph{double wedge}, \emph{aggregated double-wedge}, and finally \emph{aggregated bitriangles}.
The second phase is a querying phase. During the querying phase local queries can be submitted to the $\dpbt$. 
When a query arrives to the $\dpbt$, the enumeration incremental of bitriangles process --according to the criteria in the submitted  queries-- is launched. 
In this phase, enumerated bitriangles are extracted from the different aggregated bitriangles occurring in the graph index, i.e. the aggregated bitriangles stored in filter stages.
More concretely, the Dynamic Pipeline Algorithm for Enumerating Bitriangles ($\dpbt$) is defined in terms of the behavior of its four kinds stages: \textit{Source} ($\ibt$),  
\textit{Generator} ($\gbt$),  \textit{Sink} ($\obt$), and \textit{Filter}($\fbt$) stages. 
The algorithm considers a \acrlong{bt} as a convenient composition of three wedges as we can see the example in \autoref{fig:bitriangle-example}.
In order to reduce memory footprint, the algorithm aggregates results, i.e. the set of wedges having the same middle vertex is represented as a pair $\la l, W_l \ra$ where $l$ is the middle vertex and $W_l$ is the set of adjacent vertices of $l$ called \acrfull{awg} (see \dref{def:awg}).
The algorithm first collects \acrshort{awg} for every vertex in the $L$ set of the graph. Afterwards it constructs \acrfull{adwg} for every pair of distinct vertices, Finally  constructs \acrshort{abt}  for selected triples of vertices. 
The following \autoref{table:channels} describes the different channels that are connecting the stages in $\dpbt$.

\begin{table}[ht!]
\centering
\begin{tabular}{|p{0.2\linewidth}|p{0.8\linewidth}|} \hline
\textbf{Channel} & \textbf{Meaning}\\ \hline
$C = \la IC, OC \ra$ & A Channel pair that connects input and output channel\\ \hline
$OC$ & Set of Output Channels \\ \hline
$C_E$ & Channel of $e \in E$ \\ \hline
$IC_E$ & Input Channels carrying $e \in E$ \\ \hline
$OC_E$ & Output Channels carrying $e \in E$ \\ \hline
$C_{W_l1}$ & Channel of \acrshort{awg} \\ \hline
$IC_{W_l1}$ & Input Channels carrying \acrshort{awg} \\ \hline
$OC_{W_l1}$ & Output Channels carrying \acrshort{awg} \\ \hline
$C_{W_l2}$ & Channel of \acrshort{awg} \\ \hline
$IC_{W_l2}$ & Input Channels carrying \acrshort{awg} \\ \hline
$OC_{W_l2}$ & Output Channels carrying \acrshort{awg} \\ \hline
$C_Q$ & Channel of \acrshort{qo} \\ \hline
$IC_Q$ & Input Channels carrying \acrshort{qo} \\ \hline
$OC_Q$ & Output Channels carrying \acrshort{qo} \\ \hline
$C_{BT}$ & Channel of $\bt$ \\ \hline
$IC_{BT}$ & Input Channels carrying $\bt$ \\ \hline
$OC_{BT}$ & Output Channels carrying $\bt$ \\ \hline
\end{tabular}
\caption[{[\acrshort{iebt}] Summary of Channels used in \acrshort{dpbt}}]{Summary of Channels used in \acrshort{dpbt}. The subindex on the Channel name indicates the element type this channel is carrying, either producing or consuming. Channels prefixed with $C$ are a generic form of denominating a channel independently of it is a producing or consuming. Channels prefixed with $IC$ are Input Channels or Consumers. Channels prefixed with $OC$ are Output Channels or Producers.}
\label{table:channels}
\end{table}

 
\begin{figure}[h]
\centering  
\resizebox{1\textwidth}{!}{%
\inputtikz{btDP}
}
\caption[{[\acrshort{iebt}] $\dpbt$ Initial setup}]{Example of an initial setup of $\dpbt$. The text between dotted lines indicates the incoming data from an external source such as a file, socket, or any other. There are two incoming sources because one of them is carrying the edges and the other the commands \acrshort{qo}. Thick black double lines at the most right indicate the output targe that can be file, socket, screen, or any other. There is no data there because it is the initial step of \acrshort{dpbt}. At this initial step only $\ibt$, $\gbt$, and $\obt$ are set up with initial channels.}
\label{fig:btDP}
\end{figure}

The setup of $\dpbt$ can be appreciated in \autoref{fig:btDP}. $\ibt$ reads from input stream all $(u,l) \in E$ and transfers to the following stage each $(u,l)$ using $C_E$.
For every $(u,l)$ that arrives to $\gbt$, a new $\fbt$ instance with parameter $l \in L$ is spawn. $\fbt$ contains four actors. 
First, $\aaa$ receives from $IC_E$ the edges and builds aggregated wedges, when there are no more edges it downstreams it to it's neighbour $\fbt$ using $OC_{W1}$. 
Then $\ab$ receives from $IC_{W1}$ aggregated wedges from previous $\fbt$, downstream to next $\fbt$ and $\gbt$ using $OC_{W1}$ and at the same time use its information to see 
if it can build an aggregated double-wedge. If an aggregated double-wedge could be constructed, it will be stored in the filter state.
The need of bypassing all the aggregated wedges up to $\gbt$ is for retro feeding all the pipeline with all the aggregated wedges a second time, in order to find an \acrfull{awgc} as we defined in \dref{def:awgc}.
From retro feeding channel $IC_{W2}$, $\ac$ receives all the \acrshort{awg}  and builds \acrshort{abt} storing them in $\st$. 
$Q$ Commands are downstream from $\ibt$ to $\ad$ using channel $IC_Q$. $\ad$ receives all the commands, and for each of them, if there is match according to \dref{def:query:match}, it enumerates those \acrshort{bt} extracted from $\st$ and downstream to $\obt$ through $OC_{BT}$.

\section{Dynamic Pipeline for Enumerating Bitriangles}\label{sub:sec:iebt:dpalgo}
In this section, we present all the pseudo-code definitions of each of the stages described in ~\autoref{sub:sec:algo-sketch}. 
In spite of this work has been implemented using \acrshort{hs}, and in particular \acrshort{dpfh}, the pseudo-code algorithms presented here are language independent.
Before starting with the details, we introduce in \autoref{table:aux:fn} auxiliary functions that we use in the pseudo-code to help understanding better the desired behavior of the algorithm.

\begin{table}[!htp]
\centering
\begin{tabular}{|p{0.3\linewidth}|p{0.7\linewidth}|} \hline
\textbf{Function} & \textbf{Meaning}\\ \hline
$\sw(F,l,\st)$ & Spawn new filter instance with parameters $F$, $l \in L$ and $\st$ as the State of the Filter\\\hline
$\fd$ & Kill this filter instance because PostCondition is not satisfied\\ \hline
$\fid$ & State after calling $\fd$ on filter. Indicates if Filter is die or not. If it is dead, this filter instance does not participate anymore in the pipeline streaming processor\\ \hline
$\gs$ & Get Current State $\st$ for Filter Instance \\ \hline
$\us(\st)$ & Update Current State $\st$ for Filter Instance \\ \hline
$\p(\mathtt{v}, OC_x)$ & push some value $\mathtt{v}$ to some Output Channel $OC_x$ \\\hline
$\mt(Q, BT)$ & Check if a \acrshort{qo} $Q$ produces $\bt$ \\ \hline
\end{tabular}
\caption[{[\acrshort{iebt}] Summary of auxiliary functions for handling \acrshort{dpbt} internals}]{This table shows some auxiliary functions that are used to help understanding the pseudo-code general behavior. Depending on the language implementation chosen these functions might not exists at all, but we generalize here in order to describe a pseudo-code language-independent. For example in our \acrshort{hs} implementation there is not $\fd$ because functions in \acrshort{hs} clean after execution finish automatically by \acrshort{ghc}}
\label{table:aux:fn}
\end{table}
      
\begin{algorithm}[h!]
\SetKwInOut{P}{Input Data}
\SetKwInOut{Q}{Input Commands}
\SetKwInOut{IC}{Input Channels}
\SetKwInOut{OC}{Output Channels}
\SetAlgorithmName{}{src}{}
\SetAlgoRefName{[A1]}
\P{$IO_E$: File or Input Stream with Set of Edges $E$}
\Q{$IO_Q$: File or Input Stream with \acrlong{qo} $Q$}
\IC{$IC = \la IC_{W_l2} \ra$}
\OC{$ OC = \la OC_E, OC_{W_l1}, OC_{W_l2}, OC_Q, OC_{BT} \ra $}
\ForAll(\tcp*[f]{Edges to Generator/Filter}){$(u,l) \in IO_E$}
{$\p((u,l), OC_E)$ \label{algo:source:1}
}
\ForAll(\tcp*[f]{Feedback from Generator to Filter}){$\la l, W_l \ra \in IC_{W_l2}$}
{$\p(\la l, W_l \ra, OC_{W_l2})$ \label{algo:source:2}
}
\ForAll(\tcp*[f]{Send Query Commands}){$Q \in IO_Q$}
{$\p(Q, OC_Q)$ \label{algo:source:3}
}
\caption[Source ($\ibt$)]{Source ($\ibt$): It process all the edges from the file or input channel $IO_E$ and send to the following stages. It also receives from $IC_{W_l2}$ all the feedback \acrshort{awg} that is sending back the $\gbt$. At the end it process also from other File or Input stream the Query Command $Q$ to be sent to th filters}
\label{algo:source}
\end{algorithm}

\paragraph{Source $\ibt$} In \autoref{algo:source} we can see in \autoref{algo:source:1} how the edges arriving from the input stream of the graph are downstream to the pipeline. 
Another important part as well is \autoref{algo:source:2}, the $\ibt$ is retro feed with \acrshort{awg} stream that is generated during pipeline execution.
This is important to finally build the \acrshort{bt} as we have describe in \autoref{sub:sec:algo-sketch}. Finally, \autoref{algo:source:3} shows how all the queries are downstream as well.

\begin{algorithm}[h!]
\SetKwInOut{P}{Parameter}
\SetKwInOut{IC}{Input Channels}
\SetKwInOut{OC}{Output Channels}
\SetAlgorithmName{}{gen}{}
\SetAlgoRefName{[A2]}
\P{$F$}
\IC{$ IC = \la IC_E, IC_{W_l1}, IC_{W_l2}, IC_Q, IC_{BT} \ra $}
\OC{$ OC = \la OC_{W_l2}, OC_{BT} \ra$}
\ForAll{$(u,l) \in IC_E$}
{$\sw(F, l, \la l, \{u\} \ra)$ \label{algo:gen:1}
}
\ForAll(\tcp*[f]{Feedback channel to retrofit Source}){$\la l, W_l \ra \in IC_{W_l1}$}
{$\p(\la l, W_l \ra, OC_{W_l2})$ \label{algo:gen:2}
}
\ForAll{$\bti \in IC_{BT}$}
{$\p(\bti, OC_{BT})$ \label{algo:gen:3}
}
\caption[Generator ($\gbt$)]{Generator ($\gbt$): For each edge $(u,l)$ it receives from previous stage, it spawn a new filter using $l$ as parameter of the Filter and $\{u\}$ as the state. It also receives all the \acrshort{awg} that previous filters built and sends back to $\ibt$. Finally it sends to the $\obt$ the \acrshort{bt} matched by filters according to Command Query $Q$}
\label{algo:gen}
\end{algorithm}

\begin{figure}[h]
\centering  
\resizebox{1\textwidth}{!}{%
\inputtikz{btDP_actor1}
}
\caption[{[\acrshort{iebt}] ]$\dpbt$ With Filter instances}]{This is the evolving state of the $\dpbt$ shows in \autoref{fig:btDP}. This image is showing what happen when a $\fbt$ is spawned. We can see how all the channels are set up in the middle of the spawned $\fbt$ and also between $\fbt$ and $\gbt$.}
\label{fig:btDP_actor1}
\end{figure}

\paragraph{Generator $\gbt$} $\gbt$ also have three main loops. In \autoref{algo:gen:1} it receives each edge $(u,l)$ not consumed by any already spawn  $\fbt$, and 
spawns a new $\fbt$ using $l \in L$ as filter parameter and initializing $\st = \la l, \{u\} \ra$ as it can be seen in \autoref{fig:btDP_actor1}. 
Spawn assumes that, the implementation will connect the channels to keep the downstream correct. Defines all the input channels of $\gbt$ as input channels of the newly spawn $\fbt$ and set $\fbt$'s output channels as $\gbt$ input channels.
In \autoref{algo:gen:2} we can see how the algorithm is receiving and retro feeding $\ibt$ with all the \acrshort{awg} produced by the $\fbt$ . This also assumes that $OC_{W2}$ in $\gbt$ 
is connected with $IC_{W2}$ in $\ibt$.
Finally, \autoref{algo:gen:3} sends all the results that $Q$ matches in the different filters to $\obt$.

\begin{algorithm}[h!]
\SetKwInOut{IC}{Input Channels}
\SetKwInOut{O}{Output}
\SetAlgorithmName{}{sink}{}
\SetAlgoRefName{[A3]}
\O{$IO_{BT}$: File or Output Stream with $BT$}
\IC{$IC = \la IC_{BT} \ra$}
\ForAll(\tcp*[f]{Read Results and put in $IO_{BT}$}){$\bti \in IC_{BT}$}
{$put(\bti, IO_{BT})$
}
\caption[Sink ($\obt$)]{Sink ($\obt$): It receives \acrshort{bt} from $\gbt$ and send to the File or Output Stream}
\label{algo:sink}
\end{algorithm}

\paragraph{Sink $\obt$} In \autoref{algo:sink} shows a simple stage that receives all results and sends them to some output handler (file, standard output, etc.). 

\begin{algorithm}[h!]
\SetKwInOut{P}{Filter Parameter}
\SetKwInOut{FS}{Filter State}
\SetKwInOut{IC}{Input Channels}
\SetKwInOut{OC}{Output Channels}
\SetKwFunction{actora}{actor1}
\SetKwFunction{actorb}{actor2}
\SetKwFunction{actorc}{actor3}
\SetKwFunction{actord}{actor4}
\SetKwFunction{filter}{filter}
\SetKwProg{df}{def}{:}{end}
\SetAlgorithmName{}{fil}{}
\SetAlgoRefName{[A4]}
\P{$l \in L$}
\FS{$\st = \aw + \mathcal{P}(\dw) + \mathcal{P}(\at)$}
\IC{$ IC = \la IC_E, IC_{W_l1}, IC_{W_l2}, IC_Q, IC_{BT} \ra $}
\OC{$ OC = \la OC_E, OC_{W_l1}, OC_{W_l2}, OC_Q, OC_{BT} \ra $}
\df{\filter{}}{
      $\actora()$\\
      $\actorb()$\\
      $\actorc()$\\
      $\actord()$\\
}
\caption[Filter ($\fbt$)]{Filter ($\fbt$): Call sequentially to all the actors in this filter}
\label{algo:fil}
\end{algorithm}

\paragraph{Filter $\fbt$} In \autoref{algo:fil} we can see the simple call sequence over all the actor functions of the filter template. 
The values stored in the \textit{state of the filter} with parameter $l$ are of the sum type $\st = \aw + \mathcal{P}(\dw) + \mathcal{P}(\at)$.

\begin{algorithm}[h!]
\SetKwInOut{P}{Parameter}
\SetKwInOut{IC}{Input Channels}
\SetKwInOut{OC}{Output Channels}
\SetKwInOut{FS}{$\st$}
\SetKwInOut{PC}{Post-Cond}
\SetKwInOut{PrC}{Pre-Cond}
\SetKwFunction{acta}{actor1}
\SetKwProg{df}{def}{:}{end}
\SetAlgorithmName{}{fa1}{}
\SetAlgoRefName{[A5]}
\P{$l \in L$}
\FS{$\la l, W_l \ra$}
\IC{$ IC = \la IC_E, IC_{W_l1}, IC_{W_l2}, IC_Q, IC_{BT} \ra $}
\OC{$ OC = \la OC_E, OC_{W_l1}, OC_{W_l2}, OC_Q, OC_{BT} \ra $}
\PC{$|W_l| > 1 \lor \fid$}
\BlankLine
\df{\acta{}}{
$\la l, W_l \ra \leftarrow \gs$\\
\ForAll{$(u',l') \in IC_E$}
{\uIf{$l = l'$}{
      $W_l \leftarrow W_l \cup \{u'\}$\label{algo:act-1:1}
}\Else{$\p((u',l'),OC_E)$}
}
\uIf{$|W_l| > 1$}{
      $\us(\la l, W_l \ra)$\\ \label{algo:act-1:2}
      $\p(\la l, W_l \ra, OC_{W_l1})$\\
}\Else{$\fd$}
}
\caption[Actor1 ($actor_1$)]{Actor1 ($actor_1$): Build a set of aggregated wedges. This is, $W_l \subseteq U$ adjacent to $l$ Filter parameter. For each received edge $(u',l')$ which $l \neq l'$ by pass the edge to next filters. It updates the State of the filter with $W_l$ if it could build a $W_l$ with more than 1 vertex in $U$.}
\label{algo:act-1}
\end{algorithm}

\paragraph{Filter-$\aaa$} 
$\aaa$ receives all edges not consumed by previous $\fbt$.
If the received edge $(u',l')$ is incident to l i.e. $l = l'$,  then $u'$ will be added to the list of \acrshort{awg}. Otherwise the edge is downstream to the next stage.
In \autoref{algo:act-1:2} the state is updated and \acrshort{awg} downstream, if and only if at least 1 wedge was collected. Otherwise, the filter is marked as dead using the  $\fd$ function.

\begin{algorithm}[h!]
\SetKwInOut{P}{Parameter}
\SetKwInOut{IC}{Input Channels}
\SetKwInOut{OC}{Output Channels}
\SetKwInOut{FS}{$\st$}
\SetKwInOut{PC}{Post-Cond}
\SetKwInOut{PrC}{Pre-Cond}
\SetKwFunction{actb}{actor2}
\SetKwProg{df}{def}{:}{end}
\SetAlgorithmName{}{fa2}{}
\SetAlgoRefName{[A6]}
\P{$l \in L$}
\FS{$\la l, W_l \ra$}
\IC{$ IC = \la IC_E, IC_{W_l1}, IC_{W_l2}, IC_Q, IC_{BT} \ra $}
\OC{$ OC = \la OC_E, OC_{W_l1}, OC_{W_l2}, OC_Q, OC_{BT} \ra $}
\BlankLine
\PrC{$W_l \subseteq U, |W_l| > 1$}
\PC{$|\dwi| \geq 1 \lor \fid$}
\df{\actb{}}{
$\la l, W_l \ra \leftarrow \gs$\\
\ForAll{$\la l', W_l' \ra \in IC_{W_l1}$}{
      \tcp*[h]{Send Wedge from previous filters to next one}
      $\p(\la l', W_l \ra, OC_{W_l1})$\\
      $\dwi \leftarrow \emptyset$\\
      \If{$W_l' \cap W_l \neq \emptyset$}{ \label{algo:act-2:1}
            $(l_l, l_u) \leftarrow (\argmin_{l,l'}, \argmax_{l,l'})$\\
            \uIf{$l < l'$}{
                  $(W_{l_l}, W_{l_u}) \leftarrow (W_l, W_l')$
            }\Else{
                  $(W_{l_l}, W_{l_u}) \leftarrow (W_l', W_l)$
            }
            $I \leftarrow W_{l_l} \setminus W_{l_u}$\\ \label{algo:act-2:2}
            $J \leftarrow W_{l_l} \cap W_{l_u}$\\
            $K \leftarrow W_{l_u} \setminus W_{l_l}$\\ \label{algo:act-2:3}
            $U_l \leftarrow \la I, J, K\ra$\\
            $\dwi \leftarrow dw \cup \{\la (l_l, l_u), U_l \ra\}$
      }
}
\uIf{$\dwi = \emptyset$}{$\fd$}
\Else{$\us(\dwi)$}
}
\caption[Actor2 ($actor_2$)]{Actor2 ($actor_2$): Receiving all aggregated wedges from previous filters, build a set of all possible aggregated double-wedges $\dwi = \{\la (l,l'), U_l \ra\}, \dwi \subseteq \dw$, which first component $l$ is smallest between the Parameter of the Filter and the vertices from the incoming wedges. At the end, it updates the State of the filter with $\dwi$ if $\dwi \neq \emptyset$}
\label{algo:act-2}
\end{algorithm}

\paragraph{Filter-$\ab$} In \autoref{algo:act-2} \acrshort{adwg} is built. Following that idea and according to the \dref{def:adwg}, this algorithm will collect all the \acrshort{awg} 
from previous filters if an only if the condition in \autoref{algo:act-2:1} is met. Note that the two \acrshort{awg}, $W_l$ and $W_l'$ have different vertices from the set $L$. $W_l$ intersection check is mandatory since if \acrshort{awg} are disjoint, we cannot aggregate them.
After this checking from \autoref{algo:act-2:2} to \autoref{algo:act-2:3}, the algorithm builds three disjoint Sets to separate upper edges in three subsets; those which are incident only of 
$l$ and $l'$ which are $I$ and $K$ and those that are shared by both lower layer vertices. This can be appreciated in \autoref{fig:agg-double-wedge-example}.
Once \acrshort{adwg} is built $\st = \dwi$ updating the state for the next $\ac$.


\begin{algorithm}[h!]
\DontPrintSemicolon
\SetKwInOut{P}{Parameter}
\SetKwInOut{IC}{Input Channels}
\SetKwInOut{OC}{Output Channels}
\SetKwInOut{FS}{$\st$}
\SetKwInOut{PC}{Post-Cond}
\SetKwInOut{PrC}{Pre-Cond}
\SetKwFunction{actc}{actor3}
\SetKwProg{df}{def}{:}{end}
\SetAlgorithmName{}{fa3}{}
\SetAlgoRefName{[A7]}
\P{$l \in L$}
\FS{$\dwi \subseteq \dw$}
\IC{$ IC = \la IC_E, IC_{W_l1}, IC_{W_l2}, IC_Q, IC_{BT} \ra $}
\OC{$ OC = \la OC_E, OC_{W_l1}, OC_{W_l2}, OC_Q, OC_{BT} \ra $}  
\BlankLine
\PrC{$|\dwi| \geq 1$}
\PC{$|\ati| \geq 1 \lor \fid$}
\df{\actc{}}{
      $\dwi \leftarrow \gs$\\
      $\ati \leftarrow \emptyset$\\
      \ForAll{$\la l', W_l \ra \in IC_{W_l2}$}{
            \tcp*[h]{By pass to be used by following Filters}
            $\p(\la l', W_l \ra, OC_{W_l2})$\\ 
            \tcp*[h]{For each double wedge in State}\\
            \ForEach{$\la (l_l, l_u), \la I, J, K \ra \ra \in \dwi, l_l < l' \land l_u > l'$}{
                  $I' \leftarrow I \cup J$\\
                  $K' \leftarrow K \cup J$\\
                  \If{$W_l \cap I' \neq \emptyset \land W_l \cap K' \neq \emptyset$}{
                        $I' \leftarrow I' \cap W_l$\\
                        $K' \leftarrow K' \cap W_l$\\
                        $\hat{U}_l  \leftarrow \la I', J, K' \ra$\\
                        $\ati \leftarrow \ati \cup \big\{\la (l_l, l', l_u), \hat{U}_l \ra\big\}$
                  }
            }
      }
      \uIf{$\ati = \emptyset$}{$\fd$}
      \Else{$\us(\ati)$\\}
}
\caption[Actor3 ($actor_3$)]{Actor3 ($actor_3$): Receiving all aggregated wedges that came from feedback channel, build a Set of all possible Aggregated bitriangles $\ati = \{\la (l_l, l_m, l_u), U_l \ra\}, \ati \subseteq \at$, , such that $l = l_l \lor l = l_u$, where $l$ is the Filter Parameter and $l_m$ is the middle vertex of the incoming wedge. At the end, it updates the State of the filter with $\ati$ if $\ati \neq \emptyset$}
\label{algo:act-3}
\end{algorithm}

\paragraph{Filter-$\ac$} $\ac$ focuses on treating elements from feedback channel $IC_{W2}$ which is going to downstream all the \acrshort{awg} of all filters.
This is because in order to build \acrshort{abt} finding all the possibles \acrshort{awgc}. This is what is doing \autoref{algo:act-3} according to definition 
\dref{def:awgc} and \dref{def:abt}. If that can be achieved, algorithm sets $\st = \ati$ and $\ad$ can be executed.

\begin{algorithm}[h!]
\SetKwInOut{P}{Parameter}
\SetKwInOut{IC}{Input Channels}
\SetKwInOut{OC}{Output Channels}
\SetKwInOut{FS}{$\st$}
\SetKwInOut{PC}{Post-Cond}
\SetKwInOut{PrC}{Pre-Cond}
\SetKwFunction{actc}{actor4}
\SetKwFunction{butV}{buildBtVertex}
\SetKwFunction{butE}{buildBtEdge}
\SetKwProg{df}{def}{:}{end}
\SetAlgorithmName{}{fa4}{}
\SetAlgoRefName{[A8]}
\P{$l \in L$}
\FS{$\ati \subseteq \at$}
\IC{$ IC = \la IC_E, IC_{W_l1}, IC_{W_l2}, IC_Q, IC_{BT} \ra $}
\OC{$ OC = \la OC_E, OC_{W_l1}, OC_{W_l2}, OC_Q, OC_{BT} \ra $}  
\BlankLine
\PrC{$|\ati| \geq 1$}
\df{\actc{}}{
$\ati \leftarrow \gs$\\
\ForAll{$Q \in IC_Q$}{
      \ForEach{$\la (l_l, l_m, l_u), \la I,J,K \ra \ra \in \ati$}{
            \Switch{$Q$}{
                  \Case{$\mathcal{P}(U + L)$}{
                        \If{$\mathcal{P}(U + L) \cap \{l_l, l_m, l_u\} \neq \emptyset \lor \mathcal{P}(U + L) \cap (I \cup J \cup K) \neq \emptyset$}{
                              $\btii \leftarrow \butV(\la (l_l, l_m, l_u), \la I,J,K \ra \ra), \mathcal{P}(U + L))$\\
                              \ForAll{$\bti \in \btii$}{
                                    $\p(\bti, OC_{BT})$
                              }
                        }
                  }
                  \Case{$\mathcal{P}(E)$}{
                        \ForEach{$(u,l) \in \mathcal{P}(E)$}{
                              \If{$(l = l_l \lor l = l_m \lor l = l_u) \land u \in (I \cup J \cup K)$}{
                                    $\btii \leftarrow \butE(\la (l_l, l_m, l_u), \la I,J,K \ra \ra), (u,l))$\\
                                    \ForAll{$\bti \in \btii$}{
                                          $\p(\bti, OC_{BT})$
                                    }
                              }
                        }
                  }
            }
      }
}
}
\caption[Actor4 ($actor_4$)]{Actor4 ($actor_4$): Receives all the \acrlong{qo} $Q$ that arrives from channel $IC_Q$. For each \acrshort{qo} $Q$ it builds the bitriangles from $\ati \subseteq at$ according to the query, and downstream to channel $OC_{BT}$ to be processed by the Sink}

\label{algo:act-4}
\end{algorithm}

\paragraph{Filter-$\ad$} Once the execution reaches $\ad$, it is ready for processing $Q$ \acrlong{qo}. 
Since we have a compressed representation of \acrshort{bt}, which is a similar idea exposed here~\cite{Lai}, we need to enumerate incrementally all the \acrshort{bt} present in the compressed format and filter the ones that match the command.
The matches proceeds in the following manner. Given a \acrshort{abt} with the form $\la \ell, \hat{U}_l \ra$, such that $\ell = (l_1,l_2,l_3)$ and $\hat{U}_l = \la I,J,K \ra$, where $I \subseteq U, J \subseteq U, K \subseteq U$, if the \acrshort{qo} contains a command with vertices, those vertices are searched in $\ell$ and $\hat{U}_l$. 
There are two posibilities. On the first case, if any of those vertices that belongs to $Q$ matches $\ell$, then forall $u_i \in I, u_j \in J, u_k \in K$ where $ u_i \neq u_j \neq u_k$, a \acrshort{bt} is built using a 6-cycle $(u_i, l_1, u_j, l_3, u_k, l_2, u_i)$ path.
On the second case, if any of those vertices that belongs to $Q$ matches on some vertex in $\hat{U}_l$, build only the \acrshort{bt} in which 6-cycle $(u_i, l_1, u_j, l_3, u_k, l_2, u_i)$ path contains that vertex in $Q$.
Using the same construction process a \acrshort{bt} is built if \acrshort{qo} contains a command with edges and any of those edges matches with any of the possible \acrshort{bt} 6-cycle path.

The following pseudo-code definitions on \autoref{algo:buildBtVertex} and \autoref{algo:buildBtEdge} are auxiliary functions that are called
from $\ad$ if $Q$ pattern match either with $\mathcal{P}(U + L)$ or $\mathcal{P}(E)$.

\begin{algorithm}[h!]
\SetKwInOut{I}{Input}
\SetKwInOut{O}{Output}
\SetKwFunction{but}{buildBtVertex}
\SetKwProg{df}{def}{:}{end}
\SetAlgorithmName{}{a1}{}
\SetAlgoRefName{[A9]}
\I{$\la (l_l, l_m, l_u), \la I,J,K \ra \ra \in \ati, \ati \subseteq \at$}
\I{$\mathcal{P}(U + L)$}
\O{$\btii \subseteq \bt$ or $\emptyset$ if cannot build any $\btii$}
\df{\but{$\la (l_l, l_m, l_u), \la I,J,K \ra \ra$, $\mathcal{P}(U + L)$}}{
      $\btii \leftarrow \emptyset$\\
      \uIf{$\mathcal{P}(U + L) \cap \{l_l, l_m, l_u\}$}{
            \tcp*[h]{If it is in lower i need to build all for this lower triplet}\\
            \ForEach{$i \in I$}{
                  \ForEach{$j \in J, j \neq i$}{
                        \ForEach{$k \in K, k \neq i \land k \neq j$}{
                              $\btii \leftarrow \btii \cup \{BT_{(l_l, l_m,l_u)}^{(i, j, k)}\}$
                        }
                  }
            }
      }\Else{
            \tcp*[h]{Otherwise just build those that are in the this upper $v$}\\
            \ForEach{$i \in I$}{
                  \ForEach{$j \in J, j \neq i$}{
                        \ForEach{$k \in K, k \neq i \land k \neq j \land (\mathcal{P}(U + L) \cap \{i,j,k\} \neq \emptyset$}{
                              $\btii \leftarrow \btii \cup \{BT_{(l_l, l_m,l_u)}^{(i, j, k)}\}$
                        }
                  }
            }
      }
      \Return{$\btii$}
}
\caption[Function \mintinline{shell}{buildBtVertex}]{Function \mintinline{shell}{buildBtVertex}: Given a Set of Vertex in $\mathcal{P}(U + L)$, if any of those vertices matches with some vertex in the parameter of the function $\la (l_l, l_m, l_u), \la I,J,K \ra \ra$, build the set of bitriangles that matches that query}
\label{algo:buildBtVertex}
\end{algorithm}

\begin{algorithm}[h!]
\SetKwInOut{I}{Input}
\SetKwInOut{O}{Output}
\SetKwFunction{but}{buildBtEdge}
\SetKwFunction{he}{hasEdge}
\SetKwProg{df}{def}{:}{end}
\SetAlgorithmName{}{a2}{}
\SetAlgoRefName{[A10]}
\I{$\la (l_l, l_m, l_u), \la I,J,K \ra \ra \in \ati, \ati \subseteq \at$}
\I{$(u,l)$}
\O{$\btii \subseteq \bt$ or $\emptyset$ if cannot build any $\btii$}
\df{\but{$\la (l_l, l_m, l_u), \la I,J,K \ra \ra$, $\mathcal{P}(U + L)$}}{
      $\btii \leftarrow \emptyset$\\
      \ForEach{$i \in I$}{
            \ForEach{$j \in J, j \neq i$}{
                  \ForEach{$k \in K, k \neq i \land k \neq j \land \he((u,l), (l_l,l_m,l_u),(i,j,k))$}{
                        $\btii \leftarrow \btii \cup \{BT_{(l_l, l_m,l_u)}^{(i, j, k)}\}$
                  }
            }
      }
      \Return{$\btii$}
}
\df{\he{$(u,l), (l_l,l_m,l_u), (i,j,k)$}}{
      \If{$(u,l) = (l_l, i) \lor (u,l) = (l_l, j) \lor (u,l) = (l_u, j) \lor (u,l) = (l_u, k) \lor (u,l) = (l_m, i) \lor (u,l) = (l_m, k)$}{
            \Return{True}
      }\Else{
            \Return{False}
      }
}
\caption[Function \mintinline{shell}{buildBtEdge}]{Function \mintinline{shell}{buildBtEdge}: Given a Set of Edges in $\mathcal{P}(E)$, if any of those edges matches with some edge in the parameter of the function $\la (l_l, l_m, l_u), \la I,J,K \ra \ra$, build the set of bitriangles that matches that query}
\label{algo:buildBtEdge}
\end{algorithm}

\iffalse
\clearpage
\section{DRAFT - Complexity Analysis of Algorithm}
The complexity analysis provided in this section will cover each of the principal pseudo-code algorithms described in the previous \autoref{sub:sec:iebt:dpalgo}, this is \autoref{algo:source}, \autoref{algo:gen}, \autoref{algo:sink} and \autoref{algo:fil} (including all actors). 

\begin{theorem}[$|\aw| \leq |L|$]\label{theorem:awg}
The size of all possible aggregated wedges is upper bound by $|L|$.
\end{theorem}
\begin{proof}[Anlaysis]
This can be deduced by \dref{def:awg}, because at most each $l \in L$ is participating on a wedge.
\end{proof}

\begin{theorem}[$|\dw| \leq \binom{|L|}{2}$]\label{theorem:adwg}
The size of all possible aggregated double-wedges is upper bound by $\binom{|L|}{2}$.
\end{theorem}
\begin{proof}[Anlaysis]
This can be deduced by \dref{def:adwg}, and by \autoref{theorem:awg}. If we can combine vertices from $L$ taken by $2$ without repetition that forms an \acrshort{adwg}, we have at most that amount of possible aggregated double-wedges combinations.
\end{proof}
      
\begin{complexity}[Source $\ibt$]\label{prop:comp-src}
The total complexity of $\ibt$ algorithm in worst-case is $O(|E|)$.
\end{complexity}
\begin{proof}[Anlaysis]
In the case of this stage, we have three \texttt{for} loops. The first one is done in the number of edges of the graph $|E|$. The second \texttt{for} loop iterates over  
all possible \acrshort{awg} which is at most $|L|$ according to \autoref{theorem:awg}.
And the last \texttt{for} loop is taking the size of all commands $Q$ provided by the user. Then, worst-case complexity is $O(|E| + |L| + |Q|)$.
Since $|L| \leq |E|$ by \dref{def:bt} and $|Q| << |E|$ because otherwise the user will be able to manually inspect the whole graph. Since the algorithm implements a \emph{pay-as-you-go} model
as we have described in \autoref{relate-work}, it has a high cost for the user to ask for an amount of commands near to $|E|$.
Therefore, the dominant term is $|E|$, and the complexity of this part of the algorithm is $O(|E|)$.
\end{proof}

\begin{complexity}[Generator $\gbt$]\label{comp:anal:gen}
The total complexity of $\gbt$ algorithm in worst-case is $O(2|L| + |Q||\ati|)$.
\end{complexity}
\begin{proof}[Anlaysis]
$\gbt$ receives at most $|L|$ edges in in \autoref{algo:gen:1} first loop, since $\aaa$ will collect all the upper vertices from the same $l$ without bypassing those collected edges (\autoref{theorem:awg}).
In \autoref{algo:gen:2}, the second loop, it will receive at most $|L|$ aggregated wedges $W_l$, which also can be deduced by \autoref{theorem:awg}.
Finally, it will receive all matched $\btii \subseteq \bt$. By \autoref{algo:act-4}, \autoref{algo:buildBtEdge} and \autoref{algo:buildBtVertex} the amount of possible $\btii$ matched is $|Q| \times |\ati|$ (see \dref{def:abt}).
Then worst-case complexity is $O(|L| + |L| + |Q||\ati|) = O(2|L| + |Q||\ati|)$.
\end{proof}

\begin{complexity}[Sink $\obt$]
The total complexity of $\obt$ algorithm in worst-case is $O(|Q||\ati|)$.
\end{complexity}
\begin{proof}[Anlaysis]
Obvious by definition of \autoref{algo:sink} and by complexity analysis in \autoref{comp:anal:gen}.
\end{proof}

\paragraph{Filter Consideration}\label{comp:an:filter:note} In the case of complexity analysis of the Filter instances $\fbt$ algorithms, which involves the four actors, we are going to take into consideration the worst-case.
The worst-case is the first filter, because it is the one that is going to receive all the edges from the $\ibt$ at least as we can see in \autoref{sub:sec:algo-sketch}. 
If the second filter is spawned, is going to receive at most $|E|-2$ edges, the third $|E|-3$, and so on. 

\begin{complexity}[Filter $\fbt$]
The total complexity of $\fbt$ algorithm in worst-case is $O(|E| + |L|^2 \times 6|U| + |Q||\ati|)$.
\end{complexity}
\begin{proof}[Anlaysis]
Since actors in filters are executed sequentially, lets analyze each actor separately. 
The complexity of $\aaa$ is $O(|E|)$ by \dref{algo:act-1}. 
This worst-case complexity of $|E|$ for $\aaa$ only happens in the first filter as we have explained before in \ref{comp:an:filter:note}. 
Then for $\ab$ complexity is $O(|L| \times 6|U|)$. $|L|$ is the upper bound by \autoref{theorem:awg} since the \texttt{for} loop is iterating over all $W_l$ received from the channel.
Inside that \texttt{for} loop there 3 sets operations are taking place. Each of those operations are upper bounded by $|U|$ in the worst-case, since $W_l \subseteq U$ by \dref{def:awg}. 
Then, the cost of each loop is $O(6|U|)$ worst-case.  
The complexity of $\ac$ is $O(|L| \times |L| \times 6|U|)$. The first $|L|$ is the outer \texttt{for} loop which receives all the possible \acrshort{awg}, which is upper bounded by $|L|$ according to \autoref{theorem:awg}.
$|L|$ is the second \texttt{for} loop that is iterating over all the \acrshort{adwg} that are in this filter, and it is upper bounded by $|L|-1$ because all possible $\dwi \subseteq \dw$ in one filter has the $l \in L$ parameter of the filter, which can be possible combine with the rest of $|L|-1$ lower layer verticesin the worst case. 
Then, $6|U|$ is inside of the second \texttt{for} loop and follows the same analisys for the set operations provided in $\ab$. Finally, $\ad$ takes $O(|Q||\ati|)$ as we have described in $\gbt$ complexity analysis.
Gluing everything together the worst-case complexity of four actors in the filter instance is $O(|E| + |L| \times 6|U| + |L| \times |L| \times 6|U| + |Q||\ati|)$. 
Since $|L| \times |L| \times 6|U| > |L| \times 6|U|$, we can eliminate that term which is not dominating. Therefore $O(|E| + |L|^2 \times 6|U| + |Q||\ati|)$.

\end{proof}
\fi   
      
\clearpage
\section{Correctness of the Algorithm}
Given a bipartite graph $G$ we prove the algorithm enumerates all the bitriangles in $G$  without duplicating them. 
First, we prove that if $BT_{\{l_1,l_2,l_3\}}^{\{u_1,u_2,u_3\}} = (u_1,l_1,u_2,l_3,u_3,l_2,u_1)$ is a bitriangle occurring in $G$, then only one of the feasible permutations of this 6-cycle is contained in an aggregated bitriangle of $\at$  (\autoref{TH-uniqueness}). Second, we prove that  all the bitriangles occuring in $G$ are contained in an aggregate bitriangles of $\at$ (\autoref{TH-all}). 
%
\begin{theorem}[Uniqueness] \label{TH-uniqueness} 
Given a bipartite graph $G = ((U\cup L),E)$, $\forall \btii\in\bt$  \acrshort{iebt} stores $\btii$ in an $\ati\in \at$ only once.

\end{theorem}
\begin{proof}
Let $\btii\in \bt$, $\btii = (u_1,l_1,u_2,l_3,u_3,l_2,u_1)$. Let us suppose that \acrshort{iebt} stores two different feasible permutations of $\btii$, $\btii$$_1= (u_1',l_1',u_2',l_3',u_3',l_2',u_1')$ and $\btii$$_2 = (u_1'',l_1'',u_2'',l_3'',$ $u_3'',l_2'',u_1'')$ in some aggregated bitriangles. By Definition \ref{def:abt} and $\ac$ line 7, $l_1'<l_2'<l_3'$ and $l_1''<l_2''<l_3''$. The fact that $\btii$$_1 \neq \btii$$_2$ implies that $(l_1',l_2',l_3') \neq (l_1'',l_2'',l_3'')$. This means that  $(l_1',l_2',l_3')$ is a permutation of $(l_1'',l_2'',l_3'')$ (and viceversa). Thus, because of the strict order defined on $L$,  just one, either   $l_1'<l_2'<l_3'$ or $l_1''<l_2''<l_3''$ holds. Therefore, only one of the triples $(l_1',l_2',l_3')$ or $(l_1'',l_2'',l_3'')$  is used in lines 7--16 in $\ac$ to include $\btii = (u_1,l_1,u_2,l_3,u_3,l_2,u_1)$ in an aggregated bitriangle
\end{proof}
 
\begin{theorem}\label{TH-all}Given a bipartite graph $G$, if the bitriangle $\btii = (u_1,l_1,u_2,l_3,$ $u_3,l_2,u_1)\in \bt$, then $\btii$  can be enumerated.
\end{theorem}

\begin{proof}
We proceed in a constructive way. Let $\btii = (u_1,l_1,u_2,l_3,u_3,l_2,u_1) \in \bt$. 
When $\aaa$ running in a filter $\sfilter$ with parameter $l_1$, $F_{l_1}$, ends reading all the edges, the state of $F_{l_1} = \langle l_1,  W_{l_1}\rangle$ and  $\{u_1,u_2\} \subseteq W_{l_1}$. Similarly,  when $\aaa$ running in filter $F_{l_3}$ ends reading all the edges,  $F_{l_3} = \langle l_3,  W_{l_3}\rangle$ and $\{u_2,u_3\} \subseteq W_{l_3}$. This is, wedges $(u_1,l_1,u_2)$ and $(u_2,l_3,u_3)$ are stored in the aggregated wedges (states) of $F_{l_1}$ and $F_{l_3}$, respectively. 
When running  $\ab$ in filter $F_{l_1}$ or in filter $F_{l_3}$, in lines 7-12 the pair 
$(l_l, l_u) \equiv (\argmin_{l_1,l_3}, \argmax_{l_1,l_3})$ is added to $\mathsf{dw}$. This is, the double wedge $(u_1,l_l,u_2,l_u,u_3)$ is stored either, in the state $F_{l_1}$ or the state of $F_{l_3}$. When running $\ac$, in filter $F_{l_1}$ or in filter $F_{l_3}$, $\ac$ receives in $IC_{W_l2}$, line 4, the connector wedge $(l_2, W_{l_2})$ from the incoming  aggregated BT-connector. In lines 7-16, if the condition in line 10 holds, i.e. non empty intersection, since edges $(u_1,l_2)$ and $(u_3,l_2)$ are in $E$, $\{u_1,u_3\} \subseteq W_{l_2}$. Therefore, $\ac$ adds  $\btii$  to the aggregated bitriangle of the filter. This is, the permutation  $(u_1,l_l,u_2,l_u,u_3,l_2,u_1)$ of the bitriangle $\btii$ is stored in $\ati$ and therefore it can be listed by $\ad$
\end{proof}        
 
\section{\acrshort{dpbt} implementation}\label{sec:iebt:hs:imp}
As we have seen in the previous \autoref{dp-hs}, first we need to define the $\dpbt$ using the \acrshort{dsl}.

\begin{listing}[htp!]
\begin{minted}[fontsize=\small,numbers=left,breaklines,frame=lines,framerule=2pt,framesep=2mm,baselinestretch=1.2,highlightlines={3}]{haskell}

type DPBT = Source (Channel (Edge :<+> W :<+> Q :<+> BT :<+> BTResult :<+> W :<+> Eof))
      :=> Generator (Channel (Edge :<+> W :<+> Q :<+> BT :<+> BTResult :<+> Eof))
      :=> FeedbackChannel (W :<+> Eof)
      :=> Sink

\end{minted}
\caption{[\mintinline{shell}{BTriangle.hs}] Enconding of \acrshort{dpbt}}
\label{src:dpbt:1}
\end{listing}

In \autoref{src:dpbt:1} can be appreciated the use of feedback channel in the highlighted line. 
Automatically \acrshort{dpbt} is going to connect this with the $\ibt$.

$\ibt, \obt, \gbt$ are not going to be covered because they are straightforward to follow from the code. 
The most sofisticated part of the algorithm, as we have seen in \autoref{sub:sec:iebt:dpalgo}, relies on actor filter.

\begin{listing}[htp!]
\begin{minted}[fontsize=\small,numbers=left,breaklines,frame=lines,framerule=2pt,framesep=2mm,baselinestretch=1.2,highlightlines={16,21}]{haskell}

actor1 :: Edge
      -> ReadChannel (UpperVertex, LowerVertex)
      -> ReadChannel W
      -> ReadChannel Q
      -> ReadChannel BT
      -> ReadChannel BTResult
      -> ReadChannel W
      -> WriteChannel (UpperVertex, LowerVertex)
      -> WriteChannel W
      -> WriteChannel Q
      -> WriteChannel BT
      -> WriteChannel BTResult
      -> WriteChannel W
      -> StateT FilterState (DP st) ()
actor1 (_, l) redges _ _ _ _ _ we ww1 _ _ _ _ = do
 foldM_ redges $ \e@(u', l') -> do
   e `seq` if l' == l then modify $ flip modifyWState u' else push e we
 finish we
 state' <- get
 case state' of
   Adj w@(W _ ws) -> when (IS.size ws > 1) $ push w ww1
   _              -> pure ()

\end{minted}
\caption{[\mintinline{shell}{BTriangle.hs}] $\aaa$}
\label{src:dpbt:2}
\end{listing}

In \autoref{src:dpbt:2} $\aaa$ source code can be appreciated. Since all the actors are inside the same filter context, all of them have access to read and write
channels of all the filters. That explains the number of parameters which is generated by the \acrshort{idl}.
The first highlighted line is the catamorphism (\mintinline{haskell}{foldM_}) of the $IC_E$ channel, which corresponds to \autoref{algo:act-1:1}.
At the last highlighted line the code is downstream the aggregated wedge collected in this filter.

\begin{listing}[htp!]
\begin{minted}[fontsize=\small,numbers=left,breaklines,frame=lines,framerule=2pt,framesep=2mm,baselinestretch=1.2,highlightlines={22,30,36}]{haskell}

actor2 :: Edge
      -> ReadChannel (UpperVertex, LowerVertex)
      -> ReadChannel W
      -> ReadChannel Q
      -> ReadChannel BT
      -> ReadChannel BTResult
      -> ReadChannel W
      -> WriteChannel (UpperVertex, LowerVertex)
      -> WriteChannel W
      -> WriteChannel Q
      -> WriteChannel BT
      -> WriteChannel BTResult
      -> WriteChannel W
      -> StateT FilterState (DP st) ()
actor2 (_, l) _ rw1 _ _ _ _ _ ww1 _ _ _ _ = do
 state' <- get
 case state' of
   Adj (W _ w_t) -> do
     modify $ const $ DoubleWedges mempty
     foldM_ rw1 $ \w@(W l' w_t') -> do
       push w ww1
       buildDW w_t w_t' l l'
     finish ww1
   _ -> pure ()

buildDW :: IntSet -> IntSet -> LowerVertex -> LowerVertex -> StateT FilterState (DP st) ()
buildDW w_t w_t' l l' =
let pair       = Pair (min l l') (max l l')
      paramBuild = if l < l' then (w_t, w_t') else (w_t', w_t)
      ut         = uncurry buildDW' paramBuild
in  if (IS.size w_t > 1) && l /= l' && not (IS.null (IS.intersection w_t w_t')) && not (nullUT ut)
      then modify $ flip modifyDWState (DW pair ut)
      else pure ()

buildDW' :: IntSet -> IntSet -> UT
buildDW' !w_t !w_t' = (w_t IS.\\ w_t', IS.intersection w_t w_t', w_t' IS.\\ w_t)

\end{minted}
\caption{[\mintinline{shell}{BTriangle.hs}] $\ab$}
\label{src:dpbt:3}
\end{listing}

In \autoref{src:dpbt:3}, the interesting part of $\ab$ is the construction of $\dwi$. That is done by the function \mintinline{haskell}{buildDW}
and \mintinline{haskell}{buildDW'} which are building the three subsets required to create aggregated double-wedges $\dwi$.

\begin{listing}[htp!]
\begin{minted}[fontsize=\small,numbers=left,breaklines,frame=lines,framerule=2pt,framesep=2mm,baselinestretch=1.2,highlightlines={27,34-40}]{haskell}      
actor3 :: Edge
       -> ReadChannel (UpperVertex, LowerVertex)
       -> ReadChannel W
       -> ReadChannel Q
       -> ReadChannel BT
       -> ReadChannel BTResult
       -> ReadChannel W
       -> WriteChannel (UpperVertex, LowerVertex)
       -> WriteChannel W
       -> WriteChannel Q
       -> WriteChannel BT
       -> WriteChannel BTResult
       -> WriteChannel W
       -> StateT FilterState (DP st) ()
actor3 (_, l) _ _ _ _ _ rfb _ _ _ _ _ wfb = do
  state' <- get
  case state' of
    DoubleWedges dwtt -> do
      modify $ const $ BiTriangles mempty
      foldM_ rfb $ \w@(W l' w_t') -> do
        push w wfb
        when (hasDW dwtt) $ do
          let (DWTT dtlist) = dwtt
          forM_ dtlist $ \(DW (Pair l_l l_u) ut) ->
            let triple = Triplet l_l l' l_u
                result =
                  if l' < l_u && l' > l_l then filterUt w_t' ut else Nothing
            in  maybe (pure ()) (modify . flip modifyBTState . BT triple) result
      finish wfb
    _ -> pure ()

filterUt :: IntSet -> UT -> Maybe UT
filterUt wt (si, sj, sk) =
  let si' = IS.filter (`IS.member` wt) si 
      sj' = IS.filter (`IS.member` wt) sj
      sk' = IS.filter (`IS.member` wt) sk
      sij' = si' `IS.union` sj'
      sjk' = sk' `IS.union` sj'
      wtInSome = not (IS.null sij' || IS.null sjk')
  in  if wtInSome then Just (sij', sj, sjk') else Nothing
\end{minted}
\caption{[\mintinline{shell}{BTriangle.hs}] $\ac$}
\label{src:dpbt:4}
\end{listing}

$\ac$ in \autoref{src:dpbt:4}, shows the process of collecting aggregated bitriangles $\ati$. In highlighted lines on \autoref{src:dpbt:4} we can see the algorithm to detect when the \acrshort{awg}
received in the feedback channel are \acrshort{awgc} according to \dref{def:awgc}.

\begin{listing}[htp!]
\begin{minted}[fontsize=\small,numbers=left,breaklines,frame=lines,framerule=2pt,framesep=2mm,baselinestretch=1.2,highlightlines={21-22,27-31}]{haskell}            
actor4 :: Edge
       -> ReadChannel (UpperVertex, LowerVertex)
       -> ReadChannel W
       -> ReadChannel Q
       -> ReadChannel BT
       -> ReadChannel BTResult
       -> ReadChannel W
       -> WriteChannel (UpperVertex, LowerVertex)
       -> WriteChannel W
       -> WriteChannel Q
       -> WriteChannel BT
       -> WriteChannel BTResult
       -> WriteChannel W
       -> StateT FilterState (DP st) ()
actor4 (_, l) _ _ query _ rbtr _ _ _ wq _ wbtr _ = do
  state' <- get
  case state' of
    BiTriangles bttt -> do
      rbtr |=> wbtr
      foldM_ query $ \e -> do
        push e wq
        unless (hasNotBT bttt) $ sendBts bttt e wbtr
    _ -> pure ()

sendBts :: MonadIO m => BTTT -> Q -> WriteChannel BTResult -> m ()
sendBts (BTTT bttt) q@(Q c _ _) wbtr = case c of
  ByVertex vx    -> forM_ bttt (\bt -> filterBTByVertex bt vx (flip push wbtr . RBT q))
  ByEdge   edges -> forM_ bttt (\bt -> filterBTByEdge bt edges (flip push wbtr . RBT q))
  AllBT          -> forM_ bttt (R.mapM_ (flip push wbtr . RBT q) . buildBT)
  Count          -> forM_ bttt (flip push wbtr . RC q . R.length . buildBT)
  _              -> pure ()
\end{minted}
\caption{[\mintinline{shell}{BTriangle.hs}] $\ad$}
\label{src:dpbt:5}
\end{listing}

$\ad$ in \autoref{src:dpbt:5} shows the pattern match done over the query $Q$ sum type and the construction
of the final \acrshort{bt} to be downstream to the $\obt$. 

\begin{listing}[htp!]
\begin{minted}[fontsize=\small,numbers=left,breaklines,frame=lines,framerule=2pt,framesep=2mm,baselinestretch=1.2,highlightlines={}]{haskell}            
filterBTByVertex :: MonadIO m => BT -> IntSet -> ((Int, Int, Int, Int, Int, Int, Int) -> IO ()) -> m ()
filterBTByVertex bt vertices f =
  if inLower bt vertices then
     liftIO
     . mapConcurrently_ f
     . buildBT
     $ bt
  else 
    if inUpper bt vertices 
      then buildBT' bt vertices f
      else pure ()

filterBTByEdge :: MonadIO m => BT -> Set Edge -> ((Int, Int, Int, Int, Int, Int, Int) -> IO ()) -> m ()
filterBTByEdge bt edges f =
  when (getAny $ foldMap (`hasEdge` bt) edges)
   $ liftIO . mapConcurrently_ f . R.filter (isInSetEdge edges) . buildBT $ bt
      
\end{minted}
\caption{[\mintinline{shell}{Edges.hs}] \texttt{filterBTByVertex} and \texttt{filterBTByEdge}}
\label{src:dpbt:6}
\end{listing}
      
Finally, definition of \mintinline{haskell}{filterBTByEdge} and \mintinline{haskell}{filterBTByVertex} can be seen in \autoref{src:dpbt:6} and differs 
from the pseudo-code presented in \autoref{algo:buildBtEdge} and \autoref{algo:buildBtVertex} because it is using \acrshort{hs} specific combinators and 
for-comprehension lists to take advantage of the non-strictness of the language as well as some concurrency primitives.

\section{Chapter Summary}
In this chapter we first introduced some basic definitions. These definitions are the foundation to represent the graph index created from the input bipartite graph and to define the \acrshort{iebt} algorithm. Second, we presented a general view of the algorithm. Third, we deeply described the definitions of the  stages  of the $\dpbt$. Then, we provided a correctness proof  of the \acrshort{iebt} algorithm. Finally, we give the main details of the implementation of the \acrshort{iebt} algorithm using \acrshort{dpfh}. 


\chapter{Empirical Evaluation}\label{experiments}
This chapter reports on the experimental study conducted to assess the efficiency of \acrshort{dpbt}. This study aims at answering the research questions that emerged from the motivation of this work. 

The fundamental part of this work focuses on incremental enumeration \acrshort{bt} in \acrshort{bg}. 
In spite of this, there are other important areas for doing empirical analysis such as memory consumption, thread scheduling, and execution time.

Regarding that, we have asked ourselves the following research questions that guide the empirical evaluation analysis, and we try to answer them with the conducted experiments:
\begin{inparaenum}[\bf {\bf RQ}1\upshape)]
\label{res:bt:question}
    \item Does \acrshort{dpbt} generate incremental results regardless of the size of the graph?
    \item Does the type of query $Q$ impact on the execution of \acrshort{dpbt}?
    \item How effectively \acrshort{dpbt} implements a \emph{pay-as-you-go} model?
    \item Does \acrshort{dpbt} handle memory and threads efficiently?
\end{inparaenum}
  
\section{Experiments Configuration}
We have conducted different kinds of experiments to answer our research questions, and verify the behavior of the approach in various benchmarks.
First, we have performed a \emph{Continuous behavior Analysis} using \acrfull{dm}~\cite{diefpaper} in order to assess the continuous behavior capabilities of the implemented algorithm (generation of incremental results). 
Then, we have performed a \emph{Benchmark Analysis} to identify how the behavior of \acrshort{dpbt} varies depending on the type of query command defined in \dref{def:query:match}.
Finally, we have executed a \textit{Performance Analysis} in which we have to gather profiling data from \acrfull{ghc} for one of the graphs, 
to measure how the program performs regarding multithreading and memory allocation.
In the following subsections, we detail the different aspects of the configuration such as hardware, \acrshort{hs} compilation flags, metrics, and benchmark to conduct these experiments.


\subsection{Benchmark}\label{data:set}
The experiments have been evaluated over the networks that composed the benchmark Konect Networks~\cite{konect}. 
Specifically, the networks used in the literature have been selected \cite{konect:2017:dbpedia-recordlabel,konect:2017:moreno_crime,konect:2017:opsahl-ucforum,konect:2017:wang-amazon}.

\begin{table}[H]
  \centering
  \begin{tabular}{|p{0.25\linewidth}|c|c|c|c|c|}
    \hline
   \textbf{Network} & \textbf{$|U|$} & \textbf{$|L|$} & \textbf{$|E|$} & \textbf{Wedges} & \textbf{\#\acrshort{bt}} \\
   \hline
   Dbpedia & 18422 & 168338 & 233286 & $1.45 \times 10^8$ & $3.62 \times 10^8$\\
   \hline
   Moreno Crime & 829 & 551 & 1476 & 4816 & 211\\
   \hline
   Opsahl UC Forum  & 899 & 522 & 33720 & 174069 & $2.2 \times 10^7$ \\
   \hline
   Wang Amazon & 26112 & 799 & 29062 & $3.4 \times 10^6$ & 110269\\
   \hline
  \end{tabular}
 \caption[{[EE] Selected Networks of \acrlong{bg}}]{This table shows the different networks used in the experiments. We provide some metrics of the networks used in order to understand a little more about the topology of each \acrshort{bg}. In particular, we are showing in the last column $2$ metrics that are important and could affect results which are a number of wedges and bi-triangles}
 \label{table:exp:data-set}
 \end{table}
 
The criteria for selecting those networks have followed the idea of conducting the analysis on one of the big networks~\cite{konect:2017:dbpedia-recordlabel} used on the \acrshort{bt} counting work~\cite{btcount}.
The rest of the networks, from the same data source~\cite{konect}, have been selected randomly but taking into consideration different sizes and topologies.

\subsection{Metrics}\label{sub:metric}
We report the following metrics \begin{inparaenum}[\bf a\upshape)]
      \item {\bf $dief@t$}: a measurement for continuous behavior in the first $t$ time units of the results generated by \acrshort{dpbt}~\cite{diefpaper}. Time $t$ in our experiments represents the number of nanoseconds elapsed to deliver that result from the moment the \acrshort{dpbt} finishes the execution of $\ac$ and it started executing $\ad$, so $\ad$ is able to start processing commands. 
      Therefore, time $0$ is equal to the start of $\ad$ execution.
      \item \emph{Average Running Time}: The average of the total running time of $1000$ resamples using \texttt{criterion} tool~\cite{criterion}. Per sample, the running time is measure from the beginning of the execution of the program until when the last answer is produced.
      \item \emph{Total Running Time}: Total running time of one execution set over each experiment setup and benchmark. The time is measure from the beginning of the execution of the program until the end, i.e., the last answer is produced.
      \item \emph{GHC productivity}: Measures the proportion of \acrfull{mut} execution time vs. \acrfull{gc} time. This is provided by default in \acrshort{hs} by enabling flag \texttt{-s} in the execution command line argument.
      \item \emph{Distribution of Threads per Core}: Measures the amount of threads per processor on each time slot of execution. This metric is gathered by \texttt{TreadScope} \cite{threadscope} tool.
      \item \emph{Distribution of Allocated Memory per Data Type}: Measure the amount of allocated memory per \acrshort{hs} Data Type. This metric is gathered by \texttt{eventlog2html} \cite{eventlog2html} tool.
  \end{inparaenum}
    
\subsection{Scenarios}\label{sub:exp:exp-data-setup}
An experimental scenario is a specific configuration of the \acrfull{qo} that we have defined in \dref{def:query:match}. 
That means, that for each network that we run an experiment with a different configuration of the \acrshort{qo} to obtain different results.

\begin{definition}[Incidence Level]\label{def:exp:incidence}
Let $G$ be a \acrlong{bg}.
Let $v$ be a vertex such that $v \in V$ of $G$.
Let $e$ be an edge such that $e \in E$ of $G$.
A \emph{low, medium, or high incidence level} for vertices is defined as following:
\begin{inparaenum}
  \item[Low] A vertex $v$ has low incidence if its degree is less than $1\%$ of $|V|$,
  \item[Medium] A vertex $v$ has medium incidence if its degree is between $1\%-25\%$ of $|V|$, and
  \item[High] A vertex $v$ has high incidence if its degree is more than $25\%$ of $|V|$
\end{inparaenum}
A \emph{low, medium, or high incidence level} for edges is defined as following:
\begin{inparaenum}
  \item[Low] A edge $e = (u, l)$ has low incidence if any of its vertices $u$ or $l$ has degree less than $1\%$ of $|V|$,
  \item[Medium] A edge $e = (u, l)$ has medium incidence if any of its vertices $u$ or $l$ has degree between $1\%-25\%$ of $|V|$, and
  \item[High] A edge $e = (u, l)$ has high incidence if any of its vertices $u$ or $l$ has degree more than $25\%$ of $|V|$
\end{inparaenum}
\end{definition}


Having the previous \dref{def:exp:incidence}, we define the following scenarios to conduct all the experiments regardless of the network.

\begin{table}[H]
  \centering
  \begin{tabular}{|l|c|c|}
    \hline
    \textbf{Scenario ID} & \textbf{Name} & \textbf{Search by}\\
    \hline
    E-H & Edge High & edge with high incidence \\
    \hline
    E-L & Edge Low & edge with low incidence \\
    \hline
    E-M & Edge Medium & edge with medium incidence \\
    \hline
    VL-H & $l \in L$ High & vertex in lower layer with high incidence \\
    \hline
    VL-L & $l \in L$ Low & vertex in lower layer with low incidence \\
    \hline
    VL-M & $l \in L$ Medium & vertex in lower layer with medium incidence \\
    \hline
    VU-H & $u \in U$ High & vertex in upper layer with high incidence \\
    \hline
    VU-L & $u \in U$ Low & vertex in upper layer with low incidence \\
    \hline
    VU-M & $u \in U$ Medium & vertex in upper layer with medium incidence \\
    \hline
  \end{tabular}
  \caption[{[EE] Experiment Data Setup for experiments}]{The first column of the table is an identifier to be reference in the rest of the section. $E$ and $V$ indicate if the \acrshort{qo} scenario contains edge or vertex query, respectively. $VL$ or $VU$ indicates if those vertices belongs to $L$ (lower layer) or $U$ upper layer. After the $-$ symbol, the letters $L,M,H$ indicate the incidence level defined in \dref{def:exp:incidence}}
  \label{table:exp:data-setup}
  \end{table}

\paragraph{Selection of values for \acrshort{qo}}\label{sub:exp:sel-vals} The selection of values, either vertices $V$ or edges $E$, has been done pseudo-randomly, i.e., given \dref{def:exp:incidence} the next steps were followed: \begin{inparaenum}[\bf i\upshape)]
  \item Sort the vertices by its degree.
  \item Randomly select following a uniform distribution,  a vertex or edge depending on the scenario, from the subset of vertices or edges that fulfill the \dref{def:exp:incidence}.
  \item Execution of a sample of experiments to check if that selections provides results or not. If not, the test case is eliminated.
\end{inparaenum}
  
\subsection{Implementation}
\paragraph{Hardware Platform} All the experiments have been executed in the \emph{HPC Cluster at UPC}. The nodes' architecture running in the cluster is $x86$ $64$ bits with a \textit{$24$-Core Intel(R) Xeon(R) CPU X5650} processor of $2.67$ GHz. 
Regarding memory, the allocated nodes have been requested from $40 GB$ up to $120 GB$ of RAM for the biggest \acrshort{dbpedia} graph. These machines also have $256\ KB$ of L2 cache memory, and $12\ MB$ of L3 cache.

\paragraph{Haskell Setup} The implementation uses \acrshort{ghc} $8.10.4$ plus the following set of \acrshort{hs} libraries:
\begin{inparaenum}[]
  \item \texttt{dyanmic-pipeline} $0.3.2.0$ \cite{dynamic-pipeline},
  \item \texttt{bytestring} $0.10.12.0$ \cite{bytestring},
  \item \texttt{containers} $0.6.2.1$ \cite{containers}, 
  \item\texttt{relude} $1.0.0.1$ \cite{relude}
  \item and\texttt{unagi-chan} $0.4.1.3$ \cite{unagi} 
\end{inparaenum}. The \texttt{relude} library is utilized because \texttt{Prelude} was disabled from the project with the language extension \texttt{NoImplicitPrelude} \cite{extensions}. 
We have compiled our program using \texttt{stack} version $2.5.1$ \cite{stack} with the following command and option flags\footnote{For more information about package.yaml or cabal file, please check https://github.com/jproyo/upc-miri-tfm/tree/main/bt-graph-dp}:
\mintinline[fontsize=\small, breaklines]{bash}{stack build --ghc-options "-threaded -O3 -rtsopts -with-rtsopts=-N"}.
Flag \texttt{threaded} indicates \acrshort{ghc} to compile the program with thread support enable. \texttt{-O3} is the highest optimization level for the compiler.
Regarding \texttt{-with-rtsopts=-N}, it allows us to change dynamically on each runtime execution command, the number of processors, and other execution flags that we will explain in \autoref{par:ex:param}. 

\paragraph{Optimal Execution Parameters}\label{par:ex:param} \acrshort{ghc} enables different flags parameters to speed up the execution. Unfortunately there is no recipe to tune those parameters in the correct way and each program needs to be analyzed to take advantage of \acrshort{ghc} capabilities.\footnote{Execution parameters details are explained in \aref{apx:running:experiments}}
The most important parameters to be tuned are memory and the number of processors. In the case of the number of processors, we have run all the experiments between 6 and 12 cores. The detail of each run can be found in \autoref{table:e1:def}.
Setting up memory allocation is not a straightforward task, because the combination of two parameters needs to be considered. \texttt{-A} flag which indicates the allocation area for the garbage collector, which is fixed and never resized, and \texttt{-H} which is the heap size. 
We used the tool \texttt{ghc-gc-tune} \cite{ghctune} to find the best combination of those parameters; 
it implements a heuristic algorithm that tries different setups until it finds an \emph{suboptimal} combination for memory allocation. \emph{Suboptimal combination} here refers to find the fastest total execution time with the less amount of allocated memory, as it can be seen in \autoref{fig:exp:opt-mem}.
We have run \acrshort{dpbt} with that tool, obtaining the following result that can be appreciated in \autoref{fig:exp:opt-mem}

\begin{figure}[h!]
  \centering  
  \resizebox{1\textwidth}{!}{%
  \includegraphics{bt-graph-dp-time-gc-space}
  }
\caption[{[EE] $\dpbt$ Finding Optimal Memory Setup}]{This figure shows the results obtained after running \acrshort{dpbt} with the \texttt{ghc-gc-tune} tool in order to find the suboptimal configuration for Memory allocation. $y$ axis shows the total execution time, $x$ axis is the \texttt{-A} configuration flag and $z$ axis is the \texttt{-H} configuration flag.}
\label{fig:exp:opt-mem}
\end{figure}

In \autoref{fig:exp:opt-mem} we can appreciate that the tool runs sereval times the same program with different configurations on flags \texttt{-A} and \texttt{-H} until if it finds a minimum. The dark blue shows the better performance where the curve find its minimum execution time. 
This indicates two possible suboptimal setup, either when \texttt{-A} is equal to \texttt{-H}, or either when \texttt{-A} is $\frac{1}{4}$ of the \texttt{-H}. For all our experiments we have selected the first option which \texttt{-A} is equal to \texttt{-H}, because choosing any of the two best combinations provides the same results as it can be seen in \autoref{fig:exp:opt-mem}.
It is important to remind the reader that \texttt{ghc-gc-tune} \cite{ghctune} uses an heuristic algorithm, providing suboptimal results.

\section{Experimental Results}\label{sec:exp:observed-results}
\subsection{E1: Continuous behavior Analysis}\label{sub:sec:exp-1} 
\paragraph{Goal} In this experiment, we assess the ability of \acrshort{dpbt} to generate results incrementally.
In order to do that, we use the \acrfull{dm} Tool \emph{diefpy}~\cite{diefpy} which implements \emph{Diefficiency Metric}~\cite{diefpaper} measurement analysis.
This experiment allows us to answer research questions [R1] and [R3] defined in \qref{res:bt:question}. 

\paragraph{Procedure} We execute this experiment for each of the networks described in \autoref{data:set} and for each scenario described in \autoref{sub:exp:exp-data-setup}.
This experiment has been executed five times on each case until we found the proper vertex or edges in the selection described in \autoref{sub:exp:sel-vals}. The criteria followed by the selection of the vertices or edges are detailed in \autoref{sub:exp:sel-vals}.
The metric  \acrfull{dt}-described in \autoref{sub:metric}- is used to measure the continuous behavior of the proposed approach in a given time frame.
\autoref{table:e1:def} depicts the different configurations evaluated in this experiment.

\begin{table}[H]
  \centering
  \resizebox{1\textwidth}{!}{%
  \begin{tabular}{|p{0.25\linewidth}|c|c|c|c|}
    \hline
   \textbf{Network} & \textbf{Scenario ID} & \textbf{Exec Flags} & \textbf{Query} & \textbf{Timeout}\\
   \hline
   \multirow{9}{*}{Dbpedia} & E-H & \texttt{+RTS -A5G -N8 -c -H5G -RTS} & \texttt{by-edge (921, 4)} & 48 hours\\
   & E-L & \texttt{+RTS -A5G -N8 -c -H5G -RTS} & \texttt{by-edge (383, 397)} & 48 hours\\
   & E-M & \texttt{+RTS -A5G -N8 -c -H5G -RTS} & \texttt{by-edge (540, 60)} & 48 hours\\
   & VL-H & \texttt{+RTS -A5G -N8 -c -H5G -RTS} & \texttt{by-vertex 9} & 48 hours\\
   & VL-L & \texttt{+RTS -A5G -N8 -c -H5G -RTS} & \texttt{by-vertex 809} & 48 hours\\
   & VL-M & \texttt{+RTS -A5G -N8 -c -H5G -RTS} & \texttt{by-vertex 511} & 48 hours\\
   & VU-H & \texttt{+RTS -A5G -N8 -c -H5G -RTS} & \texttt{by-vertex 921} & 48 hours\\
   & VU-L & \texttt{+RTS -A5G -N8 -c -H5G -RTS} & \texttt{by-vertex 93} & 48 hours\\
   & VU-M & \texttt{+RTS -A5G -N8 -c -H5G -RTS} & \texttt{by-vertex 540} & 48 hours\\
   \hline
   \multirow{9}{*}{Moreno Crime} & E-H & \texttt{+RTS -A5G -N6 -c -H5G -RTS} & \texttt{by-edge (413, 419)} & 48 hours\\
   & E-L & \texttt{+RTS -A5G -N6 -c -H5G -RTS} & \texttt{by-edge (361, 19)} & 48 hours\\
   & E-M & \texttt{+RTS -A5G -N6 -c -H5G -RTS} & \texttt{by-edge (531, 196)} & 48 hours\\
   & VL-H & \texttt{+RTS -A5G -N6 -c -H5G -RTS} & \texttt{by-vertex 95} & 48 hours\\
   & VL-L & \texttt{+RTS -A5G -N6 -c -H5G -RTS} & \texttt{by-vertex 187} & 48 hours\\
   & VL-M & \texttt{+RTS -A5G -N6 -c -H5G -RTS} & \texttt{by-vertex 97} & 48 hours\\
   & VU-H & \texttt{+RTS -A5G -N8 -c -H5G -RTS} & \texttt{by-vertex 2} & 48 hours\\
   & VU-L & \texttt{+RTS -A5G -N6 -c -H5G -RTS} & \texttt{by-vertex 793} & 48 hours\\
   & VU-M & \texttt{+RTS -A5G -N6 -c -H5G -RTS} & \texttt{by-vertex 533} & 48 hours\\
   \hline
   \multirow{9}{*}{Opsahl UC Forum} & E-H & \texttt{+RTS -A5G -N6 -c -H5G -RTS} & \texttt{by-edge (213, 33)} & 48 hours\\
   & E-L & \texttt{+RTS -A5G -N6 -c -H5G -RTS} & \texttt{by-edge (398, 10)} & 48 hours\\
   & E-M & \texttt{+RTS -A5G -N6 -c -H5G -RTS} & \texttt{by-edge (129, 171)} & 48 hours\\
   & VL-H & \texttt{+RTS -A5G -N6 -c -H5G -RTS} & \texttt{by-vertex 289} & 48 hours\\
   & VL-L & \texttt{+RTS -A5G -N6 -c -H5G -RTS} & \texttt{by-vertex 258} & 48 hours\\
   & VL-M & \texttt{+RTS -A5G -N6 -c -H5G -RTS} & \texttt{by-vertex 433} & 48 hours\\
   & VU-H & \texttt{+RTS -A5G -N8 -c -H5G -RTS} & \texttt{by-vertex 395} & 48 hours\\
   & VU-L & \texttt{+RTS -A5G -N6 -c -H5G -RTS} & \texttt{by-vertex 390} & 48 hours\\
   & VU-M & \texttt{+RTS -A5G -N6 -c -H5G -RTS} & \texttt{by-vertex 207} & 48 hours\\
   \hline
   \multirow{9}{*}{Wang Amazon} & E-H & \texttt{+RTS -A5G -N6 -c -H5G -RTS} & \texttt{by-edge (839, 9)} & 48 hours\\
   & E-L & \texttt{+RTS -A5G -N6 -c -H5G -RTS} & \texttt{by-edge (10987, 36)} & 48 hours\\
   & E-M & \texttt{+RTS -A5G -N6 -c -H5G -RTS} & \texttt{by-edge (19630, 84)} & 48 hours\\
   & VL-H & \texttt{+RTS -A5G -N6 -c -H5G -RTS} & \texttt{by-vertex 124} & 48 hours\\
   & VL-L & \texttt{+RTS -A5G -N6 -c -H5G -RTS} & \texttt{by-vertex 321} & 48 hours\\
   & VL-M & \texttt{+RTS -A5G -N6 -c -H5G -RTS} & \texttt{by-vertex 64} & 48 hours\\
   & VU-H & \texttt{+RTS -A5G -N8 -c -H5G -RTS} & \texttt{by-vertex 1727} & 48 hours\\
   & VU-L & \texttt{+RTS -A5G -N6 -c -H5G -RTS} & \texttt{by-vertex 9970} & 48 hours\\
   & VU-M & \texttt{+RTS -A5G -N6 -c -H5G -RTS} & \texttt{by-vertex 73} & 48 hours\\
   \hline
  \end{tabular}
  }
 \caption[{[EE] E1 Procedure}]{This table shows all the different experiments setups conducted for E1. Execution flags are the Runtime execution flags which needs to be set on the execution command as it is detailed in \autoref{apx:running:experiments}. On the last column we can see for each scenario, the query command executed for \acrshort{bt} enumeration (see \dref{def:query:match})}
 \label{table:e1:def}
 \end{table}

 \paragraph{Observed Results}\label{sub:sec:res:e1}
 As observed in the results above in \autoref{fig:dief:dbpedia}, \autoref{fig:dief:moreno}, \autoref{fig:dief:opsahl} and \autoref{fig:dief:wang} in all the networks and experiments setups, \acrshort{bt} are incrementally enumerated and delivered. 

 \begin{figure}[!htp]
  \centering
  \begin{subfigure}[t]{0.45\textwidth}
   \includegraphics[width=1\linewidth, height=0.2\textheight]{experiments/diepfy/dbpedia.png}
    \caption[{[EE] \acrshort{dm} Results: \acrshort{dbpedia}}]{\acrshort{dm} metric results after running all experiment scenarios described in \autoref{table:e1:def} on the DBpedia network. Each color represents one scenario}
    \label{fig:dief:dbpedia}
  \end{subfigure}\hfill
  \begin{subfigure}[t]{0.45\textwidth}
   \includegraphics[width=1\linewidth, height=0.2\textheight]{experiments/diepfy/moreno_crime.png}
    \caption[{[EE] \acrshort{dm} Results: Moreno Crime}]{\acrshort{dm} metric results after running all experiment scenarios described in \autoref{table:e1:def} on Moreno Crime network. Each color represents one scenario}
    \label{fig:dief:moreno}
  \end{subfigure}
  \vspace{0.5cm}

  \begin{subfigure}[t]{0.45\textwidth}
   \includegraphics[width=1\linewidth, height=0.2\textheight]{experiments/diepfy/opsahl-ucforum.png}
    \caption[{[EE] \acrshort{dm} Results: Opsahl UC Forum}]{\acrshort{dm} metric results after running all experiment scenarios described in \autoref{table:e1:def} on Opsahl UC Forum network. Each color represents one scenario}
    \label{fig:dief:opsahl}
  \end{subfigure}\hfill
  \begin{subfigure}[t]{0.45\textwidth}
    \includegraphics[width=1\linewidth, height=0.2\textheight]{experiments/diepfy/wang-amazon.png}
     \caption[{[EE] \acrshort{dm} Results: Wang Amazon}]{\acrshort{dm} metric results after running all experiment scenarios described in \autoref{table:e1:def} on Wang Amazon network. Each color represents one scenario}
     \label{fig:dief:wang}
   \end{subfigure}
   \caption[{[EE] \acrshort{dm} General Results}]{These figures show \acrshort{dm} observed results after running all the scenarios described in \autoref{table:e1:def} for each network. $y$ axis represents the number of Answers produced and $x$ axis is the $t$ time of the \acrshort{dm} metric describe in \autoref{sub:metric}. For example in \autoref{fig:dief:dbpedia} we can see that dark green color shows VL-H scenario on DBpedia network. The more data points distributed throughout the $x$ axis, the higher, the continuous behavior.}
   \label{fig:dief:all}
 \end{figure}

The continuous behavior can be appreciated in \autoref{fig:dief:all} having the fact that any of those points draw a straight vertical line. 
If that were the case, the plot would have indicated that all the $t$ times, in \acrshort{dm} metric, are being produced at the same $t$.
Moreover we can observed \acrshort{dm} values for dbpedia of $4, 1.22, 6, 9.8, 1.14, 6.32, 4.11, 8.15$ and $3.63$ for the nine scenarios indicating continuous behavior. 
In the case of Opsahl UC Forum \acrshort{dm} of $5, 1.51, 10, 7.68, 4.30, 3.23, 1.06, 8.48, 5.77$ indicating continuous behavior as well.
Having those values, we can ensure that all the results are continuously generated by \acrshort{dpbt}.
We can also observe how the results obtained in experiments VU-M for Opshal with $10$, e-L for Moreno with $28.42$, VL-M for dbpedia with $9.8$ and VL-M for Wang Amazon with $11766.99$ (see \autoref{table:exp:data-setup}), are more continuous compare to the rest of the experiments setups. 
The only experiment setup that cannot be appreciated with the same level of continuous results as the rest is \emph{Wang Amazon} which for the rest of the scenarios has value $0$ of \acrshort{dm} metric. 
Moreover, it is observed how Moreno Crime in \autoref{fig:dief:moreno} that between time $100$ and $150$, scenarios VL-H, VL-M, and VL-L are plotting in the same $t$ all the enumerated \acrshort{bt}, with a \acrshort{dm} value of $0$ indicating low continuous behavior. 
This network is the only one exhibiting that behavior in more than two scenarios.

\begin{figure}[!htp]
  \centering
  \begin{subfigure}[t]{0.45\textwidth}
  \includegraphics[width=1\linewidth, height=0.25\textheight]{experiments/diepfy/dbpedia_radial.png}
    \caption[{[EE] \acrshort{dm} Results (Radial): \acrshort{dbpedia}}]{\acrshort{dm} metric results in radial plot after running all experiment scenarios described in \autoref{table:e1:def} on DBpedia network. Each color represents one scenario. Each dimension of the radial represents an observed result for that scenario on that dimension}
    \label{fig:dief:dbpedia-radial}
  \end{subfigure}\hfill
  \begin{subfigure}[t]{0.45\textwidth}
  \includegraphics[width=1\linewidth, height=0.25\textheight]{experiments/diepfy/moreno_crime_radial.png}
    \caption[{[EE] \acrshort{dm} Results (Radial): Moreno Crime}]{\acrshort{dm} metric results in radial plot after running all experiment scenarios described in \autoref{table:e1:def} on Moreno Crime network. Each color represents one scenario. Each dimension of the radial represents an observed result for that scenario on that dimension}
    \label{fig:dief:moreno-radial}
  \end{subfigure}
  \vspace{0.5cm}
  %
  \begin{subfigure}[t]{0.45\textwidth}
  \includegraphics[width=1\linewidth, height=0.25\textheight]{experiments/diepfy/opsahl-ucforum_radial.png}
    \caption[{[EE] \acrshort{dm} Results (Radial): Opsahl UC Forum}]{\acrshort{dm} metric results in radial plot after running all experiment scenarios described in \autoref{table:e1:def} on Opsahl UC Forum network. Each color represents one scenario. Each dimension of the radial represents an observed result for that scenario on that dimension}
    \label{fig:dief:opsahl-radial}
  \end{subfigure}\hfill
  \begin{subfigure}[t]{0.45\textwidth}
    \includegraphics[width=1\linewidth, height=0.25\textheight]{experiments/diepfy/wang-amazon_radial.png}
    \caption[{[EE] \acrshort{dm} Results (Radial): Wang Amazon}]{\acrshort{dm} metric results in radial plot after running all experiment scenarios described in \autoref{table:e1:def} on Wang Amazon network. Each color represents one scenario. Each dimension of the radial represents an observed result for that scenario on that dimension}
    \label{fig:dief:wang-radial}
  \end{subfigure}
  \caption[{[EE] \acrshort{dm} General Results (Radial)}]{Radial plots show how the different dimensions values provided by \acrshort{dm} tool such as \acrfull{tt}, \acrfull{tfft}, \acrfull{dt}, \acrfull{et} and \acrfull{comp} are related each other for each experimental case. These figures show radial plot observed results after running all the scenarios described in \autoref{table:e1:def} for each network. \acrshort{dm} is described in \\ \autoref{sub:metric}.}
\end{figure}

Another plots that \acrshort{dm} tool provides are shown in \autoref{fig:dief:dbpedia-radial}, \autoref{fig:dief:moreno-radial}, \autoref{fig:dief:opsahl-radial} and \autoref{fig:dief:wang-radial}.
Here we can observe how all the networks under the VL-H scenario cover \acrshort{dt} metrics with the dark green area, which indicates that it is continuously delivering results for each network. 
The rest of the data setup experiments indicates that the level of throughput, completeness, and execution time is less than \acrfull{dt}, and the results can be delivered faster without appreciating continuous behavior properly in the plot. 
These radial plots obtained from \acrshort{dm} shows how \acrfull{et}, \acrfull{tfft}, \acrfull{comp}, \acrfull{tt} and \acrshort{dm} measured by tool, relate each other in the same setup. 
For our case analysis, the higher the area that is cover in \acrfull{dt} the better, indicating a high level of continuous behavior.

\subsection{E2: Benchmark Analysis}\label{sub:sec:exp-2} 
\paragraph{Goal} Regarding benchmarking, we aim at answering research question [R2] and assess  the impact of the type of command query $Q$ on the  \acrshort{bt} enumeration.
The results of this evaluation will also provide evidence to answer [R3] since we have provided evidence with the benchmarking that the execution time varies depending on the command query $q$ proving that we are effectively implemented a \emph{pay-as-you-go} model. 

\paragraph{Procedure} This experiment measures \emph{Average Running Time} and \emph{Total Running Time} as  described in \autoref{sub:metric}. 
For gathering \emph{Average Running Time}, the experiment scenarios described in \autoref{table:e2:def} were conducted; the \emph{\acrshort{dbpedia}} network was not considered, but the \mintinline{shell}{criterion} \cite{criterion} benchmark tool were followed.
The command run for \texttt{criterion} benchmark analysis is \mintinline[fontsize=\small, breaklines]{bash}{stack exec benchmark}. 

\begin{table}[H]
\centering
\begin{tabular}{|c|c|c|c|}
  \hline
  \textbf{Network} & \textbf{Scenario ID} & \textbf{Query} & \textbf{Timeout}\\
  \hline
  \multirow{9}{*}{Moreno Crime} & E-H & \texttt{by-edge (413, 419)} & 48 hours \\
  & E-L & \texttt{by-edge (361, 19)} & 48 hours \\
  & E-M & \texttt{by-edge (531, 196)} & 48 hours \\
  & VL-H & \texttt{by-vertex 95} & 48 hours \\
  & VL-L & \texttt{by-vertex 187} & 48 hours \\
  & VL-M & \texttt{by-vertex 97} & 48 hours \\
  & VU-H & \texttt{by-vertex 2} & 48 hours \\
  & VU-L & \texttt{by-vertex 793} & 48 hours \\
  & VU-M & \texttt{by-vertex 533} & 48 hours \\
  \hline
  \multirow{9}{*}{Opsahl UC Forum} & E-H & \texttt{by-edge (213, 33)} & 48 hours \\
  & E-L & \texttt{by-edge (398, 10)} & 48 hours \\
  & E-M & \texttt{by-edge (129, 171)} & 48 hours \\
  & VL-H & \texttt{by-vertex 289} & 48 hours \\
  & VL-L & \texttt{by-vertex 258} & 48 hours \\
  & VL-M & \texttt{by-vertex 433} & 48 hours \\
  & VU-H & \texttt{by-vertex 395} & 48 hours \\
  & VU-L & \texttt{by-vertex 390} & 48 hours \\
  & VU-M & \texttt{by-vertex 207} & 48 hours \\
  \hline
  \multirow{9}{*}{Wang Amazon} & E-H & \texttt{by-edge (839, 9)} & 48 hours \\
  & E-L & \texttt{by-edge (10987, 36)} & 48 hours \\
  & E-M & \texttt{by-edge (19630, 84)} & 48 hours \\
  & VL-H & \texttt{by-vertex 124} & 48 hours \\
  & VL-L & \texttt{by-vertex 321} & 48 hours \\
  & VL-M & \texttt{by-vertex 64} & 48 hours \\
  & VU-H & \texttt{by-vertex 1727} & 48 hours \\
  & VU-L & \texttt{by-vertex 9970} & 48 hours \\
  & VU-M & \texttt{by-vertex 73} & 48 hours \\
  \hline
\end{tabular}
\caption[{[EE] E2 Procedure}]{This table shows all the different experiments setups that we have conducted for E2. Execution command is the same for each experiment because it is handled by \texttt{criterion} tool. On the last column we can see for each scenario the query command executed for \acrshort{bt} enumeration (see \dref{def:query:match}), although the output is not used in this case, because we focus on average running time}
\label{table:e2:def}
\end{table}

Furthermore, \emph{Total Running Time} is gathered with the time measured after running all the scenarios described in \autoref{table:e1:def}.


\paragraph{Observed Results}\label{sub:sec:res:e2} As it can be seen in \autoref{fig:exp:bench}, yellow bars are all the experiments related to Lower Layer vertices, blue bars are related to Upper Layer, and red bars are the experiments related to edge search. 
The longest bars are from network opsahl-ucforum, which we already know by \autoref{table:exp:data-set} it is the biggest of the four networks in terms of the number of \acrshort{bt} and wedges. So, it needs to enumerate more \acrshort{bt}. 
Further, almost all the high incidence vertices and edges queries are taken longer as well. 
There is only one scenario that behaves differently, which is edges (see \autoref{table:exp:data-setup}) for opsahl-ucforum, where execution time is not proportional to incidence level.

\begin{figure}[!htb]
 \begin{center}
    \includegraphics[width=1\textwidth] {experiments/bench_1}
      \end{center}
    \caption[{[EE] Benchmark Results: Criterion Plot}]{This plot depicts the results after running \texttt{criterion} the tool over the experimental setup described in \autoref{table:e2:def}. Yellow bars are the experiments over Opsahl UC Forum network. Blue light bars represents the experiments on Moreno Crime network, and red bars on Wang Amazon}
    \label{fig:exp:bench}
\end{figure}

 The \emph{Total Running Time}, which can be seen in \autoref{fig:exp:bench:2}, is also reported.

\begin{figure}[!htb]
  \begin{center}
     \includegraphics[width=1\textwidth] {experiments/execution_time_by_experiments}
       \end{center}
     \caption[{[EE] Total Execution Time: Comparison between all setups}]{This plot shows all the experiment scenarios run in \autoref{table:e1:def} and the comparison of the Total running execution time for each case. $y$ axis shows the total time in $\ln$ scale and $x$ axis is each Scenario ID describe in \autoref{table:exp:data-setup}}
     \label{fig:exp:bench:2}
 \end{figure}

Regarding \autoref{fig:exp:bench:2}, it is observed with the red bars that \acrshort{dbpedia} was the one which took more time in all scenarios.
We can also see in \autoref{fig:exp:bench:2} that scenario E-H took $8756$ seconds in \acrshort{dbpedia} network compare with the E-L and E-M, which took $596$ seconds and $577$ seconds, respectively. 
Also, in \autoref{fig:exp:bench:2}, we can see that Opsahl UC Forum took $1660$ seconds for VL-H scenario, $11$ seconds for VL-L, and $22$ seconds for VL-M, also showing the same behavior as \acrshort{dbpedia}.
Moreover, in \autoref{fig:exp:bench:2} it can also be appreciated how Wang Amazon network and Moreno Crime are showing the same behavior for VL-H, VL-L, and VL-M scenarios, where Wang Amazon networks report $21, 8$ and $8$ seconds, respectively and Moreno Crime $3, 3$ and $2$ seconds.
The same behavior repeats for all networks and all scenarios, indicating that the experiments with higher incidence take more time to finalize the execution compared to the experiments with lower incidence. This means that the type of query command $Q$ impacts on the execution of the program; the higher the incidence, the more bi-triangles to be enumerated and the longest the program will take to finish.

\subsection{E3: Performance Analysis}\label{sub:sec:exp-3} 
\paragraph{Goal} In this experiment, we take measurements on one network, to gather data about the use of memory allocation and threads on \acrshort{ghc} during the execution of \acrshort{dpbt}.

\paragraph{Procedure} This experiment gathers three metrics describe in \autoref{sub:metric}: \emph{GHC Productivity}, \emph{Distribution of Threads per Core} and \emph{Distribution of Allocated Memory per Data Type}.
Using \mintinline{shell}{ThreadScope} \cite{threadscope} tool we measure \emph{GHC Productivity}, \emph{Distribution of Threads per Core}, and using \mintinline{shell}{eventlog2html} \cite{eventlog2html} tool, we measure \emph{Distribution of Allocated Memory per Data Type}.
For both cases, we have run these tools using one network only and one scenario. This is because both tools enable profiling flags at compilation level on \acrshort{ghc}, penalizing performance.

\begin{table}[H]
\centering
\resizebox{1\textwidth}{!}{%
\begin{tabular}{|c|c|c|c|c|c|}
  \hline
  \textbf{Tool} & \textbf{Network} & \textbf{Scenario ID} & \textbf{Exec Flags} & \textbf{Query} & \textbf{Timeout}\\
  \hline
  \texttt{ThreadScope} & Dbpedia & VU-L & \texttt{+RTS -A10G -H10G -c -N8 -l -s -RTS} & \texttt{by-vertex 93} & 24 hours\\
  \hline
  \texttt{eventlog2html} & Dbpedia & VU-L & \texttt{+RTS -A10G -H10G -c -N8 -l -s -RTS} & \texttt{by-vertex 93} & 24 hours\\
  \hline
\end{tabular}
}
\caption[{[EE] E3 Procedure}]{This table shows the experiments scenario run for each of the tools. Notice the increase of \texttt{-A} and \texttt{-H} to support more memory allocation due to the profiling analysis}
\label{table:e3:def}
\end{table}

As observed in \autoref{table:e3:def}, although we have selected the biggest network, we are only running the tools to gather data using VU-L, which enumerate less \acrshort{bt}. Therefore, this allows for the retrieval of profiling information, such as multithreading details and memory allocation.

\paragraph{Observed Results}\label{sub:sec:res:e3}
We can divide the analysis of this section into two: memory consumption and multithreading.

\paragraph{Multithreading} Regarding multithreading, we have gathered multithreading metrics of different time slots of the total execution time of Moreno Crime network run.
We could not analyze bigger networks due to the huge amount of data gathered that make the program timeout for this experiment. The timeout on this case was set on $24$ hours and the program timed out running out of memory.
As we can see in the overview, execution in \autoref{fig:exp:perf:1} all the cores (8) are running \acrfull{mut} time in threads almost during the whole execution of the program.
Running \acrfull{mut} most of the execution of the program, also indicates that there are few GC pauses, and running time is overtaken by \acrshort{mut} time and not GC. In fact, \acrshort{ghc} productivity on this run indicated $99.8\%$.
\begin{figure}[!htb]
  \centering
  \includegraphics[width=1\textwidth]{experiments/thread/general_overview}
  \caption[{[EE] Thread Metrics: General overview}]{This is a general overview of ThreadScope results over the experiments running. Green bar indicates \acrshort{mut} time. The distribution between different 8 green bars means that it is executing on the 8 assigned cores.}
  \label{fig:exp:perf:1}
 \end{figure}

ThreadScope~\cite{threadscope} output allows us to zoom in different portions of the execution time to analyze the results better. 
If we zoom in on the execution threads at the beginning and at the end, we are going to see that there is a moment when only one core is executing. 
At the middle of the execution, we are seeing more processing distributing evenly among cores with less use GC and higher \acrshort{mut} time.

\begin{figure}[!htb]
  \centering
  \begin{subfigure}[t]{0.3\textwidth}
   \includegraphics[width=1\linewidth, height=0.2\textheight]{experiments/thread/init}
   \caption{{[EE] Thread Metrics: Initial execution}}
   \label{fig:exp:perf:2}
  \end{subfigure}%
  \hspace{.3cm}%
  \begin{subfigure}[t]{0.3\textwidth}
    \includegraphics[width=1\linewidth, height=0.2\textheight]{experiments/thread/middle}
    \caption{{[EE] Thread Metrics: Middle of execution}}
    \label{fig:exp:perf:4}
   \end{subfigure}% 
   \hspace{.3cm}%
   \begin{subfigure}[t]{0.3\textwidth}
   \includegraphics[width=1\linewidth, height=0.2\textheight]{experiments/thread/end}
   \caption{{[EE] Thread Metrics: End of execution}}
   \label{fig:exp:perf:3}
  \end{subfigure}%
  \caption[{[EE] Thread Metrics: Partitioned}]{These plots depict different moments of the execution of the program after doing a zoom of the images allowing to scale down execution time view. The left image indicates the beginning of the execution. Middle image is the middle of the execution and right image is the end of the execution.}
\end{figure}

\paragraph{Memory Consumption} In the case of memory consumption, we have been able to measure the memory consumption for the biggest graph, \acrshort{dbpedia}. 
As it is known, enabling profiling downgrades the performance of execution time. Because of that, the program runs out of memory after $24$ hs., as we are going to see in the image. Although this, we have still been able to gather memory allocation data to conduct the analysis.

\begin{figure}[!htb]
  \begin{center}
     \includegraphics[width=1\textwidth] {experiments/mem/overview}
       \end{center}
     \caption[{[EE] Memory Metrics: Allocation by Data Type}]{This plot is showing the accumulated memory allocation size of each \acrshort{hs} Data Type throughout the execution of the program.}
     \label{fig:exp:mem:1}
 \end{figure}

As we can appreciate in \autoref{fig:exp:mem:1} the darkest blue area belongs to \texttt{MUT\_ARR\_PTRS\_CLEAN}. These types of objects are pointers to function.
It is clear that \texttt{MUT\_ARR\_PTRS\_CLEAN} is allocating $2G$ at most, and it is more than the rest of the objects. This also represents more than $50\%$ of the accumulated memory allocation during the whole execution time.
Below dark blue \texttt{MUT\_ARR\_PTRS\_CLEAN}, there is a lighter blue area, which represents the allocation of \texttt{Maybe} type. \texttt{Maybe} is the type that transfers data between stages through channels. It also represents the $25\%$ of the total allocated memory with less than $1G$ in total.
Continue below \texttt{Maybe} type, it is \texttt{IntSet} data type memory allocation, which is the type for storing the \acrshort{adwg} and \acrshort{abt} as we have described on \autoref{sec:prem:def}. This is even lower than the previous one, allocating up to $0.3G$ of memory that does not represent more than $7\%$ of the total memory. 
Alongside this, it is the \texttt{W} type which is storing \acrshort{awg}; it represents $0.5G$ which is a $10\%$ of the total allocation.
The rest $10\%$ of the total memory is evenly distributed among other data types that are not part of the specific \acrshort{dpfh} implementation, but from \acrshort{hs} in general.
 
\section{Discussion}\label{sec:exp:discussion}
\paragraph{E1: Continuous behavior} \autoref{sub:sec:res:e1} reports show that continuous behavior can be appreciated better in some scenarios and networks than others. For example, the scenarios containing queries with lower and medium incident vertices. 
The behavior on these cases can be explained because bi-triangles are aggregated based on triple $\ell = (l_l,l_m,l_u)$ (see \dref{def:abt}). Then, if the requested $l \in L$ matches with some of these vertices in $\ell$, all the $\hat{U}_l$ cartesian product need to be enumerated in a 6-cycle path \acrshort{bt} (see \dref{def:bt}). 
Therefore, if there are few \acrshort{bt} because the incidence level is lower, this is going to be delivered faster and continuously.
Moreover, the cases that exhibit less continuous behavior are Wang Amazon and Moreno Crime networks in the majority of their scenarios. In the case of Wang Amazon $8$ out of $9$ scenarios shows a \acrshort{dm} value of $0$. In the case of Moreno crime networks, it is reported a \acrshort{dm} value of $0$ in $5$ out of $9$ scenarios. 
We provide two arguments that support the explanation of this low continuous level in Wang Amazon and Moreno Crime. 
Firstly, the selection of the command $Q$ value (vertex or edge) for the different experimental scenario is \emph{pseudo-random}-- as it is commented in \autoref{sub:exp:sel-vals}. 
We have detected that for this particular experiment setup, even when the \emph{pseudo-randomly} chosen vertex belongs to the Lower Layer $L$, it is not a vertex with a high incidence as it should be.
Secondly, it could be explained by the fact of the topology of both graphs. Wang Amazon and Moreno Crime networks are the smallest of all the graphs used, in terms of the number of \acrshort{bt} as we can see in \autoref{table:exp:data-set}.
Therefore, results are delivered extremely fast and incremental results only can be appreciated in the vertices with high incidence.
In conclusion, we are able to answer [R1] and asses that we have built an incremental algorithm for enumerating \acrlong{bt}. 
The same conclusion can be obtained regarding question [R3]. We verify that, depending on the incidence of the vertex or edge, \acrshort{dpbt} is enumerating \acrshort{bt} in a continuous manner. This shows us \acrshort{dpbt} effectively implements a \emph{pay-as-you-go} model.

\paragraph{E2: Benchmark} Regarding \autoref{sub:sec:res:e2} we have noticed that for the case of edges scenarios in Opshal UC Forum network, the execution time is not consistent with what we expected regarding the incidence level. This could be explained by the \emph{pseudo-randomly} selection detailed in \autoref{sub:exp:sel-vals}.
In what relates to \emph{Total Execution time}, we have pointed out that DBpedia network is the one taking longer. This is perfectly explained by the characteristics and topology of the graph, since it is both the biggest graph in terms of edges and vertex, and it is the graph which proportionally has more \acrshort{bt}. 
We can answer the question [R2] because it can clearly be seen in the benchmark analysis and in \autoref{fig:exp:bench:2} that as long as the user request for command queries $Q$ that have more incidence in the graph and participates in more \acrshort{bt},
the execution time increases.


\paragraph{E3: Performance} In \autoref{sub:sec:res:e3} we have noticed the suitable distribution of threads among cores during the execution time. 
That exhibits a perfect fit with the \acrshort{dp} model since at the beginning, it considers all the filters and starts reading the input file, where there is less distribution of threads among the core. 
After that, we can see an even distribution and all cores busy when \acrshort{dp} executes the filters. Remember that each stage runs on its own thread. At the end, we observe a reduction in the distribution of the cores, because  the last part of the execution is done by $\obt$.
Regarding memory allocation, we have seen that most of the allocation is done by \texttt{MUT\_ARR\_PTRS\_CLEANER} data type.
In \acrlong{hs} \acrshort{mut} is the acronym of a thread evaluating an \acrshort{hs} expression.
Therefore, having $2G$ of allocated memory for \texttt{MUT\_ARR\_PTRS\_CLEANER} type means that there are many pointers allocated waiting for evaluating expressions. \acrshort{dpbt} implementation explains this behavior because it is spawning one
thread per stage, and in particular, that means one thread per filter instance as well. In the case of \acrshort{dbpedia} which contains $168.338$ vertices in 
$L$ according to \autoref{table:exp:data-set}, in the worst case \acrshort{dpbt} is spawning the same amount of threads for every run of this network. Since the execution of all these stages will not be released until it finishes the last $\ad$ (processing queries), all of them are waiting for the queries to be processed and executed.
Another important part of the allocated memory as we have seen in \autoref{sub:sec:exp-3} is distributed between \texttt{Maybe}, \texttt{IntSet} and \texttt{W} data types. This behavior is expected because \texttt{Maybe} carries data between stages which cannot be avoided due to the \acrshort{dp} model. Additionally,  both \texttt{IntSet} and \texttt{W} are the 
compressed form \acrshort{dpbt} uses for storing intermediate structures to build \acrshort{bt}, as we have described in \autoref{sec:prem:def}.
One of the proposed solutions for future work is to reduce the number of $\fbt$ for bigger graphs in order to reduce the number of allocated pointers waiting for commands.
Although memory allocation shows a linear growth in \autoref{fig:exp:mem:1} for the \texttt{MUT\_ARR\_PTRS\_CLEANER} type in this experiment, the rest of the memory allocation is not growing linearly, as we can see also in \autoref{fig:exp:mem:1}. This is a key factor on memory analysis since it is showing how \acrshort{dpfh} is compressing intermediate objects like \acrshort{awg}, \acrshort{adwg}, and \acrshort{abt} without penalizing the rest. 
In conclusion, we can answer the question [R4] as we have shown that threads are efficiently handled by \acrshort{hs} \acrshort{ghc} scheduler supporting the parallelization level that \acrshort{dp} requires. 
We can also state that memory management is efficiently handled as well by the analysis exposed before.
Additionally, memory allocation can be also improved by implementing a better matching algorithm for searching and deliver \acrshort{bt} like for example the ones describe by Lai et al.~\cite{Lai}. 
It was out of the scope of this work to solve the efficiency of the queries, as well as the underlying data structures improvements. 


\section{Chapter Summary}
In this chapter, we have explained the experiments conducted in order to answer our research question.
We first started presenting a summary of the research questions. After that, we have described the experimental configuration.
Then, the experimental procedure and results are reported. Finally, we have discussed the observed outcomes in terms of the main properties of our proposed approach.
To summarize, the main points observed during the experimentation are:
\begin{itemize}
  \item High values of the metric dief\@t that is indicating the continuous behavior of \acrshort{dpbt} exhibiting its capability of incrementally enumerates \acrshort{bt}.
  \item High values of the metrics \emph{Average Running Time} and \emph{Total Running Time} for scenarios which enumerates more bi-triangles, are suggesting an effective implementation of a \emph{pay-as-you-go} model of \acrshort{dpbt}.
  \item Results captured by the \texttt{ThreadScope} tool indicating an even distribution of the threads among processors, showing efficient use of the parallel model.
  \item Results gathered by the \texttt{eventlog2html} tool suggesting that memory consumption is efficiently handled in the intermediate objects that \acrshort{dpfh} collects on each filter and transfers between them. A proposal on how also improve this part was exposed in \autoref{sec:exp:discussion}.
\end{itemize} 

\chapter{Conclusions and Future Work}\label{conclusions}
In this chapter we present the future work that we think it is interesting to address as a result of this work, 
as well as the conclusions obtained.

\section{Future Work}
On the first thing that has been out of the scope of this work and it will be tremendously important to address is 
the selection of better data structures for handling the search in the compress \acrshort{bt} once the query command arrives to the $\ad$.
We have seen in \autoref{sub:sec:res:e3} how there is a memory bottleneck that is not allowing the execution of even bigger graphs, without 
paying so much extra cost for that. Performing an indexing over the edges and vertices could lead to important improvements in search, although there
is a trade of in terms of memory allocation.
On the other hand and continuing with the same idea, an important work to be addressed in the future as well is the distribution of the stages not only between threads,
but between machines as well. We believe that ramping up from a multithreading model to a distributed model will allow \acrlong{dp} to reduce the gap of memory consumption
for bigger instance graphs as well. The same as before, there is a trade of in terms of transfer data and the delay that this kind of distributed computations brings, but 
since we are delivering incremental results and we implement a \emph{pay-as-you-go} model the gain is bigger; without mentioning that the speed up and reliance in network communication
is higher than ever.

In other aspect of the computational model and more related with the \acrlong{hs} implementation, there are several improvements to be conducted there as well. One of them could be
to delegate the distribution to other stream processing system like Kafka~\cite{kafka} for example, and do the parallel processing of the splitted data with \acrshort{hs} consuming from them.
As well as more radical improvements can be done with \acrshort{dpfh}, there are some improvements that can be done to the framework itself. From more abstraction designed to help the user to write
pipelines with less effort and errors, to helping the thread schedule and memory management to even perform better than it is, for example unboxing some Boxed types~\cite{hs-unbox} which reduce Memory
footprint.

\section{Conclusions}
\acrfull{dp} has shown to be a quite efficient model for building parallel algorithms for solving complex problem like Enumeration of Bi-triangles in Bi-partite graph.
We have seen the strong of the paradigm in companion with a powerful language like \acrfull{hs} and the capabilities of combining both to not only implement the concepts
but also we have empirically shown that can deal with big networks like the \acrlong{dbpedia} which contains more than $300$ millons of bi-triangles.
In the experimental analysis we have also shown the capabilities to deliver incremental results that this solution offer. 
In conclusion we believe that the achieve results open a wide range of possibilities not only to improve the existing framework and algorithm, as well as to implement 
other complex problems where the use case does require a \emph{pay-as-you-go} model.


\appendix
\chapter{Appendix}
\section{Experiments}
\subsection{Benchmark Analysis}\label{app:exp:bench}
\begin{longtable}{|l|c|c|}
  \caption{$R^2$ goodness of the fit - Regression model}
  \label{table:app:exp:bench}\\
    \hline
   \textbf{Graph} & \textbf{Experiment} & \textbf{$R^2$}\\
   \hline
   opsahl-ucforum & VL-L & 0.954 \\
   \hline
   opsahl-ucforum & VL-M & 0.812 \\
   \hline
   opsahl-ucforum & VL-H & 0.999 \\
   \hline
   wang-amazon & VL-L & 0.992 \\
   \hline
   wang-amazon & VL-M & 0.933 \\
   \hline
   wang-amazon & VL-H & 0.849 \\
   \hline
   moreno-crime & VL-L & 0.999 \\
   \hline
   moreno-crime & VL-M & 0.997 \\
   \hline
   moreno-crime & VL-H & 0.97 \\
   \hline
   opsahl-ucforum & VU-L & 0.718 \\
   \hline
   opsahl-ucforum & VU-M & 0.972 \\
   \hline
   opsahl-ucforum & VU-H & 0.997 \\
   \hline
   wang-amazon & VU-L & 0.932 \\
   \hline
   wang-amazon & VU-M & 0.896 \\
   \hline
   wang-amazon & VU-H & 0.992 \\
   \hline
   moreno-crime & VU-L & 0.999 \\
   \hline
   moreno-crime & VU-M & 0.998 \\
   \hline
   moreno-crime & VU-H & 0.997 \\
   \hline
   opsahl-ucforum & E-L & 0.996 \\
   \hline
   opsahl-ucforum & E-M & 0.987 \\
   \hline
   opsahl-ucforum & E-H & 0.929 \\
   \hline
   wang-amazon & E-L & 0.954 \\
   \hline
   wang-amazon & E-M & 0.955 \\
   \hline
   wang-amazon & E-H & 0.979 \\
   \hline
   moreno-crime & E-L & 1.00 \\
   \hline
   moreno-crime & E-M & 1.00 \\
   \hline
   moreno-crime & E-H & 0.997 \\
   \hline
 \end{longtable}



\bibliographystyle{unsrtnat}
\bibliography{Thesis}

\end{document}

